El siguiente código es una implementación literal de los combinadores \( \bs{S} \), \( \bs{K} \) e \( \bs{I} \) descritos en \ref{defn:ski}; los valores de verdad y las operaciones booleanas descritas en \ref{sec:algebra-booleana}; los numerales de Church y las operaciones aritméticas descritas en \ref{sec:aritmetica}.

Se presentan dos implementaciones, en \texttt{Scheme} y en \texttt{Haskell}, sin embargo, es sencillo adaptar el código a lenguajes de programación que cuenten con lambdas (también llamadas clausuras).


\section{Scheme}
\label{ap:1:scheme}

En este listado de código se ha estilizado la palabra \texttt{lambda} con \( \bs{λ} \).

\begin{lstlisting}[language=scheme]
;; 
;; Combinadores SKI
;;
(define I (lambda (x) x))
(define K (lambda (x) (lambda (y) x)))
(define S (lambda (x) (lambda (y) (lambda (z) ((x z) (y z))))))

;; 
;; Álgebra booleana
;;
(define T K)
(define F (lambda (x) (lambda (y) y)))
(define IF (lambda (p) (lambda (m) (lambda (n) ((p m) n)))))
(define NOT (lambda (p) (((IF p) F) T)))
(define OR (lambda (p1) (lambda (p2) (((IF p1) T)(((IF p2) T) F)))))
(define AND (lambda (p1) (lambda (p2) (((IF p1) (((IF p2) T) F) F)))))

;; 
;; Aritmética
;; 
(define N:0 F)
(define N:0? (lambda (n) ((n (K F)) T)))
(define SUCC (lambda (n) (lambda (x) (lambda (y) (x ((n x) y))))))
(define N:1 (SUCC N:0))
(define N:2 (SUCC N:1))
(define N:3 (SUCC N:2))
(define N:4 (SUCC N:3))
(define + (lambda (m) (lambda (n) ((n SUCC) m))))
(define * (lambda (m) (lambda (n) ((n (+ m)) N:0))))
(define ^ (lambda (m) (lambda (n) ((n (* m)) N:1))))
(define N:0* (lambda (x) (lambda (y) (lambda (z) y))))
(define SUCC* (lambda (n) (lambda (x) (lambda (y) (lambda (z) (((n x) (z y)) x))))))
(define PRED (lambda (n) (lambda (x) (lambda (y) (((((n SUCC*) N:0*) x) y) I)))))
(define - (lambda (m) (lambda (n) ((n PRED) m))))
\end{lstlisting}

\section{Haskell}
\label{ap:1:haskell}

En este listado de código se ha estilizado el caracter \texttt{\textbackslash} como \( \bs{λ} \) y la secuencia \texttt{->} como \( \bs{\rightarrow} \).

\begin{lstlisting}[language=chaskell]
-- 
-- Combinadores SKI
--
C_I x = x
C_K x y = x
C_S x y z = x z (y z)

--
-- Álgebra booleana
--
B_T = C_K
B_F x y = y
B_If p m n = p m n
B_Not p = B_If p B_F B_T
B_Or p q = B_If p B_T (B_If q B_T B_F)
B_And p q = B_If p (B_If q B_T B_F) B_F

---
--- Aritmética
---
N_0 = B_F
N_0p n = n (C_K B_F) B_T
N_Succ n = \x y -> x (n x y)
N_1 = N_Succ N_0
N_2 = N_Succ N_1
N_3 = N_Succ N_2
N_4 = N_Succ N_3
N_Sum m n = n N_Succ m
N_Mul m n = n (N_Sum m) N_0
N_Exp m n = n (N_Mul m) N_1
N_02 x y z = y
N_Succ2 n = \x y z -> n x (z y) x
N_Pred n = \x y -> n N_Succ2 N_02 x y C_I
N_Sub m n = n N_Pred m
\end{lstlisting}

%%% Local Variables:
%%% mode: latex
%%% TeX-master: "main"
%%% End:
