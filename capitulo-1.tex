\section{Introducción}

El cálculo lambda es un sistema formal inventado en la década de 1920 por el
matemático y lógico Alonzo Church con la finalidad de describir las propiedades
mas básicas de la abstracción, aplicación y sustitución de funciones en un
contexto muy general~\cite{CardoneHindley:History}.\\

Cardone y Hindley mencionan en su artículo publicado en ``Handbook of the
History of Logic'' que el cálculo lambda es utilizado ampliamente en el cómputo
y la lógica de orden superior. Versiones modificadas del cálculo lambda
fundamentan importantes sistemas lógicos y lenguajes de programación. Ejemplos
de esto se puede encontrar en el desarrollo de los lenguajes
Lisp~\cite{Lisp:first}, Scheme~\cite{Scheme:first} y Haskell~\cite{Haskell:first};
así como en demostradores automáticos de teoremas como LCF~\cite{ML:first}.\\

El concepto de \emph{función} en el cálculo lambda es tratado como \emph{reglas}
en lugar de como gráficas, la noción de funciones como reglas se refiere, en
este sistema formal, al proceso de obtener un valor a partir de un argumento,
sin imponer restricciones sobre éste. En la época en la que Church formuló el
cálculo lambda, sistemas formales con la noción de función como la teoría de
conjuntos Zermelo-Fraenkel impedían que algunas funciones pudieran ser definidas
\cite{Barendregt:Bible}, un ejemplo de esto es la función \emph{identidad}, es
usual restringir el dominio de esta función dependiendo de el contexto en la que
es utilizada, sin embargo, en principio pudiera tener como dominio el conjunto
universal, lo cual no es permitido en la teoría de conjuntos tradicional ya que
dicho conjunto sería elemento de si mismo.\\

\section{Revisión histórica}
La siguiente revisión histórica resume el desarrollo del cálculo lambda de
acuerdo a una compilación de artículos sobre la historia de la lógica matemática
\cite{CardoneHindley:History}.\\

El sistema formal conocido hoy en día como el \emph{cálculo lambda} fue inventado por
el matemático y lógico Alonzo Church en 1928, sin embargo el primer
trabajo publicado en donde se hace mención de este sistema fue en el
artículo titulado \emph{A set of postulates for the foundation of
  logic}~\cite{Church:FoundationsLogic} en 1932, en donde presentó el desarrollo
de un sistema lógico a partir del cálculo lambda.\\

De 1931 a 1934 en la universidad de Princeton, Alonzo Church recibió
ayuda de dos estudiantes: Stephen Kleene y Barkley Rosser. Entre ellos
colaboraron e hicieron varias aportaciones, tanto en la lógica de Church como en
el sistema detrás de ella, el cual era el cálculo lambda. Uno de estos descubrimientos
fue la inconsistencia de la lógica propuesta por Church en su trabajo
de 1932.\\

En años siguientes se desarrolló el estudio de la teoría de reducción
en el cálculo lambda, lo cual llevó a la demostración del teorema de
confluencia de Church-Rosser en 1936 y aseguró la consistencia del
cálculo lambda puro, es decir, el sistema simple del cálculo lambda no contenía
inconsistencias. En este año también se encontró la equivalencia del
sistema con la \emph{lógica combinatoria}, un sistema inventado por
Haskell Curry basado en ideas similares a las de Church.\\

Trabajos sobre la representación de los números naturales en el cálculo lambda
se desarrollaron y se descubrió que las funciones del sistema que
operaban sobre estas representaciones eran mas poderosas que lo
anticipado, debido a una demostración de equivalencia entre esta clase
de funciones, las funciones recursivas de Herbrand-Gödel y las
funciones computables de Turing las cuales intentaban formalizar el
concepto informal de cómputo. Alonzo Church conjeturó en la tesis de
Church-Turing que las funciones definibles en el cálculo lambda capturaban
exactamente el concepto informal de cómputo. El clímax de esta serie
de trabajos sobre lo que es posible definir en el cálculo lambda fue cuando
Church dio una solución negativa al problema de decisión planteado por
Hilbert para la lógica de primer orden.\\

Por muchos años el cálculo lambda falló en atraer a otros estudiosos lógicos y
a pesar de haber servido para la primer demostración para
el problema de decisión, Turing poco tiempo después consiguió
desarrollar otra demostración utilizando un modelo mas transparente al
cómputo, incluso fuertes personajes que contribuyeron al desarrollo
del cálculo lambda como Kleene parecían favorecer otros sistemas. Sin embargo
Alonzo Church escribió un libro introductorio titulado \emph{Los
  cálculos de la conversión lambda} dirigido a una demografía menos
especializada lo cual amplió el estudio del sistema a otras ramas de
las matemáticas.\\

Durante la década de los cuarentas y cincuentas, los avances
relacionados con el cálculo lambda se enfocaron en cuatro temas: teoría de
tipos simples (simple type theory), teoría abstracta de reducción
(abstract reduction theory), reducciones en la lógica
combinatoria y en el cálculo lambda y sistemas ilativos (illative systems). En
su extensa revisión de la historia del cálculo lambda, Cardone y Hindley
expresan que  los avances de estas cuatro vertientes de estudio fueron
fundamentales para el desarrollo del cálculo lambda en las siguientes
décadas.\\

En los primeros sistemas basados en el cálculo lambda la aplicación de
funciones no tenía restricciones, una función podía ser evaluada en si
misma o en cualquier otra expresión válida del cálculo lambda. Versiones del
cálculo lambda en donde se restringe la aplicación de funciones utilizando
tipos son llamados cálculo lambda con tipos. Estos tipos son incrustados
incrustados en la definición de funciones como anotaciones para
restringir el rango de dominio.\\

Basándose en las ideas de Frank Ramsey y Leon Chwistek, Church
desarrolló una teoría de tipos basada en funciones para simplificar la
teoría de tipos de Russel y Whitehead. Su sistema de tipos fue
analizada y extendida por una serie de estudiantes de doctorado de
Princeton, cuyas contribuciones mas relevantes fueron dar definiciones
de estructuras algebraicas para el cálculo lambda con tipos y la extensión del
sistema para tipos transfinitos.\\

Alonzo Church no se involucró mucho en las teorías abstractas de
reducción, sin embargo, fuertes contribuciones fueron realizadas en la
década de los sesenta. El concepto central detrás de estas teorías fue
el de la propiedad de confluencia asociada a ciertos procedimientos
para la manipulación de expresiones en el cálculo lambda. Con el pasar de los
años el interés por estas teorías abstractas siguió desarrollándose y
llegó a extenderse mas allá de los sistemas formales del cálculo lambda y la
lógica combinatoria.\\

Los primeros trabajos de Kleene y Rosser con el cálculo lambda involucraban un
procedimiento de manipulación de expresiones llamado \betaredu. Haskell
Curry estudió mas a fondo esta reducción, sin embargo su trabajo (como
al igual que muchos científicos de la época) se vio interrumpido por
la segunda guerra mundial. Fue hasta después de 1950 que Curry retomó
esta línea de investigación y por 20 años trabajó tanto con el cálculo lambda
como con la lógica combinatoria vaciando una gran cantidad de
resultados y teoremas en dos monografías que se convirtieron en
referencias bibliográficas clave para futuros matemáticos.\\

A partir de la década de los sesenta, el interés por el cálculo lambda llegó a
varios científicos trabajando en ciencias de la computación, en
particular  en el área de lenguajes de programación.\\

Hubo tres figuras clave en el desarrollo del cálculo lambda en el área de la
programación: John McCarthy con el desarrollo del lenguaje de
programación LISP cuya notación de abstracción tiene muchas
similitudes con la del cálculo lambda; Peter Landin el cual propuso varias
construcciones para el lenguaje ALGOL utilizando expresiones del
cálculo lambda y subsecuentemente desarrollando la máquina abstracta SECD la cual
trataba expresiones del cálculo lambda como programas de computadora y
finalmente Corrado Böhm quien desarrolló el lenguaje CUCH en donde se
utilizaban los combinadores de Haskell Curry y el cálculo lambda.\\

A pesar de que los trabajos de McCarthy y Landin llegaron a ser mas
populares en la comunidad de programación. El trabajo que realizó Böhm
y los estudiantes que formó fue el que contribuyó mas al cálculo lambda desde
una perspectiva de ciencias de la computación. En sus años de estudio
del sistema puro del cálculo lambda, ellos plantearon y resolvieron varios
problemas asociados a la sintaxis del cálculo lambda y la lógica combinatoria.\\

El cálculo lambda ha sido reciclado por varias generaciones convirtiéndose en
tema de interés para personas estudiando otras ramas de las
matemáticas. A pesar de que estos trabajos se han alejado de los
objetivos originales de Church, las  nociones básicas desarrolladas
por el trabajo previo se han convertido en pilares fundamentales de
las ciencias de la computación.\\

El ganador del premio Turing de 1976 Dana Scott menciona en una
presentación de la celebración del centenario de Alan Turing
\cite{Scott:TuringCentenary} algunos
trabajos que han sido relevantes para el cálculo lambda en este milenio,
incluyo aquí algunos títulos de artículos que a pesar de tratar temas
mas avanzados que los presentados en este trabajo, pueden proveer una
perspectiva moderna del estudio del cálculo lambda.\\

\begin{itemize}
\item Moerdijk-Palmgren (2000) Predicative topos.
\item Ehrhard, Regnier (2003) Differential \(\lambda\)-calculus.
\item Mosses (2004) Modular Structural Operational Semantics.
\item Taylor (2005) A \(\lambda\)-calculus for real analysis.
\item Awodey-Warren (2006) Homotopy type theory.
\item Hudak, etal. (2010) Functional Reactive Programming.
\end{itemize}



\section{Enfoques del cálculo lambda}

De acuerdo a Barendregt~\cite{Barendregt:Bible} el estudio del cálculo lambda
tiene tres enfoques:\\

\begin{enumerate}[I]
\item Fundamentos de matemáticas.
\item Cómputo.
\item Cálculo lambda puro.
\end{enumerate}

\subsection{Fundamentos de matemáticas}

Los objetivos que se tenían cuando se dio origen al cálculo lambda eran
desarrollar una teoría general de funciones y extender esta teoría con nociones
lógicas para proveer fundamentos a la lógica y a partes de las
matemáticas~\cite{Barendregt:Bible}.\\

El primer objetivo es expresado por Church en~\cite{Church:LambdaConversion}, en
donde dice

\begin{quote}
El estudio de las propiedades generales de funciones, independientemente de su
aparición en cualquier dominio matemático, pertenece a la lógica formal o se
encuentra en la línea divisoria entre la lógica y la matemática. Este estudio es
la motivación original para los cálculos --- pero estos están formulados de tal
manera que es posible abstraer a partir del significado pretendido y
considerarlos simplemente como sistemas formales.
\end{quote}

Desde la publicación del artículo~\cite{Church:FoundationsLogic} se intentó
lograr el segundo objetivo, sin embargo, todos los intentos de proveer un
fundamento para las matemáticas fallaron. Casi inmediatamente después de la
publicación de Church se encontró una contradicción. Kleene y Rosser, dos
estudiantes de Church, formularon una variante de la paradoja de Richard para el
sistema lógico planteado~\cite{KleeneRosser:paradox}.\\

Después del descubrimiento de la paradoja Kleene-Rosser Church se sintió
desalentado en el estudio de los fundamentos de las matemáticas. En 1941 Church
publicó una teoría que corregía las inconsistencias observadas
en~\cite{KleeneRosser:paradox}, ésta fue llamada cálculo-\(\lambda I\) y era menos
ambiciosa y mas limitada que sus anteriores sistemas. Otros matemáticos y
lógicos continúan utilizando variaciones del cálculo lambda en sus
investigaciones sobre los fundamentos de las matemáticas.\\

\subsection{Cómputo}

La parte del cálculo lambda que solo trata con funciones resultó ser bastante
exitosa en el estudio del cómputo. Utilizando esta teoría, Church propuso una
formalización de la noción ``efectivamente computable'' la cual es llamada
\emph{lambda definibilidad}. Esta formalización resultó ser equivalente a el
concepto de \emph{computabilidad} propuesto por Turing.\\

Debido a que el cálculo lambda fue la inspiración de muchos lenguajes de
programación, investigación en la semántica de los lenguajes se realizó
utilizando este sistema. Peter Landin en la década de 1960 realizó una
traducción del lenguaje Algol al cálculo lambda, lo cuál le permitió expresar de
manera formal la semántica operacional del lenguaje, es decir, cómo debía
ejecutarse un programa escrito en Algol.\\

En 1969 Dana Scott logró expresar la semántica denotacional de los lenguajes de
programación utilizando el cálculo lambda, esto permitió poder expresar de
manera formal el significado de los programas escritos en un lenguaje de
programación.\\

\subsection{Cálculo lambda puro}

Los anteriores dos enfoques del cálculo lambda se desarrollaron extendiendo la
teoría básica con sistemas lógicos y computacionales para utilizar al cálculo
lambda como herramienta o como un lenguaje básico. Otro enfoque es que el objeto
de estudio sea el sistema en sí.\\

El estudio del cálculo lambda en sí consiste principalmente en estudiar las
funciones como reglas partiendo de la identificación de las fórmulas bien
formadas o \emph{términos} que pueden ser expresados en el sistema. Las
investigaciones en el estudio del cálculo lambda permiten resolver problemas
como la identificación de funciones sobre términos que se pueden definir como
términos en el cálculo lambda o mecanismos para poder manipular un término para
obtener otro y la relación entre ellos.\\


%%% Local Variables:
%%% coding: utf-8
%%% mode: latex
%%% TeX-master: "main"
%%% End: