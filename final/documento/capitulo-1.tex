El cálculo \( λ \) fue creado por Alonzo Church en 1932 con la finalidad de expresar, manipular y estudiar funciones para el desarrollo de los fundamentos de la lógica y las matemáticas \cite{Church:FoundationsLogic}. A lo largo de la historia, este sistema se ha adaptado para el estudio de los fundamentos de la computación y como sustento teórico para el desarrollo de lenguajes de programación.

En este capítulo se presentan de manera informal los aspectos más elementales del cálculo \( λ \), en la primer sección se describe la notación funcional usada en el cálculo en comparación a la notación matemática usual; en la segunda sección se describen los objetos matemáticos válidos del cálculo, sus estructuras sintácticas y las expresiones que se pueden construir con estas estructuras; en la tercer sección se describen los mecanismos básicos para manipular y transformar los objetos del cálculo y en la cuarta sección se presentan las nociones de equivalencia y el significado de las expresiones.

El contenido de este capítulo está basado en los primeros dos capítulos del libro ``The Lambda Calculus, Its Syntax and Semantics'' por H.P. Barendregt \cite{Barendregt:Bible}; los capítulos 1 y 3 del libro ``Lambda Calculus and Combinators, an Introduction'' por J.R. Hindley y J.P. Seldin \cite{HindleySeldin:LambdaCalculusAndCombinators}; y la monografía titulada ``The Calculi of Lambda-Conversion'' por Alonzo Church \cite{Church:LambdaConversion}.

\section{Notación funcional}
\label{sec:notacion-funcional}

El concepto de \emph{lenguaje} es de suma importancia en el cálculo \( λ \), informalmente el lenguaje se puede asociar con el concepto de \emph{notación}, es decir, una manera de escribir cosas. La notación del cálculo \( λ \) consiste en una manera alternativa de escribir funciones.

La definición de función en la matemática clásica es el de una relación entre un conjunto de entradas, llamado \emph{dominio} y un conjunto de salidas, llamado \emph{codominio}. Esta relación tiene además la propiedad de que cada elemento del dominio se asocia exactamente con un elemento del codominio, formalmente, sean \( A \) y \( B \) dos conjuntos, una función \( f \) con dominio \( A \) y codominio \( B \) es un subconjunto del producto cartesiano \( A \times B \), tal que para toda \( a \in A \), existe \( b \in B \) tal que \( (a,\ b) \in f \) y si \( (a,\ b^{\prime}) \in f \) con \( b^{\prime} \in B \), entonces \( b=b^{\prime} \).

Las funciones tienen varias maneras de ser representadas. En la definición anterior la representación es la de pares ordenados, en donde la primer componente del par es un elemento en el dominio y la segunda es un elemento en el codominio. Dependiendo del uso que se le da a las funciones, puede ser conveniente representarlas simbólicamente con expresiones, gráficamente con dibujos, numéricamente con tablas o incluso verbalmente con palabras.

Por ejemplo, la función \( f : \mathbb{Z} \to \mathbb{Z} \) tal que \[ f = \{ (x,\ 2x+y) \mid \text{ para todo } x\in\mathbb{Z} \} \] se escribe con la notación convencional como \[ f(x) = 2x+y \] En la notación funcional usada en el cálculo \( λ \), \( f \) se escribe como \[ f = λx.2x+y \] Para denotar la evaluación de una función para un valor en concreto, por ejemplo el \( 3 \), se escribe con la notación convencional como \[ f(3) \] mientras que en la notación \( λ \) se escribe \[ (f\ 3) \] A pesar de ser similares, la notación \( λ \) hace que sea posible denotar funciones anónimas, es decir, funciones que no están asociadas a un nombre como \( f \). Expresar funciones anónimas con la notación usual mencionada es inconveniente ya que al leer \( 2x+y \) no se puede estar seguros si la variable es \( x \) o \( y \), por otra parte, con \( λx.2x+y \) no existe esta ambigüedad.

La capacidad de denotar simbólicamente funciones anónimas resulta ser de utilidad ya que nos permite escribir funcionales o funciones de orden superior, es decir, aquellas funciones cuyo dominio o codominio son a su vez funciones. Por ejemplo \[ f = λg.λx.2(g\ x) \] denota una función que toma como argumento una función \( g \) y resulta en una función anónima que toma como argumento un valor \( x \) y resulta en el doble de la evaluación de \( g \) de \( x \). Desarrollando la evaluación de \( f \) en \( λy.3y \) se tiene que
\begin{align*}
(f\ λy.3y) 	&= ((λg.λx.2(g\ x))\ λy.3y) \\
			&= λx.2((λy.3y)\ x) \\
            &= λx.2(3x) \\
            &= λx.6x
\end{align*}

En varias áreas de las matemáticas se utilizan otras notaciones funcionales para expresar funciones anónimas. Por ejemplo, la notación \( \mapsto \) es similar a la notación \( λ \) y puede ser usada para denotar los ejemplos anteriores. Con esta otra notación, la función \( f = λg.λx.2(g\ x) \) es escrita \[ f = g \mapsto (x \mapsto 2g(x)) \] mientras que el desarrollo de la evaluación de \( f \) en \( y \mapsto 3y \) es
\begin{align*}
f(y \mapsto 3y) 	&= [g \mapsto (x \mapsto 2g(x))](y \mapsto 3y) \\
					&= x \mapsto 2[y \mapsto 3y](x) \\
            		&= x \mapsto 2(3x) \\
            		&= x \mapsto 6x
\end{align*}

A pesar de poder utilizar notaciones funcionales equivalentes pero sintácticamente diferentes a la del cálculo \( λ \), en este trabajo se opta por utilizar la notación \( λ \) debido a su ubicuidad en los textos que tratan temas relacionados con este cálculo.

Poniendo a un lado las cuestiones superficiales, la diferencia conceptual más importante entre el cálculo \( λ \) y el enfoque clásico del estudio de funciones es que este cálculo permite expresar únicamente funciones, evaluaciones y variables. Otros objetos matemáticas como los números y los conjuntos no son permitidos en el cálculo \( λ \). Las implicaciones de esto son fuertes ya que en la matemática clásica una función es representada como un conjunto y la notación utilizada al tratar funciones es únicamente ``azúcar sintáctico'' sobre la notación usada en teoría de conjuntos, mientras que en el cálculo \( λ \) todo es representado como cadenas de símbolos de un lenguaje formal.

En el estudio del cálculo \( λ \) se distinguen dos lenguajes: el lenguaje de las expresiones y el \emph{metalenguaje}. El lenguaje de las expresiones es un \emph{lenguaje formal} que especifica las secuencias de símbolos que representan expresiones válidas y se relaciona con las clases de objetos del cálculo que son válidos manipular, comparar y representar. Por otro lado, el metalenguaje es un lenguaje informal que permite describir cómo es que estas expresiones son manipuladas y analizadas, así como los mecanismos para representar conceptos y objetos matemáticos en forma de expresiones.

Es posible utilizar la definición de función basada en conjuntos para describir operaciones o transformaciones de expresiones y utilizar lógica de primer orden o aritmética para aseverar o cuantificar propiedades, sin embargo estos objetos matemáticos son parte del metalenguaje y no del lenguaje del cálculo \( λ \).

\section{Expresiones}
\label{sec:expresiones}

Existen tres clases de expresiones en el cálculo \( λ \): los \emph{átomos}, las \emph{abstracciones} y las \emph{aplicaciones}.

Las expresiones más simples son los \emph{átomos}, estos son objetos sintácticos usualmente representados con un símbolo como \( x \), \( y \) o \( z \). Los átomos son la representación simbólica de las variables \cite[p.~577]{Barendregt:Bible}.

Cuando se tiene una expresión que se conforma de varias \( x \) se refiere al símbolo como ``el átomo \( x \)'', sin embargo es posible tener diferentes variables representadas con el mismo símbolo en la expresión y para referirse a una de ellas en particular se debe especificar en que parte de la expresión se encuentra la variable \( x \) a la que se hace referencia.

En matemáticas y en computación es usual \emph{nombrar} valores, por ejemplo, declarar que \( x = 8 \) y expresar a \( x \) en diferentes contextos que no necesariamente son expresiones lógicas y funciones. En el cálculo \( λ \), los nombres son parte del metalenguaje y no del lenguaje de las expresiones.

Al igual que en la matemática clásica y en la lógica de primer orden, las variables más relevantes son las que se \emph{cuantifican}, por ejemplo en el estudio de funciones, la \( x \) en \( f(x) = M \) y en el estudio de enunciados lógicos, la \( y \) en \( \forall y\ M \) o \( \exists y\ M \). En el cálculo \( λ \) el cuantificador de variables es el símbolo \( λ \) de las abstracciones.

Las abstracciones son expresiones con \emph{estructura}, es decir, se conforman de \emph{partes} identificables. Las expresiones que son abstracciones representan la generalización de una expresión y son usualmente asociadas al concepto de \emph{función}. En el cálculo \( λ \) las abstracciones son representadas simbólicamente con un átomo y con otra expresión, se escriben de la forma \[ (λx.M) \] donde \( x \) es algún átomo llamado \emph{argumento} y \( M \) es alguna expresión ya sea otra abstracción, una aplicación o un átomo a la cual llamamos \emph{cuerpo} de la abstracción.

Un ejemplo de abstracción es \( (λx.x) \), esta se conforma por dos variables representadas con el mismo átomo, la expresión \( x \) por sí sola no es de mucha utilidad, sin embargo, al ser parte de esta abstracción toma importancia ya que si pensamos en las abstracciones como funciones, la \( x \) podrá ser cambiada por alguna otra expresión.

Las aplicaciones, al igual que las abstracciones, son expresiones con estructura. Se conforman por otras dos expresiones y se escriben de la forma \[ (M\, N) \] donde \( M \) y \( N \) son cualesquiera átomos, abstracciones o aplicaciones.

El concepto relacionado con las aplicaciones en la matemática clásica es el de \emph{evaluación de funciones}. En el cálculo \( λ \) se hace una distinción entre la estructura sintáctica para denotar una evaluación como \( f(x) \) y el acto de encontrar el valor asociado a \( x \) en \( f \). En el cálculo \( λ \) el primero se refiere a las aplicaciones y el segundo se refiere a las \emph{reducciones} las cuales serán abordadas más a detalle en la siguiente sección.

A pesar de estar asociado a la evaluación de funciones, una aplicación válida pudiera ser \[ (y\, (λx.x)) \] Esto puede resultar extraño ya que la parte izquierda de una evaluación es una función, las funciones están asociadas a las abstracciones y en este ejemplo \( y \) es una variable. Sin embargo, la \( y \) en la expresión pudiera estar cuantificada en la expresión \[ (λy.(y\, (λx.x)) \] y ahora al reducir esta expresión con otra abstracción como por ejemplo \( (λw.w) \), la \( y \) en el cuerpo se referirá a \( (λw.w) \) la cuál pudiera ser a su vez reducida con \( (λx.x) \). Es por esto que en el cálculo \( λ \) si una expresión es válida, sus partes también lo son.

Todas las combinaciones válidas de expresiones nos permiten formar estructuras sintácticas diversas y asignarles significados interesantes. Por ejemplo, considerando la expresión \( (λx.(x\, x)) \) y teniendo en cuenta la analogía entre abstracciones, funciones, aplicaciones, reducciones y evaluaciones se puede considerar que esta cadena de símbolos representan el concepto de duplicación de expresiones, ya que reducir \[ ((λx.(x\, x))\, y) \] resulta en \( (y\, y) \) y reducir \[ ((λx.(x\, x))\, (y\, y)) \] resulta en \( ((y\, y)\, (y\, y)) \). Por otro lado, reducir \[ ((λx.(x\, x))\, (λx.(x\, x))) \] resulta en sí misma. A este tipo de expresiones se les llaman ``quines'' \cite[pp.~431--437]{Hofstadter:GEB} término originalmente asociado a una paradoja sobre sistemas lógicos \cite{Quine:Paradox}. En la actualidad, el término ``quine'' hace referencia a un programa cuya \emph{salida} es el programa mismo.

Otro ejemplo de una expresión interesante es\[ (λf.(λx.(f\, (f\, x)))) \]Al reducirla con una expresión cualquiera \( M \) se obtiene \( (λx.(M\, (M\, x))) \) si esta expresión resultante es reducida con otra expresión cualquiera \( N \) el resultado es \( (M\, (M\, N)) \). Este proceso puede pensarse como si una función de dos argumentos fuera evaluada en dos valores, bajo esta analogía una abstracción cuyo cuerpo sea otra abstracción se pudiera considerar como una abstracción con dos argumentos.

Asignarle significados a expresiones del cálculo \( λ \) es una tarea que debe realizarse con cuidado. Debido a que usualmente se trabaja con este cálculo en contextos rigurosos es de extrema importancia ser consistentes con la interpretación que se le está dando a las expresiones. Cuando se desea representar en el cálculo \( λ \) alguna función, se deben \emph{codificar} como expresiones del lenguaje los elementos del dominio y el codominio de la función, así como las operaciones entre elementos de ambos conjuntos.

Por ejemplo, para representar la función \( f \colon \mathbb{N} \to \mathbb{N} \) dada por \( f(x)=x^{2} \) primero se deben codificar los números naturales con expresiones del cálculo \( λ \), esta codificación debe ser acompañada de la codificación de las operaciones aritméticas elementales como la suma y resta así como de los predicados sobre números naturales como discriminar entre el mayor de dos números o si un número es cero; posteriormente se debe expresar la operación de exponenciación de cualquier número natural como base y el número \( 2 \) como exponente. La codificación es similar a la implementación de algoritmos y estructuras de datos en lenguajes de programación.

El hecho de tener un lenguaje tan reducido y minimalista para las expresiones del cálculo \( λ \) nos obliga a entender con detalle y precisión todos los procesos de manipulación y transformación de expresiones y siendo que todo lo que se representa con el cálculo \( λ \) debe ser codificado como expresiones, los objetos representados pueden ser entendidos de la misma manera.

\section{Operaciones}
\label{sec:operaciones}

En el cálculo \( λ \) se pueden realizar algunas operaciones para transformar expresiones, estas operaciones son parte del metalenguaje y consisten de una serie de cambios mecánicos a la estructura de las expresiones de acuerdo a un criterio particular.

En la sección \ref{sec:expresiones} se mencionó que las \emph{reducciones} son análogas al acto de evaluar una función en un valor. Considerando la función \( f(x) = 2x+3 \) la evaluación de \( f(4) \) resulta en \( 11 \) el cuál puede ser obtenido sustituyendo en \( 2x+3 \) la \( x \) por \( 4 \). Las operaciones más importantes del cálculo \( λ \) se basan en este proceso de sustitución y serán abordadas en esta sección.

\subsection{Sustitución}
\label{sec:op-sustitucion}

La \emph{sustitución} es la operación que nos permite transformar una expresión cualquiera \( M \) cambiando las apariciones de un átomo \( x \) por alguna otra expresión \( N \), este procedimiento se denota

\[ \subst{M}{x}{N} \]

En muchos casos la operación de sustitución se puede realizar fácilmente, consideremos por ejemplo la operación\[ \subst{x}{x}{y} \]ya que \( x \) es la expresión en donde \( x \) es cambiada por \( y \), el resultado es \( y \). De manera similar, se pueden involucrar expresiones más complejas en donde es sencillo encontrar el resultado de la sustitución. Por ejemplo\[ \subst{(x(x(λy.y)))}{x}{z} \] resulta en \( (z(z(λy.y))) \) y \[ \subst{(x\, x)}{x}{(λw.w)} \] resulta en \( ((λw.w)(λw.w)) \)

Sin embargo, existen algunos detalles sutiles de la sustitución que se deben tomar en cuenta para evitar obtener expresiones erróneas, en particular cuando se sustituye en expresiones que contienen abstracciones. Para ilustrar estos casos especiales, consideremos la abstracción análoga a la función constante \( f(x)=y \): \[ (λx.y) \] Reducir esta expresión en cualquier otra expresión resulta siempre en el átomo \( y \). Si se realiza la sustitución \[ \subst{(λx.y)}{y}{z} \] se obtiene la expresión \( (λx.z) \), la cual también es análoga a una función constante pero con el átomo \( z \). Si no se tiene cuidado, sustituir un átomo por otro en esta abstracción puede resultar en una expresión con diferente \emph{interpretación}. Por ejemplo, cuando se considera la misma abstracción pero se desea sustituir la \( y \) por una \( x \): \[ \subst{(λx.y)}{y}{x} \] el resultado de simplemente cambiar el átomo \( y \) por el átomo \( x \) resultaría en\[ (λx.x) \] la cuál no tiene la interpretación de la función constante, esta expresión  se pudiera considerar análoga a la función identidad.

La operación de sustitución en el cálculo \( λ \) así como en otras áreas de la matemática y la lógica no admite cambiar las expresiones de esta manera. Para entender la operación de sustitución se tiene que pensar que lo que le da sentido a una variable \( x \) es una \( λ x \). Consideremos la expresión \[ (λx.(λy.((x\, y)\, z))) \] el átomo \( x \) que aparece en el cuerpo de la expresión se dice ser una variable \emph{ligada} a la \( λ x \), la cual se puede pensar como una especie de ``referencia'' a la expresión a la que la abstracción es aplicada, esto limita a la operación de sustitución a no \emph{romper} la referencia de una variable ligada. De igual manera, el átomo \( y \) es una variable ligada a la \( λ y \) y debe mantener su referencia bajo la operación de sustitución. Sin embargo, el átomo \( z \) es lo que se llama una variable \emph{libre}: No está en el \emph{alcance} de alguna \( λ z \) y puede ser libremente sustituida por alguna otra expresión.

En el caso de \( \subst{(λx.y)}{y}{x} \) se pretende sustituir la variable libre \( y \) por una expresión \( x \), un cambio de átomos de \( y \) por \( x \) \emph{introduciría} una referencia a la \( λ x \) de la expresión, la cuál no existía previamente. Con esto se identifica que la operación de sustitución \( \subst{M}{x}{N} \) no debe introducir o eliminar referencias a alguna \( λ \) en \( M \).

\subsection{Cambio de variable ligada}
\label{sec:op-cambio-var-ligada}

Para definir el comportamiento de la operación de sustitución cuando se presentan los conflictos mencionados se debe considerar otra operación llamada \emph{cambio de variable ligada}. Se parte de la observación que en una expresión del cálculo \( λ \), las referencias entre \( λ x \) y las variables \( x \) son más importantes que el símbolo con el que se representa el átomo. En las expresiones simbólicas de funciones sucede lo mismo, al expresar \( f(x)=x^{2} \) y \( f(y)=y^{2} \) hacemos referencia a la misma regla de correspondencia y por lo tanto a la misma función. En el cálculo \( λ \), cambiar el símbolo que representa el átomo \( x \) en la expresión \( (λx.y) \) por otro símbolo no utilizado nos permite realizar la sustitución sin abordar los conflictos mencionados.

Para realizar la sustitución \( \subst{(λx.y)}{y}{x} \) primero se realiza un cambio de variable ligada en \( (λx.y) \) para obtener, por ejemplo, \( (λz.y) \), con la cuál se puede replantear la operación como \[ \subst{(λz.y)}{y}{x} \] la cuál resulta en \( (λz.x) \) y mantiene la interpretación de una función constante.

Cuando se realiza un cambio de variable ligada sobre una abstracción \( (λx.M) \) se cambia tanto el átomo \( x \) acompañado por la \( λ \), llamada variable \emph{vinculada} como todas las variables ligadas a \( λx \) en el cuerpo de la abstracción, también llamado \emph{alcance de} \( λ x \) a menos que en \( M \) se encuentre una expresión de la forma \( (λx.N) \), ya que las \( x \) en \( N \) hacen referencia a la \( λx \) de \( (λx.N) \) no de \( (λx.M) \).

La operación de cambiar una variable ligada \( x \) a \( y \) en una abstracción \( (λx.M) \) se denota \( \specialcontract{α}{y} \) y se define en base a la operación de sustitución como \[ (λx.M) \specialcontract{α}{y} (λy.\subst{M}{x}{y}) \]

La definición de la operación de sustitución es recursiva y hace uso de la operación de cambio de variable ligada, considerando a \( x \), \( y \), \( z \) como átomos diferentes y \( M \), \( N \) y \( P \) como expresiones cualquiera se define la sustitución de acuerdo a los siguientes casos:
\begin{itemize}
\item \( \subst{x}{x}{M} = M\);
\item \( \subst{y}{x}{M} = y\);
\item \( \subst{(M\, N)}{x}{P} = (\subst{M}{x}{P} \subst{N}{x}{P}) \);
\item \( \subst{(λx.M)}{x}{N} = (λx.M) \) debido a que las referencias a \( x \) no deben eliminarse;
\item \( \subst{(λy.M)}{x}{N} \) resulta en:
  \begin{itemize}
  \item \( (λy.M) \) cuando \( x \) no es una variable libre en \( M \),
  \item \( (λy.\subst{M}{x}{N}) \) cuando \( x \) es una variable libre en \( M \) pero \( y \) no es una variable libre en \( N \) debido a que esto introduciría una referencia a \( λ y \),
  \item \( (λz.\subst{\subst{M}{y}{z}}{x}{N}) \) cuando \( x \) es una variable libre en \( M \) y \( y \) es una variable libre en \( N \).
  \end{itemize}
\end{itemize}

\subsection{Reducción de aplicaciones}
\label{sec:op-reduccion}

La operación de \emph{reducción de aplicaciones} es el mecanismo mediante el cual se puede ``concretar'' una abstracción haciendo uso de otra expresión como valor de la variable enlazada. De manera similar a como se pudiera realizar la evaluación de funciones, el concretar una función consiste en sustituir todas las apariciones del argumento por el valor en el que la función es aplicada.

La reducción de aplicaciones se puede realizar sobre aplicaciones en donde la expresión izquierda es una abstracción, se denota \( \contract{β} \) y al igual que el cambio de variable ligada, también se define en base a la operación de sustitución. Sea \( (λx.M) \) una abstracción cualquiera y \( N \) una expresión cualquiera, la reducción de \( ((λx.M)\, N) \) se define como \[ ((λx.M)\, N) \contract{β} \subst{M}{x}{N} \]

Cuando la aplicación que se reduce es parte de otra expresión, se suele denotar la aplicación sobre la \( \contract{β} \). Por ejemplo el procedimiento para reducir la expresión \( (λx.((λw.w)\, x)) \) a \( (λx.x) \) es
\begin{align*}
(λx.((λw.w)\, x)) \specialcontract{β}{((λw.w)\, x)} (λx.\subst{w}{w}{x}) = (λx.x)
\end{align*}
En la mayoría de los casos, la aplicación reducida se puede inferir observando la transformación de expresiones, por lo tanto esta especificidad será omitida en este trabajo.

Las operaciones de sustitución, cambio de variable ligada y reducción se pueden aplicar en varios pasos de un proceso de transformación sin especificar qué operación en particular se realiza, denotando el paso de una expresión a otra con el símbolo \( \contract{} \). Por ejemplo la expresión \( ((λy.(λx.y))\, x) \) se puede transformar a \( (λz.x) \) siguendo los siguientes pasos:
\begin{align*}
((λy.(λx.y))\, x) 	&\contract{} ((λy.(λz.y))\, x) \\
					&\contract{} \subst{(λz.y)}{y}{x} \\
                    &\contract{} (λz.x)
\end{align*}

El cálculo \( λ \) es un sistema maleable y se permite definir operaciones arbitrarias sobre expresiones para estudiar cómo el sistema se comporta en diferentes contextos, por ejemplo, se puede considerar una operación similar a la sustitución que permite introducir referencias a una o más \( λ \) en una expresión, sin embargo, el presente trabajo está constituido para entender plenamente las ideas centrales del cálculo \( λ \) haciendo uso principalmente de las operaciones de \emph{sustitución}, \emph{cambio de variable ligada} y \emph{reducción de aplicaciones}.

\section{Equivalencias}
\label{sec:equivalencias}

El cálculo \( λ \) se considera formalmente como una \emph{teoría ecuacional}, esto significa que los axiomas de su teoría son ecuaciones que relacionan expresiones del lenguaje. Esto hace que el concepto de \emph{equivalencia} de expresiones sea de suma importancia.

Es tan relevante la formalización de las nociones de equivalencia que considerar alguna equivalencia entre dos expresiones que se escriben diferente puede cambiar por completo el sistema formal que se estudia. En el desarrollo histórico del cálculo \( λ \), el estudio de los criterios que permiten establecer que dos expresiones son equivalentes ha dado pie a una gran diversidad de variantes de la teoría original; es por ello que en la literatura se suele hablar de \emph{los cálculos \( λ \)} y no únicamente de un cálculo \( λ \).

Como se aborda en la sección \ref{sec:operaciones}, con la operación de sustitución se puede transformar expresiones del cálculo \( λ \) y definir otras operaciones como el cambio de variable ligada y la reducción de aplicaciones. Usualmente, las transformaciones de expresiones se pueden asociar a nociones de equivalencia. En terminología del cálculo \( λ \), las nociones de equivalencia entre expresiones son asociadas a la propiedad de \emph{convertibilidad}, la cual significa que si dos expresiones \( M \) y \( N \) son equivalentes en el sistema, es posible transformar \( M \) a \( N \) y viceversa por medio de un número finito de operaciones.

En esta sección se describen algunos criterios de equivalencia entre expresiones del cálculo \( λ \) y las maneras en las que las equivalencias se relacionan entre sí.

\subsection{Equivalencia sintáctica}
\label{sec:equivalencia-sintactica}

La \emph{equivalencia sintáctica} es una relación binaria entre expresiones que no está asociada a una transformación. Se considera como una equivalencia trivial, ya que asevera la igualdad entre dos expresiones que son escritas exactamente igual, símbolo por símbolo a excepción de abusos de notación. Por ejemplo, la expresión \( \sin^{2}(x) \) es un abuso de notación de \( \left( \sin(x) \right)^{2} \) y ambas se consideran exactamente iguales. En el cálculo \( λ \), la equivalencia sintáctica es denotada como

\[ M \synteq N \]

cuando \( M \) es sintácticamente la misma expresión que \( N \).

Todos los cálculos \( λ \), al igual que la mayoría de los sistemas formales, comprenden la noción de equivalencia sintáctica. Sin embargo las equivalencias más interesantes son las que involucran transformaciones entre expresiones.

\subsection{\texorpdfstring{\( α \)-convertibilidad}{alfa-convertibilidad}}
\label{sec:alfa-convertibilidad}

La operación de cambio de variable ligada se relaciona con una equivalencia estructural entre dos expresiones. Cuando se realiza esta operación no se modifica la estructura de la expresión, únicamente se modifica el átomo usado para representar una variable vinculada y las variables enlazadas a ella.

Considerando la expresión análoga a la función identidad \( (λx.x) \) se observa que tiene la misma estructura que \( (λy.y) \) y que \( (λz.z) \), estas tres representan el mismo concepto. De igual manera otras expresiones como \( ((x\, y)z) \) o \( (λw.x) \) son estructuralmente equivalentes a \( ((a\, b)c) \) y \( (λf.h) \) respectivamente. A pesar de que no se escriben sintácticamente igual, la correspondencia que hay entre las posiciones de los átomos en una y otra expresión nos permite considerarlas como equivalentes. Sin embargo, la operación de cambio de variable ligada no considera cambios de nombres a átomos que sean variables libres.

El criterio de equivalencia relacionado con la operación de cambio de variable ligada es llamada \( α \)-convertibilidad y se denota como \[ M \convertible{α} N \] para dos expresiones \( M \) y \( N \) en donde a partir de una cantidad finita de cambios de variable ligada en \( M \) o parte de \( M \) se pueda obtener \( N \).

Una técnica utilizada por algoritmos que verifican si dos expresiones \( M \) y \( N \) son \( α \)-convertibles es la de \emph{índices de De Bruijn}, esta transformación cambia la aparición de átomos por números naturales que representan la ``distancia'' de los átomos a las \( λ \) que hacen referencia. Por ejemplo, consideremos la expresión \[ (λz.((λy.(y(λx.x)))(λx.(z\, x)))) \] utilizando índices de De Bruijn se escribe como \[ λ (λ 1 (λ 1)) (λ 2\, 1) \] En la figura \ref{fig:debrujn} se puede observar de manera gráfica la transformación de una notación a otra para este ejemplo, visualizando las expresiones del cálculo \( λ \) como árboles.

\begin{figure}
  \centering
  \begin{tikzpicture}[level/.style={sibling distance=40mm/#1}]
    \node [draw] (term) {\( (λz.((λy.(y (λx.x))) (λx.(z\, x)))) \)};
    \node [below=2pt of term] (arrow1) {\( \Downarrow \)};
    \node [circle,draw,below=2pt of arrow1] (z) {\( λ z \)}
      child {
        node [circle,draw] (a) {\( λ y \)}
        child {
          node [circle,draw] (c) {\( y \)}
        }
        child {
          node [circle,draw] (d) {\( λ x \)}
          child {
            node [circle,draw] (g) {\( x \)}
          }
        }
      }
      child {
        node [circle,draw] (b) {\( λ x \)} 
        child {
          node [circle,draw] (e) {\( z \)}
        } 
        child {
          node [circle,draw] (f) {\( x \)}
        }
      };
      \node [below=120pt of z] (arrow2) {\( \Downarrow \)};
      \node [circle,draw,below=2pt of arrow2] (z2) {\( λ \)}
      child {
        node [circle,draw] (a2) {\( λ \)} 
        child {
          node [circle,draw] (c2) {\( 1 \)}
        }
        child {
          node [circle,draw] (d2) {\( λ \)} 
          child {
            node [circle,draw] (g2) {\( 1 \)}
          }
        }
      }
      child {
        node [circle,draw] (b2) {\( λ \)} 
        child {
          node [circle,draw] (e2) {\( 2 \)}
        }
        child {
          node [circle,draw] (f2) {\( 1 \)}
        }
      };
      \node [below=120pt of z2] (arrow3) {\( \Downarrow \)};
      \node [draw,below=2pt of arrow3](bruijn) {\( λ (λ 1 (λ 1)) (λ 2\, 1) \)};
  \end{tikzpicture}
  \caption{Transformación gráfica para notación de De Bruijn}
  \label{fig:debrujn}
\end{figure}

Una desventaja de utilizar la notación de De Bruijn es que ciertas expresiones del cálculo \( λ \) no pueden ser escritas, en particular, los átomos no pueden ser variables libres para que esta notación pueda ser utilizada.

\subsection{\texorpdfstring{\( β \)-convertibilidad}{beta-convertibilidad}}
\label{sec:beta-convertibildad}

Al igual que el cambio de variable ligada, la operación de reducción de aplicaciones es utilizada para describir un criterio de equivalencia entre expresiones. La idea básica de este criterio consiste en observar que al aplicar una abstracción \( (λx.M) \) a una expresión \( N \), el resultado de su reducción siempre es el mismo. De manera similar a la aplicación de funciones, cuando se define una función \( f(x)=x^{2} \), la aplicación \( f(3) \) se suele igualar al resultado de la aplicación: \( f(3)=8 \).

Esta relación de equivalencia es llamada \( β \)-convertibilidad y se denota como \[ M \convertible{β} N \] para dos expresiones \( M \) y \( N \) en donde \( N \) puede ser obtenida a partir de \( M \) por una cantidad finita de reducciones de aplicaciones, reducciones inversas y cambios de variable ligada.

La reducción inversa de \( P \) a \( P' \) significa que \( P' \contract{β} P \). Este proceso de transformación inverso a la reducción de aplicaciones se puede ilustrar considerando que \[ x \operatorname{\not\rightarrow_{β}} ((λy.y)\, x) \] pero \[ ((λy.y)\, x) \contract{β} x \] por lo tanto hay una reducción inversa de \( x \) a \( ((λy.y)\, x) \).

\subsection{Relaciones de equivalencia}
\label{sec:relaciones-de-equivalencia}

Todas las nociones de convertibilidad son relaciones de equivalencia, las cuales por definición cumplen con tres propiedades. Sea \( \sim \) una relación de equivalencia
\begin{enumerate}
\item Toda expresión \( M \) es equivalente a sí misma, es decir, \( M \sim M \). \label{enum:rela:a}
\item Si una expresión \( M \) es relacionada con una equivalencia a otra expresión \( N \), entonces \( N \) también es relacionada a \( M \), es decir \( M \sim N \implies N \sim M \). \label{enum:rela:b}
\item Si una expresión \( M \) se relaciona con una equivalencia a otra expresión \( N \) y \( N \) se relaciona con la misma equivalencia a \( P \), entonces, \( M \) y \( P \) se relacionan con esta equivalencia, es decir, \( M \sim N,\ N \sim P \implies M \sim P \). \label{enum:rela:c}
\end{enumerate}

La equivalencia sintáctica corresponde a la propiedad \ref{enum:rela:a} la cuál es llamada \emph{reflexividad}; al igual que la \( α \)-conversión y la \( β \)-conversión, la equivalencia sintáctica no está asociada a una regla de inferencia. En las propiedades \ref{enum:rela:b} y \ref{enum:rela:c} se tienen inferencias que parten de expresiones equivalentes y basado en si estas expresiones son equivalentes o no, ciertas propiedades se deben cumplir. La propiedad \ref{enum:rela:b} es llamada \emph{simetría}, mientras que la propiedad \ref{enum:rela:c} es llamada \emph{transitividad}.

La \( α \)-conversión y la \( β \)-conversión fueron definidas como criterios de equivalencia por separado y su definición cumple con las tres propiedades mencionadas a pesar de ser definidas en base a un procedimiento y no en una regla declarativa, sin embargo, es deseable referirse a una sola equivalencia de expresiones que tenga las propiedades de \emph{reflexividad}, \emph{simetría} y \emph{transitividad} y posteriormente considerar otras propiedades que la equivalencia deba de cumplir.

Al igual que Haskell Curry en \cite[p.~59]{Curry:CombinatoryLogicI} se utilizan las letras griegas \( α \) y \( β \) para referirse a las ecuaciones relacionadas con el cambio de variable ligada y la reducción de aplicaciones, mientras que las letras \( ρ \), \( σ \) y \( τ \) se refieren a las ecuaciones relacionadas con las propiedades de reflexividad, simetría y transitividad respectivamente, se retoma esta convención para elaborar la siguiente definición de una relación de equivalencia \( = \):
\begin{subequations}
  \begin{align}
    \label{sim:alpha} \tag{\( α \)}
    (λx.M) &= (λy.\subst{M}{x}{y}) \\
    \label{sim:beta} \tag{\( β \)}
    ((λx.M)N) &= \subst{M}{x}{N} \\
    \label{sim:rho} \tag{\( ρ \)}
    M &= M \\
    \label{sim:sigma} \tag{\( σ \)}
    M = N & \implies N = M \\
    \label{sim:tau} \tag{\( τ \)}
    M = N,\ N = P & \implies M = P
  \end{align}
\end{subequations}
Estas ecuaciones resultan describir una equivalencia muy parecida a la \( β \)-conversión: 
\begin{itemize}
\item Si \( M \contract{α} N \) entonces \( M = N \) por la ecuación \eqref{sim:alpha}
\item Si \( M \contract{β} N \) entonces \( M = N \) por la ecuación \eqref{sim:beta}
\item Si \( M \synteq N \) entonces \( M = N \) por la ecuación \eqref{sim:rho}
\item Ya que \( \contract{α} \) y \( \synteq \) son simétricas y la \( β \)-conversión contempla reducciones inversas: Si \( M \convertible{β} N \), entonces \( M = N \) por la ecuación \eqref{sim:sigma}
\item Ya que la \( β \)-conversión contempla una cantidad finita de transformaciones: Si \( M \convertible{β} N \), entonces \( M = N \) por la ecuación \eqref{sim:tau}
\end{itemize}
Sin embargo \( M = N \) no implica que \( M \convertible{β} N \) ya que la \( β \)-conversión contempla la transformación de partes de una expresión y \( = \) no. Para ilustrar esta diferencia consideremos la convertibilidad \[ (λf.((λx.(f\, x))\, y)) \convertible{β} (λf.(f\, y)) \] La secuencia detallada de transformaciones realizadas para pasar de la expresión de la izquierda a la de la derecha es \[ (λf.((λx.(f\, x))\, y))  \specialcontract{β}{((λx.(f\, x))\, y)} (λf.\subst{(f\, x)}{x}{y}) \contract{} (λf.(f\, y)) \]
En el primer paso se reduce una aplicación interna a la expresión, sin embargo, ni una ecuación de \( = \) establece una equivalencia entre la expresión original y el resultado de la primer transformación.

Para capturar la definición de \( β \)-convertibilidad con ecuaciones, es necesario considerar otras ecuaciones en \( = \). Las siguientes reglas, nombradas por Curry \cite[p.~59]{Curry:CombinatoryLogicI} como \( ν \), \( μ \) y \( ξ \), junto con las anteriores reglas de \( = \) completan la definición declarativa de \( β \)-convertibilidad:
\begin{subequations}
  \begin{align}
    \label{simbeta:nu} \tag{\( ν \)}
    M = N & \implies (M\, Z) = (N\, Z) \\
    \label{simbeta:mu} \tag{\( μ \)}
    M = N & \implies (Z\, M) = (Z\, N) \\
    \label{simbeta:xi} \tag{\( ξ \)}
    M = N & \implies (λx.M) = (λx.N)
  \end{align}
\end{subequations}

Con estas reglas y a partir de un razonamiento lógico, podemos demostrar la \( β \)-equivalencia entre dos expresiones.

\begin{align}\label{eq:es-convertible}
((λx.(f\, x)) y) &\convertible{β} (f\, y) &\text{por } β \\
(λf.((λx.(f\, x)) y)) &\convertible{β} (λf.(f\, y)) &\text{por } \eqref{eq:es-convertible}
\end{align}

Es posible incluir aún más reglas de equivalencia cuando se estudia el cálculo \( λ \), a pesar de poder trabajar con expresiones en este sistema a partir de equivalencias arbitrarias, usualmente cada regla de equivalencia se asocia con alguna argumentación basada en la noción de función.

Por ejemplo, se pueden considerar dos abstracciones diferentes \( (λx.M) \) y \( (λy.N) \) que al ser aplicadas a cualquier expresión \( Z \) sean \( β \)-convertibles a una misma expresión \( W \). Si se relacionan las abstracciones del cálculo \( λ \) con funciones, es natural pensar que \( M \) y \( N \) sean equivalentes, ya que por definición, dos funciones \( f \) y \( g \) son equivalentes si para toda \( x \) en su dominio \( f(x)=g(x) \). Por ejemplo, las funciones \( f(n)=\sum_{i=0}^{n}i \) y \( g(n)=\frac{n(n+1)}{2} \) a pesar de describir dos procedimientos diferentes para el cálculo de la suma de los primeros \( n \) números naturales son ``funcionalmente'' equivalentes ya que para todo natural \( f(n)=g(n) \). Por otro lado, si se relacionan las abstracciones del cálculo \( λ \) con algoritmos, \( M \) y \( N \) no pudieran ser consideradas equivalentes ya que en el estudio de la complejidad algorítmica, el énfasis en la comparación entre dos procedimientos no es las entradas y salidas, si no el proceso que describen. Por ejemplo, el algoritmo de ordenamiento \emph{merge sort} logra ordenar una secuencia de \( n \) números de menor a mayor en \( \mathcal{O}(n \log n) \) mientras que el algoritmo \emph{bubble sort} computa el mismo resultado pero en \( \mathcal{O}(n^2) \).  La equivalencia ``funcional'' se pudiera incluir como ecuación de \( = \) de la siguiente manera:

\[ (M\, P) \sim (N\, P) \implies M \sim N \]

Con esto se termina la introducción informal al cálculo \( λ \), las ideas que se han manejado en esta sección son formalizadas y definidas de manera rigurosa en el capítulo \ref{ch:formalizacion}.

%%% Local Variables:
%%% mode: latex
%%% TeX-master: "main"
%%% End:
