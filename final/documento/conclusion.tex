En este trabajo se presentaron las ideas generales del cálculo \( λ \), se formalizaron utilizando como herramientas matemáticas las teorías formales y los sistemas de reducción, también se desarrollaron varias maneras para representar información compleja y algoritmos tanto iterativos como recursivos utilizando únicamente funciones y aplicación de funciones. Sin embargo, este trabajo comprende sólo una pequeña parte del estudio del cálculo \( λ \).

Este trabajo puede ser ameno de leer para algunas personas y para otras no. Intenté mantener un balance entre los aspectos conceptuales y formales, pero el contenido del trabajo terminó siendo más técnico de lo que esperaba. Los aspectos pragmáticos y las aplicaciones del cálculo \( λ \) están ausentes del trabajo y esto puede resultan contraproducente para interesar a un lector que nunca ha sido expuesto al cálculo \( λ \).

Algunos aspectos que se pueden estudiar después de leer este trabajo son las extensiones del cálculo con lógica ilativa, teoría de tipos, semántica denotacional y programación funcional.


%%% Local Variables:
%%% mode: latex
%%% TeX-master: "main"
%%% End:
