El cálculo \( λ \) es un lenguaje y herramienta para el estudio del cómputo, ha sido utilizado en una gran variedad de áreas de la computación, desde los aspectos más fundamentales como en establecer los límites del cómputo y la semántica de lenguajes, hasta aspectos muy prácticos como en algoritmos para la inferencia de tipos y demostradores automáticos de teoremas.

Este trabajo presenta una introducción al estudio del cálculo \( λ \) puro desde una perspectiva matemática y computacional. El contenido del trabajo es producto de una revisión de la literatura sobre este cálculo y temas relacionados de las ciencias de la computación.

La estructura del trabajo se conforma de tres capítulos y dos apéndices:

\begin{itemize}
\item El primer capítulo aborda las ideas elementales del cálculo \( λ \) de manera informal relacionando los conceptos de este cálculo con otras áreas de estudio más convencionales de las matemáticas.

\item El segundo capítulo presenta la perspectiva matemática del cálculo \( λ \), se aborda la formalización de la sintaxis del lenguaje del cálculo y después haciendo uso de sistemas formales y sistemas de reducción se formalizan el resto de los conceptos introducidos en el primer capítulo.

\item El tercer capítulo presenta la perspectiva computacional del cálculo \( λ \), se aborda la codificación de objetos matemáticos y algoritmos del álgebra booleana y la aritmética elemental, después se explora la representación de procesos y estructuras recursivas.

\item Los apéndices son implementaciones de programas para ser ejecutados en la computadora. El primer apéndice presenta una manera de programar las codificaciones del tercer capítulo en los lenguajes de programación Scheme y Haskell. El segundo apéndice presenta el intérprete y editor estructural de un lenguaje diseñado para explorar los temas de este trabajo de manera interactiva.
\end{itemize}

Las aportaciones de este trabajo son las codificaciones de algoritmos desarrolladas en el tercer capítulo y el programa del segundo apéndice.