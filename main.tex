% Eduardo Acuña Yeomans 2016
% Stripped down version of my dissertation.
%

% Document class and language
\documentclass[letterpaper,twoside,openright,12pt]{book}

\usepackage{setspace}
% \onehalfspacing
\doublespacing

\usepackage[lmargin=1.4 in, rmargin=.8 in, tmargin=1 in, bmargin=1 in]{geometry}
\usepackage[utf8]{inputenc}
\usepackage[spanish]{babel}
\usepackage[T1]{fontenc}

\usepackage[sumlimits]{amsmath}
\usepackage{amssymb}
\usepackage{amsthm}
\usepackage[full]{textcomp}
\usepackage[osf]{newpxtext}
\usepackage{cabin}
\usepackage[varqu,varl]{inconsolata}
\usepackage[bigdelims,vvarbb]{newpxmath}
\usepackage[cal=boondoxo]{mathalfa} % mathcal
\usepackage{proof}

\newcommand{\bs}{\boldsymbol}

\theoremstyle{plain}%
\newtheorem{thm}{Teorema}[section]
\newtheorem{lem}[thm]{Lema}
\newtheorem{prop}[thm]{Proposición}
\newtheorem*{cor}{Corolario}
\theoremstyle{definition}
\newtheorem{defn}{Definición}[section]
\newtheorem{exmp}{Ejemplo}[section]
\theoremstyle{remark}
\newtheorem*{rem}{Observación}
\newtheorem*{note}{Nota}
\newtheorem{case}{Caso}

\usepackage{algorithmic}
\usepackage[chapter]{algorithm}
\floatname{algorithm}{Algoritmo}
\renewcommand{\algorithmicrequire}{\textbf{Entrada:}} 
\renewcommand{\algorithmicensure}{\textbf{Salida:}} 
\renewcommand{\algorithmicend}{\textbf{fin}} 
\renewcommand{\algorithmicif}{\textbf{si}} 
\renewcommand{\algorithmicthen}{\textbf{entonces}} 
\renewcommand{\algorithmicelse}{\textbf{de lo contrario}} 
\renewcommand{\algorithmicelsif}{\algorithmicelse\ y \algorithmicif} 
\renewcommand{\algorithmicendif}{\algorithmicend\ \algorithmicif} 
\renewcommand{\algorithmicfor}{\textbf{para}} 
\renewcommand{\algorithmicforall}{\textbf{para todo}} 
\renewcommand{\algorithmicdo}{\textbf{hacer}} 
\renewcommand{\algorithmicendfor}{\algorithmicend\ \algorithmicfor} 
\renewcommand{\algorithmicwhile}{\textbf{mientras}} 
\renewcommand{\algorithmicendwhile}{\algorithmicend\ \algorithmicwhile} 
\renewcommand{\algorithmicloop}{\textbf{lazo}} 
\renewcommand{\algorithmicendloop}{\algorithmicend\ \algorithmicloop} 
\renewcommand{\algorithmicrepeat}{\textbf{repetir}} 
\renewcommand{\algorithmicuntil}{\textbf{hasta}} 
\renewcommand{\algorithmicprint}{\textbf{imprimir}} 
\renewcommand{\algorithmicreturn}{\textbf{regresar}} 
\renewcommand{\algorithmictrue}{\textbf{verdadero}} 
\renewcommand{\algorithmicfalse}{\textbf{falso}} 
\renewcommand{\algorithmiccomment}[1]{$\quad\vartriangleright$ #1}
\renewcommand{\algorithmicto}{\textbf{hasta}}


% Pictures
\usepackage{tikz}
\usetikzlibrary{positioning, arrows, shadows}

% Lists
\usepackage{enumerate}

% Unicode lambda
\usepackage{newunicodechar}
\newunicodechar{λ}{\lambda}
\newunicodechar{Λ}{\Lambda}
\newunicodechar{α}{\alpha}
\newunicodechar{β}{\beta}
\newunicodechar{Γ}{\Gamma}
\newunicodechar{ρ}{\rho}
\newunicodechar{σ}{\sigma}
\newunicodechar{Σ}{\Sigma}
\newunicodechar{τ}{\tau}
\newunicodechar{ν}{\nu}
\newunicodechar{μ}{\mu}
\newunicodechar{ξ}{\xi}
\newunicodechar{Ξ}{\Xi}
\newunicodechar{ζ}{\zeta}
\newunicodechar{η}{\eta}
\newunicodechar{φ}{\phi}
\newunicodechar{Φ}{\Phi}
\newunicodechar{π}{\pi}
\newunicodechar{Π}{\Pi}

% λ-calculus commands
\newcommand{\subst}[3]{#1 \left[ #2 \operatorname{:=} #3 \right]}
\newcommand{\synteq}{\equiv}
\newcommand{\contract}[1]{\operatorname{\rightarrow_{#1}}}
\newcommand{\reduce}[1]{\operatorname{\twoheadrightarrow_{#1}}}
\newcommand{\convertible}[1]{\operatorname{=_{#1}}}
\DeclareMathOperator{\Sub}{Sub}
\DeclareMathOperator{\FV}{FV}

% Links in PDF
\usepackage{hyperref}
\hypersetup{
    colorlinks=true,
    pdftitle={El cálculo lambda y los fundamentos de la computación},
    pdfauthor={Eduardo Acuña Yeomans}
    bookmarks=true,
}

\begin{document}

\title{El cálculo lambda y los fundamentos de la computación}
\author{Eduardo Acuña Yeomans}
\date{2016}

\maketitle

\frontmatter
\tableofcontents

\mainmatter
\chapter{Noción informal del cálculo lambda}
\label{ch:nocion-informal}
El cálculo lambda es un sistema formal creado con la finalidad de expresar,
manipular y estudiar funciones. La manera en la que se trabaja con funciones en
este sistema es diferente a como es usual en la matemática clásica. Por este
motivo, se presenta una introducción informal que tiene como objetivo esclarecer
estas diferencias.\\

La estructura de este capítulo se conforma de tres secciones: en la primera se
aborda la introducción informal al cálculo lambda, en donde se presenta la
terminología utilizada en las explicaciones y ejemplos, se describen algunas de
las diferencias tanto conceptuales como de notación entre las funciones en este
sistema y las funciones en la matemática clásica; en la segunda sección se
presenta la formalización del cálculo lambda y en base a ésta se definen con
precisión los conceptos abordados en la primer sección; en la tercer sección se
exploran diferentes maneras de representar en el cálculo lambda algunos objetos
y operaciones matemáticas utilizadas en el estudio de la computación.\\

\section{Introducción} \label{sec:1.1}

La definición de función en la matemática clásica es el de una relación entre un
conjunto de entradas, llamado \emph{dominio} y un conjunto de salidas, llamado
\emph{codominio}. Esta relación tiene además la propiedad de que cada elemento
del dominio se relaciona exactamente con un elemento del codominio,
formalmente:\\

Sean \(A\) y \(B\) dos conjuntos, una función \(f\) con dominio \(A\) y
codominio \(B\) es un subconjunto del producto cartesiano \(A\times B\), tal que
para toda \(a\in A\), existe \(b\in B\) tal que \((a,\ b)\in f\) y si \((a,\
b^\prime)\in f\) con \(b^\prime \in B\), entonces \(b=b^\prime\).\\

Las funciones tienen varias maneras de ser representadas. En la definición
anterior la representación es la de pares ordenados, en donde la primer
componente del par es un elemento en el dominio y la segunda es un elemento en
el codominio. Dependiendo del uso que se le dá a las funciones,
puede ser conveniente representarlas simbolicamente con expresiones,
graficamente con dibujos, numéricamente con tablas o incluso verbalmente con
palabras.\\

Es posible utilizar esta definición para \emph{describir} las funciones en el
cálculo lambda, para esto se tiene que establecer cuál es el dominio y codominio
de cada función; después presentar una representación conveniente para las
reglas de correspondencia en el sistema. Sin embargo, hay algunas propiedades
del cálculo lambda que hacen que esta definición no pueda ser directamente
aplicada. En particular, el cálculo lambda como sistema formal es una
\emph{teoría ecuacional}, esto significa que la teoría formal del cálculo lambda
solo se conforma de reglas de igualdad entre expresiones, estas reglas
consideran únicamente la estructura de las expresiones y en el caso del cálculo
lambda, las expresiones no se componen de conjuntos, conectivos o
cuantificadores lógicos.\\

En el cálculo lambda existen expresiones para representar \emph{variables},
\emph{funciones} y \emph{aplicaciones}. El concepto de aplicación hace alusión a
la \emph{aplicación de funciones}, es decir, el acto de obtener un elemento del
codominio de una función, a partir de un elemento en su dominio, por ejemplo,
considerando la función \(f(x)=x^2\), aplicar \(f\) a 4 es \(f(4)=16\).\\

Las expresiones de funciones y aplicaciones en el cálculo lambda son en algúnos
aspectos mas restrictivas que en la matemática clásica, ya que en las
expresiones no se pueden escribir directamente números como el 2, ni operaciones
como la exponenciación, por lo tanto, no es posible escribir directamente la
función \(f(x)=x^2\) como una expresión válida del cálculo lambda. Por otro
lado, la aplicación en el cálculo lambda es menos restrictiva que la aplicación
de funciones de la matemática clásica, ya que el cálculo lambda permite aplicar
cualquier expresión válida a otra, no únicamente funciones a valores.\\

En general, las expresiones en el cálculo lambda asociadas a los conceptos de
función y de aplicación de funciones se pueden escribir únicamente en términos
de otras expresiones, las cuales a su vez pueden ser solo variables, funciones o
aplicaciones. Esto no significa que al trabajar con este sistema no podamos
trabajar también con teoría de conjuntos, aritmética, lógica o algúna otra rama
de las matemáticas, esto solo significa que las expresiones en el
\emph{lenguaje} del sistema formal son restrictivas en la manera en la que se
escriben. Sin embargo, al igual que las palabras en el español, en el lenguaje
utilizado para examinar y describir el cálculo lambda es válido hacer uso de
cualquier herramienta, ya sea matemática o computacional, a este otro lenguaje
se le llama \emph{metalenguaje}.\\

Al tratar con funciones en el cálculo lambda, se omite hablar de su dominio y
codominio, esto es debido a que todas las funciones válidas tienen como dominio
y codominio al conjunto de todas las expresiones válidas del cálculo lambda.
Este detalle debe ser tratado con cuidado cuando se representan objetos y
operaciones matemáticas en el cálculo lambda, ya que el dominio y codominio de
estas operaciones segue siendo el del conjunto de todas las expresiones válidas
del cálculo lambda, sin importar la operación que se represente. Por ejemplo, es
posible representar cualquier número natural con expresiones válidas del cálculo
lambda y también es posible tener una representación de la operación de
exponenciación; se puede \emph{emular} la función \(f : \mathbb{N} \to
\mathbb{N}\), \(f(x,y)=x^y\), mas sin embargo la función del cálculo lambda que
representa esta operación segue teniendo como dominio y codominio el conjunto de
todas las expresiones válidas del cálculo lambda, esto significa que será valido
aplicar esta representación de exponenciación a expresiones que no sean
representaciones de números naturales y el resultado de dicha aplicación no
necesariamente es una expresión que represente a un número natural.\\

El hecho de tener un lenguaje tan reducido y minimalista para las expresiones
nos permite poder entender de manera clara y precisa todos los procesos de
manipulación y transformación de la estructura de una expresión, a tal grado que
todas las operaciones que se realizan sobre las expresiones pueden reproducirse
paso a paso de manera mecánica, manipulando los símbolos que las conforman.\\

\subsection{Notación} \label{sec:1.1.1}

La notación utilizada en la matemática clásica para escribir la definición y
aplicación de funciones suele ser la de expresar una regla de correspondencia
como una expresión simbólica. En el cálculo lambda, también se utiliza ésta
representación, pero los símbolos empleados para escribir las expresiones son
definidos con precisión de antemano, en contraste con las expresiones
matemáticas, en donde la notación de las reglas de correspondencia puede ser
extendida de manera arbitraria ya sea para incluír operaciones sobre distintos
objetos matemáticos, compactar repeticiones de operaciones como \(\sum_{i=0}^n\)
o incluso incrustar en la notación procesos no finitos como límites al infinito
\(\lim_{x\to \infty}\).\\

Para introducir la notación del cálculo lambda, consideramos la función
identidad \(I : \mathbb{N} \to \mathbb{N}\) definida como \(I(x)=x\).\\

En la notación clásica, \(I\) se compone de la especificación de su dominio y
codominio, en este caso \(\mathbb{N}\), después se establece la regla de
correspondencia la cual indica que, al aplicar \(I\) a un argumento \(x\in
\mathbb{N}\), el resultado es equivalente a la expresión del lado derecho de la
ecuación, en donde toda aparición de la variable \(x\) hace referencia al
argumento particular al que le fué aplicado \(I\).\\

En el cálculo lambda, no se considera el dominio ni el codominio de las
funciones, e incluso, no se considera el nombre con el que nos referimos a
ellas. La manera en como \(I\) es escrita en este sistema es \[\lc{\x.x}\] el
símbolo ``\(\lambda\)'' nos indica que la expresión es una función, y el símbolo
``\(.\)'' separa la variable que hace referencia al argumento al que la función
es aplicada y la expresión del lado derecho de la igualdad.\\

La aplicación de expresiones se denota de manera diferente también, mientras que
en la notación clásica se escribe \(I(y)\) considerando que \(y\in \mathbb{N}\),
en el cálculo lambda, debido a que no se nombran las funciones, se escribe
explícitamente la función a la que hacemos referencia \[\lc{(\x.x)y}\] En ambos
casos, \emph{realizar} la aplicación consiste en sustituír las apariciones de
\(x\) por \(y\) en la función, dando como resultado \(y\). Sin embargo, no
podemos afirmar que \(\lc{(\x.x)y}=y\) sin antes mencionar de manera explícita
el significado que se le dá a la igualdad entre dos expresiones.\\

Como se mencionó anteriormente, lo único que se puede escribir en el cálculo
lambda son variables, funciones definidas en términos de otra expresión y
aplicaciones entre dos expresiones. Todas las partes de la aplicación del
ejemplo anterior también son expresiones válidas: ``\(\lc{(\x.x)y}\)'',
``\(\lc{\x.x}\)'', ``\(\lc{y}\)'' y ``\(\lc{x}\)''. Estas expresiones muestran
la manera en como se escriben las aplicaciones, las funciones y en el caso de
\(\lc{x}\) y \(\lc{y}\), las variables.\\

A pesar de ser aparentemente una notación mas inconveniente debido a que se
limita a tratar solo con tres clases de expresiones, esta notación nos permite
ser mas explícitos en la descripción de expresiones y provee uniformidad en el
lenguaje formal. Esta notación también permite tener mas control sobre la manera
en que las expresiones son transformadas.\\

La estructura de las expresiones hace que sea mas directa la relación entre una
expresión o una parte de la expreción y un significado. El significado de una
expresión puede referirse a lo que la expresión represente conceptualmente
hablando o a la manera en la que la expresión puede ser operada.\\

Un ejemplo de la importancia de la asignación explícita del significado
operacional de las expresiones es el de los posibles problemas que se pueden
encontrar cuando se realiza la sustitución al momento de aplicar una función a
una expresión, desde la perspectiva de la matemática clásica, consideremos la
función factorial \(f : \mathbb{N} \to \mathbb{N}\) definida como
\[f(n)=
\begin{cases} 
  1 &\mbox{si } n=1\\
  n\times f(n-1) & \mbox{en otro caso.}
\end{cases}
\]

Para obtener el resultado correcto la aplicación de \(f\) en 5, primero
verificamos si \(5=0\), en donde, si fuera el caso, el resultado sería 1, pero
ya que \(5\not= 0\), el resultado es \(5\times f(4)\), el proceso mecánico de
sustituír el argumento en la expresión de la regla de correspondencia consiste
en primero verificar si la condición es cierta antes de proceder en sustituír el
valor del argumento en el consecuente correspondiente. Si este modelo de
sustitución no se especifica para el uso de la notación del análisis casos
presente en el ejemplo, se pudieran contemplar otras maneras de sustituír al 5
en la expresión, por ejemplo, sustituyendo el argumento en todas las apariciones
de la variable \(n\), luego ``expandir'' el valor de la aplicación de funciones
y posteriormente decidir el resultado final verificando si el argumento cumple
la condición. Sin embargo, utilizar este modelo de sustitución en el ejemplo
resulta en realizar una infinidad de sustituciones debido a la naturaleza
recursiva de la definición.\\

En el uso cotidiano de las matemáticas, no se suele analizar el proceso de
sustitución, sin embargo, en el cálculo lambda es de suma importancia. Esta
diferencia se debe a que en la matemática clásica las comparaciones entre dos
funciones o dos expresiones tienden a ser \emph{declarativas}, es decir, se
declaran las relaciones, aseverando que la expresión es cierta; mientras que en
el cálculo lambda son \emph{imperativas}, es decir, toda relación o equivalencia
entre dos expresiones expresa un mecanismo para construir una expresión a partir
de otra.\\

Un ejemplo de esta distinción es el manejo del concepto de función inversa desde
ambas perspectivas, una definición declarativa es: Sea \(f\) una función cuyo
dominio es \(A\) y cuya imágen es \(B\), la función inversa de \(f\) es la
función \(f^{-1}\) con dominio \(B\) e imagen \(A\) tal que, \(f(a)=b\) si y
sólo si \(f^{-1}(b)=a\) con \(a\in A\) y \(b\in B\). En el cálculo lambda no es
usual trabajar con este tipo de definiciones debido a que no describen un
procedimiento mediante el cual se puede obtener \(f^{-1}\) a partir de \(f\).\\

Al referirse a una expresión del cálculo lambda, usualmente, se conoce
parcialmente su estructura, es decir, algúna descripción de sus partes. En el
resto de esta sección nos referiremos a una variable entre la \(\lambda\) y el
punto de una función como \emph{argumento de la función} y a la expresión
después del punto y antes del paréntesis cerrado como el \emph{cuerpo de la
función}. A continuación se muestran algunos ejemplos de expresiones:

\begin{align*}
  \text{a) }\ &\lc{x}\\ 
  \text{b) }\ &\lc{\x.x}\\
  \text{c) }\ &\lc{y (\x.x)}\\
  \text{d) }\ &\lc{(\y.y(\x.x))(\w.w)}\\
  \text{e) }\ &\lc{\x.x x}\\
  \text{f) }\ &\lc{\f x.f x}\\
\end{align*}

Las variables en el cálculo lambda son expresiones válidas, en el inciso
\emph{a} aparece la variable \(\lc{x}\) la cual no es ni una función ni una
aplicación; las variables por si solas en el cálculo lambda casi no tienen
utilidad, pero al ser partes de otra expresión, puede aumentar su importancia:
en el caso del inciso \emph{b} la misma variable \(\lc{x}\) es el cuerpo de la
función y como es también el argumento, esta variable tiene el potencial de
convertirse en cualquier otra expresión a partir de la aplicación de la función
\(\lc{\x.x}\).\\

En el inciso \emph{c} se tiene una aplicación inusual, es la variable \(\lc{y}\)
siendo aplicada a una función. Comunmente se trabaja con expresiones en donde lo
que se aplica es una función, sin embargo si \(\lc{y (\x.x)}\) fuera el cuerpo
de una función, entonces \(\lc{y}\) jugaría un papel mas relevante. Esto se
puede apreciar en el inciso \emph{d}, en donde la expresión del inciso \emph{c}
es el cuerpo de una función con argumento \(\lc{y}\) y esta función está siendo
aplicada a otra función. Este ejemplo nos permite abordar dos ideas importantes,
primero, las funciones pueden ser aplicadas a funciones y segundo el realizar la
aplicación del ejemplo \emph{d}, la variable \(\lc{y}\) toma el valor de
\(\lc{\w.w}\) y es ahora aplicada a la función \(\lc{\x.x}\):

\begin{align*} 
  \text{1. } &\lc{(\y.y(\x.x))(\w.w)} & &\text{ expresión del inciso \emph{d}}\\ 
  \text{2. } &\lc{(\w.w)(\x.x)} & &\text{ al aplicar } \lc{\y.y(\x.x)} \text{ a } \lc{\w.w}\\ 
  \text{3. } &\lc{\x.x} & &\text{ al aplicar } \lc{\w.w} \text{ a } \lc{\x.x}
\end{align*}

En este último ejemplo se describe una secuencia de transformaciones mecánicas
sobre los símbolos de la expresión, este procedimiento tiene algunos detalles
que son importantes recalcar pero se abordan cuando se describa la formalización
del cálculo lambda en la sección \ref{sec:1.2}. Por el momento se describen los
últimos dos incisos los cuales presentan dos conceptos interesantes.\\

En el inciso \emph{e} se tiene una función cuyo cuerpo es la aplicación de su
argumento sobre sí mismo. Lo interesante de esta expresión es que encapsula la
idea de replicar cualquier expresión a la que se aplique. Por ejemplo, si
aplicamos la expresión a la variable \(\lc{y}\) y realizamos el proceso de
aplicación similar al mostrado con el anterior ejemplo, obtendremos \(\lc{y y}\)
como resultado; si aplicamos la expresión a sí misma obtendremos un
``\emph{quine}'' \cite{Hofstadter:GEB}:

\begin{align*} 
  \text{1. } &\lc{(\x.x x)(\x.x x)} & &\text{ expresión del inciso \emph{e} aplicada a si misma}\\
  \text{2. } &\lc{x} \leftarrow \lc{\x.x x} & &\text{ valor que toma } \lc{x} \text{ en el cuerpo de } \lc{\x.x x}\\ 
  \text{3. } &\lc{x x} & &\text{ expresión en donde se sustituye } \lc{x}\\
  \text{4. } &\lc{(\x.x x)(\x.x x)} & &\text{ al completar la sustitución}
\end{align*}

Como podemos observar, el resultado de la aplicación es la expresión inicial, a
pesar de que el término quine se asoció originalmente a una paradoja sobre
valores de verdad \cite{Quine:Paradox}, hoy en día hace referencia a un programa
que tiene como resultado el código fuente de él mismo.\\

El inciso \emph{f} es una función cuyo cuerpo es otra función, en donde el
cuerpo de esta última es la aplicación del argumento de la primer función al
argumento de la segunda. El concepto interesante que ilustra esta expresión es
el de funciones de varias variables: Aplicar esta expresión a una expresión
cualquiera \(M\) y posteriormente aplicar este resultado a otra expresión
cualquiera \(N\) produce el mismo resultado a que si tuvieramos una función de
dos argumentos \(\lc{f}\) y \(\lc{x}\) cuyo cuerpo es \(\lc{f x}\) y aplicaramos
esta expresión hipotética a \(M\) y \(N\).\\

Otra manera de trabajar con funciones de varias variables es la de representar a
tuplas en el el cálculo lambda y tener expresiones para obtener cada elemento de
una tupla. Sin embargo, representar tuplas es un mecanismo mas complejo que se
aborda en la sección \ref{sec:1.3}.\\

\subsection{El concepto de igualdad} \label{sec:1.1.2}

El concepto de igualdad es muy importante en el cálculo lambda. En el
desarrollo histórico del este sistema, el estudio de los criterios que permiten
establecer que dos expresiones son iguales dió pié a una grán diversidad de
variantes de la teoría original.\\

De la mano al concepto de igualdad, están los mecanismos de transformación de
expresiones, estos mecanismos de transformación no son operadores dentro del
lenguaje del cálculo lambda, si no mas bien, son transformaciones que permiten
explorar las diferentes estructuras de las expresiones del cálculo lambda y son
usados como metalenguaje para referirse a equivalencias entre dos expresiones.
Hay una grán variedad de operaciones que pueden transformar expresiones, sin
embargo, en esta subsección abordaremos las mas elementales.\\

\subsubsection{Sustitución}

Un concepto que es de grán importancia para describir y definir las
transformaciones que realizemos sobre expresiones es el de \emph{sustitución}.
Cuando se describió informalmente el proceso mecánico de la aplicación de
funciones en los ejemplos anteriores se mencionó este concepto. A continuación
se presenta una descripción mas detallada de este concepto abordado como
operación de transformación. \\

La sustitución involucra dos expresiones \(\lc{M}\) y \(\lc{N}\) cualesquiera y una
variable \(\lc{x}\), el proceso de transformación consiste en intercambiar todas las
apariciones de la variable \(\lc{x}\) en la expresión \(\lc{M}\) por la expresión \(\lc{N}\),
denotado \(\lc{q[subst[M,x,N]]}\). Es usual que en los sistemas formales, se tenga
cuidado al definir la transformación de sustitución, los detalles de la
transformación son pospuestos por el resto de la sección y se presentan ejemplos
en donde esta definición provisional es suficiente. \\

A manera de ejemplo, consideremos la sustitución de la variable \(\lc{y}\) por \(\lc{x}\)
en la expresión \(\lc{y w}\), se escribe \[\lc{q[subst[y w, y, x]]}\]  y la
expresión resultante es \(\lc{x w}\). \\

Retomando el proceso descrito en el ejemplo de la expresión del inciso \emph{d},
el acto de aplicar la función \(\lc{\y.y(\x.x)}\) en \(\lc{\w.w}\) se escribe
con esta notación como:
\begin{align*} 
  &\lc{(\y.y(\x.x))(\w.w)} & &\lc{q[subst[y(\x.x), y, \w.w]]} \\
  &\lc{(\w.w)(\x.x)} & &\lc{q[subst[w,w,\x.x]]} \\
  &\lc{\x.x} & &
\end{align*}

A diferencia de las transformaciones que se abordan a continuación, la
sustitución no está relacionada directamente con algún concepto de igualdad mas
que el que describe la operación por si misma. Sin embargo es la transformación
fundamental sobre la cual se describe el resto. \\

\subsubsection{Equivalencia de expresiones}

Con la sustitución se pueden transformar expresiones del cálculo lambda
independientemente del contexto en el que se utilizan y sin prestar atención a
lo que la expresión representa conceptualmente. A excepción de la función
identidad y una representación del número 1, no se han abordado expresiones que
representen algún concepto mas allá de su estructura simbólica, y a pesar de ser
tentador asignarle un significado preciso a cada expresión presentada, es
importante seguir analizando únicamente su estructura para tratar el tema de
equivalencia de expresiones. \\

En esta subsección se exploran diferentes criterios para determinar si dos
expresiones son iguales. \\

Usualmente podemos afirmar que dos expresiones son iguales cuando entendemos el
contexto y el nivel de abstracción en el que se refiere a ellas. Por ejemplo,
dos números se suelen considerar iguales si representan el mismo concepto con el
que se desea trabajar, si observamos una ecuación como \(3=\frac{6}{2}\) sabemos
de inmediato que la ecuación es cierta, a pesar de que \(frac{6}{2}\)
explícitamente hace referencia a una división y los dos lados de la igualdad se
escriban diferente simbolicamente. Cuando se involucran variables y operaciones
determinar si dos expresiones son iguales involucra mas información:
considerando la ecuación \(x\cdot y=y\cdot x\), es imposible poder aseverar si
la ecuación es cierta sin establecer los valores que ``\(x\)'', ``\(y\)'' y
``\(\cdot\)'' representan, en caso que sean números naturales y la operación
aritmética de multiplicación la ecuación es cierta, pero en caso que sean
matrices y la operación de producto matricial, la ecuación es falsa. \\

Al abordar expresiones matemáticas, el contexto en el que se expresan casi
siempre se puede inferir por la manera en como las expresiones son utilizadas y
lo mas común es tener a la mano la definición de las estructuras matemáticas y
operaciones utilizadas en las expresiones, sin embargo, en el estudio del
cálculo lambda, se tiene que ser mas explícito en especificar el concepto de
igualdad con el que se trabaja, es tan importante que modificar su significado,
modifica los axiomas de la teoría. \\


Un criterio trivial de equivalencia es el de considerar a dos expresiones como
equivalentes cuando se escriben exactamente igual símbolo por símbolo. A esta
equivalencia se le llama sintáctica y es denotada con el símbolo \(\synteq\). \\

Otro criterio que se puede apreciar al dar un vistazo a dos expresiones es el de
equivalencia estructural. Esta equivalencia toma en cuenta el hecho que la
relevancia de las variables no reside en su nombre o representación
sintáctica, si no en su posición en la estructura de la expresión. \\

Considerndo la función identidad \(\lc{\x.x}\) se puede observar que tiene la
misma estructura que \(\lc{\y.y}\) la cual representa el mismo concepto. A pesar
de no estar escritas exactamente igual, la correspondencia que hay de la
posición de la variable \(x\) en la primera expresión con la posición de la
variable \(y\) en la segunda y el hecho de que ambas tienen la misma estructura
nos permite considerarlas como equivalentes. \\ 

Considerando dos expresiones un poco mas complejas como \(\lc{\f x.f x}\) y
\(\lc{\g y.g y}\) podemos notar que también son equivalentes en este sentido. \\

Una notación utilizada para corroborar la equivalencia estructural es el
\emph{índice de De Bruijn}, esta notación evita la aparición de variables en las
expresiones y en su lugar utiliza números que representan la ``distancia'' de una
variable a la \(\lambda\) de la función en donde aparece como argumento. De tal
manera que una expresión como

\begin{equation}\label{eq:2.1} \lc{\z.(\y.y (\x.x))(\x. z x)}
\end{equation}

se escribe usando el índice de De Bruijn como

\begin{equation}\label{eq:2.2} \lambda (\lambda 1 (\lambda 1)) (\lambda 2 1)
\end{equation}

En la figura~\ref{fig:DeBruijn-transformation} se puede observar de manera
gráfica la transformación de una notación a otra para este ejemplo en
particular. \\

\begin{figure}[h!]
  \centering
  
  \begin{tikzpicture}[level/.style={sibling distance=60mm/#1}] 
    \node (term) {
      \(\lc{\z.(\y.y (\x.x))(\x. z x)}\)
    }; 
    \node [below of=term] (arrow1) {
      \(\Downarrow\)
    }; 
    \node [circle,draw,below of= arrow1] (z) {
      \(\lambda z\)
    } child {
      node [circle,draw] (a) {
        \(\lambda y\)
      } child {
        node [circle,draw] (c) {
          \(y\)
        }
      }
      child {
        node [circle,draw] (d) {
          \(\lambda x\)
        }
        child {
          node [circle,draw] (g) {
            \(x\)
          }
        } 
      } 
    } 
    child {
      node [circle,draw] (b) {
        \(\lambda x\)
      } 
      child {node [circle,draw] (e) {
          \(z\)
        }
      } 
      child {
        node [circle,draw] (f) {
          \(x\)
        }
      } 
    };
    \node [below=4cm of z] (arrow2) {
      \(\Downarrow\)
    };
    \node [circle,draw,below of= arrow2] (z2) {
      \(\lambda\)
    }
    child {
      node [circle,draw] (a2) {
        \(\lambda\)
      } 
      child {
        node [circle,draw] (c2) {
          \(1\)
        }
      } 
      child {
        node [circle,draw] (d2) {
          \(\lambda\)
        } 
        child {
          node [circle,draw] (g2) {
            \(1\)
          }
        } 
      } 
    } 
    child {
      node [circle,draw] (b2) {
        \(\lambda\)
      } 
      child {
        node [circle,draw] (e2) {
          \(2\)
        }
      } 
      child {
        node [circle,draw] (f2) {
          \(1\)
        }
      }
    };
    \node [below=4cm of z2] (arrow3) {
      \(\Downarrow\)
    };
    \node [below of=arrow3](bruijn) {
      \(\lambda (\lambda 1 (\lambda 1)) (\lambda 2 1)\)
    };
  \end{tikzpicture}
  \caption{Transformación de~\eqref{eq:2.1} a~\eqref{eq:2.2}.}
  \label{fig:DeBruijn-transformation}
\end{figure}

Una desventaja de utilizar la notación de De Bruijn es que ciertas expresiones
del cálculo lambda no pueden ser escritas, en particular, toda variable tiene
que estar asociada a una \(\lambda\) para que esta notación pueda ser utilizada.
Sin embargo como veremos más adelante, la mayoría de los usos del cálculo lambda
asocian a todas las variables en las expresiones. \\

A este criterio de equivalencia se le llama \(\alpha\)-convertibilidad. \\

Otra equivalencia que podemos encontrar en las expresiones es la de aplicación
de funciones, esta hace referencia a que la aplicación de una función a una
expresión es equivalente al resultado de evaluar la función con dicha expresión
como argumento. Para entender mejor este concepto, consideramos la función en
notación tradicional \(f(x)=x^2\), si se evalúa \(f(3)\) el resultado es 8, por
lo tanto podemos decir que \(f(3)\) y 8 son equivalentes. \\

Si consideramos la expresión de la función identidad \(\lc{\x.x}\) podemos
afirmar que para cualquier expresión \(M\), \(\lc{(\x.x) M}\) es equivalente a
\(M\).

A esta equivalencia se le llama \(\beta\)-convertibilidad.\\

En la notación tradicional, estas tres equivalencias se denotan con el mísmo
símbolo \(=\), de tal manera que si dos expresiones son equivalentes ya sea
sintácticamente, estructuralmente o aplicativamente, entonces son consideradas
iguales. En el cálculo lambda es importante diferenciar estas equivalencias ya
que el manejo de las funciones no se aborda desde el punto de vista de una
relación entre el dominio y codominio, si no como una expresión que puede ser
manipulada y transformada de manera mecáncia.\\

\paragraph{Equivalencia de redundancia}

Otro tipo de equivalencia es la de redundancia, consideremos la expresión
\(\lc{\x.(\y.y) x}\), el papel que puede jugar es el de ser aplicada en otra
expresión \(M\), la cual resulta igual a aplicar la expresión interna
\(\lc{\y.y}\) en \(M\). Por las equivalencias descritas previamente podemos
observar que \(\lc{\x.(\y.y) x}\) y \(\lc{\y.y}\) no son sintácticamente
equivalentes, ni estructuralmente equivalentes, ni siquiera aplicativamente
equivalentes. El trabajar con la función que envuelve a \(\lc{\y.y}\) resulta
redundante al momento de aplicar las funciones en expresiones, ésto nos permite
considerar un criterio de equivalencia.\\

En el cálculo lambda, la equivalencia de redundancia se denomina
\(\eta\)\emph{-equivalencia} y nos permite considerar como iguales las
expresiones de la forma \(\lc{\x.M x}\) y \(M\).\\

\paragraph{Equivalencia computacional}

En el estudio de la lógica, se hace la distinción que una equivalencia puede ser
extensional o intensional. La equivalencia extensional hace referencia a las
propiedades externas de los objetos, mientras que la equivalencia intensional
hace referencia a la definición o representación interna de los objetos.\\

Las equivalencias sintáctica y estructural son equivalencias intencionales,
mientras que las equivalencias de aplicación y redundancia son equivalencias
extensionales, debido a que se juzgan dos objetos a partir de su evaluación. Sin
embargo, las equivalencias de aplicación y de redundancia no comprenden el caso
mencionado al inicio de esta subsección. Suponiendo que tenemos dos expresiones
\(M\) y \(N\) que describen el mismo algoritmo o la misma función, la
equivalencia de aplicación no los considera equivalente.\\

En la notación tradicional, la igualdad de funciones es una equivalencia
extensional, por ejemplo \(f(x) = e^{i\pi}\times x\) y \(g(x) = x\) describen la
función identidad y podemos aseverar que \(f=g\) sin necesidad de evaluar ambas
funciones con un argumento en particular.\\

En el cálculo lambda se puede hablar de este tipo de igualdad funcional si
consideramos que para toda expresión del cálculo lambda \(P\), si \(\lc{M P}\)
es equivalente a \(\lc{N P}\), entonces las expresiones \(M\) y \(N\) se dice
que son \(ext\) equivalentes.\\

\paragraph{Una simple regla con fuertes implicaciones}

Existe una regla, llamada regla \(\xi\), la cual establece una equivalencia muy
sencilla: si dos expresiones \(M\) y \(N\) son equivalentes, entonces las
expresiones \(\lc{\x.M}\) y \(\lc{\x.N}\) también lo son.\\

Aunque esta regla aparente aportar poco y pueda ser considerada innecesaria si
combinamos todas las equivalencias previamente descritas, es suficiente para
eliminar la equivalencia de redundancia y la equivalencia computacional de la
formalización del cálculo lambda, la cuál es abordada en la sección
\ref{sec:1.2}.\\


\subsubsection{Transformación de expresiones}

A cada equivalencia diferente a la sintáctica se le puede asociar una operación
de transformación la cual nos permita pasar de una expresión \(M\) a otra
expresión \(N\) de tal manera que estas dos expresiones sean equivalentes bajo
algún criterio específico.\\

En el caso de la \(\alpha\)-conversión la operación correspondiente consiste en cambiar
nombres de variables, en la \(\beta\)-\emph{convertibilidad} la operación
consiste en realizar una secuencia de sustituciones de las variables de una
función por expresiones a las que la función es aplicada y en la
\(\eta\)\emph{-equivalencia} la operación consiste en la eliminación de
funciones redundantes.\\

Estas operaciones se definen de manera formal más adelante y aunque puedan
parecer operaciones sencillas de definir a partir de la operación de
sustitución, se tiene que tener mucho cuidado en no obtener expresiones que
rompan la equivalencia asociada.\\

\section{Formalización de la teoría
\texorpdfstring{$\boldsymbol\lambda$}{lambda}} \label{sec:1.2}

La teoría \(\boldsymbol\lambda\) es el conjunto de axiomas que definen
formalmente al cálculo lambda como sistema formal, el objeto de estudio
principal de esta teoría es el del conjunto cociente formado a partir de un
conjunto de expresiones bien formadas y una relación de equivalencia. En las
siguientes subsecciones se definen estos conceptos, los cuales nos permitirán
comenzar el estudio formal del cálculo lambda.\\

\subsection{Expresiones bien formadas} \label{sec:1.2.1}

Una expresión bien formada es un objeto formal sintáctico. Para definir las
expresiones bien formadas de un lenguaje es necesario expresar de manera
rigurosa cómo se constituyen simbolicamente. \\

El conjunto de expresiones bien formadas del cálculo lambda es llamado
\emph{términos lambda}, denotado como \(\Lambda\). Este conjunto tiene elementos
que son expresiones construidas a partir del alfabeto \(\Sigma\). Éste alfabeto
es un conjunto que se conforma por los símbolos \((\) , \()\) , \(.\) ,
\(\lambda\) y una infinidad de símbolos \(v ,\ v^{\prime} ,\ v^{\prime\prime} ,\
\dots\ \), etc. A esta secuencia infinita de símbolos \(v^i\) se denota como
\(V\), de tal manera que \(\Sigma = \left\{\ (\ ,\ )\ ,\ .\ ,\ \lambda\ \right\}
\cup V\).\\

\(\Lambda\) es el conjunto mas pequeño tal que:

\begin{align}
  \label{eq:2.3} \lc{x} \in V &\Rightarrow \lc{x} \in \Lambda\\
  \label{eq:2.4} \lc{M},\ \lc{N} \in \Lambda &\Rightarrow \lc{M N} \in \Lambda\\
  \label{eq:2.5} \lc{M} \in \Lambda,\ \lc{x} \in V &\Rightarrow \lc{\x.M} \in
\Lambda
\end{align}

Cada una de estas tres reglas corresponde a los tres tipos de términos lambda.
La regla \eqref{eq:2.3} implica que los símbolos en \(V\) son términos lambda,
estos símbolos son llamados \emph{átomos}; la regla \eqref{eq:2.4} implica que
dos términos lambda entre paréntesis también son términos lambda, a este tipo de
términos se les llama \emph{aplicaciones}; la regla \eqref{eq:2.5} implica que
si se tiene entre paréntesis el símbolo \(\lambda\) seguido de un átomo, un
punto y un término lambda cualquiera, entonces ésta expresión también es un
término lambda, a este tipo de términos lambda se les llama
\emph{abstracciones}.\\

Desde la perspectiva de lenguajes formales, \(\Lambda = L(G)\), donde \(G\) es
una gramática libre de contexto con categorías sintácticas \(T\) (términos
lambda), \(E\) (aplicaciones), \(F\) (abstracciones) y \(A\) (átomos); símbolos
terminales \(\left\{\ (\ ,\ )\ ,\ .\ ,\ \lambda\ ,\ v,\ {}^{\prime}\ \right\}\);
símbolo inicial \(T\) y con las siguientes reglas de producción:

\begin{align*} \text{1. }\ T &\rightarrow E\ \mid\ F\ \mid\ V\\ \text{2. }\ E
&\rightarrow (\ T\ T\ )\\ \text{3. }\ F &\rightarrow (\ \lambda\ A\ .\ T\ )\\
\text{4. }\ A &\rightarrow v\ \mid\ E\ {}^{\prime}
\end{align*}

Para facilitar la escritura de términos lambda, en este trabajo se realizan las
siguientes consideraciones sobre la notación:

\begin{itemize}
\item[I.] Cuando se hace referencia a cualquier término lambda se utilizan las
letras mayúsculas \(M,\ N,\ O,\ \) etc. Es importante establecer que si en un
ejemplo, explicación, teorema o demostración hacemos referencia a un término
lambda con una letra mayúscula, cualquier otra aparición de esta letra hará
referencia a este mismo término.
\item[II.] Cuando se hace referencia a cualquier átomo se utilizan las letras
minúsculas \(x,\ y,\ z,\ w,\ \) etc. Al igual que en el punto anterior, la
aparición de una letra minúscula en un ejemplo, explicación, teorema o
demostración hace referencia al mismo término.
\item[III.] Los paréntesis son omitidos de acuerdo a las siguientes
equivalencias sintácticas:
  \begin{itemize}
  \item[a)] \(\lc{M N O} \synteq \lc*{M N O}\), en general, se considera la
aplicación de términos lambda como una operación con asociación a la izquierda.
Se tiene que tener cuidado con respetar la asociación, por ejemplo \(\lc{M(N(O
P))} \synteq \lc*{M(N(O P))} \not\synteq \lc*{M N O P}\).
  \item[b)] \(\lc{\x.M N} \synteq \lc*{\x.M N}\), en general, se puede escribir
una abstracción omitiendo los paréntesis externos siempre y cuando no se escriba
un término sintácticamente diferente. Por ejemplo \(\lc{(\x.M N) O} \synteq
\lc*{(\x.M N) O} \not\synteq \lc*{\x.M N O}\) ya que el lado derecho de la
equivalencia es sintácticamente equivalente a \(\lc*{\x.M N O}\).
  \item[c)] \(\lc{\x y z.M} \synteq \lc*{\x y z.M}\), en general, si el
subtérmino a la derecha del punto en una abstracción es también una abstracción,
se pueden agrupar los åtomos antes del punto de ambas abstracciones después de
una \(\lambda\) y antes que el punto, dejando el subtérmino después del punto de
la segunda abstracción, como el del nuevo término.
  \end{itemize}
\end{itemize}

La notación explicada en \emph{III.a)} proviene de la reducción usada por
Schönfinkel, en donde funciones de varias variables se transformn a funciones de
una sola variable \cite{Schonfinkel:Varargs}.\\

\subsection{Relación de equivalencia} \label{sec:1.2.2}

Una relación de equivalencia es una relación binaria \(=\) sobre elementos de
un conjunto \(X\), donde \(=\) es reflexiva, simétrica y transitiva, es
decir:

\begin{itemize}
\item[\S] \(a\in X \Rightarrow a= a\)
\item[\S] \(a,\ b\in X,\ a= b \Rightarrow b= a \)
\item[\S] \(a,\ b,\ c\in X, a= b,\ b= c \Rightarrow a= c\)
\end{itemize}

En el estudio formal del cálculo lambda, la relación de equivalencia asociada a
los términos lambda es llamada \emph{convertibilidad}. Ésta relación es generada
a partir de axiomas y para formular estos axiomas es necesario formalizar el
concepto de \emph{sustitución}.\\

\textbf{DEFINIR LOS CONCEPTOS NECESARIOS PARA HABLAR DE LO QUE SIGUE}

Todas las variables son términos lambda y son llamados átomos \\

Si \(\lc{M}\) y \(\lc{N}\) son términos lambda, entonces \(\lc{(M N)}\) es un
término lambda llamado aplicación \\

Si \(\lc{M}\) es cualquier término lambda y \(\lc{x}\) es cualquier variable,
entonces \(\lc{(\x.M)}\) es un término lambda llamado abstracción \\

La longitud de un término \(\lc{M}\), denotada \(lgh(M)\) es la cantidad total
de apariciones de átomos en \(\lc{M}\) \\

\(\lc{P}\) aparece en \(\lc{Q}\) o \(\lc{P}\) es un subtérmino de \(\lc{Q}\) o
\(\lc{Q}\) contiene a \(\lc{P}\) \\

El alcance de \(\lambda x\) en \(\lc{λx.M}\) es \(\lc{M}\) \\

La aparición de una variable \(\lc{x}\) en un término \(\lc{P}\) es llamada
ligada si está en el alcance de \(\lambda x\) en \(\lc{P}\), ligada y enlazada
si y sólo si es la \(\lc{x}\) en \(\lambda x\) y libre en otro caso \\

Un término cerrado es un término que no contiene variables libres. \\

Un cambio de variable ligada o \(\alpha\)-conversión \\

\(\lc{P}\) es congruente a \(\lc{Q}\) o \(\lc{P}\) se \(\alpha\)-convierte a
\(\lc{Q}\) o \(\lc{P} \convertible{\alpha} Q\) \\

\(\lc{P}\) y \(\lc{Q}\) son términos \(\alpha\)-convertibles \\

Cualquier término de la forma \(\lc{(\x.M)N}\) es llamado \(\beta\)-redex \\

El término correspondiente \(\lc{q[subst[M,x,N]]}\) es llamado su contracción \\

Si y sólo si un término \(P\) contiene una aparición de \(\lc{(\x.M)N}\) y
reemplazamos esa aparición por \(\lc{q[subst[M,x,N]]}\), y el resultado es
\(\lc{q[prime[P,1]]}\), decimos que contraemos la aparición redex en \(\lc{P}\),
y \(\lc{P}\) se \(\beta\)-contrae a \(\lc{q[prime[P,1]]}\) \\

Si y sólo si \(\lc{P}\) puede ser transformada a un término \(\lc{Q}\) a través
de una serie finita o vacía de \(\beta\)-contracciones y cambios de variable
ligada, decimos que \(\lc{P}\) se \(\beta\)-reduce a \(\lc{Q}\) \\

Un término \(\lc{Q}\) que no contiene \(\beta\)-redex es llamada una
\(\beta\)-forma normal o \(\beta\)-fn \\

Decimos que \(\lc{P}\) es \(\beta\)-igual o \(\beta\)-convertible a \(\lc{Q}\)
si y sólo si \(\lc{Q}\) puede ser obtenido a partir de \(\lc{P}\) por una serie
finita o vacía de \(\beta\)-contracciones, \(\beta\)-contracciones inversas y
cambios de variable ligada \\


\textbf{DEFINIR SUSTITUCIÓN}\\

\begin{align} 
  \lc{x}[\lc{x}:=\lc{M}] &\synteq M &\\
  \lc{y}[\lc{x}:=\lc{M}] &\synteq y &y \not\synteq x\\
  \lc{M N}[\lc{x}:=P] &\synteq (M[x:=P]\ N[x:=P]) &\\
  \lc{\x.M}[x:=N] &\synteq \lc{\x.M} &\\
  \lc{\y.M}[x:=N] &\synteq \lc{\y.M} &x \not\in FV(P)\\
  \lc{\y.M}[x:=N] &\synteq (\lambda\ y\ .\ M[x:=N]) &x\in FV(M),\ y\not\in FV(N)\\
  \lc{\y.M}[x:=N] &\synteq (\lambda\ z\ .\ M[y:=z][x:=N]) &x\in FV(M),\ y\in FV(N)
\end{align}

En las ecuaciones e, f y g, la variable $y$ debe de ser diferente a $x$ y en el
inciso g, la variable $z \in (FV(N) \cup FV(M))^c$.


\textbf{DEFINIR ALPHA CONVERSIÓN Y SUS DETALLES FINOS}\\

\textbf{DEFINIR BETA CONVERSIÓN Y SUS DETALLES FINOS}\\

\textbf{DEFINIR REGLA XI Y SUS DETALLES FINOS}\\

\subsection{Conjunto cociente y clases de equivalencia}\label{sec:1.2.3}

Considerando los términos lambda y la relación de equivalencia descritos en las
anteriores subsecciones, su conjunto cociente, llamado \(\Lambda\) modulo \(=\)
y denotado \(\Lambda / =\) es el conjunto de todas las clases de equivalencia de
\(\Lambda\) con respecto a \(=\), donde la clase de equivalencia de un elemento
\(\lc{M}\in \Lambda\) es \(\left\{ \lc{N} \in \Lambda \mid \lc{M} = \lc{N}
  \right\}\). \\

Al tener como objeto de estudio de la teoría \(\boldsymbol{\lambda}\) a
este conjunto, se establece que cada término lambda es representativo de una
clase de términos lambda.\\



\subsection{Axiomas de
\texorpdfstring{$\boldsymbol\lambda$}{lambda}} \label{sec:1.2.4}

Sean \(M,\ N,\ Z\in \Lambda\) y \(x,\ y\in V\), la convertibilidad en la teoría
\(\boldsymbol\lambda\) se genera a partir de los siguientes axiomas:

\begin{align}
  (\alpha) &  & \lc{\x.M}=\lc{\y.q[subst[M,x,y]]M} \\
  (\beta)  &  & \lc{(\x.M)N}=\lc{q[subst[M,x,N]]} \\
  (\rho)   &  & \lc{M}=\lc{M} \\
  (\sigma) &  & \infer{\lc{N}=\lc{M}}{\lc{M}=\lc{N}} \\
  (\tau)   &  & \infer{\lc{M}=\lc{P}}{\lc{M}=\lc{N},\ \lc{N}=\lc{P}} \\
  (\nu)    &  & \infer{\lc{M Z}=\lc{N Z}}{\lc{M}=\lc{N}} \\
  (\mu)    &  & \infer{\lc{Z M}=\lc{Z N}}{\lc{M}=\lc{N}} \\
  (\xi)    &  & \infer{\lc{\x.M}=\lc{\x.N}}{\lc{M}=\lc{N}}
\end{align}

\begin{align} 
  &\lc{M} = \lc{M} &(\text{reflexividad})\\
  &\lc{M} = \lc{N} \Rightarrow \lc{N} = \lc{M} &(\text{simetría})\\ 
  &\lc{M} = \lc{N},\ \lc{N} = \lc{L} \Rightarrow \lc{M} = \lc{L} &(\text{transitividad})\\ 
  &\lc{M} = \lc{N} \Rightarrow \lc{M Z} = \lc{N Z}\\
  &\lc{M} = \lc{N} \Rightarrow \lc{Z M} = \lc{Z N}\\
  &\lc{\x.M} = \lc{\y.M}[\lc{x}:=\lc{y}] &(\alpha\text{-conversión})\\ 
  &\lc{(\x.M)N} = \lc{M}[\lc{x}:=\lc{N}] &(\beta\text{-conversión})\\
  &\lc{M} = \lc{N} \Rightarrow \lc{\x.M} = \lc{\x.N} &(\text{regla } \xi)
\end{align}

\textbf{CLARIFICAR DIFERENCIAS SOBRE LAS TEORÍAS Y SUS NOMBRES}\\

\textbf{DESCRIBIR LÓGICA COMBINATORIA Y LA TEORIA LAMBDA I}

\section{Representaciones} \label{sec:1.3}

\subsection{Álgebra booleana} \label{sec:1.3.1}

\subsection{Aritmética} \label{sec:1.3.2}

\subsection{Estructuras complejas} \label{sec:1.3.3}

\subsection{Técnicas de representación} \label{sec:1.3.4}

%%% Local Variables: %%% coding: utf-8 %%% mode: latex %%% TeX-master: "main"
%%% End:

\chapter{Formalización del cálculo lambda}
\label{ch:formalizacion}
La noción de \emph{generalización} es de suma importancia en el estudio general de funciones, operaciones o transformaciones. Los predicados en la lógica de primer orden, las funciones en la matemática clásica, los algoritmos en la computación y las abstracciones en el cálculo lambda pueden ser considerados como implementaciones del concepto de generalización para los sistemas de los que forman parte y en algunos casos son la motivación original para el desarrollo de las teorías que los fundamentan.

El estudio de las propiedades generales de las funciones es una de las motivaciones originales del cálculo lambda, sin embargo, este cálculo se formuló de tal manera que es posible abstraer de su propósito original y ser tratado meramente como un sistema formal \cite{Church:LambdaConversion}.

El presente capítulo tiene el objetivo de formalizar las ideas presentadas en el capítulo \ref{ch:nocion-informal}. La formalización del cálculo lambda se realiza desde dos perspectivas:

\begin{enumerate}
\item Construyendo una \emph{teoría formal}, en donde un conjunto de axiomas y reglas de inferencia permiten plantear razonamientos lógicos para demostrar propiedades del cálculo lambda.
\item Formulando nociones de \emph{reducción}, de tal manera que mediante procedimientos de transformación de expresiones del cálculo lambda, se puedan estudiar sus propiedades.
\end{enumerate}

Independientemente de la perspectiva de la formalización, los conceptos son similares, se describe la misma idea general del cálculo lambda y ambos trabajan con el lenguaje formal de sus expresiones.

De acuerdo a Barendregt \cite[p.~22]{Barendregt:Bible}, el objeto de estudio del cálculo lambda es el conjunto de términos lambda módulo convertibilidad. Estos conceptos serán presentados a lo largo de este capítulo.

El contenido de este capítulo está basado en los primeros cuatro capítulos del libro \emph{The Lambda Calculus, Its Syntax and Semantics} de H.P. Barendregt \cite{Barendregt:Bible} y los capítulos 1, 3, 6, 7 y 8 del libro \emph{Lambda Calculus and Combinators, an Introduction} de J.R. Hindley y J.P. Seldin \cite{HindleySeldin:LambdaCalculusAndCombinators} así como el artículo \emph{A Set of Postulates for the Foundation of Logic} y la monografía \emph{The Calculi of Lambda-Conversion} de Alonzo Church \cite{Church:FoundationsLogic,Church:LambdaConversion}.

\section{Términos lambda}
\label{sec:terminos-lambda}

Esta subsección está basada principalmente en el capítulo 2 de \cite{Barendregt:Bible}.

Los \emph{términos lambda} son la formalización de las expresiones descritas en la sección \ref{sec:expresiones}. El conjunto de todos los términos lambda es un lenguaje formal \( Λ \) en donde sus elementos son cadenas compuestas de símbolos de un alfabeto \cite{Hopcroft:Automata}.

El lenguaje \( Λ \) se puede definir de diferentes maneras, a continuación se presenta una definición inductiva y posteriormente una definición basada en una gramática libre de contexto.

\begin{rem}[Notación]\
  \begin{itemize}
  \item El símbolo \( \implies \) denota una implicación lógica, \( P \implies Q \) se lee ``Si \( P \), entonces \( Q \)''.
  \item El símbolo \( \, \longrightarrow\, \) denota una producción en una gramática, \( P \longrightarrow Q \) se lee ``\( P \) produce \( Q \)''.
  \item El símbolo \( \, \Rightarrow\, \) denota un paso en la derivación de una cadena.
  \end{itemize}
\end{rem}

\begin{defn}[Términos lambda]
  El conjunto \( Λ \) tiene elementos que son cadenas conformadas por símbolos en el alfabeto \( Σ=\{\mathtt{(},\ \mathtt{)},\ \mathtt{.},\ λ\} \cup V \), donde \( V \) es un conjunto infinito \( \{v_{0},\ v_{00},\ ... \} \) de variables. \( Λ \) es el conjunto más pequeño que satisface:
  \label{defn:terminos}
  \begin{subequations}
    \begin{align}
      \label{terminos:atomos} \tag{a}
      x \in V & \implies x \in Λ \\
      \label{terminos:abstracciones} \tag{b}
      M \in Λ,\ x \in V & \implies (λx.M) \in Λ \\
      \label{terminos:aplicaciones} \tag{c}
      M,\ N \in Λ & \implies (M\, N) \in Λ
    \end{align}
  \end{subequations}
\end{defn}

Cada uno de estos tres incisos corresponde a las tres clases de términos lambda:

\begin{description}
\item[\eqref{terminos:atomos}] establece que todo elemento de \( V \) es un término lambda a los cuales se les llama \emph{átomos};
\item[\eqref{terminos:abstracciones}] establece que las cadenas de la forma \( (λx.M) \) son términos lambda, donde \( x \) es un átomo y \( M \) es cualquier término lambda, a estos términos se les llama \emph{abstracciones};
\item[\eqref{terminos:aplicaciones}] establece que las cadenas de la forma \( (M\, N) \) son términos lambda, donde \( M \) y \( N \) son términos lambda cualesquiera, a estos términos se les llama \emph{aplicaciones}.
\end{description}

En el estudio usual de lenguajes formales \cite{Hopcroft:Automata}, \( Λ \) pertenece a la clase de lenguajes libres de contexto y puede ser definido de la siguiente manera:

\begin{defn}[Términos lambda]
  \label{defn:terminos-cfg}
  El conjunto de términos lambda es el lenguaje generado por la gramática libre de contexto \( G \) conformado por

  \begin{description}
  \item[categorías sintácticas] \( T \), \( E \), \( F \) y \( A \), las cuales denotan las reglas para derivar términos lambda, aplicaciones, abstracciones y átomos respectivamente;
  \item[símbolos terminales] \( \{\mathtt{(},\ \mathtt{)},\ \mathtt{.},\ λ,\ v,\ {}_{0}\} \), los cuales son los símbolos que conforman a las cadenas en \( Λ \);
  \item[símbolo inicial] \( T \), el cual es el símbolo del que se derivan todos los términos lambda;
  \item[reglas de producción]
    \begin{subequations}
    \begin{align}
      \label{terminos-cfg:terminos} \tag{a}
      T & \rightarrow E\ \mid\ F\ \mid\ A \\
      \label{terminos-cfg:atomos} \tag{b}
      A & \rightarrow \mathtt{v}_{0}\ \mid\ A {}_{0} \\
      \label{terminos-cfg:abstracciones} \tag{c}
      F & \rightarrow \mathtt{(} λ\ A\ \mathtt{.}\ T\ \mathtt{)} \\
      \label{terminos-cfg:aplicaciones} \tag{d}
      E & \rightarrow \mathtt{(}\ T\ T\ \mathtt{)}
    \end{align}
  \end{subequations}
  \end{description}
\end{defn}

Dada una secuencia de símbolos \( M \), se pueden utilizar estas dos definiciones para verificar si \( M \) es o no un término lambda. En el caso de la definición inductiva, se debe presentar un razonamiento que pruebe que las partes de la cadena satisface la definición \ref{defn:terminos}. En el caso de la gramática libre de contexto \ref{defn:terminos-cfg} se debe presentar una derivación de la cadena a partir de la categoría sintáctica \( T \).

\begin{exmp} Sea \( M = (λv_{0}.(v_{00} (λv_{00}.v_{000}))) \), la cadena \( M \) es un término lambda ya que

  \begin{description}
  \item[Por definición inductiva]
    \begin{align*}
      v_{000} \in V &\implies v_{000} \in Λ; \\
      v_{00} \in V,\ v_{000} \in Λ &\implies (λv_{00}.v_{000}) \in Λ,\ v_{00} \in Λ; \\
      v_{00},\ (λv_{00}.v_{000}) \in Λ &\implies (v_{00} (λv_{00}.v_{000})) \in Λ; \\
      v_{0} \in V,\ (v_{00} (λv_{00}.v_{000})) \in Λ &\implies (λv_{0}.(v_{00} (λv_{00}.v_{000}))).
    \end{align*}
  \item[Por gramática] Se mantienen los espacios en los lados derechos de las producciones de la gramática para ser consistentes, sin embargo, el espacio en blanco no es un símbolo terminal, por lo tanto pueden ser omitidos.
    \begin{align*}
      T &\Rightarrow F \Rightarrow (\ λ\ A\ .\ T\ ) \Rightarrow (\ λ\ v_{0}\ .\ T\ ) \Rightarrow (\ λ\ v_{0}\ .\ E\ ) \Rightarrow (\ λ\ v_{0}\ .\ (\ T\ T\ )\ ) \\
        &\Rightarrow (\ λ\ v_{0}\ .\ (\ A\ T\ )\ ) \Rightarrow (\ λ\ v_{0}\ .\ (\ A_{0}\ T\ )\ ) \Rightarrow (\ λ\ v_{0}\ .\ (\ v_{00}\ T\ )\ ) \\
        &\Rightarrow (\ λ\ v_{0}\ .\ (\ v_{00}\ F\ )\ ) \Rightarrow (\ λ\ v_{0}\ .\ (\ v_{00}\ (\ λ\ A\ .\ T\ )\ )\ ) \\
        &\Rightarrow (\ λ\ v_{0}\ .\ (\ v_{00}\ (\ λ\ A_{0}\ .\ T\ )\ )\ ) \Rightarrow (\ λ\ v_{0}\ .\ (\ v_{00}\ (\ λ\ v_{00}\ .\ T\ )\ )\ ) \\
        &\Rightarrow (\ λ\ v_{0}\ .\ (\ v_{00}\ (\ λ\ v_{00}\ .\ A\ )\ )\ ) \Rightarrow (\ λ\ v_{0}\ .\ (\ v_{00}\ (\ λ\ v_{00}\ .\ A_{0}\ )\ )\ ) \\
        &\Rightarrow (\ λ\ v_{0}\ .\ (\ v_{00}\ (\ λ\ v_{00}\ .\ A_{00}\ )\ )\ ) \Rightarrow (\ λ\ v_{0}\ .\ (\ v_{00}\ (\ λ\ v_{00}\ .\ v_{000}\ )\ )\ ).
    \end{align*}
  \end{description}
\end{exmp}

\begin{exmp} Sea \( N = ((λv_{00}.v_{0}\, v_{00}) v_{0}) \), la cadena \( N \) no es un término lambda ya que

  \begin{description}
  \item[Por definición inductiva] Ya que \( Λ \) se definió como el \emph{conjunto más pequeño}, se demuestra que \( N \not\in Λ \) de la siguiente manera
    \begin{align*}
      (λv_{00}.v_{0}\, v_{00}), v_{0} \in Λ &\implies ((λv_{00}.v_{0}\, v_{00}) v_{0}) \in Λ; \\
      v_{00} \in V,\ v_{0}\, v_{00} \in Λ &\implies (λv_{00}.v_{0}\, v_{00}) \in Λ; \\
      v_{0}\, v_{00} \not\in Λ &\therefore ((λv_{00}.v_{0}\, v_{00}) v_{0}) \not\in Λ.
    \end{align*}
  \item[Por gramática] La gramática no es ambigua, realizando una derivación por la izquierda
    \begin{align*}
      T &\Rightarrow E \Rightarrow (\ T\ T\ ) \Rightarrow (\ F\ T\ ) \Rightarrow (\ (\ λ\ A\ .\ T\ )\ T\ ) \\
        &\Rightarrow (\ (\ λ\ A_{0}\ .\ T\ )\ T\ ) \Rightarrow (\ (\ λ\ v_{00}\ .\ T\ )\ T\ ) \\
        &\nRightarrow (\ (\ λ\ A\ .\ v_{0}\, v_{00}\ )\ T\ ).
    \end{align*}
  \end{description}
\end{exmp}

La sintaxis del cálculo lambda es uniforme, lo cual permite identificar su estructura con facilidad y evitar ambigüedades, sin embargo, suele ser tedioso escribir términos largos debido al extenso uso de paréntesis. Es por esto que en este trabajo se hacen las siguientes consideraciones sobre la notación:

\begin{enumerate}
\item \label{enum:notacion:1} El símbolo \( \synteq \) denota la equivalencia sintáctica entre dos términos lambda, esta equivalencia contempla las consideraciones de este listado.
\item \label{enum:notacion:2} Cuando se hace referencia a \emph{cualquier} término lambda se utilizan las letras mayúsculas \( M \), \( N \), \( P \), etc. Es importante establecer que si en un ejemplo, explicación, teorema o demostración se hace referencia a un término lambda con una letra mayúscula, cualquier otra aparición de esta letra hace referencia a este mismo término dentro de ese contexto.
\item \label{enum:notacion:3} Cuando se hace referencia a \emph{cualquier} átomo se utilizan las letras minúsculas \( x \), \( y \), \( z \), etc. Al igual que en el punto anterior, la aparición de una letra minúscula en un ejemplo, explicación, teorema o demostración hace referencia al mismo átomo.
\item \label{enum:notacion:4} Los paréntesis son omitidos de acuerdo a las siguientes equivalencias sintácticas:
  \begin{enumerate}
  \item \label{enum:notacion:4a} \( ((M\, N) P) \synteq M\, N\, P\), en general, se considera la aplicación de términos lambda con asociación a la izquierda. Se tiene que tener cuidado con respetar esta regla, por ejemplo \( (M(N(O\, P))) \synteq M(N(O\, P)) \not\synteq M\, N\, O\, P \).
  \item \label{enum:notacion:4b} \( (λx.(M N)) \synteq λx.(M N) \), en general, se puede escribir una abstracción omitiendo los paréntesis externos. Es necesario escribir de manera explícita los paréntesis en algunos casos, por ejemplo \( ((λx.(M\, N)) O) \synteq (λx.(M\, N)) O \not\synteq λx.(M\, N)O \) ya que el lado derecho de la equivalencia es sintácticamente equivalente a \( (λx.((M\, N)O)) \).
  \item \label{enum:notacion:4c} \( (λx.(λy.(λz.M))) \synteq (λx\, y\, z.M) \), en general, si el cuerpo de una abstracción es también una abstracción, se pueden agrupar las variables ligadas y enlazadas. Éste abuso de notación es consistente con la representación de funciones de varias variables usada por Schönfinkel \cite{Schonfinkel:Varargs}.
  \end{enumerate}
\item \label{enum:notacion:5} Para hacer referencia a una secuencia con una cantidad arbitraria de términos lambda se usa la notación \( \vec{x}=x_{1},...,x_{n} \) cuando es secuencia de átomos y \( \vec{M}=M_{1},...,M_{n} \) cuando es secuencia de términos lambda en general. Con esta notación se puede abreviar la consideración de \ref{enum:notacion:4a} como
  \[ ((\, ...\, ((M_{1}\, M_{2}) M_{3})\, ...\, ) M_{n}) \synteq \vec{M} \]
  y la consideración de \ref{enum:notacion:4c} como
  \[ (λx_{1}.(λx_{2}.(λx_{3}.\, ...\, (λx_{n}.M)\, ...\, ))) \synteq (λ\vec{x}.M) \]
  Ya que la notación no indica la cantidad de términos en la secuencia, se suele decir que \( \vec{M} \) cabe en \( \vec{N} \) cuando son secuencias con la misma cantidad de elementos.
\item Al escribir términos lambda con repetición de aplicaciones suele ser conveniente utilizar una notación más compacta. Cuando se aplica \( n \) veces un término \( F \) por la izquierda a otro término \( M \) se denota \( F^{n}\, M \). Cuando se aplica \( n \) veces un término \( M \) por la derecha a otro término \( F \) se denota \( F\, M^{\sim n}\). Por ejemplo, el término \( (f(f(f(f\, x)))) \) se puede denotar como \( (f^{4}\, x) \) y el término \( (f\, x\, x\, x\, x) \) se puede denotar como \( (f\, x^{\sim 4}) \). La definición inductiva de esta notación es:
  \begin{align}
    \label{eq:abuso:F}
    \begin{split}
      F^{n+1}\, M & \synteq F (F^{n}\, M) \\
      F^{0}\, M & \synteq M
    \end{split}
  \end{align}
  \begin{align}
    \label{eq:abuso:M}
    \begin{split}
      F\, M^{\sim n+1} & \synteq (F\, M^{\sim n}) M \\
      F\, M^{\sim 0} & \synteq F
    \end{split}
  \end{align}
\end{enumerate}

Inicialmente, estos abusos de notación pueden resultar confusos, sin embargo, al escribir términos lambda complejos resulta conveniente acortarlos. A continuación se muestran ejemplos de términos lambda asociados a términos sintácticamente equivalentes pero escritos con abuso de notación:

\begin{exmp}
  \label{exmp:notacion}
  \begin{align*}
    (((x\, y)z) (y\, x)) & \synteq x\, y\, z (y\, x) \\
    (λx.((u\, x)y)) & \synteq λx.u\, x\, y \\
    (λy.(u(λx.y))) & \synteq λy.u(λx.y) \\
    (((λy.((v\, u)u))z)y) & \synteq (λy.v\, u\, u) z\, y \\
    (((u\, x)(y\, z))(λv.(v\, y))) & \synteq u\, x(y\, z)(λv.v\, y) \\
    ((((λx.(λy.(λz.((x\, z) (y\, z))))) u) v) w) & \synteq (λx\, y\, z.x\, z(y\, z)) u\, v\, w
  \end{align*}
\end{exmp}

\subsection{Estructura}

Dado un término lambda \( M \) es deseable poder cuantificar algunas propiedades de acuerdo a su estructura, la medida más común es la de \emph{longitud}. Esta propiedad resulta importante en los razonamientos inductivos, por ejemplo, al plantear una demostración se suele usar la expresión ``por inducción sobre \( M \)'' la cual técnicamente se refiere a una inducción sobre la longitud de \( M \).

\begin{defn}[Longitud]
  La longitud de un término lambda \( M \), denotada como \( \| M \| \), es la cantidad de \emph{apariciones} de átomos en \( M \), se determina a partir de la estructura del término lambda como:
  \label{defn:longitud}
  \begin{align*}
    \|x\| & = 1 \\
    \|M\, N\| & = \|M\| + \|N\| \\
    \|λx.M\| & = 1 + \|M\|
  \end{align*}
\end{defn}

Debido a que la definición considera la cantidad de átomos en \( M \) y la longitud de un átomo es \( 1 \), se infiere que para cualquier término lambda \( M \), su longitud será estrictamente mayor a cero. Una implicación de esta observación es que al ``desbaratar'' la longitud de un término lambda de acuerdo a su estructura, en el caso de que \( M \) sea una aplicación o una abstracción, la longitud de sus partes es estríctamente menor a su longitud.

\begin{exmp} A continuación se presenta el procedimiento para calcular la longitud del término \( M \synteq (x(λy.y\, u\, x)) \) siguiendo la definición \ref{defn:longitud}
  \begin{align*}
    \| M \| &= \| (x(λy.y\, u\, x)) \| = \| (x (λy.((y\, u) x))) \\
            &= \| x \| + \| (λy.((y\, u) x)) \| = 1 + \| (λy.((y\, u) x)) \| \\
            &= 1 + ( 1 + \| ((y\, u) x) \|  ) = 2 + \| ((y\, u) x) \| \\
            &= 2 + \| (y\, u) \| + \| x \| = 2 + \| (y\, u) \| + 1 = 3 + \| (y\, u) \| \\
            &= 3 + \| y \| + \| u \| = 3 + 1 + 1 \\
            &= 5
  \end{align*}
\end{exmp}

Una cuestión importante al momento de demostrar un teorema o definir un concepto por inducción sobre un término lambda es que usualmente la inducción matemática relaciona proposiciones con números naturales. Sin embargo es posible tener dos términos diferentes \( M \) y \( N \) tal que \( \|M\| = \|N\| \), por ejemplo \( (λx.x) \) y \( (z\, z) \) tienen longitud \( 2 \).

La inducción sobre la longitud de un término lambda considera también la estructura del término, de tal manera que para una proposición \( P \) sobre un término lambda \( M \), los casos base de la inducción son aquellos en donde la estructura no es compuesta (en átomos cuya longitud siempre es \( 1 \)) y la hipótesis de inducción considera que \( P \) se cumple para los subtérminos de \( M \) cuya longitud siempre es estrictamente menor que \( \|M\| \).

En la definición de longitud se menciona de la cantidad de \emph{apariciones} de átomos en \( M \), el concepto de aparición de \( M \) en \( N \) para cualesquiera \( M \) y \( N \) se formaliza a partir del concepto de \emph{subtérmino}.

\begin{defn}[Subtérmino]
  \( M \) es un subtérmino de \( N \), denotado \( M \subset N \) si \( M \in \Sub(N) \), donde \( \Sub(N) \) es la colección de subtérminos de \( N \) definida de manera inductiva como
  \label{defn:subtermino}
  \begin{align*}
    \Sub(x) & = \{ x \} \\
    \Sub(λx.M) & = \Sub(M) \cup \{ λx.M \} \\
    \Sub(M\, N) & = \Sub(M) \cup \Sub(N) \cup \{ M\, N \}
  \end{align*}
\end{defn}

\begin{defn}[Aparición]
  La aparición de \( M \) en \( N \) implica que \( M \subset N \) o que \( M \) es \emph{el} argumento de una abstracción en \( N \).
  \label{defn:aparicion}
\end{defn}

Usualmente se habla de la aparición de \( M \) en \( N \) para referirse a una subtérmino en particular en \( N \), sin embargo, un subtérmino pude \emph{aparecer} varias veces en un término. Algunas clasificaciones de subtérminos son:

\begin{itemize}
\item Si \( M_{1} \) y \( M_{2} \) son subtérminos de \( N \) y no tienen átomos en común, se dice que son términos \emph{disjuntos} de \( N \), ya que si esta condición se cumple \( \Sub(M_{1}) \cap \Sub(M_{2}) = \emptyset \);
\item Si \( M \subset N \) y \( (M\, Z) \subset N \) se dice que \( M \) es un término \emph{activo} en \( N \), de lo contrario, se le llama \emph{pasivo};
\item Si \( M \subset N \) y \( (λx.M) \subset N \), se dice que la aparición \( M \) es el \emph{alcance} de la aparición del átomo \( x \) que acompaña a la \( λ \).
\end{itemize}

\begin{exmp}
  Sea \( M \synteq λx.x\, y (λz.y) \):
  \label{exmp:subterminos-apariciones}
  \begin{itemize}
  \item el término \( (x\, y) \subset M \);
  \item el átomo \( z \not\subset M \) pero si aparece en \( M \), debido a que \( z \) acompaña a una \( λ \);
  \item el término \( y(λz.y) \) a pesar de parecer ser un subtérmino de \( M \) no lo es, esto se puede corroborar escribiendo los términos sin el abuso de notación: \( y(λz.y) \synteq (y(λz.y)) \) y \( M \synteq λx.x y(λz.y) \synteq (λx.((x\, y)(λz.y))) \), en este caso, la clave está en observar la estructura de la aplicación \( (x\, y(λz.y)) \).
  \item Las apariciones de \( x \) y \( (λz.y) \) en \( M \) son disjuntas.
  \item Los términos \( x \) y \( (x\, y) \) son subtérminos activos de \( M \), mientras que \( y \) y \( (λz.y) \) son subtérminos pasivos.
  \end{itemize}
\end{exmp}

Los conceptos de longitud y de subtérmino nos permiten razonar de manera clara sobre la estructura de los términos lambda y con la clasificación de los subtérminos se puede caracterizar el rol que juegan las partes de un término en la estructura general.

\subsection{Clasificación}

A continuación se presentan algunos criterios para clasificar partes de los términos lambda y las propiedades que tienen los términos de acuerdo a su clasificación.

Al considerar las apariciones de átomos en un término lambda, es conveniente diferenciar a los átomos sintácticamente iguales dependiendo de el papel que juegan en el término.

\begin{defn}[Clasificación de variables]\label{defn:clasifvar}
  La aparición de un átomo \( x \) en un término \( P \) es llamada:
  \begin{itemize}
  \item \emph{variable ligada} si es un subtérmino de \( M \) en una abstracción \( (λx.M) \) en \( P \);
  \item \emph{variable enlazada} si y sólo si es la \( x \) que acompaña la \( λ \) de \( (λx.M) \) en \( P \);
  \item \emph{variable libre} en otro caso.
  \end{itemize}
\end{defn}

La diferencia entre un átomo \( x \subset M \) y una aparición de \( x \) en \( M \) es que la aparición se refiere a una variable en particular nombrada \( x \) en una parte específica de la estructura de \( M \). Por ejemplo, en el término lambda \( ((λx.x) x) \) la primera aparición del átomo \( x \) es una variable enlazada, la segunda aparición es una variable ligada y la tercera aparición es una variable libre.

Cuando se abordó el concepto de reducción en la sección \ref{sec:op-reduccion} la distinción entre una variable libre y una ligada era importante ya que las variables libres nunca son sustituídas en una reducción ya que el procedimiento relacionaba únicamente a las variables ligadas en el alcance de una abstracción activa.

\begin{exmp}
  Sea \( M \synteq x(λy.x\, y) \):
  \label{exmp:clasifvar}
  \begin{itemize}
  \item El átomo \( x \) aparece como variable libre dos veces en \( M \);
  \item El átomo \( y \) aparece como variable ligada en \( M \);
  \item El átomo \( y \) aparece como la variable enlazada de la abstracción.
  \end{itemize}
\end{exmp}

En la definición formal de algunos conceptos es conveniente hacer referencia a todas las variables libres de un término lambda.

\begin{defn}[Variables libres]
  El conjunto de variables libres de un término lambda \( M \) se denota \( \FV(M) \) y se define de manera inductiva como:
  \label{defn:varlib}
  \begin{align*}
    \FV(x) & = \{ x \} \\
    \FV(λx.M) & = \FV(M) \setminus \{ x \} \\
    \FV(M\, N) & = \FV(M) \cup \FV(N)
  \end{align*}
  Cuando \( \FV(M)=\emptyset \) se dice que \( M \) es un \emph{combinador} o \emph{término cerrado}.
\end{defn}

\begin{exmp}
  Consideremos los términos \( (x(λx.x\, y\, z)) \), \( (λx\, y\, z.y) \) y \( ((λy.x)λx.y) \).
  \label{exmp:varlib}
  \begin{itemize}
  \item \( \FV(x(λx.x\, y\, z)) = \{x,\ y,\ z\} \);
  \item \( \FV(λx\, y\, z.y)=\emptyset \), por lo tanto es un combinador;
  \item \( \FV((λy.x)λx.y)=\{ x,\ y \} \).
  \end{itemize}
\end{exmp}

En ocaciones es importante distinguir los términos lambda cerrados de aquellos que contienen variables libres, para ello se identifica el subconjunto de \( Λ \) que contiene a todos los términos cerrados:

\begin{defn}[Términos cerrados]
  Se denota como \( Λ^{0} \) al conjunto
  \label{defn:termcerr}
  \[ \{ M \in Λ \mid M \text{ es un término cerrado} \} \]
\end{defn}

La notación \( Λ^{0} \) se puede generalizar para identificar diferentes subconjuntos de \( Λ \) a partir de las variables libres de los términos lambda:

\[ Λ^{0}(\vec{x})=\{ M \in Λ \mid \FV(M) \subseteq \{ \vec{x} \} \} \]

De tal manera que:

\[ Λ^{0}=Λ^{0}(\emptyset) \]

Si consideramos un término \( M \) con variables libres, se puede encontrar otro término \( N \in Λ^{0} \) similar a \( M \), al cual se le llama clausura de \( M \).

\begin{defn}[Clausura] \label{defn:clausura}
  La clausura de un término lambda \( M \) con \( \FV(M) \not= \emptyset \) es un término lambda
  \[ (λ\vec{x}.M) \]
  con \( \vec{x}=\FV(M) \)
\end{defn}

\begin{exmp} \label{exmp:clausura}
  Sea \( M \synteq λz.x\, y\, z \)
  \begin{itemize}
  \item \( (λx\, y.λz.x\, y\, z) \) es una clausura de \( M \);
  \item \( (λy\, x\, z.x\, y\, z) \) es una clausura de \( M \);
  \item \( (λz\, x\, y.λz.x\, y\, z) \) no es una clausura de \( M \).
  \end{itemize}
\end{exmp}

\subsection{Sustitución de términos}
\paragraph{Pendiente explicar el resto de la sección}

\begin{defn}[Sustitución]
  \label{defn:sustitucion}
  Para cualesquiera términos lambda \( M \), \( N \) y \( x \), se define \( M[x:=N] \) como el resultado de sustituir cada aparición libre de \( x \) por \( N \) en \( M \) de acuerdo a las siguientes reglas:
  \begin{align*}
    x[x:=N] & \synteq N; \\
    a[x:=N] & \synteq a && a \not \synteq x; \\
    (P\, Q)[x:=N] & \synteq P[x:=N]\, Q[x:=N]; \\
    (λx.P)[x:=N] & \synteq λx.P; \\
    (λy.P)[x:=N] & \synteq λy.P && x \not\synteq y,\ x \not\in \FV(P); \\
    (λy.P)[x:=N] & \synteq λy.P[x:=N] && x \not\synteq y,\ x \in \FV(P),\ y \not\in \FV(N); \\
    (λy.P)[x:=N] & \synteq λz.P[y:=z][x:=N] && x \not\synteq y,\ x \in \FV(P),\ y \in \FV(N),\ z \not\in \FV(N P).
  \end{align*}
\end{defn}

\begin{exmp} \label{exmp:sustitucion}
  Procedimientos de sustituciones para cada uno de los casos de la definición \ref{defn:sustitucion}:
  \begin{itemize}
  \item Caso \( x[x:=N] \)
    \begin{align*}
      y[y:=λx.x] \synteq λx.x
    \end{align*}
  \item Caso \( a[x:=N] \), donde \( a \not\synteq x \)
    \begin{align*}
      z[w:=x\, x] \synteq z
    \end{align*}
  \item Caso \( (P\, Q)[x:=N] \)
    \begin{align*}
      (y\, x\, x)[x:=y] & \synteq ((y\, x) x)[x:=y] \\
                        & \synteq (y\, x)[x:=y]\, x[x:=y] \\
                        & \synteq (y[x:=y]\, x[x:=y]) y \\
                        & \synteq y\, y\, y
    \end{align*}
  \item Caso \( (λx.P)[x:=N] \)
    \begin{align*}
      (λf\, x.f\, f\, x)[f:=g] \synteq λf\, x.f\, f\, x
    \end{align*}
  \item Caso \( (λy.P)[x:=N] \), donde  \( x \not\synteq y \), \( x \not\in \FV(P) \)
    \begin{align*}
      (λf\, x.f\, f\, x)[f:=g] \synteq λf\, x.f\, f\, x
    \end{align*}
  \item Caso \( (λy.P)[x:=N] \), donde \( x \not\synteq y \), \( x \in \FV(P) \), \( y \not\in \FV(N) \)
    \begin{align*}
      (λf.x\, λx.f\, f\, x)[x:=y] & \synteq λf.(x\, λx.f\, f\, x)[x:=y] \\
                                  & \synteq λf.x[x:=y]\, (λx.f\, f\, x)[x:=y] \\
                                  & \synteq λf.y\, λx.f\, f\, x
    \end{align*}
  \item Caso \( (λy.P)[x:=N] \), donde \( x \not\synteq y \), \( x \in \FV(P) \), \( y \in \FV(N) \) y \( z \not\in \FV(N P) \)
    \begin{align*}
      (λf.x\, λx.f\, f\, x)[x:=f] & \synteq λg.(x\, λx.f\, f\, x)[f:=g][x:=f] \\
                                  & \synteq λg.(x[f:=g](λx.f\, f\, x)[f:=g])[x:=f] \\
                                  & \synteq λg.(x\, λx.(f\, f\, x)[f:=g])[x:=f] \\
                                  & \synteq λg.(x\, λx.((f\, f)[f:=g] x[f:=g]))[x:=f] \\
                                  & \synteq λg.(x\, λx.((f[f:=g]\, f[f:=g]) x))[x:=f] \\
                                  & \synteq λg.(x\, λx.g\, g\, x)[x:=f] \\
                                  & \synteq λg.x[x:=f] (λx.g\, g\, x)[x:=f] \\
                                  & \synteq λg.f\, λx.g\, g\, x
    \end{align*}
  \end{itemize}
\end{exmp}

En el último caso es importante observar que las apariciones ligadas de \( x \) no se sustituyen.

\begin{lem}
  Si \( (y\, x) \not\in \FV(L) \) y \( x \not\synteq y \), entonces
  \[ M[x:=N][y:=L] \synteq M[y:=L][x:=N[y:=L]] \]
\end{lem}

En contraste a la operación de sustitución en donde no se permite introducir o quitar referencias a variables enlazadas, el \emph{contexto} es un término con ``hoyos'':

\begin{defn}[Contexto]
  \label{defn:contexto}
  Un contexto es un término lambda denotado \( C[\quad] \) definido de manera inductiva:
  \begin{itemize}
  \item \( x \) es un contexto;
  \item \( [\quad] \) es un contexto;
  \item Si \( C_{1}[\quad] \) y \( C_{2}[\quad] \) son contextos, entonces \( C_{1}[\quad]\, C_{2}[\quad] \) y \( λx.C_{1}[\quad] \) también lo son.
  \end{itemize}
\end{defn}

Si \( C[\quad] \) es un contexto y \( M \in Λ \), entonces \( C[M] \) denota el resultado de reemplazar por \( M \) los hoyos de \( C[\quad] \). Al realizar esto, las variables libres de \( M \) pueden convertirse en variables ligadas de \( C[M] \).

\begin{exmp}
  Consideremos el contexto \( C[\quad] \synteq λx.x\, λy.[\quad] \) y el término lambda \( M \synteq (x y) \).
  
  \begin{align*}
    C[M] & \synteq (λx.x\, λy.[\quad])[(x\, y)] \\
         & \synteq (λx.x\, λy.(x\, y))
  \end{align*}
  
  El caso análogo con la sustitución es

  \begin{align*}
    (λx.x\, λy.w)[w:=(x\, y)] & \synteq λz.(x\, λy.w)[x:=z][w:=(x\, y)] \\
                              & \synteq λz.(x[x:=z] (λy.w)[x:=z])[w:=(x\, y)] \\
                              & \synteq λz.(z\, λy.w)[w:=(x y)] \\
                              & \synteq λz.z[w:=(x\, y)] (λy.w)[w:=(x\, y)] \\
                              & \synteq λz.z\, λv.w[w:=(x\, y)] \\
                              & \synteq λz.z\, λv.(x\, y)
  \end{align*}
\end{exmp}

\section{Los cálculos de la conversión lambda}
\label{sec:conversion-lambda}

El objetivo principal de esta subsección es presentar una formalización del cálculo lambda descrito en el capítulo \ref{ch:nocion-informal} desde el punto de vista de \emph{teorías formales}. El nombre técnico de la teoría formal principal de este trabajo es \( \bs{λKβ} \), se pueden realizar modificaciones y extensiones a esta teoría y los siguientes conceptos permiten estudiar las implicaciones de estos cambios.

\subsection{Teorías formales}
\label{sec:teorias-formales}

Una \emph{teoría formal} \( \mathcal{T} \) es una tripleta \( (\mathcal{F},\mathcal{A},\mathcal{R}) \) donde

\begin{itemize}
\item \( \mathcal{F} \) es el conjunto de todas las \emph{fórmulas} \( X = Y \) con \( X \) y \( Y \) elementos de un lenguaje formal;
\item \( \mathcal{A} \) es un conjunto de \emph{axiomas} y \( \mathcal{A} \subseteq \mathcal{F} \);
\item \( \mathcal{R} \) es un conjunto de \emph{reglas}.
\end{itemize}

Una regla es una función \( φ \colon \mathcal{F}^{n} \to \mathcal{F} \) con \( n \geq 1 \). Si se consideran \( n \) fórmulas \( A_{1},\ ...\ ,\ A_{n} \) tal que

\[ φ(A_{1},\ ...\ ,\ A_{n})=B \]

Se dice que la secuencia \( \langle A_{1},\ ...\ ,\ A_{n},\ B \rangle \) es una \emph{instancia} de la regla \( φ \). Las primeras \( n \) fórmulas de una instancia son llamadas \emph{premisas} y la última fórmula es llamada \emph{conclusión}. Para escribir una instancia de una regla se utiliza la notación

\[ \infer{B}{A_{1} & ... & A_{n}} \]

\begin{rem}
  En la literatura se pueden encuentrar diferentes maneras de trabajar con teorías formales, dependiendo de su ``estilo'' y definición, por ejemplo en \cite{Troelstra:ProofTheory} las reglas se definen como conjuntos de secuencias \( \langle A_{1},...,A_{n+1} \rangle \) con \( n \) premisas y una conclusión, en donde los axiomas se definen como elementos de \( \mathcal{R} \) con cero premisas. La definición de teoría formal presentada en este trabajo es del estilo Hilbert y está basada en \cite[pp.~69--70]{HindleySeldin:LambdaCalculusAndCombinators}.
\end{rem}

Si consideramos un conjunto de \emph{suposiciones} \( Γ \subseteq \mathcal{F} \), una \emph{deducción} de una fórmula \( B \) desde \( Γ \) es un árbol dirigido de fórmulas en donde los vértices de un extremo son elementos de \( \mathcal{A} \) o \( Γ \), los vértices intermedios son deducidos a partir de los vértices que inciden en ellos a partir de una regla y el vértice de el otro extremo siendo \( B \). Si y solo si existe una deducción para una fórmula \( B \), se dice que \( B \) es \emph{demostrable} en \( \mathcal{T} \) suponiendo \( Γ \), denotado

\[ \mathcal{T},Γ \vdash B \]

En caso que la deducción no tenga suposiciones, se dice que es una \emph{demostración} y que \( B \) es un \emph{teorema}. Cuando una deducción no tiene suposiciones, es decir, \( Γ = \emptyset \) se denota

\[ \mathcal{T} \vdash B \]

La relación binaria \( = \) en las fórmulas de una teoría es una relación de equivalencia, la cual por definición es \emph{reflexiva}, \emph{simétrica} y \emph{transitiva}. La \emph{clase de equivalencia} de un objeto \( x \) con respecto a \( = \) de una teoría formal \( \mc{T} \), denotado \( [x]_{\mc{T}} \), es el conjunto de todos los objetos \( y \) tal que \( x = y \) es una fórmula de \( \mc{T} \).

En el contexto de las teorias que formalizan los cálculos lambda, los objetos que se relacionan son términos lambda. La frase ``módulo convertibilidad'' se refiere al conjunto de todas las clases de equivalencia de \( Λ \) considerando la relación de equivalencia de la teoría formal con la que se esté trabajando.

Que este conjunto sea el objeto de estudio de una teoría \( \bs{λ} \) del cálculo lambda significa que cada elemento de \( Λ \) módulo convertibilidad, denotado \( Λ/\!=_{\bs{λ}} \), es distinto y representa una clase de términos lambda considerados en la teoría \( \bs{λ} \) como equivalentes. Cuando \( \bs{λ} \vdash M = N \) se dice que \( M \) y \( N \) son términos \emph{convertibles}, también denotado \( M =_{\bs{λ}} N \).

Habiendo definido una teoría \( \bs{λ} \), el interés de estudiarla es

\begin{itemize}
\item determinar los términos que son convertibles en \( \bs{λ} \);
\item estudiar las propiedades que comparten dos términos convertibles;
\item modificar a \( \bs{λ} \) y comparar la teoría modificada con la original.
\end{itemize}

La comparación entre teorías usualmente consiste en partir de una teoría \( \bs{λ} \), modificar sus fórmulas, axiomas o reglas para obtener otra teoría \( \bs{λ}^{\prime} \) y determinar si \( \bs{λ} \) y \( \bs{λ}^{\prime} \) son equivalentes. Para poder realizar esto, se debe definir formalmente cómo se modifica una teoría y cómo se demuestra que dos teorías son equivalentes.

Modificar una teoría \( \bs{λ} \) puede implicar cambiar la definición de sus términos, es decir, utilizar un lenguaje formal diferente al de \( \bs{λ} \) para expresar las fórmulas \( M=N \). Hacer cambios al lenguaje formal suele requerir modificar al conjunto \( \mc{F} \), \( \mc{A} \) y \( \mc{R} \) de la teoría.

La modificación al lenguaje formal puede ser únicamente de relevancia sintáctica, por ejemplo modificar una teoría \( \bs{λ} \) cuyo lenguaje de términos es el conjunto \( Λ^{0} \) para que los términos sean escritos con el índice de DeBruijn mostrado en \ref{exmp:debrujn} no tendría implicaciones fuertes en la convertibilidad de la teoría, ya que hay una correspondencia uno a uno entre estas dos notaciones. Por otro lado, modificar una teoría \( \bs{λ} \) con términos \( Λ \) de tal manera que se consideren únicamente los términos cerrados \( Λ^{0} \) si puede tener fuertes implicaciones en la convertibilidad de la teoría ya que habrá términos lambda no admitidos en fórmulas.

Otra manera de modificar una teoría \( \bs{λ} \) es añadir o quitar axiomas y reglas de inferencia. Para abordar la modificación de estas dos componentes de una teoría consideramos que los axiomas son reglas sin premisas.

Cuando se considera extender una teoría \( \bs{λ} \) con una nueva regla \( φ \) lo primero que se debe estudiar es si \( φ \) es \emph{derivable} en \( \bs{λ} \), es decir, si para cada instancia de \( φ \), su conclusión es deducible en \( \bs{λ} \) considerando sus premisas como suposiciones. Formalmente, para cada instancia \( \langle A_{1},\ ...\ ,\ A_{n},\ B \rangle \) de \( φ \), \( φ \) es derivable en \( \bs{λ} \) si y sólo si

\begin{equation}
  \label{eq:teorias-derivable}
  \bs{λ},\ \{ A_{1},\ ...\ ,\ A_{n} \} \vdash B
\end{equation}

Cuando añadir una regla \( φ \) a una teoría \( \bs{λ} \) no cambia el conjunto de teoremas se dice que la regla es \emph{admisible}, por ejemplo si \( φ \) es utilizada en la demostración de un teorema, pero este teorema se puede demostrar sin suponer las premisas de \( φ \), entonces añadir a \( φ \) no afecta el hecho de que el teorema exista en la teoría.

Otra manera de verificar si una regla \( φ \) es admisible en \( \bs{λ} \) es demostrando que la regla es \emph{correcta}. Una regla se dice ser correcta en una teoría \( \bs{λ} \) si y sólo si, para cada instancia \( \langle A_{1},\ ...\ ,\ A_{n},\ B \rangle \) de \( φ \):

\begin{equation}
  \label{eq:teorias-correcta}
  (\bs{λ} \vdash A_{1}),\ ...\ ,\ (\bs{λ} \vdash A_{n}) \implies (\bs{λ} \vdash B)
\end{equation}

Si una regla es derivable, entonces es admisible, sin embargo, una regla admisible no siempre es derivable. Consideremos una instancia \( \mc{r} \) de una regla admisible en \( \bs{λ} \) tal que ni las premisas, ni la conclusión de \( \mc{r} \) son demostrables en la teoría, entonces la implicación \eqref{eq:teorias-correcta} es verdadera para \( \mc{r} \), sin embargo esto no implica que se pueda demostrar la conclusión suponiendo las premisas. Por otro lado, si consideramos una instancia \( \mc{r} \) de una regla derivable en \( \bs{λ} \), entonces ya que la conclusión es demostrable suponiendo las premisas, demostrar las premisas asegura que se puede demostrar la conclusión.

Con estos conceptos se pueden definir dos criterios de equivalencia entre teorías: \emph{equivalentes en teoremas} y \emph{equivalentes en reglas}.

\begin{defn}[Equivalencia de teorías] \label{defn:teorias-equivalentes}
  Sean \( \bs{λ} \) y \( \bs{λ}^{\prime} \) dos teorías formales con el mismo conjunto de fórmulas.

  Se dice que las teorías son \emph{equivalentes en teoremas} cuando cada regla y axioma de \( \bs{λ} \) es admisible en \( \bs{λ}^{\prime} \) y viceversa.

  Se dice que las teorías son \emph{equivalentes en reglas} cuando cada regla y axioma de \( \bs{λ} \) es derivable en \( \bs{λ}^{\prime} \) y viceversa.

  La equivalencia en teoremas es una equivalencia más débil que la equivalencia en reglas.
\end{defn}

\subsection{Teoría \( \bs{λKβ} \)}
\label{sec:teorialambda}

La teoría \( \bs{λKβ} \) es la formalización del cálculo lambda que se ha tratado desde el inicio de este trabajo. Ya que es la teoría principal, a partir de este punto cuando se hable de \emph{la} teoría \( \bs{λ} \) se estará refiriendo a la teoría \( \bs{λKβ} \) y cuando se hable de \emph{las} teorías \( \bs{λ} \) se estará refiriendo a la familia de teorías que formalicen los cálculos lambda.

\begin{defn}[Teoría \( \bs{λKβ} \)]
  \label{defn:teorialambda}

  El conjunto de fórmulas \( \mc{F} \) en \( \bs{λKβ} \) tiene como elementos ecuaciones de la forma:

  \begin{align*}
    M = N & & \forall M,N \in Λ \text{ (de la definición~\ref{defn:terminos})}
  \end{align*}
  
  Los axiomas \( \mc{A} \) de \( \bs{λKβ} \) son:

  \begin{subequations}
    \begin{align}
      \label{teorialambda:alpha} \tag{\( α \)}
      λx.M & = λy.M[x:=y] & &  \forall y \not\in \FV(M) \\
      \label{teorialambda:beta} \tag{\( β \)}
      (λx.M)N & = M[x:=N] \\
      \label{teorialambda:rho} \tag{\( ρ \)}
      M & = M
    \end{align}
  \end{subequations}

  Las reglas \( \mc{R} \) de \( \bs{λKβ} \) son:

  \begin{equation}
    \label{teorialambda:mu} \tag{\( μ \)}
    \infer{Z\, M = Z\, N}{M = N}
  \end{equation}
  \begin{equation}
    \label{teorialambda:nu} \tag{\( ν \)}
    \infer{M\, Z = N\, Z}{M = N}
  \end{equation}
  \begin{equation}
    \label{teorialambda:xi} \tag{\( ξ \)}
    \infer{λx.M = λx.N}{M = N}
  \end{equation}
  \begin{equation}
    \label{teorialambda:tau} \tag{\( τ \)}
    \infer{M = P}{M = N & N = P}
  \end{equation}
  \begin{equation}
    \label{teorialambda:sigma} \tag{\( σ \)}
    \infer{N = M}{M = N}
  \end{equation}
  
\end{defn}

Consideremos la convertibilidad en \( \bs{λKβ} \) de los términos lambda
\begin{align*}
  M &\synteq (λf.x((λy.y\, f) λz.z))w \\
  N &\synteq x\, w
\end{align*}

Se demuestra que \( M=_{\bs{λKβ}}N \) construyendo un árbol de deducción como el de la figura \ref{fig:demostrabilidad}.

\begin{figure}
  \centering
  \begin{tikzpicture}[
    equat/.style={rectangle,draw},grow=up,edge from parent/.style={draw,latex-},
    level 1/.style={sibling distance=20em, level distance=5em},
    level 2/.style={sibling distance=40em},
    level 3/.style={sibling distance=20em},
    level 4/.style={sibling distance=10em}
    ]
    \node [equat] (foo1) {\( (λf.x((λy.y\, f) λz.z))w = x\, w \)}
    child {
      node [equat] (foo3) {\( (λf.x\, f)w = x\, w \)}
    }
    child {
      node [equat] (foo5) {\( (λf.x((λy.y\, f) λz.z))w = (λf.x\, f)w \)}
      child {
        node [equat] (foo7) {\( λf.x((λy.y\, f) λz.z) = λf.x\, f \)}
        child {
          node [equat] (foo9) {\( x((λy.y\, f) λz.z) = x\, f \)}
          child {
            node [equat] (foo11) {\( (λy.y\, f)λz.z = f \)}
            child {
              node [equat] (foo13) {\( (λz.z)f = f \)}
            }
            child {
              node [equat] (foo14) {\( (λy.y\, f)λz.z = (λz.z) f \)}
            }
            edge from parent [] node [right] {\( (μ) \)}
          }
          edge from parent [] node [right] {\( (ξ) \)}
        }
        edge from parent [] node [right] {\( (ν) \)}
      }
    };
    \node [above=0em of foo13] (bar1) {\( (β) \)};
    \node [above=0em of foo14] (bar2) {\( (β) \)};
    \node [above=0em of foo11] (bar3) {\( (τ) \)};
    \node [above=0em of foo1] (bar4) {\( (τ) \)};
    \node [above=0em of foo3] (bar5) {\( (β) \)};
  \end{tikzpicture}
  \caption{Árbol de deducción para demostrar la convertibilidad entre dos términos}
  \label{fig:demostrabilidad}
\end{figure}

\subsection{Combinadores \( \bs{SKI} \)}
\label{sec:combinadores-ski}

\begin{defn}[Combinadores SKI]
  \label{defn:ski}
  Tres términos lambda de suma importancia son
  \begin{align*}
    \bs{I} & \synteq λx.x \\
    \bs{K} & \synteq λx\, y.x \\
    \bs{S} & \synteq λx\, y\, z.x\, z(y\, z)
  \end{align*}
\end{defn}

\begin{cor}
  \label{cor:ski}
  Para todo término \( M,N,L \in Λ \)

  \begin{align*}
    \bs{I}\, M & =_{\bs{λ}} M \\
    \bs{K}\, M\, N & =_{\bs{λ}} M \\
    \bs{S}\, M\, N\, L & =_{\bs{λ}} M\, L (N\, L)
  \end{align*}
\end{cor}

Estos tres combinadores generan en la teoría \( \bs{λ} \) al conjunto \( Λ^{0} \) con combinaciones de aplicaciones. Debido a que \( \bs{SKK} =_{\bs{λ}} \bs{I} \), sólo es necesario combinar con aplicaciones a \( \bs{K} \) y a \( \bs{S} \) para generar cualquier término cerrado.

\begin{defn}[Bases]
  \label{defn:bases}
  \begin{enumerate}
  \item Sea \( \mathcal{X} \subset Λ \). El conjunto de términos \emph{generado} por \( \mathcal{X} \), denotado \( \mathcal{X}^{+} \), es el conjunto mas pequeño tal que
    \begin{enumerate}
    \item \( \mathcal{X} \subseteq \mathcal{X}^{+} \),
    \item \( M, N \in \mathcal{X}^{+} \implies (M N) \in \mathcal{X}^{+} \).
    \end{enumerate}
  \item Sea \( \mathcal{P} \subset Λ \). \( \mathcal{X} \subset Λ \) es una \emph{base} para \( \mathcal{P} \) si para toda \( M \in \mathcal{P} \) existe \( N \in \mathcal{X}^{+} \) tal que \( N = M \).
  \item \( \mathcal{X} \) es llamada una \emph{base} si \( \mathcal{X} \) es una base para \( Λ^{0} \).
  \end{enumerate}
\end{defn}

\begin{lem}
  \label{lem:ski}
  Sea \( λx.M \) una abstracción tal que \( \Sub(M) \) no contiene abstracciones

  \begin{enumerate}
  \item Si \( M = x \), entonces \( λx.M = \bs{I} \);
  \item Si \( x \not\in \FV(M) \), entonces \( λx.M = (\bs{K} M) \);
  \item Si \( M = P\, Q \), entonces \( λx.M = \bs{S}(λx.P)(λx.Q) \).
  \end{enumerate}
\end{lem}

\begin{proof}
  Ver \ref{defn:ski} y \ref{cor:ski}
  \begin{enumerate}
  \item
    \begin{align*}
      (\bs{I}\, N) &= N \\
                   &= ((λx.x) N) \\
                   &= ((λx.M) N)
    \end{align*}
  \item
    \begin{align*}
      ((\bs{K}\, M) N) &= (λx\, y.x)M\, N \\
                       &= (λy.M)N \\
                       &= ((λx.M)N)
    \end{align*}
  \item
    \begin{align*}
      \bs{S}(λx.P)(λx.Q) &= (λabc.(a\, c)(b\, c))(λx.P)(λx.Q) \\
                         &= λc.((λx.P)c)((λx.Q)c) \\
                         &= λc.P[x:=c]Q[x:=c] \\
                         &= λc.(P\, Q)[x:=c] \\
                         &= λx.P\, Q \\
                         &= λx.M
    \end{align*}
  \end{enumerate}
\end{proof}

\begin{prop}
  \label{prop:ski}
  \( \{ \bs{S}, \bs{K}, \bs{I} \} \) es una base, es decir, para todo término \( M \in Λ^{0} \), existe un término \( M' \) compuesto de aplicaciones de \( \bs{S} \), \( \bs{K} \) e \( \bs{I} \) tal que \( M = M' \).
\end{prop}

La demostración de la proposición \ref{prop:ski} consiste en la construcción de un algoritmo para transformar \( M \) a \( M' \).

\begin{proof}
  \label{proof:ski}
  Sea \( M \in Λ^{0} \), se construye un término \( M' \in \{ \bs{S},\bs{K},\bs{I} \}^{+} \) tal que \( M' = M \) enumerando los subtérminos en \( M \) que sean abstracciones de menor a mayor longitud.

  Sea \( λx.N \) la abstracción con menor longitud en \( M \), según la estructura de \( N \) se aplican las siguientes transformaciones:

  \begin{enumerate}
  \item Si \( N = a \)
    \begin{enumerate}
    \item \label{item:ski:1a} Si \( a = x \) se transforma \( λx.N \) a \( \bs{I} \) en \( M \).
    \item \label{item:ski:1b}Si \( a \not= x \) se transforma \( λx.N \) a \( (\bs{K}\, a) \) en \( M \).
    \end{enumerate}
  \item \( N = (P\, Q) \)
    \begin{enumerate}
    \item \label{item:ski:2a} Si \( x \not\in \FV(P) \) y \( x \not\in \FV(Q) \) se transforma \( λx.P\, Q \) a \( \bs{S} (\bs{K}\, P) (\bs{K}\, Q) \) en \( M \).
    \item \label{item:ski:2b} Si \( x \not\in \FV(P) \) y \( x \in \FV(Q) \) se transforma \( λx.P\, Q \) a \( \bs{S} (\bs{K}\, P) (λx.Q) \) en \( M \).
    \item \label{item:ski:2c} Si \( x \in \FV(P) \) y \( x \not\in \FV(Q) \) se transforma \( λx.P\, Q \) a \( \bs{S} (λx.P) (\bs{K}\, Q) \) en \( M \).
    \item \label{item:ski:2d} Si \( x \in \FV(P) \) y \( x \in \FV(Q) \) se transforma \( λx.P\, Q \) a \( \bs{S} (λx.P) (λx.Q) \) en  \( M \).
    \end{enumerate}
  \end{enumerate}

  En los casos \ref{item:ski:2a}, \ref{item:ski:2b}, \ref{item:ski:2c}, \ref{item:ski:2d} se forman abstracciones con longitud menor a \( λx.N \), por lo tanto serán las que se transformarán después. Ya que la longitud de estas abstracciones es estrictamente menor a \( λx.N \) y los casos base \ref{item:ski:1a} y \ref{item:ski:1b} de la transformación no introducen abstracciones, en una cantidad finita de pasos el término \( M \) transformado no tendrá abstracciones de la forma \( λx.N \).

  Para un término \( M \) con sólo una abstracción, \( \mathit{a} \) aplicaciones y \( \mathit{v} \) variables ligadas (no enlazadas) una cota superior para la máxima cantidad de pasos se calcula considerando que para los términos de la forma \( λx.a \) se cumple el caso \ref{item:ski:1b} (el cual aumenta la cantidad de aplicaciones en 1) y que para los términos de la forma \( λx.P\, Q \) se cumple el caso \ref{item:ski:2d} (el cual aumenta la cantidad de aplicaciones en 2) ya que en estos casos se produce el término con mas aplicaciones, las cuales determinan la cantidad de veces que se repite el algoritmo por cada abstracción en \( M \). Para calcular la cota superior de la cantidad de aplicaciones \( a' \) que produce el algoritmo para un término con \( n \) abstracciones se plantea la siguiente relación de recurrencia:

  \begin{align*}
    \mathit{a}'_{0} &= \mathit{a} \\
    \mathit{a}'_{n} &= 2 \times \mathit{a}'_{n-1} + \mathit{v}
  \end{align*}

  Esta recurrencia describe la función \( \mathit{a}' \colon \mathbb{N} \to \mathbb{N} \):

  \[ \mathit{a}'(n) = 2^{n} \times \mathit{a} + (2^{n}-1) \times \mathit{v} \]

  Para la cota superior de la cantidad de pasos realizados por el algoritmo para un término \( M \) con \( n \) abstracciones, se plantea la siguiente relación de recurrencia basada en \( \mathit{a}' \) y en el hecho de que la cantidad de variables ligadas y no enlazadas no aumenta en los pasos del algoritmo:

  \begin{align*}
    \mathit{p}_{0} &= 0 \\
    \mathit{p}_{n} &= \mathit{p}_{n-1} + \mathit{a}'(n-1) + \mathit{v}
  \end{align*}

  Esta recurrencia describe la función \( \mathit{p} \colon \mathbb{N} \to \mathbb{N} \):

  \begin{align*}
    \mathit{p}(n) &= (\mathit{a} + \mathit{v}) \times \sum_{i=0}^{n-1} 2^{i} \\
                  &= (\mathit{a} + \mathit{v}) \times (2^{n}-1)
  \end{align*}
  
\end{proof}

\paragraph{Algoritmo para compilar \( Λ^{0} \bs{\mapsto} \{ \bs{S},\ \bs{K},\ \bs{I} \}^{+} \)}

\begin{algorithm}
  \caption{SKI}
  \begin{algorithmic}
    \REQUIRE \( M \in Λ^{0} \)
    \ENSURE \( M' \in \{ \bs{S},\ \bs{K},\ \bs{I} \} \)
    
    \STATE \( M' \leftarrow M \)
    \STATE \( \mathcal{L} \leftarrow \{ A \in \Sub(M') \mid A \synteq λx.N \} \)
    
    \WHILE{\( \mathcal{L} \not= \emptyset \)}
    
    \STATE \( A \leftarrow λx.N \in \mathcal{L} \mid \| λx.N \| \leq A', \forall A' \in \mathcal{L} \)
    
    \IF{\( A \synteq λx.a \)}
    \IF{\( a \synteq x \)}
    \STATE \( M'[A] \leftarrow \bs{I} \)
    \ELSIF{\( a \not\synteq x \)}
    \STATE \( M'[A] \leftarrow \bs{K}\, a \)
    \ENDIF
    \ELSIF{\( A \synteq λx.P\, Q \)}
    \IF{\( x \not\in \FV(P) \land x \not\in \FV(Q) \)}
    \STATE \( M'[A] \leftarrow \bs{S} (\bs{K}\, P) (\bs{K}\, Q) \)
    \ELSIF{\( x \not\in \FV(P) \land x \in \FV(Q) \)}
    \STATE \( M'[A] \leftarrow \bs{S} (\bs{K}\, P) (λx.Q) \)
    \ELSIF{\( x \in \FV(P) \land x \not\in \FV(Q) \)}
    \STATE \( M'[A] \leftarrow \bs{S} (λx.P) (\bs{K}\, Q) \)
    \ELSIF{\( x \in FV(P) \land x \in \FV(Q) \)}
    \STATE \( M'[A] \leftarrow \bs{S} (λx.P) (λx.Q) \)
    \ENDIF
    \STATE \( M'[A] \leftarrow \mathrm{SKI}(\ M'[A]\ ) \)
    \ENDIF
    
    \STATE \( \mathcal{L} \leftarrow \mathcal{L} \setminus \{ A \} \)
    
    \ENDWHILE
    \RETURN \( M' \)
  \end{algorithmic}
\end{algorithm}

\subsection{Teoría \( \bs{λIβ} \)}
\label{sec:lambda-i-beta}

En el artículo \cite{Church:LambdaConversion}, Alonzo Church presenta una definición del cálculo lambda con un conjunto restringido de términos lambda. A la teoría que considera a este conjunto restringido de términos lambda (denotado \( Λ_{I} \)) y los axiomas y reglas de inferencia de la teoría \( \bs{λ} \) cambiando \( Λ \) por \( Λ_{I} \) se le conoce como teoría \( \bs{λIβ} \) (o el cálculo \( λI \)).

\begin{defn}[Términos en \( Λ_{I} \)]
  \label{defn:lambdaI}
  \begin{align*}
    x \in V & \implies x \in Λ_{I} \\
    M \in Λ_{I},\ x \in \FV{M} & \implies λx.M \in Λ_{I} \\
    M, N \in Λ_{I} & \implies M\, N \in Λ_{I}
  \end{align*}
\end{defn}

La diferencia fundamental entre las teorías \( \bs{λKβ} \) y \( \bs{λIβ} \) es el término lambda \( \bs{K} \), ya que \( \bs{K} \in Λ \setminus Λ_{I} \) pero \( \bs{K} \not\in Λ_{I} \). Esto es debido a que el subtérmino \( λy.x \) en \( \bs{K} \) de la definición \ref{defn:ski} no puede existir en \( Λ_{I} \) debido a que \( y \not\in \FV(x) \).

\subsection{Extensionalidad}
\label{sec:extensionalidad}

El concepto de igualdad de funciones usado en la mayoría de las ramas de la matemática es lo que se conoce como ``extensional'', esta propiedad de las relaciones de equivalencia hace referencia a las características externas de los objetos que compara, en el caso de las funciones, se incluye la suposición de que para funciones \( f \) y \( g \) con el mismo dominio

\[ \forall x [ f(x)=g(x) ] \implies f=g \]

Contraria a esta suposición, en la computación, el tema central son los procedimientos y procesos que describen los programas o algoritmos, cuyas igualdades ``intensional'', es decir, si dos programas computan la misma función matemática, no necesariamente se dice que son el mismo programa ya que uno pudiera ser mas eficiente que otro (la característica de eficiencia es interna a cada algorítmo).

La teoría \( \bs{λ} \) también es intensional: existen dos términos lambda \( F \) y \( G \) tales que para tódo término \( X \)

\[ \bs{λ} \vdash F\, X = G\, X \]

Pero no \( \bs{λ} \vdash F=G \). Por ejemplo, \( F \synteq y \) y \( G \synteq λx.y\, x \)

Cuando se plantea formalizar un cálculo lambda que sea extensional, surge la pregunta, ¿Qué es demostrable en el sistema extensional que no es demostrable en \( \bs{λ} \). A continuación se presentan tres diferentes agregados a la teoría \( \bs{λ} \) las cuales incluyen la propiedad de extensionalidad y que han sido propuestas en la literatura \cite{HindleySeldin:LambdaCalculusAndCombinators,Barendregt:Bible}. Las teorías extendidas son llamadas \( \bs{λζ} \), \( \bs{λ+ext} \) y \( \bs{λη} \) de acuerdo a la regla que se añade a la definición \ref{defn:teorialambda}.

\begin{defn}[Reglas de extensionalidad]
  \label{defn:extensionalidad}
  Cada una de las siguientes reglas nos permite añadir a \( \bs{λ} \) la propiedad de extensionalidad.
  \begin{description}
  \item[Reglas de inferencia]
    \begin{subequations}
      \begin{align}
        \label{extensionalidad:zeta} \tag{\( ζ \)}
        \infer{M = N}{M\, x = N\, x} & & \text{si \( x \not\in \FV(M\, N) \)} \\
        \label{extensionalidad:ext} \tag{ext}
        \infer{M = N}{M\, P = N\, P} & & \forall P \in Λ
      \end{align}
    \end{subequations}
  \item[Axiomas]
    \begin{align}
      \label{extensionalidad:eta} \tag{\( η \)}
      λx.M\, x = M & & \text{si \( x \not\in \FV(M) \)}
    \end{align}
  \end{description}
\end{defn}

La regla \eqref{extensionalidad:zeta} dice, de manera informal, que si \( M \) y \( N \) tienen el mismo efecto sobre un objeto no especificado \( x \), entonces \( M = N \). La regla \eqref{extensionalidad:ext} tiene una infinidad de premisas, una por cada término lambda \( P \), por lo tanto, las deducciones en donde se involucre esta regla serán árboles infinitos.

\paragraph{Pendiente} Comparación entre teorías formales, énfasis en \( η \) contra \( ζ \) contra \( ext \).

\section{Teoría de reducción}
\label{sec:teoriareduccion}

1.14 y 2.9

\subsection{Contracciones}
\label{sec:contracciones}

Transformaciones de términos con un paso.

\begin{defn}[Contracciones Hindley y Seldin]
  \label{defn:contraccion}
  Dado un término lambda \( X \), una \emph{contracción} en \( X \) es una tripleta \( \langle X,R,Y \rangle \), denotada \( X \contract{R} Y \), donde \( R \) es una aparición de un \emph{redex} en \( X \) y \( Y \) es el resultado de contraer \( R \) en \( X \).
\end{defn}

\begin{exmp}
  \begin{align*}
    (λx.(λy.y\, x)z)v &\contract{(λx.(λy.y\, x)z)v} (λy.y\, v)z ,\\
    (λx.(λy.y\, x)z)v &\contract{(λy.y\, x)z} (λx.z\, x)v.
  \end{align*}
\end{exmp}

\subsection{Reducciones}
\label{sec:reducciones}

Reducciones basadas en contracciones, de Barendregt

\begin{defn}[Relación compatible]
  \label{defn:compatible}
  Si dice que una relación binaria \( \bs{R} \) sobre \( Λ \) es:
  \begin{enumerate}
  \item Una \emph{relación compatible} cuando
    \[ (M,M') \in \bs{R} \implies (Z\, M,Z\, M') \in \bs{R},\ (M\, Z,M'\, Z) \in \bs{R},\ (λx.M,λx.M') \in \bs{R} \]
    para toda \( M, M', Z \in Λ \).
  \item Una \emph{relación de congruencia} cuando \( \bs{R} \) es compatible, reflexiva, transitiva y simétrica.
  \item Una \emph{relación de reducción} cuando \( \bs{R} \) es compatible, reflexiva y transitiva.
  \end{enumerate}
  \emph{compatible} cuando

\end{defn}

\begin{note}
  Una relación \( \bs{R} \subseteq Λ^{2} \) es compatible cuando

  \[ (M,M') \in \bs{R} \implies (C[M],C[M']) \in \bs{R} \]

  para toda \( M, M' \in Λ \) y todo contexto \( C[\quad] \), con un hoyo.
\end{note}

\begin{defn}
  \label{defn:nocion-reduccion}
  Una \emph{noción de reducción} en \( Λ \) es una relación binaria \( \bs{R} \) en \( Λ \).
\end{defn}

Sean \( \bs{R}_{1} \) y \( \bs{R}_{2} \) nociones de reducción, la relación \( \bs{R}_{1} \cup \bs{R}_{2} \) se denota \( \bs{R}_{1}\bs{R}_{2} \).


\begin{defn}[Reducción \( \bs{β} \)]
  La regla \eqref{teorialambda:beta} en la teoría \( \bs{λ} \) se puede definir como la reducción:
  
  \[ \bs{β} = \{ ((λx.M)N,M[x:=N]) : M, N \in Λ \} \]
\end{defn}

\begin{defn}
  Sea \( \bs{R} \) una noción de reducción en \( Λ \), \( \bs{R} \) introduce las relaciones binarias:

  \begin{itemize}
  \item R-reducción en un paso, denotada \( \contract{R} \) y definida de manera inductiva como:
    \begin{align*}
      \text{(1)} && (M,N) \in \bs{R} &\implies M \contract{R} N \\
      \text{(2)} && M \contract{R} N &\implies Z\, M \contract{R} Z\, N \\
      \text{(3)} && M \contract{R} N &\implies M\, Z \contract{R} N\, Z \\ 
      \text{(4)} && M \contract{R} N &\implies λx.M \contract{R} λx.N
    \end{align*}
  \item R-reducción, denotada \( \reduce{R} \) y definida de manera inductiva como:
    \begin{align*}
      \text{(1)} && M \contract{R} N &\implies M \reduce{R} N \\
      \text{(2)} && M \reduce{R} M \\
      \text{(3)} && M \reduce{R} N,\ N \reduce{R} L &\implies M \reduce{R} L
    \end{align*}
  \item R-convertibilidad, denotada \( \convertible{R} \) y definida de manera inductiva como:
    \begin{align*}
      \text{(1)} && M \reduce{R} N &\implies M \convertible{R} N \\
      \text{(2)} && M \convertible{R} N &\implies N \convertible{R} M \\
      \text{(3)} && M \convertible{R} N,\ N \convertible{R} L &\implies M \convertible{R} L
    \end{align*}
  \end{itemize}
\end{defn}

\begin{lem}
  Las relaciones \( \contract{R} \), \( \reduce{R} \) y \( \convertible{R} \) son compatibles. Por lo tanto \( \reduce{R} \) es una relación de reducción y \( \convertible{R} \) es una relación de congruencia.
\end{lem}

\begin{proof}
  
\end{proof}

Usualmente una noción de reducción se introduce de la siguiente manera: ``Sea \( \bs{R} \) definida por las siguientes \emph{reglas de contracción} \ \( \bs{R} \colon M \contract{} N \text{ dado que } ... \)''.

Esto significa que \( \bs{R} = \{ (M,N) : ... \} \), por ejemplo, \( \bs{β} \) pudo haber sido introducida por la siguiente regla de contracción

\[ \bs{β} : (λx.M)N \contract{} M[x:=N] \]

\begin{defn}
  \begin{enumerate}
  \item Un \( R \)-redex es un término \( M \) tal que \( (M,N) \in R \) para algún término \( N \). En este caso \( N \) es llamado un \( R \)-contractum de \( M \).
  \item Un término \( M \) es llamado una forma normal de \( R \), denotado \( R-fn \), si \( M \) no contiene algún \( R \)-redex.
  \item Un término \( N \) es una \( R-fn \) de \( M \) (o \( M \) tiene la \( R-fn \) \( N \)) si \( N \) es una \( R-fn \) y \( M \convertible{R} N \).
  \end{enumerate}
\end{defn}

El proceso de pasar de un redex a su contractum es llamado \emph{contracción}. En lugar de escribir ``\( M \) es una \( R-fn \)'' usualmente se escribe ``\( M \) está en \( R-fn \)'', pensando en una máquina que ha llegado a su estado final.

\begin{exmp}
  \( (λx.x\, x)λy.y \) es un \( β \)-redex. Por lo tanto \( (λx.x\, x)(λy.y)z \) no está en \( β-fn \); sin embargo este término tiene la \( β-fn \) \( z \).
\end{exmp}

Plantear las ideas para pasar del concepto de contracción al de reducción y posteriormente al de convertibilidad.

Gráficas de reducción.

Teorema de Church-Rosser y toda la magia necesaria para abordarlo y la magia que valga la pena mencionar en la que CR es importante va aquí.

Probablemente es buena idea separar la sección ``Noción informal del cálculo lambda'' y ``Formalización del cálculo lambda'' en dos capítulos diferentes. ¿Valdrá la pena abordar árboles de Böhm?


%%% Local Variables:
%%% mode: latex
%%% TeX-master: "main"
%%% End:


\chapter{Codificación de objetos}
\label{ch:codificacion}

\section{Álgebra Booleana}
\label{sec:algebra-booleana}

El álgebra booleana es una rama del álgebra en donde las expresiones tienen asociado un valor de \emph{falso} o \emph{verdadero}. Estas expresiones son fundamentales en el estudio de circuitos y programas escritos en lenguajes de programación.

Los términos lambda no tienen asignados un valor de verdad y las operaciones que se plantearon en los primeros dos capítulos involucraron el concepto de falso y verdadero únicamente en el metalenguaje y asociando estos valores no a los términos lambda en sí, si no a propiedades de estos, por ejemplo, es falso que \( \| λx.x \| = 5 \) y es verdadero que \( (\bs{K}\, x) \reduce{β} (λx.y) \). Sin embargo es posible codificar los valores de verdad como elementos de \( Λ \) y construir abstracciones que emulen las propiedades de las operaciones booleanas bajo la \( β \)-reducción. De esta manera se pueden escribir términos que, de acuerdo con la codificación establecida, representen expresiones booleanas y términos lambda al mismo tiempo.

En los lenguajes de programación usualmente se mezclan las expresiones booleanas con otras expresiones y objetos a partir de \emph{predicados}, éstos son funciones con algún dominio \( X \) y codominio \( \{ \mathrm{falso},\ \mathrm{verdadero} \} \). Por ejemplo, al escribir un programa en donde se necesite tomar una desición a partir de si un número \verb!n! es positivo o negativo se escribiría (en pseudocódigo):

\begin{verbatim}
Si esPositivo(n), entonces:
    ...
De lo contrario:
    ...
Fin
\end{verbatim}

En este ejemplo \verb!esPositivo! es un predicado que es evaluado a falso si \verb!n! no es positivo y a verdadero si lo es.

La codificación de valores de verdad y operaciones booleanas es común incluso en lenguajes de programación populares, por ejemplo en C, el tipo \verb!bool! es codificado como un entero, en donde falso es 0 y verdadero cualquier otro entero, a su vez, los enteros son codificados usualmente como secuencias de 32 bits en complemento a dos. Por lo tanto, si \verb!esPositivo! fuera una función de C: \verb!esPositivo(8)! sería evaluado a 1 y \verb!esPositivo(-8)! sería evaluado a 0.

Al igual que el cálculo lambda, otras teorías que fundamentan las ciencias de la computación también carecen de expresiones y operaciones booleanas. En el caso de la máquina de Turing los cambios de estado en la ejecución de un programa se determinan a partir de su función de transición y predicados simples de igualdad entre símbolos del alfabeto de cinta se realizan en un paso, sin embargo, predicados mas complejos requieren ser codificados con estados, transiciones y anotaciones en su cinta.

\subsection{Valores de verdad}
\label{sec:valores-de-verdad}

En el álgebra booleana, los valores de las expresiones son falso y verdadero. El nombre de estos valores no es de relevancia y usualmente falso se representa como 0 y verdadero como 1. El aspecto importante de estos valores es que son distintos y si un valor \( x \) no es uno, entonces es el otro.

Podemos ignorar la representación concreta de estos valores y pensar en una situación hipotética: Una persona omnisciente y muda llamada \( P \) puede decirme si una oración que le digo es falsa o verdadera dándole una manzana y una pera; si me regresa la manzana significa que la oración es verdadera y si me regresa la pera significa que la oración es falsa. En este planteamiento irreal e hipotético, no fué necesario conocer la estructura de la verdad y la falsedad, solo fué necesario tener a alguien que tomara una desición (en este caso \( P \)) y proveer dos objetos que podemos distinguir entre sí (en este caso la manzana y la pera). Las desiciones de esta persona pueden ser los conceptos de falso y verdadero si nunca podemos conocer los valores booleanos.

Detrás del concepto de falso y verdadero, está el concepto de \emph{desición}, la codificación que se desarrolla está basada en este concepto y aparece en \cite[p.~133]{Barendregt:Bible}.

Supongamos que \( P \) es un término lambda el cual puede ser aplicado a una oración \( O \), al \( β \)-reducir \( (P\, O) \) se obtiene una decisión \( D \) la cual al ser aplicada a dos términos lambda \( M \) y \( N \) se \( β \)-reduce a \( M \) si la oración \( O \) es verdadera y a \( M \) si es falsa:

\[ P\, O \reduce{β} D, \]
\[ D\, M\, N \reduce{β} \begin{cases} M & \text{si \( O \) es verdadera}\\ N & \text{si \( O \) es falsa}\end{cases} .\]

Para fines prácticos no es necesario saber cómo es \( P \) ni \( O \), lo importante es que cuando \( O \) es cierta, \( D \) eligirá \( M \) y si \( O \) es falsa, eligirá \( N \). Por lo tanto, \( (P\, O) = D \) es un término lambda de la forma

\[ λx\, y.Q \]

Si \( D \) es una desición tomada por que \( O \) es verdadera, podemos asegurar que \( (D\, M\, N) = M \), por lo tanto:

\[ D \synteq λx\, y.x \]

Si \( D \) es una desición tomada por que \( O \) es falsa, podemos asegurar que \( (D\, M\, N) = N \), por lo tanto:

\[ D \synteq λx\, y.y \]

Teniendo los términos lambda que representan la desición de \( P \) ante una oración falsa y ante una oración verdadera, se puede considerar que estos términos representan el concepto de falso y verdadero.

\begin{defn}[Valores de verdad]
  \label{defn:valores-verdad}
  El concepto de falso y verdadero es codificado en el cálculo lambda como los términos \( \bs{T} \) y \( \bs{F} \) respectivamente.
  \begin{align*}
    \bs{T} &\synteq λx\, y.x & \bs{F} &\synteq λx\, y.y
  \end{align*}
\end{defn}

Utilizar \( \bs{T} \) y \( \bs{F} \) en términos lambda es similar a imitar a \( P \) y determinar cuando \( O \) es verdadera o falsa. Esto es debido a que se pueden plantear predicados que sean conceptualmente ilógicos, por ejemplo, si \verb!esPositivo! se define de tal manera que sin importar en que valor sea evaluado siempre resulte en falso, los programas que se escriban no van a funcionar suponiendo que \verb!esPositivo! calcula lo que debe de calcular, sin embargo lo importante de codificar el álgebra booleana es poder manipular los valores de falso y verdadero, no representar un término \( P \) que determine verdades absolutas.

\subsection{Expresiones booleanas}

Las expresiones booleanas se conforman de operaciones y valores de verdad. Las operaciones más básicas son la conjunción, la disyunción y la negación, también llamadas \( AND \), \( OR \), \( NOT \) y denotadas \( \land \), \( \lor \) y \( \lnot \) respectivamente.

La conjunción y la disyunción son operaciones binarias definidas en

\[ \{ \mathrm{falso},\ \mathrm{verdadero} \}^{2} \to \{ \mathrm{falso},\ \mathrm{verdadero} \} \]

y la negación es una operación unaria definida en

\[ \{ \mathrm{falso},\ \mathrm{verdadero} \} \to \{ \mathrm{falso},\ \mathrm{verdadero} \}. \]

Las tablas de verdad en el cuadro \ref{tab:and-or-not} establecen los resultados de estas tres operaciones para cada valor en su dominio.

\begin{table}[h!]
  \centering
  \small
  \begin{tabular}{|c|c|c|c|}
    \hline
    \( x \) & \( y \) & \( x \land y \) & \( x \lor y \) \\ [0.5ex]
    \hline\hline
    falso & falso & falso & falso \\
    falso & verdadero & falso & verdadero \\
    verdadero & falso & falso & verdadero \\
    verdadero & verdadero & verdadero & verdadero \\
    \hline
  \end{tabular}
  \hfill
  \begin{tabular}{|c|c|}
    \hline
    \( x \) & \( \lnot x \) \\ [0.5ex]
    \hline\hline
    falso & verdadero  \\
    verdadero & falso \\
    \hline
  \end{tabular}
  \caption{Tablas de verdad para \( \land \), \( \lor \) y \( \lnot \)}
  \label{tab:and-or-not}
\end{table}

En el álgebra booleana, las expresiones se escriben en notación de infijo, utilizan paréntesis para agrupar expresiones y cuando los paréntesis son omitidos la negación tiene mayor presedencia que la conjunción y la conjunción tiene mayor presedencia que la disyunción, por ejemplo:

\[ \mathrm{verdadero} \land \mathrm{falso} \lor \lnot \mathrm{falso} \]
\[ \lnot (\mathrm{falso} \lor \mathrm{falso}) \]
\[ \mathrm{verdadero} \land (\mathrm{falso} \lor \mathrm{falso}) \]

Esta notación es conveniente para escribir expresiones booleanas de manera concisa, pero es únicamente una conveniencia sintáctica del álgebra booleana. La codificación que se desarrolla de las operaciones seguirá las convenciones sintácticas del cálculo lambda, por ejemplo, suponiendo que \( \bs{\land} \), \( \bs{\lor} \), \( \bs{\lnot} \) son términos lambda, las expresiones mencionadas escribirían con notación de prefijo:

\[ \bs{\lor} (\bs{\land}\, \bs{T}\, \bs{F}) \bs{F} \]
\[ \bs{\lnot} (\bs{\lor}\, \bs{F}\, \bs{F}) \]
\[ \bs{\land}\, \bs{T} (\bs{\lor}\, \bs{F}\, \bs{F}) \]

Al igual que los valores de verdad, las operaciones básicas son codificadas como abstracciones del cálculo lambda. Hay varias metodologías para derivar términos lambda para las operaciones booleanas a partir de \( \bs{T} \) y \( \bs{F} \), en esta sección se abordarán dos:

\begin{itemize}
\item Combinando valores de verdad
\item Programando las operaciones
\end{itemize}

La primer metodología parte de la observación de que la codificación de falso y verdadero son abstracciones, por lo tanto, es posible \( β \)-reducirlas al aplicarlas a otros términos; se explora la clase de términos lambda en \( \{ \bs{T},\ \bs{F} \}^{+} \).

La segunda metodología presenta la construcción del operador condicional, a partir del cual se derivan las operaciones booleanas como si fueran programas de computadora.

\subsubsection{Combinaciones de valores de verdad}
\label{sec:combinacion-valores}

Una manera de obtener términos lambda a partir de \( \bs{F} \) y \( \bs{T} \) es \( β \)-reducir combinaciones de aplicaciones entre estos valores. En el cuadro \ref{tab:verdad-pares} se muestran los términos obtenidos al reducir combinaciones de dos valores de verdad.

\begin{table}[h!]
  \centering
  \begin{tabular}{|c||l|}
    \hline
    \( \bs{F}\, \bs{F} \) & \( (λx\, y.y)\bs{F} \reduce{β} λy.y \synteq \bs{I} \) \\
    \hline
    \( \bs{F}\, \bs{T} \) & \( (λx\, y.y)\bs{T} \reduce{β} λy.y \synteq \bs{I} \) \\
    \hline
    \( \bs{T}\, \bs{F} \) & \( (λx\, y.x)\bs{F} \reduce{β} λy.\bs{F} \synteq \bs{K}\, \bs{F} \) \\
    \hline
    \( \bs{T}\, \bs{T} \) & \( (λx\, y.x)\bs{T} \reduce{β} λy.\bs{T} \synteq \bs{K}\, \bs{T} \) \\
    \hline
  \end{tabular}
  \caption{Posibles combinaciones de valores de verdad por pares.}
  \label{tab:verdad-pares}
\end{table}

En las reducciones de \ref{tab:verdad-pares} se pueden observar cuatro términos, a partir de estos se puede descubrir la operación de negación:

\begin{itemize}
\item \( (\bs{F}\, \bs{F}) \) se reduce a la abstracción identidad, esto significa que para cualquier término \( M \in Λ \)
  \[ \bs{λ} \vdash (\bs{F}\, \bs{F}\, M) = M \]
\item Al igual que la primer reducción \( (\bs{F}\, \bs{T}) \) se reduce a \( \bs{I} \), por lo tanto se concluye que para cualesquiera términos \( M \in Λ \), \( N \in \{ \bs{F},\ \bs{T} \} \)
  \[ \bs{λ} \vdash (\bs{F}\, N\, M) = M \]
\item \( (\bs{T}\, \bs{F}) \) se reduce a la abstracción constante de \( \bs{F} \), esto significa que para cualquier término \( M \in Λ \)
  \[ \bs{λ} \vdash (\bs{T}\, \bs{F}\, M) = \bs{F} \]
\item Al igual que la tercer reducción \( (\bs{T}\, \bs{T}) \) se reduce a \( (\bs{K}\, \bs{T}) \), por lo tanto se concluye que para cualesquiera términos \( M \in Λ \), \( N \in \{ \bs{F},\ \bs{T} \} \)
  \[ \bs{λ} \vdash (\bs{T}\, N\, M) = N \]
\end{itemize}

Debido a las reducciones mostradas en el cuadro \ref{tab:verdad-pares} se puede analizar que a partir de un témino \( \bs{F} \), se puede obtener \( \bs{T} \) al reducir \( (\bs{F}\, N\, \bs{T}) \) y que a partir de un término \( \bs{T} \), se puede obtener \( \bs{F} \) al reducir \( (\bs{T}\, \bs{F}\, M) \). Considerando que \( N \synteq \bs{F} \) y \( M \synteq \bs{T} \) las reducciones serían:

\[ \bs{F}\, \bs{F}\, \bs{T} \reduce{β} \bs{T} \]
\[ \bs{T}\, \bs{F}\, \bs{T} \reduce{β} \bs{F} \]

Si se considera que \( P \in \{ \bs{F},\ \bs{T} \} \)

\[ P\, \bs{F}\, \bs{T} \reduce{β} \bs{\lnot}\, P \]

\begin{rem}[Sobre la \( β \)-reducción]
  En el tratamiento de la codificación del álgebra booleana en el cálculo lambda, cuando se \( β \)-reducen términos lambda que tienen como subtérminos valores que suponemos son \( \bs{F} \) o \( \bs{T} \) se extiende la teoría \( \bs{λ} \) con la siguiente ecuación:

  \begin{align*}
    P\, \bs{T}\, \bs{F} &= P && \text{si \( P \in \{ \bs{F},\ \bs{T} \} \)}
  \end{align*}
\end{rem}

\begin{defn}[Operación de negación]
  \label{defn:negacion}
  El término lambda \( \bs{\lnot} \synteq (λp.p\, \bs{F}\, \bs{T}) \) se reduce a \( \bs{T} \) cuando es aplicado a \( \bs{F} \) y viceversa

  \begin{align*}
    \bs{\lnot}\, \bs{F} &\synteq (λp.p\, \bs{F}\, \bs{T}) \bs{F} \\
                        &\contract{β} \bs{F}\, \bs{F}\, \bs{T} \\
                        &\reduce{β} \bs{T}
  \end{align*}
  \begin{align*}
    \bs{\lnot}\, \bs{T} &\synteq (λp.p\, \bs{F}\, \bs{T}) \bs{T} \\
                        &\contract{β} \bs{T}\, \bs{F}\, \bs{T} \\
                        &\reduce{β} \bs{F}
  \end{align*}
\end{defn}

Las reducciones del cuadro \ref{tab:verdad-pares} se pueden aplicar a \( \bs{F} \) y \( \bs{T} \) para obtener todas las posibles combinaciones de aplicaciones de valores de verdad de la forma \( ((P\, M) N) \), en el cuadro \ref{tab:verdad-tripletas} se muestran las reducciones de las nuevas aplicaciones.

\begin{table}[h!]
  \centering
  \begin{tabular}{|c||l|}
    \hline
    \( \bs{F}\, \bs{F}\, \bs{F} \) & \( \bs{I}\, \bs{F} \reduce{β} \bs{F} \) \\
    \hline
    \( \bs{F}\, \bs{F}\, \bs{T} \) & \( \bs{I}\, \bs{T} \reduce{β} \bs{T} \) \\
    \hline
    \( \bs{F}\, \bs{T}\, \bs{F} \) & \( \bs{I}\, \bs{F} \reduce{β} \bs{F} \) \\
    \hline
    \( \bs{F}\, \bs{T}\, \bs{T} \) & \( \bs{I}\, \bs{T} \reduce{β} \bs{T} \) \\
    \hline
    \( \bs{T}\, \bs{F}\, \bs{F} \) & \( \bs{K}\, \bs{F}\, \bs{F} \reduce{β} \bs{F} \) \\
    \hline
    \( \bs{T}\, \bs{F}\, \bs{T} \) & \( \bs{K}\, \bs{F}\, \bs{T} \reduce{β} \bs{F} \) \\
    \hline
    \( \bs{T}\, \bs{T}\, \bs{F} \) & \( \bs{K}\, \bs{T}\, \bs{F} \reduce{β} \bs{T} \) \\
    \hline
    \( \bs{T}\, \bs{T}\, \bs{T} \) & \( \bs{K}\, \bs{T}\, \bs{T} \reduce{β} \bs{T} \) \\
    \hline
  \end{tabular}
  \caption{Posibles combinaciones de valores de verdad con asociación a la izquierda.}
  \label{tab:verdad-tripletas}
\end{table}

Al observar el cuadro \ref{tab:verdad-tripletas}, se distinguen algúnos patrones en los resultados de las reducciones, por ejemplo, si \( P \) es un valor de verdad cualquiera, \( (P\, \bs{F}\, \bs{F}) \) se reduce a \( \bs{F} \) y \( (P\, \bs{T}\, \bs{T}) \) se reduce a \( \bs{T} \), las combinaciones mas interesantes se presentan en los renglones 2, 3, 6 y 7.

En búsqueda de las operaciones binarias de conjunción y disyunción se desarrollan tablas de verdad con las posibles combinaciones de dos términos \( P, Q \in \{ \bs{F},\ \bs{T} \} \). La cantidad de combinaciones de estos valores es \( 2 \times \binom 3 2 = 2 \times \frac{3!}{2!} = 6 \) y son \( (P\, Q\, \bs{F}) ,\ (P\, Q\, \bs{T}) ,\ (P\, \bs{F}\, Q) ,\ (P\, \bs{T}\, Q) ,\ (\bs{F}\, P\, Q) ,\ (\bs{T}\, P\, Q) \). Las tablas de verdad de estas combinaciones intercambiando las posiciones de \( P \) y \( Q \) serían las mismas ya que ambos términos toman los valores de falso y verdadero en las tablas de verdad. En el cuadro \ref{tab:verdad-pq} se muestran estas tablas.

\begin{table}[h!]
  \centering
  \begin{tabular}{|c|c||c|c|c|c|c|c|}
    \hline
    \( P \) & \( Q \) & \( P\, Q\, \bs{F} \) & \( P\, Q\, \bs{T} \) & \( P\, \bs{F}\, Q \) & \( P\, \bs{T}\, Q \) & \( \bs{F}\, P\, Q \) & \( \bs{T}\, P\, Q \) \\ [0.5ex]
    \hline
    \hline
    \( \bs{F} \) & \( \bs{F} \) & \( \bs{F} \) & \( \bs{T} \) & \( \bs{F} \) & \( \bs{F} \) & \( \bs{F} \) & \( \bs{F} \) \\
    \( \bs{F} \) & \( \bs{T} \) & \( \bs{F} \) & \( \bs{T} \) & \( \bs{T} \) & \( \bs{T} \) & \( \bs{T} \) & \( \bs{F} \) \\
    \( \bs{T} \) & \( \bs{F} \) & \( \bs{F} \) & \( \bs{F} \) & \( \bs{F} \) & \( \bs{T} \) & \( \bs{F} \) & \( \bs{T} \) \\
    \( \bs{T} \) & \( \bs{T} \) & \( \bs{T} \) & \( \bs{T} \) & \( \bs{F} \) & \( \bs{T} \) & \( \bs{T} \) & \( \bs{T} \) \\
    \hline
  \end{tabular}
  \caption{Tablas de verdad considerando dos variables \( P \) y \( Q \) en aplicaciones de tres términos}
  \label{tab:verdad-pq}
\end{table}

Las columnas de la combinación \( (P\, Q\, \bs{F}) \) y \( (P\, \bs{T}\, Q) \) del cuadro \ref{tab:verdad-pq} corresponden a la operación de conjunción y disyunción respectivamente, como aparecen en el cuadro \ref{tab:and-or-not}. Las otras combinaciones corresponden a operaciones no básicas del álgebra booleana: \( (P\, Q\, \bs{T}) \) es la implicación material; \( (P\, \bs{F}\, Q) \) es la no implicación inversa; \( (\bs{F}\, P\, Q) \) es la proyección de \( Q \); y \( (\bs{T}\, P\, Q) \) es la proyección de \( P \).

Ya que  \( \bs{λ} \vdash (P\, Q\, \bs{F}) = (\bs{\land}\, P\, Q) \) y \( \bs{λ} \vdash (P\, \bs{T}\, Q) = (\bs{\lor}\, P\, Q) \), se construyen los términos \( \bs{\land} \) y \( \bs{\lor} \) abstrayendo a \( P \) y \( Q \) de las igualdades.

\begin{defn}[Operación de conjunción]
  \label{defn:conjuncion}
  El término lambda que representa la conjunción es

  \[ \bs{\land} \synteq λp\, q.p\, q\, \bs{F} \]

  y cumple las siguientes propiedades de \( β \)-reducción al ser aplicada a valores de verdad:

  \begin{align*}
    \bs{\land}\, \bs{F}\, \bs{F} &\synteq (λp\, q.p\, q\, \bs{F}) \bs{F}\, \bs{F} \\
                                 &\reduce{β} \bs{F}\, \bs{F}\, \bs{F} \\
                                 &\reduce{β} \bs{F}
  \end{align*}

  \begin{align*}
    \bs{\land}\, \bs{F}\, \bs{T} &\synteq (λp\, q.p\, q\, \bs{F}) \bs{F}\, \bs{T} \\
                                 &\reduce{β} \bs{F}\, \bs{T}\, \bs{F} \\
                                 &\reduce{β} \bs{F}
  \end{align*}

  \begin{align*}
    \bs{\land}\, \bs{T}\, \bs{F} &\synteq (λp\, q.p\, q\, \bs{F}) \bs{T}\, \bs{F} \\
                                 &\reduce{β} \bs{T}\, \bs{F}\, \bs{F} \\
                                 &\reduce{β} \bs{F}
  \end{align*}

  \begin{align*}
    \bs{\land}\, \bs{T}\, \bs{T} &\synteq (λp\, q.p\, q\, \bs{F}) \bs{T}\, \bs{T} \\
                                 &\reduce{β} \bs{T}\, \bs{T}\, \bs{F} \\
                                 &\reduce{β} \bs{T}
  \end{align*}
\end{defn}

\begin{defn}[Operación de disyunción]
  \label{defn:disyuncion}
  El término lambda que representa la disyunción es

  \[ \bs{\lor} \synteq λp\, q.p\, \bs{T}\, q \]

  y cumple las siguientes propiedades de \( β \)-reducción al ser aplicada a valores de verdad:

  \begin{align*}
    \bs{\lor}\, \bs{F}\, \bs{F} &\synteq (λp\, q.p\, \bs{T}\, q) \bs{F}\, \bs{F} \\
                                &\reduce{β} \bs{F}\, \bs{T}\, \bs{F} \\
                                &\reduce{β} \bs{F}
  \end{align*}

  \begin{align*}
    \bs{\lor}\, \bs{F}\, \bs{T} &\synteq (λp\, q.p\, \bs{T}\, q) \bs{F}\, \bs{T} \\
                                &\reduce{β} \bs{F}\, \bs{T}\, \bs{T} \\
                                &\reduce{β} \bs{T}
  \end{align*}

  \begin{align*}
    \bs{\lor}\, \bs{T}\, \bs{F} &\synteq (λp\, q.p\, \bs{T}\, q) \bs{T}\, \bs{F} \\
                                &\reduce{β} \bs{T}\, \bs{T}\, \bs{F} \\
                                &\reduce{β} \bs{T}
  \end{align*}

  \begin{align*}
    \bs{\lor}\, \bs{T}\, \bs{T} &\synteq (λp\, q.p\, \bs{T}\, q) \bs{T}\, \bs{T} \\
                                &\reduce{β} \bs{T}\, \bs{T}\, \bs{T} \\
                                &\reduce{β} \bs{T}
  \end{align*}
\end{defn}

Esta metodología para encontrar operaciones del álgebra booleana, aplicando los términos codificados de los valores de verdad, es tediosa pero hasta cierto grado efectiva. Como en los casos de las operaciones no básicas mostradas en el cuadro \ref{tab:verdad-pq}, operaciones del álgebra booleana pueden ser ``descubiertas'' y no construídas. Ya que la negación, la conjunción y la disyunción fueron descubiertas con este método, cualquier operación booleana eventualmente será encontrada como combinación de valores de verdad. Sin embargo, descubrir la codificación de una operación booleana complicada utilizando este método es un proceso muy tardado.

\subsubsection{Programación de operaciones booleanas}
\label{sec:programacion-operaciones}

Otra metodología que permite construír las operaciones booleanas como términos lambda es la de partir de un algorítmo que las describa. Usualmente las operaciones booleanas no son definidas como procedimientos, si no como operaciones primitivas del lenguaje utilizado para describirlos.

Consideremos dos términos \( M \) y \( N \). Ya que \( (\bs{T}\, M\, N) \reduce{β} M \) y \( (\bs{F}\, M\, N) \reduce{β} N \), si \( M \reduce{β} M' \) y \( N \reduce{β} N' \), entonces

\[ \bs{T}\, M\, N \reduce{β} M' \]

\[ \bs{F}\, M\, N \reduce{β} N' \]

Es decir, si \( P \in \{ \bs{F},\ \bs{T} \} \):

\[ P\, M\, N \reduce{β} \begin{cases} M' & P \synteq \bs{T} \\ N' & P \synteq \bs{F} \end{cases} \]

Esta aplicación de un valor de verdad a dos términos lambda cualquiera permite capturar el concepto de una expresión o sentencia condicional, usualmente llamada en los lenguajes de programación como sentencia \verb!if-then-else!.

\begin{defn}[Expresión condicional]
  \label{defn:condicional}
  El término lambda que representa a la expresión condicional es

  \[ \bs{\prec} \synteq λp\, m\, n.p\, m\, n \]

  Y si \( P \) es un valor de verdad, entonces

  \begin{align*}
    \bs{\prec}\, P\, M\, N &\synteq (λp\, m\, n.p\, m\, n)P\, M\, N \\
                           &\reduce{β} P\, M\, N
  \end{align*}

  Un programa de la forma

\begin{verbatim}
Si P, entonces:
    M
De lo contrario:
    N
Fin
\end{verbatim}

  Puede ser traducido a \( (\bs{\prec}\, P\, M\, N) \)
\end{defn}

Consideremos la siguiente definición en pseudocódigo de la operación de negación:

\begin{verbatim}
Negacion (predicado) :=
    Si predicado, entonces:
        Regresa Falso
    De lo contrario:
        Regresa Verdadero
    Fin
Fin
\end{verbatim}

El pseudocódigo se traduce al cálculo lambda como

\[ \bs{\lnot} \synteq λp.\bs{\prec}\, p\, \bs{F}\, \bs{T} \]

El cuerpo de la abstracción puede ser \( β \)-reducido para obtener el término de la definición \ref{defn:negacion}

\begin{align*}
  λp.\bs{\prec}\, p\, \bs{F}\, \bs{T} &\synteq λp.(λp\, m\, n.p\, m\, n) p\, \bs{F}\, \bs{T} \\
                                      &\reduce{β} λp.p\, \bs{F}\, \bs{T}
\end{align*}

Para la operación de conjunción, se considera el siguiente pseudocódigo:

\begin{verbatim}
Conjuncion (predicado1, predicado2) :=
    Si predicado1, entonces:
        Si predicado2, entonces:
            Regresa Verdadero
        De lo contrario:
            Regresa Falso
        Fin
    De lo contrario:
        Regresa Falso
    Fin
Fin
\end{verbatim}

Traducido al cálculo lambda como

\[ \bs{\land} \synteq λp_{1}\, p_{2}.\bs{\prec}\, p_{1} (\bs{\prec}\, p_{2}\, \bs{T}\, \bs{F}) \bs{F} \]

Al \( β \)-reducir el cuerpo de la abstracción se obtiene el término de la definición \ref{defn:conjuncion}

\begin{align*}
  λp_{1}\, p_{2}.\bs{\prec}\, p_{1} (\bs{\prec}\, p_{2}\, \bs{T}\, \bs{F}) \bs{F}
  &\synteq λp_{1}\, p_{2}.(λp\, m\, n.p\, m\, n) p_{1} ((λp\, m\, n.p\, m\, n) p_{2}\, \bs{T}\, \bs{F}) \bs{F} \\
  &\reduce{β} λp_{1}\, p_{2}.p_{1} ((λp\, m\, n.p\, m\, n) p_{2}\, \bs{T}\, \bs{F}) \bs{F} \\
  &\reduce{β} λp_{1}\, p_{2}.p_{1} (p_{2}\, \bs{T}\, \bs{F}) \bs{F} \\
  &=_{\bs{λ}} λp_{1}\, p_{2}.p_{1}\, p_{2}\, \bs{F}
\end{align*}

De igual manera, considerando el siguiente pseudocódigo de la operación de disyunción:

\begin{verbatim}
Disyuncion (predicado1, predicado2) :=
    Si predicado1, entonces:
        Regresa Verdadero
    De lo contrario:
        Si predicado2, entonces:
            Regresa Verdadero
        De lo contrario:
            Regresa Falso
        Fin
    Fin
Fin
\end{verbatim}

Se traduce al cálculo lambda como

\[ \bs{\lor} \synteq λp_{1}\, p_{2}.\bs{\prec}\, p_{1}\, \bs{T} (\bs{\prec}\, p_{2}\, \bs{T}\, \bs{F}) \]

Y al \( β \)-reducir el cuerpo de la abstracción se obtiene el término de la definición \ref{defn:disyuncion}

\begin{align*}
  λp_{1}\, p_{2}.\bs{\prec}\, p_{1}\, \bs{T} (\bs{\prec}\, p_{2}\, \bs{T}\, \bs{F})
  &\synteq λp_{1}\, p_{2}.(λp\, m\, n.p\, m\, n) p_{1}\, \bs{T} ((λp\, m\, n.p\, m\, n) p_{2}\, \bs{T}\, \bs{F}) \\
  &\reduce{β} λp_{1}\, p_{2}.p_{1}\, \bs{T} ((λp\, m\, n.p\, m\, n) p_{2}\, \bs{T}\, \bs{F}) \\
  &\reduce{β} λp_{1}\, p_{2}.p_{1}\, \bs{T} (p_{2}\, \bs{T}\, \bs{F}) \\
  &=_{\bs{λ}} λp_{1}\, p_{2}.p_{1}\, \bs{T}\, p_{2}
\end{align*}

Utilizando esta técnica, se puede obtener el término lambda para una operación a partir del pseudocódigo basado en valores de verdad y la sentencia \verb!if-then-else!. Teniendo estos resultados resulta natural, generalizar el pseudocódigo para construír un término lambda que a partir de la tabla de verdad de una operación booleana binaria, resulte en la abstracción que codifica la operación.


\begin{defn}[Traducción de operaciones booleanas binarias]
  \label{defn:op-bool-bin-lambda}
  Sea \( \bs{\odot} \) una operación booleana binaria con la siguiente tabla de verdad

  \begin{center}
    \begin{tabular}{|c|c||c|}
      \hline
      \( P \) & \( Q \) & \( P \bs{\odot} Q \) \\ [0.5ex]
      \hline\hline
      \( \bs{F} \) & \( \bs{F} \) & \( x_{1} \) \\
      \hline
      \( \bs{F} \) & \( \bs{T} \) & \( x_{2} \) \\
      \hline
      \( \bs{T} \) & \( \bs{F} \) & \( x_{3} \) \\
      \hline
      \( \bs{T} \) & \( \bs{T} \) & \( x_{4} \) \\
      \hline
    \end{tabular}
  \end{center}

  El procedimiento generalizado es

\begin{verbatim}
OperacionBooleana(x1, x2, x3, x4) :=
    Binaria(predicado1, predicado2) :=
        Si predicado1, entonces:
            Si predicado2, entonces:
                Regresa x4
            De lo contrario:
                Regresa x3
            Fin
        De lo contrario:
            Si predicado2, entonces:
                Regresa x2
            De lo contrario:
                Regresa x1
            Fin
        Fin
    Fin
    Regresa Binaria
Fin
\end{verbatim}

  Y la traducción al cálculo lambda es

  \[ λx_{1}\, x_{2}\, x_{3}\, x_{4}.(λp_{1}\, p_{2}.(\bs{\prec}\, p_{1} (\bs{\prec}\, p_{2}\, x_{4}\, x_{3}) (\bs{\prec}\, p_{2}\, x_{2}\, x_{1}))) \]
  
\end{defn}

\begin{exmp}[Operaciones \emph{NAND} y \emph{NOR}]
  \label{ejmp:nand-nor}
  Estas operaciones booleanas binarias conforman los conjuntos unitarios \( \{ \mathrm{NAND} \} \) y \( \{ \mathrm{NOR} \} \) los cuales son conjuntos funcionalmene completos, es decir, únicamente con la operación NAND se puede emular cualquier operación booleana y únicamente con la operación NOR se puede emular cualquier operación booleana.

  
  La operación \emph{NAND} se denota \( P \uparrow Q \) y tiene la siguiente tabla de verdad

  \begin{center}
    \begin{tabular}{|c|c||c|}
      \hline
      \( P \) & \( Q \) & \( P \uparrow Q \) \\ [0.5ex]
      \hline\hline
      \( \bs{F} \) & \( \bs{F} \) & \( \bs{T} \) \\
      \hline
      \( \bs{F} \) & \( \bs{T} \) & \( \bs{T} \) \\
      \hline
      \( \bs{T} \) & \( \bs{F} \) & \( \bs{T} \) \\
      \hline
      \( \bs{T} \) & \( \bs{T} \) & \( \bs{F} \) \\
      \hline
    \end{tabular}
  \end{center}

  Con el proceso de traducción mostrado en la definición \ref{defn:op-bool-bin-lambda}, el término lambda \( \bs{\uparrow} \) que codifica la operación NAND sería
  
  \begin{align*}
    \bs{\uparrow}
    &\synteq λp_{1}\, p_{2}. \bs{\prec}\, p_{1} (\bs{\prec}\, p_{2}\, \bs{F}\, \bs{T}) (\bs{\prec}\, p_{2}\, \bs{T}\, \bs{T}) \\
    &\reduce{β} λp_{1}\, p_{2}.p_{1} (p_{2}\, \bs{F}\, \bs{T}) \bs{T}
  \end{align*}

  La operación \emph{NOR} se denota \( P \downarrow Q \) y tiene la siguiente tabla de verdad

  \begin{center}
    \begin{tabular}{|c|c||c|}
      \hline
      \( P \) & \( Q \) & \( P \downarrow Q \) \\ [0.5ex]
      \hline\hline
      \( \bs{F} \) & \( \bs{F} \) & \( \bs{T} \) \\
      \hline
      \( \bs{F} \) & \( \bs{T} \) & \( \bs{F} \) \\
      \hline
      \( \bs{T} \) & \( \bs{F} \) & \( \bs{F} \) \\
      \hline
      \( \bs{T} \) & \( \bs{T} \) & \( \bs{F} \) \\
      \hline
    \end{tabular}
  \end{center}

  Usando el mismo proceso de traducción que con la operación NAND, se obtiene

  \begin{align*}
    \bs{\downarrow}
    &\synteq λp_{1}\, p_{2}. \bs{\prec}\, p_{1} (\bs{\prec}\, p_{2}\, \bs{F}\, \bs{F}) (\bs{\prec}\, p_{2}\, \bs{F}\, \bs{T}) \\
    &\reduce{β} λp_{1}\, p_{2}.p_{1}\, \bs{F} (p_{2}\, \bs{F}\, \bs{T})
  \end{align*}

\end{exmp}

Cuando se generaliza el método de traducción de \ref{defn:op-bool-bin-lambda} a operaciones booleanas \( n \)-árias, se obtiene un término bosquejado de la siguiente manera

\begin{center}
  \begin{tikzpicture}[level/.style={sibling distance=100mm/#1, level distance=10mm}]
    \node (args1) {\( λx_{1}\, ...\, x_{2^{n}} \)}
    child {
      node (args2) {\( λp_{1}\, ...\, p_{n} \)}
      child {
        node (p1) {\( \bs{\prec}\, p_{1} \)}
        child {
          node (p2a) {\( \bs{\prec}\, p_{2} \)}
          child {
            node (dots1) {\( ... \)}
            child {
              node (pna) {\( \bs{\prec}\, p_{n} \)}
              child {
                node (x2n) {\( x_{2^{n}} \)}
              }
              child {
                node (x2n1) {\( x_{2^{n}-1} \)}
              }
            }
            child {
              node (dots3) {\( ... \)}
            }
          }
          child {
            node (dots2) {\( ... \)}
          }
        }
        child {
          node (p2b) {\( \bs{\prec}\, p_{2} \)}
          child {
            node (dots4) {\( ... \)}
          }
          child {
            node (dots5) {\( ... \)}
            child {
              node (dots6) {\( ... \)}
            }
            child {
              node (pnb) {\( \bs{\prec}\, p_{n} \)}
              child {
                node (x2) {\( x_{2} \)}
              }
              child {
                node (x1) {\( x_{1} \)}
              }
            }
          }
        }
      }
    };
  \end{tikzpicture}
\end{center}

\subsection{Extensiones al álgebra booleana}
\label{sec:boolean-extensiones}

Conociendo el proceso de codificación del álgebra booleana en el cálculo lambda, resulta simple adaptar la codificación.

Consideremos el caso en donde, además de tener los valores de falso y verdadero, se desea incorporar un valor ``desconocido'' utilizado para representar un valor que no es ni falso, ni verdadero. La interpretación de estos valores es similar a \ref{defn:valores-verdad}, pero en lugar de decidir sobre dos términos, se decide sobre tres.

\begin{defn}[Valores de álgebra trivalente]
  La codificación en términos lambda de los valores de ésta álgebra trivalente son

  \begin{align*}
    \bs{T} &\synteq λx\, y\, z.x \\
    \bs{F} &\synteq λx\, y\, z.y \\
    \bs{U} &\synteq λx\, y\, z.z 
  \end{align*}
\end{defn}

Al igual que en la codificación bivalente, se puede codificar un término \( \bs{\prec_{3}} \), similar a \( \bs{\prec} \) de la definición \ref{defn:condicional} pero con tres ramificaciones

\begin{defn}[Condicional trivalente]
  \[ \bs{\prec_{3}} \synteq λp\, m\, n\, o.p\, m\, n\, o \]

  De tal manera que, si \( P \in \{ \bs{T},\ \bs{F},\ \bs{U} \} \)

  \[ (\bs{\prec_{3}}\, P\, M\, N\, O) \reduce{β} \begin{cases} M & P \synteq \bs{T}; \\ N & P \synteq \bs{F}; \\ O & P \synteq \bs{U}. \end{cases} \]
\end{defn}

Sea \( \odot \) una operación trivalente binaria con la siguiente tabla de valores

\begin{center}
  \begin{tabular}{|c|c||c|}
    \hline
    \( P \) & \( Q \) & \( P \odot Q \) \\ [0.5ex] \hline\hline
    \( \bs{T} \) & \( \bs{T} \) & \( x_{1} \) \\ \hline
    \( \bs{T} \) & \( \bs{F} \) & \( x_{2} \) \\ \hline
    \( \bs{T} \) & \( \bs{U} \) & \( x_{3} \) \\ \hline
    \( \bs{F} \) & \( \bs{T} \) & \( x_{4} \) \\ \hline
    \( \bs{F} \) & \( \bs{F} \) & \( x_{5} \) \\ \hline
    \( \bs{F} \) & \( \bs{U} \) & \( x_{6} \) \\ \hline
    \( \bs{U} \) & \( \bs{T} \) & \( x_{7} \) \\ \hline
    \( \bs{U} \) & \( \bs{F} \) & \( x_{8} \) \\ \hline
    \( \bs{U} \) & \( \bs{U} \) & \( x_{9} \) \\ \hline
  \end{tabular}
\end{center}

El procedimiento en pseudocódigo que la describe es:

\begin{verbatim}
OperacionTrivalente(x1, x2, x3, x4, x5, x6, x7, x8, x9) :=
    Binaria(predicado1, predicado2) :=
        Si predicado1 es verdadero, entonces:
            Si predicado2 es verdadero, entonces:
                Regresa x1
            Si predicado2 es falso, entonces:
                Regresa x2
            Si predicado2 es desconocido, entonces:
                Regresa x3
            Fin
        Si predicado1 es falso, entonces:
            Si predicado2 es verdadero, entonces:
                Regresa x4
            Si predicado2 es falso, entonces:
                Regresa x5
            Si predicado2 es desconocido, entonces:
                Regresa x6
            Fin
        Si predicado1 es desconocido, entonces:
            Si predicado2 es verdadero, entonces:
                Regresa x7
            Si predicado2 es falso, entonces:
                Regresa x8
            Si predicado2 es desconocido, entonces:
                Regresa x9
            Fin
        Fin
    Fin
    Regresa Binaria
Fin
\end{verbatim}

Traducido al cálculo lambda como

\[ λx_{1}\, x_{2}\, x_{3}\, x_{4}\, x_{5}\, x_{6}\, x_{7}\, x_{8}\, x_{9}.(λp_{1}\, p_{2}.(\bs{\prec_{3}}\, p_{1}\, R_{1}\, R_{2}\, R_{3})) \]

Donde

\begin{align*}
  R_{1} &\synteq (\bs{\prec_{3}}\, p_{2}\, x_{1}\, x_{2}\, x_{3}) \\
  R_{2} &\synteq (\bs{\prec_{3}}\,  p_{2}\, x_{4}\, x_{5}\, x_{6}) \\
  R_{3} &\synteq (\bs{\prec_{3}}\, p_{2}\, x_{7}\, x_{8}\, x_{9})
\end{align*}

\subsection{Lógica en el cálculo lambda}
\label{sec:logica-lambda}

En esta sección se abordó la codificación de valores de verdad y operaciones como la conjunción, disyunción y negación. Usualmente en las ciencias de la computación y las matemáticas, estos objetos se abordan desde la perspectiva algebraica y desde la perspectiva lógica.

En el cálculo lambda se distinguen las dos perspectivas por el lenguaje en el que se manejan los objetos, en el caso del álgebra booleana, los valores de verdad y las operaciones booleanas se codifican como términos lambda y pueden ser combinados con otros términos lambda que no forman parte del álgebra booleana. Por otra parte, la lógica proposicional estudia proposiciones, las cuales son enunciados matemáticos a los cuales se les puede atribuír el valor de falso o verdadero y que, a partir de conectivos lógicos como \( \lnot \), \( \land \) y \( \lor \), se infieren verdades sobre los enunciados.

El cálculo lambda se puede extender de tal manera que nos permita inferir verdades sobre sus términos, en esta subsección se abordan los cálculos lambda aumentados con nociones \emph{ilativas}, denotados \( \mathit{i}\bs{λ} \). El tratamiento de esta extensión está basado en el apéndice B ``Lógica combinatoria ilativa'' de \cite[pp.~573--576]{Barendregt:Bible}.

\begin{defn}[Términos \( \mathit{i}\bs{λ} \)]
  \label{defn:ilativo-terminos}
  Los términos de la teoría \( \mathit{i}\bs{λ} \), denotados \( \mathit{i}Λ \) se definen sobre el alfabeto de \( Λ \) extendido con un conjunto \( C \) de \emph{constantes lógicas}.

  \begin{align*}
    M \in Λ &\implies M \in \mathit{i}Λ \\
    c \in C &\implies c \in \mathit{i}Λ \\
    M, N \in \mathit{i}Λ &\implies (M\, N) \in \mathit{i}Λ
  \end{align*}
\end{defn}

\begin{defn}[Fórmulas en \( \mathit{i}\bs{λ} \)]
  \label{defn:ilativo-formulas}
  Las fórmulas de la teoría \( \mathit{i}\bs{λ} \) se definen de la siguiente manera
  
  \begin{itemize}
  \item Si \( M, N \in \mathit{i}Λ \), entonces \( M = N \) es una fórmula.
  \item Si \( M \in \mathit{i}Λ \), entonces \( M \) es una fórmula.
  \end{itemize}

  En el estudio de teorías ilativas, la interpretación de \( \mathit{i}\bs{λ} \vdash M \) y \( \mathit{i}\bs{λ} \vdash M = N \) es ``La fórmula \( M \) es verdadera'' y ``La fórmula \( M = N \) es verdadera'' respectivamente.
\end{defn}

\subsubsection{Teoría \( \mathit{i}\bs{λ}_{KR} \)}

Constantes \( C = \{ \bs{N} \} \) con la interpretación \( \lnot\, M \) es \( (\bs{N}\, M) \)

Paradoja de Kleene-Rosser

\[ X \synteq λx.\bs{N} (x\, x) \]

\[ X\, X = (λx.\bs{N} (x\, x)) X = \bs{N} (X\, X) \]

Se puede derivar que una fórmula \( X\, X = \lnot (X\, X) \), es decir, en \( \mathit{i}\bs{λ}_{KR} \) se pueden derivar contradicciones.

\subsubsection{Teoría \( \mathit{i}\bs{λ}_{0} \)}

Constantes \( C = \{ \bs{F},\ \bs{Q},\ \bs{E},\ \bs{Ξ},\ \bs{Π},\ \bs{P} \} \) con las siguientes interpretaciones

\begin{itemize}
\item \( M \in N \) es \( N\, M \)
\item \( M \implies N \) es \( \bs{P}\, M\, N \)
\item \( M \subseteq N \) es \( \bs{Ξ}\, M\, N \)
\item \( N^{M} \) es \( \bs{F}\, M\, N \)
\item \( \forall x\ M \) es \( \bs{Ξ} (λx.M) \)
\end{itemize}

\subsubsection*{Pendientes}

\begin{itemize}
\item Definibilidad de \( \bs{E} \), \( \bs{F} \), \( \bs{P} \) y \( \bs{Π} \) a partir de \( \bs{Q} \), \( \bs{Ξ} \).
\item Inconsistencia de \( \mathit{i}\bs{λ}_{0} \) por la paradoja de Curry ``Una teoría lambda ilativa es inconsistente si cada \( M \) puede ser derivada'' (la paradoja de Kleene-Rosser es una consecuencia de la paradoja de Curry).
\item Mencionar la técnica utilizada por Church para ``escapar'' de la paradoja de Curry y presentar el ``tradeoff'' de tener un sistema suficientemente poderoso: Equivalencia \( \bs{λ} \) y máquina de Turing \emph{vs.} separar la noción ilativa de la abstracción.
\end{itemize}

\section{Aritmética}
\label{sec:aritmetica}

Así como se pueden representar los valores de verdad de falso y verdadero en el cálculo lambda, también podemos encontrar representaciones para los números naturales. En esta sección se aborda una representación llamada numerales de Church, también se presentan términos lambda para operar números naturales con esta representación.

Para cada \( n \in \mathbb{N} \) el numeral de Church de \( n \) es un término lambda denotado como \( \bar{n} \) definido como:

\[ \bar{n} \synteq λx\, y.x^{n}\, y \]

En la siguiente tabla se puede apreciar mejor la estructura de los numerales de Church

\begin{table}
  \centering
  \begin{tabular}{|c|c|}
    \hline
    \( n \) & \( \bar{n} \) \\ [0.5ex]
    \hline\hline
    0 & \( λx\, y.y \) \\
    \hline
    1 & \( λx\, y.x\, y \) \\
    \hline
    2 & \( λx\, y.x(x\, y) \) \\
    \hline
    3 & \( λx\, y.x(x(x\, y)) \) \\
    \hline
    ... & ... \\
    \hline
  \end{tabular}
  \caption{Numerales de Church}
  \label{tab:numerales}
\end{table}

Como se observa en la tabla, un numeral de Church es una abstracción descurrificada de dos argumentos la cual al ser evaluada es reducida a la \( n \)-ésima composición del primer argumento evaluada en el segundo argumento.

Una manera de entender esta representación es pensar en los números naturales como un conteo de uno en uno; el 0 es no contar; el 1 es contar uno mas que el 0; el 2 es uno mas que el 1, así que el 2 es uno mas que el uno mas que el 0; el 3 es uno mas que el 2, así que el 3 es uno mas que el uno mas que el uno mas que el 0 y así sucesivamente. La idea de ``el uno mas'' es la del sucesor, si consideramos a \( x \) como una función sucesor y a \( y \) como el 0, podemos expresar \( x(x(x(y))) \) como \( \mathrm{sucesor}(\mathrm{sucesor}(\mathrm{sucesor}(0))) \), así que

\begin{align*}
\mathrm{sucesor}(\mathrm{sucesor}(\mathrm{sucesor}(0))) & = \mathrm{sucesor}(\mathrm{sucesor}(1)) \\
                             & = \mathrm{sucesor}(2) \\
                             & = 3
\end{align*}

Es interesante pensar en diferentes maneras de expresar las operaciones mas elementales de la aritmétima como términos lambda que operen sobre esta representación. A continuación se presenta una exploración de los términos lambda correspondientes a algunas operaciones elementales de la aritmética: suma, multiplicación, exponenciación y resta. La suma es una repetición de la operación sucesor, la multiplicación una repetición de suma, la exponenciación una repetición de multiplicaciones y la resta una repetición de la operación predecesor. Esto nos lleva a identificar las operaciones de sucesor y predecesor como los algoritmos base para el resto de las operaciones primitivas.

El término \emph{sucesor} debe ser uno tal que al ser aplicado a un numeral \( \bar{n} \) se pueda \( β \)-reducir al numeral \( \bar{n+1} \). Considerando la definición de \( \bar{n} \synteq λx\, y.x^{n}\, y \), lo que buscamos es una manera de agregarle una \( x \) a la composición en el cuerpo de \( \bar{n} \) para obtener \( λx\, y.x^{n+1}\, y \). Se construye este término considerando primero que será aplicado a un numeral

\[ \mathrm{sucesor} \synteq λ\bar{n}.? \] 

Además el resultado de \( β \)-reducir esta aplicación deberá ser una función de dos argumentos (como lo son todos los numerales de Church):

\[ \mathrm{sucesor} \synteq λ\bar{n}.λx\, y.? \]

Tomando en cuenta que \( \bar{n}\, x\, y \synteq x^{n}\, y \) y que \( x\, x^{n} y \synteq x^{n+1}\, y \):

\[ \mathrm{sucesor} \synteq λn\, x\, y.x(n\, x\, y) \]

A continuación se \( β \)-reduce la aplicación de \( \mathrm{sucesor} \) al numeral \( \bar{4} \synteq λx\, y.x(x(x(x\, y))) \):

\begin{align*}
                & \mathrm{sucesor}\, \bar{4} \\
\synteq         & (λn\, x\, y.x(n\, x\, y)) (λx\, y.x(x(x(x\, y)))) \\
\convertible{α} & (λn\, x\, y.x(n\, x\, y)) (λf\, z.f(f(f(f\, z)))) \\
\contract{β}    & (λx\, y.x(((λf\, z.f (f (f (f\, z)))) x) y)) \\
\contract{β}    & (λx\, y.x((λz.x(x(x(x\, z)))) y)) \\
\contract{β}    & (λx\, y.x(x(x(x(x\, y))))) \\
\synteq         & \bar{5}
\end{align*}

La otra operación elemental en la aritmética es el \emph{predecesor}, el término que represente esta operación debe ser uno que cumpla con la siguiente definición:

\begin{align*}
\mathrm{predecesor}\, \bar{0} & \reduce{β} \bar{0} \\
\mathrm{predecesor}\, \bar{n} & \reduce{β} \bar{n-1}
\end{align*}

El término lambda del predecesor con la representación de numerales de Church es mucho mas compleja que la del sucesor. Se pudiera pensar que la misma idea utilizada en la derivación del sucesor aplicaría para la derivación del predecesor: si tenemos \( n \) aplicaciones de \( x \) a \( y \), al aplicar el término que buscamos a un numeral de Church se debe \( β \)-reducir a otro numeral con una aplicación de \( x \) menos, se utiliza \( y \) para añadir una \( x \) mas en el cuerpo del numeral. Sin embargo, la estructura de los numerales no nos permite quitar una \( x \) usando \( y \) facilmente ya que el numeral puede ser aplicado a dos términos, el que representa las \( x \) y el que representa a la \( y \); la variable que determina el valor del numeral es  \( x \) y la sustitución de \( x \) por otro término en esta representación se hace con \emph{cada} aparición de \( x \) en el cuerpo del numeral, por otro lado, al sustituír la \( y \) por otro término solo podemos hacer mas complejo el término o sustituírla por otra variable.

Para derivar el término del predecesor vamos a presentar un término con una estructura similar a los numerales de Church:

\[ λx\, y\, z.z\, x^{n}\, y \]

La diferencia entre este término y un numeral de Church es que podemos modificar su estructura por enfrente, por atrás y en las composiciones intermedias. Si este término representara \( \bar{n+1} \) pudieramos obtener el cuerpo de \( \bar{n} \) de la siguiente manera:

\begin{align*}
             & (λx\, y\, z.z\, x^{n}\, y) x\, y (λa.a) \\
\contract{β} & (((λy\, z.z\, x^{n}\, y) y) (λa.a)) \\
\reduce{β}   & ((λz.z\, x^{n}\, y) (λa.a)) \\
\contract{β} & (λa.a) x^{n}\, y \\
\contract{β} & x^{n}\, y
\end{align*}

Es decir, mantenemos las \( x \) y la \( y \) y le aplicamos a \( x^{n} y \) el término lambda que representa a la función identidad. Esta manera conveniente de representar a los numerales resulta ser incompleta, ya que no se podrá obtener \( x^{n-1} y \) a partir del resultado (ya que la \( z \) no aparece en el término resultante). Sin embargo, si logramos tener un término que nos genera este término modificado pudiéramos realizar esta transformación dentro de la función predecesor sería mas fácil encontrar \( \bar{n-1} \).

La estructura del sucesor sería:

\[ λn\, x\, y.?\, (λa.a) \]

Donde \( ? \) debe ser tal que al \( β \)-reducirse resulte en un término con la forma \( λz.z\, x^{n-1}\, y \). Es conveniente desmenuzar el problema de encontrar este término desconocido: primero sabemos que el numeral de Church \( \bar{n} \) puede ser aplicado a dos términos y el primer término al que sea aplicado se sustituirá en todas las apariciones de \( x \), como queremos que la \( β \)-reducción nos genere una función cuyo argumento sea la primer variable en la composición del numeral, tenemos que encontrar una manera de propagar un término de la forma \( λw.w\, x^{m}\, y \) de tal manera que al aplicarle otro término nos resulte \( λw.w\, x^{m+1}\, y \). De esta manera al aplicar este otro término una y otra vez, resulte \( λw.w\, x^{n-1}\, y \) con el cual podemos obtener el cuerpo del predecesor sustituyendo \( w \) por \( λa.a \).

Este otro término que buscamos será el que sustituirá a la \( x \) en \( \bar{n} \) para que:

\[ ?(λw.w\, x^{m}\, y) \reduce{β} λw.w\, x^{m+1}\, y \]

Lo que sucede en cada aplicación de este estilo es que se compone una \( x \) en cada aplicación y se deja explícita una \( w \) que podrá ser sustituída como valor. El término que nos permite hacer esto tiene la siguiente forma:

\[ λg\, w.w (g\, x) \]

Al ser aplicado a un término \( λw.w\, x^{m}\, y \) la variable \( g \) será sustituída por este término y el resultado será \( λw.w((λr.r\, x^{m}\, y) x) \) (nótese el cambio de nombre de la variable ligada \( w \) en el argumento), lo cual se reduce a \( λw.w\, x^{m+1}\, y \) el cual mantiene su estructura original.

El percance con esta aproximación a la solución es que el primer valor al que se le aplica el término \( λg\, w.w (g\, x) \) debe ser \( λw.w\, y \).

Para visualizar una manera de resolver el problema, es conveniente expresar cómo se verían las aplicaciones de \( λg\, w.w (g\, x) \) para un numeral de Church en particular. Si consideramos la aplicación de \( \bar{4}\, (λg\, w.w(g\, x)) \):

\begin{align*}
             & \bar{4} (λg\, w.w(g\, x)) \\
\synteq      & ( (λx\, y.x^{4}\, y) (λg.(λw.(w (g\, x)))) ) \\
\synteq      & ( (λx\, y.x(x(x(x\, y)))) (λg.(λw.(w (g\, x)))) ) \\
\contract{β} & (λy.(λg.(λw.(w (g\, x))))((λg.(λw.(w (g\, x))))((λg.(λw.(w (g\, x))))((λg.(λw.(w (g\, x)))) y))))
\end{align*}

Esto nos lleva al segundo paso para encontrar la función predecesor, en el desarrollo anterior notamos que la primer aplicación de \( λg\, w.w (g\, x) \) es en la variable \( y \) la cual está ligada por la \( λ \) del término. Sabemos que para obtener \( λw.w\, x^{3}\, y \) debemos \( β \)-reducir el término:

\[ ((λg.(λw.(w (g\, x)))) ((λg.(λw.(w (g\, x)))) ((λg.(λw.(w (g\, x)))) (λw.w\, y)))) \]

Con esto podemos encontrar el valor que tiene que tomar \( y \) en el numeral ya que:

\[ ((λg.(λw.(w (g\, x)))) y) \reduce{β} λw.w\, y \]

El término que buscamos es el que debe sustituír a la variable \( y \) en la reducción:

\begin{align*}
           & ((λg.(λw.(w (g\, x)))) ?) \\
\reduce{β} & (λw.(w (?\, x)))
\end{align*}

El término \( ? \) debe ser una función que al ser aplicada a \( x \) se reduzca a \( y \). El término \( λu.y \) cumple con esta propiedad y será el que utilizaremos.

Considerando los términos determinados en el procedimiento anterior, podemos decir cómo será la función predecesor. Primero se aplica \( (λg.(λw.(w (g\, x)))) \) a \( \bar{n} \), este término resultante se aplica a \( λu.y \), \( β \)-reducir esta aplicación nos resulta \( λw.w\, x^{n-1}\, y \) la cual puede ser aplicada a la función identidad \( λa.a \) para obtener \( x^{n-1}\, y \). Lo cual nos lleva al término completo de predecesor:

\[ (λn.(λx\, y.(((n (λg.(λw.(w (g\, x))))) (λu.y)) (λa.a)))) \]

Teniendo los términos lambda de sucesor y predecesor se puede abordar la derivación de operaciones mas complejas como la de adición, multiplicación, exponenciación y sustracción de numerales de Church siguiendo el mismo enfoque. En este trabajo no se abordan otras operaciones como la división debido al aumento de complejidad por no ser una operación interna, es decir, la división de dos naturales puede ser un racional y no se definió una representación de términos lambda para el conjunto de los racionales.

Un término lambda para la adición de dos numerales \( \bar{m} \) y \( \bar{n} \) es

\[ λm\, n.(λx\, y.n\, \mathrm{sucesor}\, m) \]

y se obtuvo a partir de la observación de que realizar la suma \( m+n \) es equivalente a computar el \( n \)-ésimo sucesor de \( m \).

Utilizando la estructura de \( \bar{n} \) podemos aplicar \( \bar{n}\, \mathrm{sucesor}\, \bar{m} \) para obtener la \( n \)-ésima composición de la función sucesor aplicada al numeral \( \bar{m} \):

\begin{align*}
             & \bar{n}\, \mathrm{sucesor}\, \bar{m} \\
\synteq      & (( (λx\, y.x^{n}\, y) \mathrm{sucesor}) (λx\, y.x^{m}\, y)) \\
\contract{β} & ( (λy.\mathrm{sucesor}^{n}\, y) (λx\, y.x^{m}\, y)) \\
\contract{β} & \mathrm{sucesor}^{n}\, λx\, y.x^{m}\, y \\
\reduce{β}   & λx\, y.x^{m+n}\, y \\
\synteq      & \bar{m+n}
\end{align*}

Un término lambda para la multiplicación de dos numerales de Church es

\[ λm\, n\, x\, y.n (m\, x) y \]

el cual aborda la idea de componer \( m\, n \) consigo mismo \( n \) veces (lo cual equivaldría a sumar \( n \) veces \( m \).

En el caso de la adición y la multiplicación, el orden en el que aplicamos el término a los numerales no es de importancia ya que son operaciones conmutativas, \( m+n = n+m \) y \( m \times n = n \times m \). Sin embargo en la sustracción y la exponenciación no se tiene esta propiedad, por lo que es importante el orden en el que se aplican los numerales a los términos, para ello consideraremos el orden como \( m-n \) y \( m^{n} \).

Basándonos en el término de la adición podemos obtener un término de la sustracción el cual es

\[ λm\, n\, x\, y.n\, \mathrm{predecesor}\, m \].

Ya que en la adición se dejó explícito el acto de aumentar \( m \) veces en 1 a \( n \), cambiamos el término de \( \mathrm{sucesor} \) por el de \( \mathrm{predecesor} \) y ahora se decrementa \( m \) veces en 1 a \( n \).

Un término lambda para la exponenciación es

\[ λm\, n.n\, m \]

es curioso tener una representación tan sencilla para una operación tan compleja como esta. A diferencia de los anteriores términos, al aplicarle a ésta exponenciación dos numerales, el numeral resultante tendrá las variables compuestas las variables que no se componen en las entradas, es decir, si \( \bar{m} \synteq λf\, g.f^{m}\, g \) y \( \bar{n} \synteq λx\, y.x^{n}\, y \), el resultado será \( \bar{m^{n}} \synteq λg\, y.g^{m^{n}}\, y \).

Para corroborar que estas representaciones calculan de manera correcta la operación correspondiente para los numerales de Church se pueden realizar varias pruebas con diferentes numerales de entrada. En este trabajo no se desarrollarán ejemplos para estos términos.

Los mecanismos que hemos utilizado para derivar las operaciones se basan en construír términos que vayan transformando entradas con una estructura determinada de tal manera que nos acerquemos poco a poco al cálculo de la operación deseada; esta labor llega a ser bastante tediosa y carece de interés algorítmico. A continuación se presenta una manera mas interesante y elegante de abordar el problema de representar operaciones aritméticas.

Se introduce el término lambda que me permite generar hiperoperaciones aritméticas:

\[ λf\, u\, m\, n.n (λw.f\, m\, w) u \]

Abstracción de la noción de repetición sobre la estructura de un numeral, considerar propiedades de conmutatividad y asociatividad en operaciones. Abordar el problema del cómputo de operaciones inversas. Determinar un término que nos genere elementos de la secuencia de hiperoperaciones.

\section{Procesos recursivos}
\label{sec:procesos-recursivos}

Combinador Y, ordenes de evaluación, funciones recursivas.

\[ \bs{Y} \synteq λf.(λx.f(x\, x))(λx.f(x\, x)) \]

{\center Derivación de \( \bs{Y} \)}
\begin{align}
  F &\synteq λn.\bs{\prec}\, (\bs{0^{?}}\, n) \bar{1} (\bs{\times}\, n (F (\bs{-}\, n\, \bar{1}))) \\
  F' &\synteq λf\, n.\bs{\prec}\, (\bs{0^{?}}\, n) \bar{1} (\bs{\times}\, n (f (\bs{-}\, n\, \bar{1})))\\
  F'' &\synteq λf'\, n.\bs{\prec}\, (\bs{0^{?}}\, n) \bar{1} (\bs{\times}\, n (f'\, f' (\bs{-}\, n\, \bar{1}))) \\
  C &\synteq λf''.f''\, f'' \\
  C\, F'' &\reduce{β} \text{ función factorial a partir de \( F'' \)} \\
  F''' &\synteq (λf'''.f' (f'''\, f''')) \\
  F'''\, F''' &\reduce{β} f'(F'''\, F''') \\
  C^{*} &\synteq λf'.(λf'''.f' (f'''\, f'''))(λf'''.f' (f'''\, f''')) \\
  C^{*}\, F' &\reduce{β} \text{ función factorial a partir de \( F' \)}
\end{align}

\[ Y \synteq C^{*} \]

Presentar una breve introducción sobre los combinadores y hablar del combinador \( \bs{Y} \) y cómo nos permite expresar funciones recursivas en el cálculo lambda.

Como ejemplos prácticos de esta subsección sería adecuado desarrollar el término para el cálculo de factoriales o algúna otra función de una sola variable que transforme un numeral de Church en otro. También pudiera expandir la recursividad a términos multivariables currificados como la función Ackermann (abstracción a la generación de hiperoperaciones, ver \emph{The Book of Numbers} de Conway).

\section{Pares ordenados}
\label{sec:pares-ordenados}

Construcción axiomática de pares ordenados, listas, \( n \)-tuplas, árboles y otras estructuras complejas.

Presentar la representación de pares ordenados para la construcción de estructuras mas complejas.

\[ \mathrm{Car}(\mathrm{Cons}(x,y)) = x \]

\[ \mathrm{Cdr}(\mathrm{Cons}(x,y)) = y \]

Esta sección es apropiada para comenzar a relacionar la teoría de autómatas, lenguajes regulares y libres de contexto con sistemas medianamente complejos que se pueden incrustar en el cálculo lambda sin modificar el sistema. Un problema pudiece ser el no determinismo, pero pudiera solventar esto con el desarrollo de operaciones funcionales sobre listas (map, filter, fold, etc).

\subsection*{Cambios en la metodología}

Expandir el concepto de valores de verdad al de pares ordenados

\subsubsection*{Constructor}

\begin{verbatim}
CreaPar (primero, segundo) =
    Elige (x) =
        Si x, entonces:
            primero
        De lo contrario:
            segundo
    Elige
\end{verbatim}

\begin{align*}
  \otimes & \synteq λa\, d.λx.B\, x\, a\, d \\
          & \contract{β} λa\, d.λx.x\, a\, d \\
          & \synteq λa\, d\, x.x\, a\, d
\end{align*}

\subsubsection*{Selectores}

\begin{verbatim}
Primero (Elige) =
    Elige(T)
\end{verbatim}

\begin{align*}
  \otimes_{1} & \synteq λx.x\, T
\end{align*}

\begin{verbatim}
Segundo (Elige) =
    Elige(F)
\end{verbatim}

\begin{align*}
  \otimes_{2} & \synteq λx.x\, F
\end{align*}


%%% Local Variables:
%%% mode: latex
%%% TeX-master: "main"
%%% End:


\appendix

\nocite{*}
\bibliographystyle{acm}
\bibliography{biblio}

\end{document}

%%% Local Variables:
%%% mode: latex
%%% TeX-master: t
%%% End:
