% Eduardo Acuña Yeomans - 2016
%

% Configuración del documento
\documentclass[letterpaper, twoside, openright, 11pt]{book}

\usepackage{emptypage}
\usepackage[lmargin = 1.4in, rmargin = 1.0in, tmargin = 1.0in, bmargin = 1.0in, includefoot, headheight=14pt]{geometry}
\usepackage[utf8]{inputenc}
\usepackage[spanish, mexico]{babel}
\decimalpoint{}
\usepackage[T1]{fontenc}
\usepackage{setspace}
\onehalfspacing
\usepackage{enumerate}
\usepackage{enumitem}
\setlist{nosep}
\usepackage{amsmath, amssymb}
\usepackage{amsthm}
\usepackage{proof}
\usepackage{centernot}
\theoremstyle{plain}%
\newtheorem{thm}{Teorema}[section]
\newcommand{\thmautorefname}{Teorema}
\newtheorem{lem}{Lema}[section]
\newcommand{\lemautorefname}{Lema}
\newtheorem{prop}{Proposición}[section]
\newcommand{\propautorefname}{Proposición}
\newtheorem*{cor}{Corolario}
\newcommand{\corautorefname}{Corolario}
\theoremstyle{definition}
\newtheorem{defn}{Definición}[section]
\newcommand{\defnautorefname}{Definición}
\newtheorem{exmp}{Ejemplo}[section]
\newcommand{\exmpautorefname}{Ejemplo}
\theoremstyle{remark}
\newtheorem*{rem}{Observación}
\newcommand{\remautorefname}{Observación}
\newtheorem*{note}{Nota}
\newcommand{\noteautorefname}{Nota}
\newtheorem{case}{Caso}
\newcommand{\caseautorefname}{Caso}
\usepackage{pdfpages}
\usepackage{tikz}
\usetikzlibrary{positioning, arrows, shadows}
\usepackage{algorithmic}
\usepackage[chapter]{algorithm}
\floatname{algorithm}{Algoritmo}
\renewcommand{\algorithmicrequire}{\textbf{Entrada:}} 
\renewcommand{\algorithmicensure}{\textbf{Salida:}} 
\renewcommand{\algorithmicend}{\textbf{fin}} 
\renewcommand{\algorithmicif}{\textbf{si}} 
\renewcommand{\algorithmicthen}{\textbf{entonces}} 
\renewcommand{\algorithmicelse}{\textbf{de lo contrario}} 
\renewcommand{\algorithmicelsif}{\algorithmicelse\ y \algorithmicif} 
\renewcommand{\algorithmicendif}{\algorithmicend\ \algorithmicif} 
\renewcommand{\algorithmicfor}{\textbf{para}} 
\renewcommand{\algorithmicforall}{\textbf{para todo}} 
\renewcommand{\algorithmicdo}{\textbf{hacer}} 
\renewcommand{\algorithmicendfor}{\algorithmicend\ \algorithmicfor} 
\renewcommand{\algorithmicwhile}{\textbf{mientras}} 
\renewcommand{\algorithmicendwhile}{\algorithmicend\ \algorithmicwhile} 
\renewcommand{\algorithmicloop}{\textbf{repetir}} 
\renewcommand{\algorithmicendloop}{\algorithmicend\ \algorithmicloop} 
\renewcommand{\algorithmicrepeat}{\textbf{repetir}} 
\renewcommand{\algorithmicuntil}{\textbf{hasta}} 
\renewcommand{\algorithmicprint}{\textbf{imprimir}} 
\renewcommand{\algorithmicreturn}{\textbf{regresar}} 
\renewcommand{\algorithmictrue}{\textbf{verdadero}} 
\renewcommand{\algorithmicfalse}{\textbf{falso}} 
\renewcommand{\algorithmiccomment}[1]{$\quad\vartriangleright$ #1}
\renewcommand{\algorithmicto}{\textbf{hasta}}
\usepackage{listings}
\usepackage{listingsutf8}
\lstset{xleftmargin=.15in,basicstyle=\footnotesize\ttfamily,escapeinside={@<}{>@}}
\renewcommand{\lstlistingname}{Código}
\renewcommand{\listalgorithmname}{Índice de Algoritmos}
\floatname{algorithm}{Algoritmo}
\newcommand{\algorithmautorefname}{Algoritmo}

\usepackage{newunicodechar}
\newunicodechar{λ}{\lambda}
\newunicodechar{Λ}{\Lambda}
\newunicodechar{α}{\alpha}
\newunicodechar{β}{\beta}
\newunicodechar{δ}{\delta}
\newunicodechar{Δ}{\Delta}
\newunicodechar{Γ}{\Gamma}
\newunicodechar{ω}{\omega}
\newunicodechar{Ω}{\Omega}
\newunicodechar{ρ}{\rho}
\newunicodechar{σ}{\sigma}
\newunicodechar{Σ}{\Sigma}
\newunicodechar{τ}{\tau}
\newunicodechar{ν}{\nu}
\newunicodechar{μ}{\mu}
\newunicodechar{ξ}{\xi}
\newunicodechar{Ξ}{\Xi}
\newunicodechar{ζ}{\zeta}
\newunicodechar{η}{\eta}
\newunicodechar{φ}{\phi}
\newunicodechar{Φ}{\Phi}
\newunicodechar{π}{\pi}
\newunicodechar{Π}{\Pi}
\newunicodechar{θ}{\theta}
\newunicodechar{Θ}{\Theta}

\usepackage{xcolor}
\makeatletter
\def\mathcolor#1#{\@mathcolor{#1}}
\def\@mathcolor#1#2#3{%
  \protect\leavevmode
  \begingroup
    \color#1{#2}#3%
  \endgroup
}
\makeatother

\usepackage{fancyhdr}
\pagestyle{fancy}
\fancyhf{}
\fancyhead{}
\fancyhead[LO]{\textsl{\leftmark}}
\fancyhead[RE]{\textsl{\rightmark}}
\fancyhead[RO,LE]{\thepage}
\fancyfoot{}
\renewcommand{\chaptermark}[1]{\markboth{{\textsl{\thechapter}. #1}}{}}
\renewcommand{\sectionmark}[1]{\markright{{\textsl{\thesection}. #1}}}
\renewcommand{\headrulewidth}{0.3pt}
\renewcommand{\footrulewidth}{0pt}

\usepackage{kpfonts}
\usepackage[cal=boondoxo]{mathalfa}
\usepackage[varqu,varl]{inconsolata}

\usepackage[activate={true,nocompatibility}, final, tracking = true, kerning = true, spacing = true, factor = 1100, stretch = 10, shrink = 10]{microtype}
\usepackage[hidelinks]{hyperref}
\hypersetup{
  pdftitle={El cálculo lambda y los fundamentos de la computación},
  pdfauthor={Eduardo Acuña Yeomans}
}

% Comandos de abreviaciones
\newcommand{\bs}{\boldsymbol}
\newcommand{\mc}{\mathcal}
\newcommand{\subst}[3]{#1 \left[ #2 \operatorname{:=} #3 \right]}
\newcommand{\synteq}{\ \equiv\ }
\newcommand{\contract}[1]{\ \operatorname{\rightarrow_{#1}}\ }
\newcommand{\specialcontract}[2]{\ \operatorname{\overset{#2}{\rightarrow_{#1}}}\ }
\newcommand{\xcontract}[2]{\ \operatorname{\overset{#2}{\rightarrow_{#1}}}}
\newcommand{\reduce}[1]{\ \operatorname{\twoheadrightarrow_{#1}}\ }
\newcommand{\xreduce}[2]{\ \operatorname{\overset{#2}{\twoheadrightarrow_{#1}}}}
\newcommand{\convertible}[1]{\ \operatorname{=_{#1}}\ }
\DeclareMathOperator{\Sub}{Sub}
\DeclareMathOperator{\FV}{FV}
\newcommand*\cn[1]{\widehat{#1}}

\begin{document}

\title{El cálculo \( \bs{λ} \) y los fundamentos de la computación}
\author{Eduardo Acuña Yeomans}
\date{2016}

\maketitle

\frontmatter

\cleardoublepage\thispagestyle{empty}
\vspace*{15 cm}
\begin{flushright}\it
  Dedicado a los hackers de LCC
\end{flushright}

\chapter*{Agradecimientos}
\thispagestyle{empty}
Agradezco a los profes que revisaron este trabajo: Olivia Gutú, Julio Waissman, Martín Frías y Jesús Espinoza. En particular al Frías por acompañarme en la exploración inicial de los temas y a Gutú por las sesiones de discusión y por su revisión matona (tanto en contenido como en forma) del presente trabajo.

Agradezco a los profesores Irene Rodríguez, Olivia Gutú, Julio Waissman y Donald Rodríguez por impartir las clases de computación que más disfruté y a los profesores Fernando Luque, Gabriela Robles, Martín Frías y Jacobo Núñez por impartir las clases de matemáticas que más disfruté.

Agradezco a mis compañeros de LCC y matemáticas por su amistad y compañía, en particular a Francisco Manuel Valle Ruiz y Diana Rivera Segundo a quienes admiro y fueron generadores de benevolencia e inspiración.

Agradezco a María José Flores por las tardes de hackeo en las banquitas de mate, por las pláticas en los cactus y por erradicar gnomos.

Agradezco a mis hermanos Esteban y Emilio y a mis progenitores Heriberto y Laura, por existir y ser como son.

\bigskip

Agradezco al Software Libre, a The Pirate Bay, a Library Genesis, a Sci-Hub y a OpenCourseWare por su razón de ser.

%%% Local Variables:
%%% mode: plain-tex
%%% TeX-master: "main"
%%% End:


\tableofcontents

\mainmatter

\chapter*{Introducción}
\addcontentsline{toc}{chapter}{Introducción}
\markright{Introducción}
\label{ch:introduccion}
\section*{Descripción del trabajo}

Este trabajo presenta una introducción al estudio del cálculo $ λ $ puro desde una perspectiva matemática y computacional. El contenido del trabajo es producto de una revisión de la literatura sobre este cálculo y temas relacionados de las ciencias de la computación.

El objetivo de este trabajo es explorar y plantear los aspectos básicos del cálculo $ λ $ relacionados con la computación a un nivel adecuado para estudiantes de la LCC de la Universidad de Sonora con la finalidad de promover el estudio de las ramas teóricas de las Ciencias de la Computación.

\section*{Aportaciones principales}

Este trabajo se basa fuertemente en varios libros sobre el cálculo $ λ $ y la aportación principal es la adaptación y redacción de los temas teniendo en cuenta los conocimientos de un estudiante de la LCC de últimos semestres. Otras aportaciones realizadas son:
\begin{itemize}
\item El \autoref{alg:ski} para compilar cualquier término cerrado a los combinadores $ \bs{SKI} $;
\item Las codificaciones de operaciones booleanas basadas en la codificación del término condicional en la \autoref{sec:expresiones-booleanas} y \autoref{sec:boolean-extensiones};
\item Las codificaciones de operaciones aritméticas basadas en numerales de Church en la \autoref{sec:aritmetica-elemental} y \autoref{sec:hiperoperaciones};
\item El mecanismo para codificar algoritmos iterativos en la \autoref{sec:iteracion};
\item La codificación de términos $ λ $ en el cálculo $ λ $ en la \autoref{sec:estructura-lambda};
\item La implementación de las codificaciones del \autoref{ch:codificacion} en Haskell y Scheme en el \autoref{ap:lambda-scheme};
\item El programa Lambda en el \autoref{ap:lambda}.
\end{itemize}

\section*{Estructura del trabajo}

La estructura del trabajo se conforma de tres capítulos y dos apéndices:

\begin{itemize}
\item El primer capítulo aborda las ideas elementales del cálculo $ λ $ de manera informal relacionando los conceptos de este cálculo con otras áreas de estudio más convencionales de las matemáticas.

\item El segundo capítulo presenta la perspectiva matemática del cálculo $ λ $, se aborda la formalización de la sintaxis del lenguaje del cálculo y después haciendo uso de sistemas formales y sistemas de reducción se formalizan el resto de los conceptos introducidos en el primer capítulo.

\item El tercer capítulo presenta la perspectiva computacional del cálculo $ λ $, se aborda la codificación de objetos matemáticos y algoritmos del álgebra booleana y la aritmética elemental, después se explora la representación de procesos y estructuras recursivas.

\item Los apéndices son implementaciones de programas para ser ejecutados en la computadora. El primer apéndice presenta una manera de programar las codificaciones del tercer capítulo en los lenguajes de programación Scheme y Haskell. El segundo apéndice presenta la implementación de un intérprete y editores para un lenguaje diseñado con la finalidad de explorar los temas de este trabajo de manera interactiva.
\end{itemize}

%%% Local Variables:
%%% mode: latex
%%% TeX-master: "main"
%%% End:


\chapter{Noción informal del cálculo \texorpdfstring{$ \bs{λ} $}{lambda}}
\label{ch:nocion-informal}
El cálculo lambda es un sistema formal creado con la finalidad de expresar,
manipular y estudiar funciones. La manera en la que se trabaja con funciones en
este sistema es diferente a como es usual en la matemática clásica. Por este
motivo, se presenta una introducción informal que tiene como objetivo esclarecer
estas diferencias.\\

La estructura de este capítulo se conforma de tres secciones: en la primera se
aborda la introducción informal al cálculo lambda, en donde se presenta la
terminología utilizada en las explicaciones y ejemplos, se describen algunas de
las diferencias tanto conceptuales como de notación entre las funciones en este
sistema y las funciones en la matemática clásica; en la segunda sección se
presenta la formalización del cálculo lambda y en base a ésta se definen con
precisión los conceptos abordados en la primer sección; en la tercer sección se
exploran diferentes maneras de representar en el cálculo lambda algunos objetos
y operaciones matemáticas utilizadas en el estudio de la computación.\\

\section{Introducción} \label{sec:1.1}

La definición de función en la matemática clásica es el de una relación entre un
conjunto de entradas, llamado \emph{dominio} y un conjunto de salidas, llamado
\emph{codominio}. Esta relación tiene además la propiedad de que cada elemento
del dominio se relaciona exactamente con un elemento del codominio,
formalmente:\\

Sean \(A\) y \(B\) dos conjuntos, una función \(f\) con dominio \(A\) y
codominio \(B\) es un subconjunto del producto cartesiano \(A\times B\), tal que
para toda \(a\in A\), existe \(b\in B\) tal que \((a,\ b)\in f\) y si \((a,\
b^\prime)\in f\) con \(b^\prime \in B\), entonces \(b=b^\prime\).\\

Las funciones tienen varias maneras de ser representadas. En la definición
anterior la representación es la de pares ordenados, en donde la primer
componente del par es un elemento en el dominio y la segunda es un elemento en
el codominio. Dependiendo del uso que se le dá a las funciones,
puede ser conveniente representarlas simbolicamente con expresiones,
graficamente con dibujos, numéricamente con tablas o incluso verbalmente con
palabras.\\

Es posible utilizar esta definición para \emph{describir} las funciones en el
cálculo lambda, para esto se tiene que establecer cuál es el dominio y codominio
de cada función; después presentar una representación conveniente para las
reglas de correspondencia en el sistema. Sin embargo, hay algunas propiedades
del cálculo lambda que hacen que esta definición no pueda ser directamente
aplicada. En particular, el cálculo lambda como sistema formal es una
\emph{teoría ecuacional}, esto significa que la teoría formal del cálculo lambda
solo se conforma de reglas de igualdad entre expresiones, estas reglas
consideran únicamente la estructura de las expresiones y en el caso del cálculo
lambda, las expresiones no se componen de conjuntos, conectivos o
cuantificadores lógicos.\\

En el cálculo lambda existen expresiones para representar \emph{variables},
\emph{funciones} y \emph{aplicaciones}. El concepto de aplicación hace alusión a
la \emph{aplicación de funciones}, es decir, el acto de obtener un elemento del
codominio de una función, a partir de un elemento en su dominio, por ejemplo,
considerando la función \(f(x)=x^2\), aplicar \(f\) a 4 es \(f(4)=16\).\\

Las expresiones de funciones y aplicaciones en el cálculo lambda son en algúnos
aspectos mas restrictivas que en la matemática clásica, ya que en las
expresiones no se pueden escribir directamente números como el 2, ni operaciones
como la exponenciación, por lo tanto, no es posible escribir directamente la
función \(f(x)=x^2\) como una expresión válida del cálculo lambda. Por otro
lado, la aplicación en el cálculo lambda es menos restrictiva que la aplicación
de funciones de la matemática clásica, ya que el cálculo lambda permite aplicar
cualquier expresión válida a otra, no únicamente funciones a valores.\\

En general, las expresiones en el cálculo lambda asociadas a los conceptos de
función y de aplicación de funciones se pueden escribir únicamente en términos
de otras expresiones, las cuales a su vez pueden ser solo variables, funciones o
aplicaciones. Esto no significa que al trabajar con este sistema no podamos
trabajar también con teoría de conjuntos, aritmética, lógica o algúna otra rama
de las matemáticas, esto solo significa que las expresiones en el
\emph{lenguaje} del sistema formal son restrictivas en la manera en la que se
escriben. Sin embargo, al igual que las palabras en el español, en el lenguaje
utilizado para examinar y describir el cálculo lambda es válido hacer uso de
cualquier herramienta, ya sea matemática o computacional, a este otro lenguaje
se le llama \emph{metalenguaje}.\\

Al tratar con funciones en el cálculo lambda, se omite hablar de su dominio y
codominio, esto es debido a que todas las funciones válidas tienen como dominio
y codominio al conjunto de todas las expresiones válidas del cálculo lambda.
Este detalle debe ser tratado con cuidado cuando se representan objetos y
operaciones matemáticas en el cálculo lambda, ya que el dominio y codominio de
estas operaciones segue siendo el del conjunto de todas las expresiones válidas
del cálculo lambda, sin importar la operación que se represente. Por ejemplo, es
posible representar cualquier número natural con expresiones válidas del cálculo
lambda y también es posible tener una representación de la operación de
exponenciación; se puede \emph{emular} la función \(f : \mathbb{N} \to
\mathbb{N}\), \(f(x,y)=x^y\), mas sin embargo la función del cálculo lambda que
representa esta operación segue teniendo como dominio y codominio el conjunto de
todas las expresiones válidas del cálculo lambda, esto significa que será valido
aplicar esta representación de exponenciación a expresiones que no sean
representaciones de números naturales y el resultado de dicha aplicación no
necesariamente es una expresión que represente a un número natural.\\

El hecho de tener un lenguaje tan reducido y minimalista para las expresiones
nos permite poder entender de manera clara y precisa todos los procesos de
manipulación y transformación de la estructura de una expresión, a tal grado que
todas las operaciones que se realizan sobre las expresiones pueden reproducirse
paso a paso de manera mecánica, manipulando los símbolos que las conforman.\\

\subsection{Notación} \label{sec:1.1.1}

La notación utilizada en la matemática clásica para escribir la definición y
aplicación de funciones suele ser la de expresar una regla de correspondencia
como una expresión simbólica. En el cálculo lambda, también se utiliza ésta
representación, pero los símbolos empleados para escribir las expresiones son
definidos con precisión de antemano, en contraste con las expresiones
matemáticas, en donde la notación de las reglas de correspondencia puede ser
extendida de manera arbitraria ya sea para incluír operaciones sobre distintos
objetos matemáticos, compactar repeticiones de operaciones como \(\sum_{i=0}^n\)
o incluso incrustar en la notación procesos no finitos como límites al infinito
\(\lim_{x\to \infty}\).\\

Para introducir la notación del cálculo lambda, consideramos la función
identidad \(I : \mathbb{N} \to \mathbb{N}\) definida como \(I(x)=x\).\\

En la notación clásica, \(I\) se compone de la especificación de su dominio y
codominio, en este caso \(\mathbb{N}\), después se establece la regla de
correspondencia la cual indica que, al aplicar \(I\) a un argumento \(x\in
\mathbb{N}\), el resultado es equivalente a la expresión del lado derecho de la
ecuación, en donde toda aparición de la variable \(x\) hace referencia al
argumento particular al que le fué aplicado \(I\).\\

En el cálculo lambda, no se considera el dominio ni el codominio de las
funciones, e incluso, no se considera el nombre con el que nos referimos a
ellas. La manera en como \(I\) es escrita en este sistema es \[\lc{\x.x}\] el
símbolo ``\(\lambda\)'' nos indica que la expresión es una función, y el símbolo
``\(.\)'' separa la variable que hace referencia al argumento al que la función
es aplicada y la expresión del lado derecho de la igualdad.\\

La aplicación de expresiones se denota de manera diferente también, mientras que
en la notación clásica se escribe \(I(y)\) considerando que \(y\in \mathbb{N}\),
en el cálculo lambda, debido a que no se nombran las funciones, se escribe
explícitamente la función a la que hacemos referencia \[\lc{(\x.x)y}\] En ambos
casos, \emph{realizar} la aplicación consiste en sustituír las apariciones de
\(x\) por \(y\) en la función, dando como resultado \(y\). Sin embargo, no
podemos afirmar que \(\lc{(\x.x)y}=y\) sin antes mencionar de manera explícita
el significado que se le dá a la igualdad entre dos expresiones.\\

Como se mencionó anteriormente, lo único que se puede escribir en el cálculo
lambda son variables, funciones definidas en términos de otra expresión y
aplicaciones entre dos expresiones. Todas las partes de la aplicación del
ejemplo anterior también son expresiones válidas: ``\(\lc{(\x.x)y}\)'',
``\(\lc{\x.x}\)'', ``\(\lc{y}\)'' y ``\(\lc{x}\)''. Estas expresiones muestran
la manera en como se escriben las aplicaciones, las funciones y en el caso de
\(\lc{x}\) y \(\lc{y}\), las variables.\\

A pesar de ser aparentemente una notación mas inconveniente debido a que se
limita a tratar solo con tres clases de expresiones, esta notación nos permite
ser mas explícitos en la descripción de expresiones y provee uniformidad en el
lenguaje formal. Esta notación también permite tener mas control sobre la manera
en que las expresiones son transformadas.\\

La estructura de las expresiones hace que sea mas directa la relación entre una
expresión o una parte de la expreción y un significado. El significado de una
expresión puede referirse a lo que la expresión represente conceptualmente
hablando o a la manera en la que la expresión puede ser operada.\\

Un ejemplo de la importancia de la asignación explícita del significado
operacional de las expresiones es el de los posibles problemas que se pueden
encontrar cuando se realiza la sustitución al momento de aplicar una función a
una expresión, desde la perspectiva de la matemática clásica, consideremos la
función factorial \(f : \mathbb{N} \to \mathbb{N}\) definida como
\[f(n)=
\begin{cases} 
  1 &\mbox{si } n=1\\
  n\times f(n-1) & \mbox{en otro caso.}
\end{cases}
\]

Para obtener el resultado correcto la aplicación de \(f\) en 5, primero
verificamos si \(5=0\), en donde, si fuera el caso, el resultado sería 1, pero
ya que \(5\not= 0\), el resultado es \(5\times f(4)\), el proceso mecánico de
sustituír el argumento en la expresión de la regla de correspondencia consiste
en primero verificar si la condición es cierta antes de proceder en sustituír el
valor del argumento en el consecuente correspondiente. Si este modelo de
sustitución no se especifica para el uso de la notación del análisis casos
presente en el ejemplo, se pudieran contemplar otras maneras de sustituír al 5
en la expresión, por ejemplo, sustituyendo el argumento en todas las apariciones
de la variable \(n\), luego ``expandir'' el valor de la aplicación de funciones
y posteriormente decidir el resultado final verificando si el argumento cumple
la condición. Sin embargo, utilizar este modelo de sustitución en el ejemplo
resulta en realizar una infinidad de sustituciones debido a la naturaleza
recursiva de la definición.\\

En el uso cotidiano de las matemáticas, no se suele analizar el proceso de
sustitución, sin embargo, en el cálculo lambda es de suma importancia. Esta
diferencia se debe a que en la matemática clásica las comparaciones entre dos
funciones o dos expresiones tienden a ser \emph{declarativas}, es decir, se
declaran las relaciones, aseverando que la expresión es cierta; mientras que en
el cálculo lambda son \emph{imperativas}, es decir, toda relación o equivalencia
entre dos expresiones expresa un mecanismo para construir una expresión a partir
de otra.\\

Un ejemplo de esta distinción es el manejo del concepto de función inversa desde
ambas perspectivas, una definición declarativa es: Sea \(f\) una función cuyo
dominio es \(A\) y cuya imágen es \(B\), la función inversa de \(f\) es la
función \(f^{-1}\) con dominio \(B\) e imagen \(A\) tal que, \(f(a)=b\) si y
sólo si \(f^{-1}(b)=a\) con \(a\in A\) y \(b\in B\). En el cálculo lambda no es
usual trabajar con este tipo de definiciones debido a que no describen un
procedimiento mediante el cual se puede obtener \(f^{-1}\) a partir de \(f\).\\

Al referirse a una expresión del cálculo lambda, usualmente, se conoce
parcialmente su estructura, es decir, algúna descripción de sus partes. En el
resto de esta sección nos referiremos a una variable entre la \(\lambda\) y el
punto de una función como \emph{argumento de la función} y a la expresión
después del punto y antes del paréntesis cerrado como el \emph{cuerpo de la
función}. A continuación se muestran algunos ejemplos de expresiones:

\begin{align*}
  \text{a) }\ &\lc{x}\\ 
  \text{b) }\ &\lc{\x.x}\\
  \text{c) }\ &\lc{y (\x.x)}\\
  \text{d) }\ &\lc{(\y.y(\x.x))(\w.w)}\\
  \text{e) }\ &\lc{\x.x x}\\
  \text{f) }\ &\lc{\f x.f x}\\
\end{align*}

Las variables en el cálculo lambda son expresiones válidas, en el inciso
\emph{a} aparece la variable \(\lc{x}\) la cual no es ni una función ni una
aplicación; las variables por si solas en el cálculo lambda casi no tienen
utilidad, pero al ser partes de otra expresión, puede aumentar su importancia:
en el caso del inciso \emph{b} la misma variable \(\lc{x}\) es el cuerpo de la
función y como es también el argumento, esta variable tiene el potencial de
convertirse en cualquier otra expresión a partir de la aplicación de la función
\(\lc{\x.x}\).\\

En el inciso \emph{c} se tiene una aplicación inusual, es la variable \(\lc{y}\)
siendo aplicada a una función. Comunmente se trabaja con expresiones en donde lo
que se aplica es una función, sin embargo si \(\lc{y (\x.x)}\) fuera el cuerpo
de una función, entonces \(\lc{y}\) jugaría un papel mas relevante. Esto se
puede apreciar en el inciso \emph{d}, en donde la expresión del inciso \emph{c}
es el cuerpo de una función con argumento \(\lc{y}\) y esta función está siendo
aplicada a otra función. Este ejemplo nos permite abordar dos ideas importantes,
primero, las funciones pueden ser aplicadas a funciones y segundo el realizar la
aplicación del ejemplo \emph{d}, la variable \(\lc{y}\) toma el valor de
\(\lc{\w.w}\) y es ahora aplicada a la función \(\lc{\x.x}\):

\begin{align*} 
  \text{1. } &\lc{(\y.y(\x.x))(\w.w)} & &\text{ expresión del inciso \emph{d}}\\ 
  \text{2. } &\lc{(\w.w)(\x.x)} & &\text{ al aplicar } \lc{\y.y(\x.x)} \text{ a } \lc{\w.w}\\ 
  \text{3. } &\lc{\x.x} & &\text{ al aplicar } \lc{\w.w} \text{ a } \lc{\x.x}
\end{align*}

En este último ejemplo se describe una secuencia de transformaciones mecánicas
sobre los símbolos de la expresión, este procedimiento tiene algunos detalles
que son importantes recalcar pero se abordan cuando se describa la formalización
del cálculo lambda en la sección \ref{sec:1.2}. Por el momento se describen los
últimos dos incisos los cuales presentan dos conceptos interesantes.\\

En el inciso \emph{e} se tiene una función cuyo cuerpo es la aplicación de su
argumento sobre sí mismo. Lo interesante de esta expresión es que encapsula la
idea de replicar cualquier expresión a la que se aplique. Por ejemplo, si
aplicamos la expresión a la variable \(\lc{y}\) y realizamos el proceso de
aplicación similar al mostrado con el anterior ejemplo, obtendremos \(\lc{y y}\)
como resultado; si aplicamos la expresión a sí misma obtendremos un
``\emph{quine}'' \cite{Hofstadter:GEB}:

\begin{align*} 
  \text{1. } &\lc{(\x.x x)(\x.x x)} & &\text{ expresión del inciso \emph{e} aplicada a si misma}\\
  \text{2. } &\lc{x} \leftarrow \lc{\x.x x} & &\text{ valor que toma } \lc{x} \text{ en el cuerpo de } \lc{\x.x x}\\ 
  \text{3. } &\lc{x x} & &\text{ expresión en donde se sustituye } \lc{x}\\
  \text{4. } &\lc{(\x.x x)(\x.x x)} & &\text{ al completar la sustitución}
\end{align*}

Como podemos observar, el resultado de la aplicación es la expresión inicial, a
pesar de que el término quine se asoció originalmente a una paradoja sobre
valores de verdad \cite{Quine:Paradox}, hoy en día hace referencia a un programa
que tiene como resultado el código fuente de él mismo.\\

El inciso \emph{f} es una función cuyo cuerpo es otra función, en donde el
cuerpo de esta última es la aplicación del argumento de la primer función al
argumento de la segunda. El concepto interesante que ilustra esta expresión es
el de funciones de varias variables: Aplicar esta expresión a una expresión
cualquiera \(M\) y posteriormente aplicar este resultado a otra expresión
cualquiera \(N\) produce el mismo resultado a que si tuvieramos una función de
dos argumentos \(\lc{f}\) y \(\lc{x}\) cuyo cuerpo es \(\lc{f x}\) y aplicaramos
esta expresión hipotética a \(M\) y \(N\).\\

Otra manera de trabajar con funciones de varias variables es la de representar a
tuplas en el el cálculo lambda y tener expresiones para obtener cada elemento de
una tupla. Sin embargo, representar tuplas es un mecanismo mas complejo que se
aborda en la sección \ref{sec:1.3}.\\

\subsection{El concepto de igualdad} \label{sec:1.1.2}

El concepto de igualdad es muy importante en el cálculo lambda. En el
desarrollo histórico del este sistema, el estudio de los criterios que permiten
establecer que dos expresiones son iguales dió pié a una grán diversidad de
variantes de la teoría original.\\

De la mano al concepto de igualdad, están los mecanismos de transformación de
expresiones, estos mecanismos de transformación no son operadores dentro del
lenguaje del cálculo lambda, si no mas bien, son transformaciones que permiten
explorar las diferentes estructuras de las expresiones del cálculo lambda y son
usados como metalenguaje para referirse a equivalencias entre dos expresiones.
Hay una grán variedad de operaciones que pueden transformar expresiones, sin
embargo, en esta subsección abordaremos las mas elementales.\\

\subsubsection{Sustitución}

Un concepto que es de grán importancia para describir y definir las
transformaciones que realizemos sobre expresiones es el de \emph{sustitución}.
Cuando se describió informalmente el proceso mecánico de la aplicación de
funciones en los ejemplos anteriores se mencionó este concepto. A continuación
se presenta una descripción mas detallada de este concepto abordado como
operación de transformación. \\

La sustitución involucra dos expresiones \(\lc{M}\) y \(\lc{N}\) cualesquiera y una
variable \(\lc{x}\), el proceso de transformación consiste en intercambiar todas las
apariciones de la variable \(\lc{x}\) en la expresión \(\lc{M}\) por la expresión \(\lc{N}\),
denotado \(\lc{q[subst[M,x,N]]}\). Es usual que en los sistemas formales, se tenga
cuidado al definir la transformación de sustitución, los detalles de la
transformación son pospuestos por el resto de la sección y se presentan ejemplos
en donde esta definición provisional es suficiente. \\

A manera de ejemplo, consideremos la sustitución de la variable \(\lc{y}\) por \(\lc{x}\)
en la expresión \(\lc{y w}\), se escribe \[\lc{q[subst[y w, y, x]]}\]  y la
expresión resultante es \(\lc{x w}\). \\

Retomando el proceso descrito en el ejemplo de la expresión del inciso \emph{d},
el acto de aplicar la función \(\lc{\y.y(\x.x)}\) en \(\lc{\w.w}\) se escribe
con esta notación como:
\begin{align*} 
  &\lc{(\y.y(\x.x))(\w.w)} & &\lc{q[subst[y(\x.x), y, \w.w]]} \\
  &\lc{(\w.w)(\x.x)} & &\lc{q[subst[w,w,\x.x]]} \\
  &\lc{\x.x} & &
\end{align*}

A diferencia de las transformaciones que se abordan a continuación, la
sustitución no está relacionada directamente con algún concepto de igualdad mas
que el que describe la operación por si misma. Sin embargo es la transformación
fundamental sobre la cual se describe el resto. \\

\subsubsection{Equivalencia de expresiones}

Con la sustitución se pueden transformar expresiones del cálculo lambda
independientemente del contexto en el que se utilizan y sin prestar atención a
lo que la expresión representa conceptualmente. A excepción de la función
identidad y una representación del número 1, no se han abordado expresiones que
representen algún concepto mas allá de su estructura simbólica, y a pesar de ser
tentador asignarle un significado preciso a cada expresión presentada, es
importante seguir analizando únicamente su estructura para tratar el tema de
equivalencia de expresiones. \\

En esta subsección se exploran diferentes criterios para determinar si dos
expresiones son iguales. \\

Usualmente podemos afirmar que dos expresiones son iguales cuando entendemos el
contexto y el nivel de abstracción en el que se refiere a ellas. Por ejemplo,
dos números se suelen considerar iguales si representan el mismo concepto con el
que se desea trabajar, si observamos una ecuación como \(3=\frac{6}{2}\) sabemos
de inmediato que la ecuación es cierta, a pesar de que \(frac{6}{2}\)
explícitamente hace referencia a una división y los dos lados de la igualdad se
escriban diferente simbolicamente. Cuando se involucran variables y operaciones
determinar si dos expresiones son iguales involucra mas información:
considerando la ecuación \(x\cdot y=y\cdot x\), es imposible poder aseverar si
la ecuación es cierta sin establecer los valores que ``\(x\)'', ``\(y\)'' y
``\(\cdot\)'' representan, en caso que sean números naturales y la operación
aritmética de multiplicación la ecuación es cierta, pero en caso que sean
matrices y la operación de producto matricial, la ecuación es falsa. \\

Al abordar expresiones matemáticas, el contexto en el que se expresan casi
siempre se puede inferir por la manera en como las expresiones son utilizadas y
lo mas común es tener a la mano la definición de las estructuras matemáticas y
operaciones utilizadas en las expresiones, sin embargo, en el estudio del
cálculo lambda, se tiene que ser mas explícito en especificar el concepto de
igualdad con el que se trabaja, es tan importante que modificar su significado,
modifica los axiomas de la teoría. \\


Un criterio trivial de equivalencia es el de considerar a dos expresiones como
equivalentes cuando se escriben exactamente igual símbolo por símbolo. A esta
equivalencia se le llama sintáctica y es denotada con el símbolo \(\synteq\). \\

Otro criterio que se puede apreciar al dar un vistazo a dos expresiones es el de
equivalencia estructural. Esta equivalencia toma en cuenta el hecho que la
relevancia de las variables no reside en su nombre o representación
sintáctica, si no en su posición en la estructura de la expresión. \\

Considerndo la función identidad \(\lc{\x.x}\) se puede observar que tiene la
misma estructura que \(\lc{\y.y}\) la cual representa el mismo concepto. A pesar
de no estar escritas exactamente igual, la correspondencia que hay de la
posición de la variable \(x\) en la primera expresión con la posición de la
variable \(y\) en la segunda y el hecho de que ambas tienen la misma estructura
nos permite considerarlas como equivalentes. \\ 

Considerando dos expresiones un poco mas complejas como \(\lc{\f x.f x}\) y
\(\lc{\g y.g y}\) podemos notar que también son equivalentes en este sentido. \\

Una notación utilizada para corroborar la equivalencia estructural es el
\emph{índice de De Bruijn}, esta notación evita la aparición de variables en las
expresiones y en su lugar utiliza números que representan la ``distancia'' de una
variable a la \(\lambda\) de la función en donde aparece como argumento. De tal
manera que una expresión como

\begin{equation}\label{eq:2.1} \lc{\z.(\y.y (\x.x))(\x. z x)}
\end{equation}

se escribe usando el índice de De Bruijn como

\begin{equation}\label{eq:2.2} \lambda (\lambda 1 (\lambda 1)) (\lambda 2 1)
\end{equation}

En la figura~\ref{fig:DeBruijn-transformation} se puede observar de manera
gráfica la transformación de una notación a otra para este ejemplo en
particular. \\

\begin{figure}[h!]
  \centering
  
  \begin{tikzpicture}[level/.style={sibling distance=60mm/#1}] 
    \node (term) {
      \(\lc{\z.(\y.y (\x.x))(\x. z x)}\)
    }; 
    \node [below of=term] (arrow1) {
      \(\Downarrow\)
    }; 
    \node [circle,draw,below of= arrow1] (z) {
      \(\lambda z\)
    } child {
      node [circle,draw] (a) {
        \(\lambda y\)
      } child {
        node [circle,draw] (c) {
          \(y\)
        }
      }
      child {
        node [circle,draw] (d) {
          \(\lambda x\)
        }
        child {
          node [circle,draw] (g) {
            \(x\)
          }
        } 
      } 
    } 
    child {
      node [circle,draw] (b) {
        \(\lambda x\)
      } 
      child {node [circle,draw] (e) {
          \(z\)
        }
      } 
      child {
        node [circle,draw] (f) {
          \(x\)
        }
      } 
    };
    \node [below=4cm of z] (arrow2) {
      \(\Downarrow\)
    };
    \node [circle,draw,below of= arrow2] (z2) {
      \(\lambda\)
    }
    child {
      node [circle,draw] (a2) {
        \(\lambda\)
      } 
      child {
        node [circle,draw] (c2) {
          \(1\)
        }
      } 
      child {
        node [circle,draw] (d2) {
          \(\lambda\)
        } 
        child {
          node [circle,draw] (g2) {
            \(1\)
          }
        } 
      } 
    } 
    child {
      node [circle,draw] (b2) {
        \(\lambda\)
      } 
      child {
        node [circle,draw] (e2) {
          \(2\)
        }
      } 
      child {
        node [circle,draw] (f2) {
          \(1\)
        }
      }
    };
    \node [below=4cm of z2] (arrow3) {
      \(\Downarrow\)
    };
    \node [below of=arrow3](bruijn) {
      \(\lambda (\lambda 1 (\lambda 1)) (\lambda 2 1)\)
    };
  \end{tikzpicture}
  \caption{Transformación de~\eqref{eq:2.1} a~\eqref{eq:2.2}.}
  \label{fig:DeBruijn-transformation}
\end{figure}

Una desventaja de utilizar la notación de De Bruijn es que ciertas expresiones
del cálculo lambda no pueden ser escritas, en particular, toda variable tiene
que estar asociada a una \(\lambda\) para que esta notación pueda ser utilizada.
Sin embargo como veremos más adelante, la mayoría de los usos del cálculo lambda
asocian a todas las variables en las expresiones. \\

A este criterio de equivalencia se le llama \(\alpha\)-convertibilidad. \\

Otra equivalencia que podemos encontrar en las expresiones es la de aplicación
de funciones, esta hace referencia a que la aplicación de una función a una
expresión es equivalente al resultado de evaluar la función con dicha expresión
como argumento. Para entender mejor este concepto, consideramos la función en
notación tradicional \(f(x)=x^2\), si se evalúa \(f(3)\) el resultado es 8, por
lo tanto podemos decir que \(f(3)\) y 8 son equivalentes. \\

Si consideramos la expresión de la función identidad \(\lc{\x.x}\) podemos
afirmar que para cualquier expresión \(M\), \(\lc{(\x.x) M}\) es equivalente a
\(M\).

A esta equivalencia se le llama \(\beta\)-convertibilidad.\\

En la notación tradicional, estas tres equivalencias se denotan con el mísmo
símbolo \(=\), de tal manera que si dos expresiones son equivalentes ya sea
sintácticamente, estructuralmente o aplicativamente, entonces son consideradas
iguales. En el cálculo lambda es importante diferenciar estas equivalencias ya
que el manejo de las funciones no se aborda desde el punto de vista de una
relación entre el dominio y codominio, si no como una expresión que puede ser
manipulada y transformada de manera mecáncia.\\

\paragraph{Equivalencia de redundancia}

Otro tipo de equivalencia es la de redundancia, consideremos la expresión
\(\lc{\x.(\y.y) x}\), el papel que puede jugar es el de ser aplicada en otra
expresión \(M\), la cual resulta igual a aplicar la expresión interna
\(\lc{\y.y}\) en \(M\). Por las equivalencias descritas previamente podemos
observar que \(\lc{\x.(\y.y) x}\) y \(\lc{\y.y}\) no son sintácticamente
equivalentes, ni estructuralmente equivalentes, ni siquiera aplicativamente
equivalentes. El trabajar con la función que envuelve a \(\lc{\y.y}\) resulta
redundante al momento de aplicar las funciones en expresiones, ésto nos permite
considerar un criterio de equivalencia.\\

En el cálculo lambda, la equivalencia de redundancia se denomina
\(\eta\)\emph{-equivalencia} y nos permite considerar como iguales las
expresiones de la forma \(\lc{\x.M x}\) y \(M\).\\

\paragraph{Equivalencia computacional}

En el estudio de la lógica, se hace la distinción que una equivalencia puede ser
extensional o intensional. La equivalencia extensional hace referencia a las
propiedades externas de los objetos, mientras que la equivalencia intensional
hace referencia a la definición o representación interna de los objetos.\\

Las equivalencias sintáctica y estructural son equivalencias intencionales,
mientras que las equivalencias de aplicación y redundancia son equivalencias
extensionales, debido a que se juzgan dos objetos a partir de su evaluación. Sin
embargo, las equivalencias de aplicación y de redundancia no comprenden el caso
mencionado al inicio de esta subsección. Suponiendo que tenemos dos expresiones
\(M\) y \(N\) que describen el mismo algoritmo o la misma función, la
equivalencia de aplicación no los considera equivalente.\\

En la notación tradicional, la igualdad de funciones es una equivalencia
extensional, por ejemplo \(f(x) = e^{i\pi}\times x\) y \(g(x) = x\) describen la
función identidad y podemos aseverar que \(f=g\) sin necesidad de evaluar ambas
funciones con un argumento en particular.\\

En el cálculo lambda se puede hablar de este tipo de igualdad funcional si
consideramos que para toda expresión del cálculo lambda \(P\), si \(\lc{M P}\)
es equivalente a \(\lc{N P}\), entonces las expresiones \(M\) y \(N\) se dice
que son \(ext\) equivalentes.\\

\paragraph{Una simple regla con fuertes implicaciones}

Existe una regla, llamada regla \(\xi\), la cual establece una equivalencia muy
sencilla: si dos expresiones \(M\) y \(N\) son equivalentes, entonces las
expresiones \(\lc{\x.M}\) y \(\lc{\x.N}\) también lo son.\\

Aunque esta regla aparente aportar poco y pueda ser considerada innecesaria si
combinamos todas las equivalencias previamente descritas, es suficiente para
eliminar la equivalencia de redundancia y la equivalencia computacional de la
formalización del cálculo lambda, la cuál es abordada en la sección
\ref{sec:1.2}.\\


\subsubsection{Transformación de expresiones}

A cada equivalencia diferente a la sintáctica se le puede asociar una operación
de transformación la cual nos permita pasar de una expresión \(M\) a otra
expresión \(N\) de tal manera que estas dos expresiones sean equivalentes bajo
algún criterio específico.\\

En el caso de la \(\alpha\)-conversión la operación correspondiente consiste en cambiar
nombres de variables, en la \(\beta\)-\emph{convertibilidad} la operación
consiste en realizar una secuencia de sustituciones de las variables de una
función por expresiones a las que la función es aplicada y en la
\(\eta\)\emph{-equivalencia} la operación consiste en la eliminación de
funciones redundantes.\\

Estas operaciones se definen de manera formal más adelante y aunque puedan
parecer operaciones sencillas de definir a partir de la operación de
sustitución, se tiene que tener mucho cuidado en no obtener expresiones que
rompan la equivalencia asociada.\\

\section{Formalización de la teoría
\texorpdfstring{$\boldsymbol\lambda$}{lambda}} \label{sec:1.2}

La teoría \(\boldsymbol\lambda\) es el conjunto de axiomas que definen
formalmente al cálculo lambda como sistema formal, el objeto de estudio
principal de esta teoría es el del conjunto cociente formado a partir de un
conjunto de expresiones bien formadas y una relación de equivalencia. En las
siguientes subsecciones se definen estos conceptos, los cuales nos permitirán
comenzar el estudio formal del cálculo lambda.\\

\subsection{Expresiones bien formadas} \label{sec:1.2.1}

Una expresión bien formada es un objeto formal sintáctico. Para definir las
expresiones bien formadas de un lenguaje es necesario expresar de manera
rigurosa cómo se constituyen simbolicamente. \\

El conjunto de expresiones bien formadas del cálculo lambda es llamado
\emph{términos lambda}, denotado como \(\Lambda\). Este conjunto tiene elementos
que son expresiones construidas a partir del alfabeto \(\Sigma\). Éste alfabeto
es un conjunto que se conforma por los símbolos \((\) , \()\) , \(.\) ,
\(\lambda\) y una infinidad de símbolos \(v ,\ v^{\prime} ,\ v^{\prime\prime} ,\
\dots\ \), etc. A esta secuencia infinita de símbolos \(v^i\) se denota como
\(V\), de tal manera que \(\Sigma = \left\{\ (\ ,\ )\ ,\ .\ ,\ \lambda\ \right\}
\cup V\).\\

\(\Lambda\) es el conjunto mas pequeño tal que:

\begin{align}
  \label{eq:2.3} \lc{x} \in V &\Rightarrow \lc{x} \in \Lambda\\
  \label{eq:2.4} \lc{M},\ \lc{N} \in \Lambda &\Rightarrow \lc{M N} \in \Lambda\\
  \label{eq:2.5} \lc{M} \in \Lambda,\ \lc{x} \in V &\Rightarrow \lc{\x.M} \in
\Lambda
\end{align}

Cada una de estas tres reglas corresponde a los tres tipos de términos lambda.
La regla \eqref{eq:2.3} implica que los símbolos en \(V\) son términos lambda,
estos símbolos son llamados \emph{átomos}; la regla \eqref{eq:2.4} implica que
dos términos lambda entre paréntesis también son términos lambda, a este tipo de
términos se les llama \emph{aplicaciones}; la regla \eqref{eq:2.5} implica que
si se tiene entre paréntesis el símbolo \(\lambda\) seguido de un átomo, un
punto y un término lambda cualquiera, entonces ésta expresión también es un
término lambda, a este tipo de términos lambda se les llama
\emph{abstracciones}.\\

Desde la perspectiva de lenguajes formales, \(\Lambda = L(G)\), donde \(G\) es
una gramática libre de contexto con categorías sintácticas \(T\) (términos
lambda), \(E\) (aplicaciones), \(F\) (abstracciones) y \(A\) (átomos); símbolos
terminales \(\left\{\ (\ ,\ )\ ,\ .\ ,\ \lambda\ ,\ v,\ {}^{\prime}\ \right\}\);
símbolo inicial \(T\) y con las siguientes reglas de producción:

\begin{align*} \text{1. }\ T &\rightarrow E\ \mid\ F\ \mid\ V\\ \text{2. }\ E
&\rightarrow (\ T\ T\ )\\ \text{3. }\ F &\rightarrow (\ \lambda\ A\ .\ T\ )\\
\text{4. }\ A &\rightarrow v\ \mid\ E\ {}^{\prime}
\end{align*}

Para facilitar la escritura de términos lambda, en este trabajo se realizan las
siguientes consideraciones sobre la notación:

\begin{itemize}
\item[I.] Cuando se hace referencia a cualquier término lambda se utilizan las
letras mayúsculas \(M,\ N,\ O,\ \) etc. Es importante establecer que si en un
ejemplo, explicación, teorema o demostración hacemos referencia a un término
lambda con una letra mayúscula, cualquier otra aparición de esta letra hará
referencia a este mismo término.
\item[II.] Cuando se hace referencia a cualquier átomo se utilizan las letras
minúsculas \(x,\ y,\ z,\ w,\ \) etc. Al igual que en el punto anterior, la
aparición de una letra minúscula en un ejemplo, explicación, teorema o
demostración hace referencia al mismo término.
\item[III.] Los paréntesis son omitidos de acuerdo a las siguientes
equivalencias sintácticas:
  \begin{itemize}
  \item[a)] \(\lc{M N O} \synteq \lc*{M N O}\), en general, se considera la
aplicación de términos lambda como una operación con asociación a la izquierda.
Se tiene que tener cuidado con respetar la asociación, por ejemplo \(\lc{M(N(O
P))} \synteq \lc*{M(N(O P))} \not\synteq \lc*{M N O P}\).
  \item[b)] \(\lc{\x.M N} \synteq \lc*{\x.M N}\), en general, se puede escribir
una abstracción omitiendo los paréntesis externos siempre y cuando no se escriba
un término sintácticamente diferente. Por ejemplo \(\lc{(\x.M N) O} \synteq
\lc*{(\x.M N) O} \not\synteq \lc*{\x.M N O}\) ya que el lado derecho de la
equivalencia es sintácticamente equivalente a \(\lc*{\x.M N O}\).
  \item[c)] \(\lc{\x y z.M} \synteq \lc*{\x y z.M}\), en general, si el
subtérmino a la derecha del punto en una abstracción es también una abstracción,
se pueden agrupar los åtomos antes del punto de ambas abstracciones después de
una \(\lambda\) y antes que el punto, dejando el subtérmino después del punto de
la segunda abstracción, como el del nuevo término.
  \end{itemize}
\end{itemize}

La notación explicada en \emph{III.a)} proviene de la reducción usada por
Schönfinkel, en donde funciones de varias variables se transformn a funciones de
una sola variable \cite{Schonfinkel:Varargs}.\\

\subsection{Relación de equivalencia} \label{sec:1.2.2}

Una relación de equivalencia es una relación binaria \(=\) sobre elementos de
un conjunto \(X\), donde \(=\) es reflexiva, simétrica y transitiva, es
decir:

\begin{itemize}
\item[\S] \(a\in X \Rightarrow a= a\)
\item[\S] \(a,\ b\in X,\ a= b \Rightarrow b= a \)
\item[\S] \(a,\ b,\ c\in X, a= b,\ b= c \Rightarrow a= c\)
\end{itemize}

En el estudio formal del cálculo lambda, la relación de equivalencia asociada a
los términos lambda es llamada \emph{convertibilidad}. Ésta relación es generada
a partir de axiomas y para formular estos axiomas es necesario formalizar el
concepto de \emph{sustitución}.\\

\textbf{DEFINIR LOS CONCEPTOS NECESARIOS PARA HABLAR DE LO QUE SIGUE}

Todas las variables son términos lambda y son llamados átomos \\

Si \(\lc{M}\) y \(\lc{N}\) son términos lambda, entonces \(\lc{(M N)}\) es un
término lambda llamado aplicación \\

Si \(\lc{M}\) es cualquier término lambda y \(\lc{x}\) es cualquier variable,
entonces \(\lc{(\x.M)}\) es un término lambda llamado abstracción \\

La longitud de un término \(\lc{M}\), denotada \(lgh(M)\) es la cantidad total
de apariciones de átomos en \(\lc{M}\) \\

\(\lc{P}\) aparece en \(\lc{Q}\) o \(\lc{P}\) es un subtérmino de \(\lc{Q}\) o
\(\lc{Q}\) contiene a \(\lc{P}\) \\

El alcance de \(\lambda x\) en \(\lc{λx.M}\) es \(\lc{M}\) \\

La aparición de una variable \(\lc{x}\) en un término \(\lc{P}\) es llamada
ligada si está en el alcance de \(\lambda x\) en \(\lc{P}\), ligada y enlazada
si y sólo si es la \(\lc{x}\) en \(\lambda x\) y libre en otro caso \\

Un término cerrado es un término que no contiene variables libres. \\

Un cambio de variable ligada o \(\alpha\)-conversión \\

\(\lc{P}\) es congruente a \(\lc{Q}\) o \(\lc{P}\) se \(\alpha\)-convierte a
\(\lc{Q}\) o \(\lc{P} \convertible{\alpha} Q\) \\

\(\lc{P}\) y \(\lc{Q}\) son términos \(\alpha\)-convertibles \\

Cualquier término de la forma \(\lc{(\x.M)N}\) es llamado \(\beta\)-redex \\

El término correspondiente \(\lc{q[subst[M,x,N]]}\) es llamado su contracción \\

Si y sólo si un término \(P\) contiene una aparición de \(\lc{(\x.M)N}\) y
reemplazamos esa aparición por \(\lc{q[subst[M,x,N]]}\), y el resultado es
\(\lc{q[prime[P,1]]}\), decimos que contraemos la aparición redex en \(\lc{P}\),
y \(\lc{P}\) se \(\beta\)-contrae a \(\lc{q[prime[P,1]]}\) \\

Si y sólo si \(\lc{P}\) puede ser transformada a un término \(\lc{Q}\) a través
de una serie finita o vacía de \(\beta\)-contracciones y cambios de variable
ligada, decimos que \(\lc{P}\) se \(\beta\)-reduce a \(\lc{Q}\) \\

Un término \(\lc{Q}\) que no contiene \(\beta\)-redex es llamada una
\(\beta\)-forma normal o \(\beta\)-fn \\

Decimos que \(\lc{P}\) es \(\beta\)-igual o \(\beta\)-convertible a \(\lc{Q}\)
si y sólo si \(\lc{Q}\) puede ser obtenido a partir de \(\lc{P}\) por una serie
finita o vacía de \(\beta\)-contracciones, \(\beta\)-contracciones inversas y
cambios de variable ligada \\


\textbf{DEFINIR SUSTITUCIÓN}\\

\begin{align} 
  \lc{x}[\lc{x}:=\lc{M}] &\synteq M &\\
  \lc{y}[\lc{x}:=\lc{M}] &\synteq y &y \not\synteq x\\
  \lc{M N}[\lc{x}:=P] &\synteq (M[x:=P]\ N[x:=P]) &\\
  \lc{\x.M}[x:=N] &\synteq \lc{\x.M} &\\
  \lc{\y.M}[x:=N] &\synteq \lc{\y.M} &x \not\in FV(P)\\
  \lc{\y.M}[x:=N] &\synteq (\lambda\ y\ .\ M[x:=N]) &x\in FV(M),\ y\not\in FV(N)\\
  \lc{\y.M}[x:=N] &\synteq (\lambda\ z\ .\ M[y:=z][x:=N]) &x\in FV(M),\ y\in FV(N)
\end{align}

En las ecuaciones e, f y g, la variable $y$ debe de ser diferente a $x$ y en el
inciso g, la variable $z \in (FV(N) \cup FV(M))^c$.


\textbf{DEFINIR ALPHA CONVERSIÓN Y SUS DETALLES FINOS}\\

\textbf{DEFINIR BETA CONVERSIÓN Y SUS DETALLES FINOS}\\

\textbf{DEFINIR REGLA XI Y SUS DETALLES FINOS}\\

\subsection{Conjunto cociente y clases de equivalencia}\label{sec:1.2.3}

Considerando los términos lambda y la relación de equivalencia descritos en las
anteriores subsecciones, su conjunto cociente, llamado \(\Lambda\) modulo \(=\)
y denotado \(\Lambda / =\) es el conjunto de todas las clases de equivalencia de
\(\Lambda\) con respecto a \(=\), donde la clase de equivalencia de un elemento
\(\lc{M}\in \Lambda\) es \(\left\{ \lc{N} \in \Lambda \mid \lc{M} = \lc{N}
  \right\}\). \\

Al tener como objeto de estudio de la teoría \(\boldsymbol{\lambda}\) a
este conjunto, se establece que cada término lambda es representativo de una
clase de términos lambda.\\



\subsection{Axiomas de
\texorpdfstring{$\boldsymbol\lambda$}{lambda}} \label{sec:1.2.4}

Sean \(M,\ N,\ Z\in \Lambda\) y \(x,\ y\in V\), la convertibilidad en la teoría
\(\boldsymbol\lambda\) se genera a partir de los siguientes axiomas:

\begin{align}
  (\alpha) &  & \lc{\x.M}=\lc{\y.q[subst[M,x,y]]M} \\
  (\beta)  &  & \lc{(\x.M)N}=\lc{q[subst[M,x,N]]} \\
  (\rho)   &  & \lc{M}=\lc{M} \\
  (\sigma) &  & \infer{\lc{N}=\lc{M}}{\lc{M}=\lc{N}} \\
  (\tau)   &  & \infer{\lc{M}=\lc{P}}{\lc{M}=\lc{N},\ \lc{N}=\lc{P}} \\
  (\nu)    &  & \infer{\lc{M Z}=\lc{N Z}}{\lc{M}=\lc{N}} \\
  (\mu)    &  & \infer{\lc{Z M}=\lc{Z N}}{\lc{M}=\lc{N}} \\
  (\xi)    &  & \infer{\lc{\x.M}=\lc{\x.N}}{\lc{M}=\lc{N}}
\end{align}

\begin{align} 
  &\lc{M} = \lc{M} &(\text{reflexividad})\\
  &\lc{M} = \lc{N} \Rightarrow \lc{N} = \lc{M} &(\text{simetría})\\ 
  &\lc{M} = \lc{N},\ \lc{N} = \lc{L} \Rightarrow \lc{M} = \lc{L} &(\text{transitividad})\\ 
  &\lc{M} = \lc{N} \Rightarrow \lc{M Z} = \lc{N Z}\\
  &\lc{M} = \lc{N} \Rightarrow \lc{Z M} = \lc{Z N}\\
  &\lc{\x.M} = \lc{\y.M}[\lc{x}:=\lc{y}] &(\alpha\text{-conversión})\\ 
  &\lc{(\x.M)N} = \lc{M}[\lc{x}:=\lc{N}] &(\beta\text{-conversión})\\
  &\lc{M} = \lc{N} \Rightarrow \lc{\x.M} = \lc{\x.N} &(\text{regla } \xi)
\end{align}

\textbf{CLARIFICAR DIFERENCIAS SOBRE LAS TEORÍAS Y SUS NOMBRES}\\

\textbf{DESCRIBIR LÓGICA COMBINATORIA Y LA TEORIA LAMBDA I}

\section{Representaciones} \label{sec:1.3}

\subsection{Álgebra booleana} \label{sec:1.3.1}

\subsection{Aritmética} \label{sec:1.3.2}

\subsection{Estructuras complejas} \label{sec:1.3.3}

\subsection{Técnicas de representación} \label{sec:1.3.4}

%%% Local Variables: %%% coding: utf-8 %%% mode: latex %%% TeX-master: "main"
%%% End:

\chapter{Formalización del cálculo \texorpdfstring{$ \bs{λ} $}{lambda}}
\label{ch:formalizacion}
La noción de \emph{generalización} es de suma importancia en el estudio general de funciones, operaciones o transformaciones. Los predicados en la lógica de primer orden, las funciones en la matemática clásica, los algoritmos en la computación y las abstracciones en el cálculo lambda pueden ser considerados como implementaciones del concepto de generalización para los sistemas de los que forman parte y en algunos casos son la motivación original para el desarrollo de las teorías que los fundamentan.

El estudio de las propiedades generales de las funciones es una de las motivaciones originales del cálculo lambda, sin embargo, este cálculo se formuló de tal manera que es posible abstraer de su propósito original y ser tratado meramente como un sistema formal \cite{Church:LambdaConversion}.

El presente capítulo tiene el objetivo de formalizar las ideas presentadas en el capítulo \ref{ch:nocion-informal}. La formalización del cálculo lambda se realiza desde dos perspectivas:

\begin{enumerate}
\item Construyendo una \emph{teoría formal}, en donde un conjunto de axiomas y reglas de inferencia permiten plantear razonamientos lógicos para demostrar propiedades del cálculo lambda.
\item Formulando nociones de \emph{reducción}, de tal manera que mediante procedimientos de transformación de expresiones del cálculo lambda, se puedan estudiar sus propiedades.
\end{enumerate}

Independientemente de la perspectiva de la formalización, los conceptos son similares, se describe la misma idea general del cálculo lambda y ambos trabajan con el lenguaje formal de sus expresiones.

De acuerdo a Barendregt \cite[p.~22]{Barendregt:Bible}, el objeto de estudio del cálculo lambda es el conjunto de términos lambda módulo convertibilidad. Estos conceptos serán presentados a lo largo de este capítulo.

El contenido de este capítulo está basado en los primeros cuatro capítulos del libro \emph{The Lambda Calculus, Its Syntax and Semantics} de H.P. Barendregt \cite{Barendregt:Bible} y los capítulos 1, 3, 6, 7 y 8 del libro \emph{Lambda Calculus and Combinators, an Introduction} de J.R. Hindley y J.P. Seldin \cite{HindleySeldin:LambdaCalculusAndCombinators} así como el artículo \emph{A Set of Postulates for the Foundation of Logic} y la monografía \emph{The Calculi of Lambda-Conversion} de Alonzo Church \cite{Church:FoundationsLogic,Church:LambdaConversion}.

\section{Términos lambda}
\label{sec:terminos-lambda}

Esta subsección está basada principalmente en el capítulo 2 de \cite{Barendregt:Bible}.

Los \emph{términos lambda} son la formalización de las expresiones descritas en la sección \ref{sec:expresiones}. El conjunto de todos los términos lambda es un lenguaje formal \( Λ \) en donde sus elementos son cadenas compuestas de símbolos de un alfabeto \cite{Hopcroft:Automata}.

El lenguaje \( Λ \) se puede definir de diferentes maneras, a continuación se presenta una definición inductiva y posteriormente una definición basada en una gramática libre de contexto.

\begin{rem}[Notación]\
  \begin{itemize}
  \item El símbolo \( \implies \) denota una implicación lógica, \( P \implies Q \) se lee ``Si \( P \), entonces \( Q \)''.
  \item El símbolo \( \, \longrightarrow\, \) denota una producción en una gramática, \( P \longrightarrow Q \) se lee ``\( P \) produce \( Q \)''.
  \item El símbolo \( \, \Rightarrow\, \) denota un paso en la derivación de una cadena.
  \end{itemize}
\end{rem}

\begin{defn}[Términos lambda]
  El conjunto \( Λ \) tiene elementos que son cadenas conformadas por símbolos en el alfabeto \( Σ=\{\mathtt{(},\ \mathtt{)},\ \mathtt{.},\ λ\} \cup V \), donde \( V \) es un conjunto infinito \( \{v_{0},\ v_{00},\ ... \} \) de variables. \( Λ \) es el conjunto más pequeño que satisface:
  \label{defn:terminos}
  \begin{subequations}
    \begin{align}
      \label{terminos:atomos} \tag{a}
      x \in V & \implies x \in Λ \\
      \label{terminos:abstracciones} \tag{b}
      M \in Λ,\ x \in V & \implies (λx.M) \in Λ \\
      \label{terminos:aplicaciones} \tag{c}
      M,\ N \in Λ & \implies (M\, N) \in Λ
    \end{align}
  \end{subequations}
\end{defn}

Cada uno de estos tres incisos corresponde a las tres clases de términos lambda:

\begin{description}
\item[\eqref{terminos:atomos}] establece que todo elemento de \( V \) es un término lambda a los cuales se les llama \emph{átomos};
\item[\eqref{terminos:abstracciones}] establece que las cadenas de la forma \( (λx.M) \) son términos lambda, donde \( x \) es un átomo y \( M \) es cualquier término lambda, a estos términos se les llama \emph{abstracciones};
\item[\eqref{terminos:aplicaciones}] establece que las cadenas de la forma \( (M\, N) \) son términos lambda, donde \( M \) y \( N \) son términos lambda cualesquiera, a estos términos se les llama \emph{aplicaciones}.
\end{description}

En el estudio usual de lenguajes formales \cite{Hopcroft:Automata}, \( Λ \) pertenece a la clase de lenguajes libres de contexto y puede ser definido de la siguiente manera:

\begin{defn}[Términos lambda]
  \label{defn:terminos-cfg}
  El conjunto de términos lambda es el lenguaje generado por la gramática libre de contexto \( G \) conformado por

  \begin{description}
  \item[categorías sintácticas] \( T \), \( E \), \( F \) y \( A \), las cuales denotan las reglas para derivar términos lambda, aplicaciones, abstracciones y átomos respectivamente;
  \item[símbolos terminales] \( \{\mathtt{(},\ \mathtt{)},\ \mathtt{.},\ λ,\ v,\ {}_{0}\} \), los cuales son los símbolos que conforman a las cadenas en \( Λ \);
  \item[símbolo inicial] \( T \), el cual es el símbolo del que se derivan todos los términos lambda;
  \item[reglas de producción]
    \begin{subequations}
    \begin{align}
      \label{terminos-cfg:terminos} \tag{a}
      T & \rightarrow E\ \mid\ F\ \mid\ A \\
      \label{terminos-cfg:atomos} \tag{b}
      A & \rightarrow \mathtt{v}_{0}\ \mid\ A {}_{0} \\
      \label{terminos-cfg:abstracciones} \tag{c}
      F & \rightarrow \mathtt{(} λ\ A\ \mathtt{.}\ T\ \mathtt{)} \\
      \label{terminos-cfg:aplicaciones} \tag{d}
      E & \rightarrow \mathtt{(}\ T\ T\ \mathtt{)}
    \end{align}
  \end{subequations}
  \end{description}
\end{defn}

Dada una secuencia de símbolos \( M \), se pueden utilizar estas dos definiciones para verificar si \( M \) es o no un término lambda. En el caso de la definición inductiva, se debe presentar un razonamiento que pruebe que las partes de la cadena satisface la definición \ref{defn:terminos}. En el caso de la gramática libre de contexto \ref{defn:terminos-cfg} se debe presentar una derivación de la cadena a partir de la categoría sintáctica \( T \).

\begin{exmp} Sea \( M = (λv_{0}.(v_{00} (λv_{00}.v_{000}))) \), la cadena \( M \) es un término lambda ya que

  \begin{description}
  \item[Por definición inductiva]
    \begin{align*}
      v_{000} \in V &\implies v_{000} \in Λ; \\
      v_{00} \in V,\ v_{000} \in Λ &\implies (λv_{00}.v_{000}) \in Λ,\ v_{00} \in Λ; \\
      v_{00},\ (λv_{00}.v_{000}) \in Λ &\implies (v_{00} (λv_{00}.v_{000})) \in Λ; \\
      v_{0} \in V,\ (v_{00} (λv_{00}.v_{000})) \in Λ &\implies (λv_{0}.(v_{00} (λv_{00}.v_{000}))).
    \end{align*}
  \item[Por gramática] Se mantienen los espacios en los lados derechos de las producciones de la gramática para ser consistentes, sin embargo, el espacio en blanco no es un símbolo terminal, por lo tanto pueden ser omitidos.
    \begin{align*}
      T &\Rightarrow F \Rightarrow (\ λ\ A\ .\ T\ ) \Rightarrow (\ λ\ v_{0}\ .\ T\ ) \Rightarrow (\ λ\ v_{0}\ .\ E\ ) \Rightarrow (\ λ\ v_{0}\ .\ (\ T\ T\ )\ ) \\
        &\Rightarrow (\ λ\ v_{0}\ .\ (\ A\ T\ )\ ) \Rightarrow (\ λ\ v_{0}\ .\ (\ A_{0}\ T\ )\ ) \Rightarrow (\ λ\ v_{0}\ .\ (\ v_{00}\ T\ )\ ) \\
        &\Rightarrow (\ λ\ v_{0}\ .\ (\ v_{00}\ F\ )\ ) \Rightarrow (\ λ\ v_{0}\ .\ (\ v_{00}\ (\ λ\ A\ .\ T\ )\ )\ ) \\
        &\Rightarrow (\ λ\ v_{0}\ .\ (\ v_{00}\ (\ λ\ A_{0}\ .\ T\ )\ )\ ) \Rightarrow (\ λ\ v_{0}\ .\ (\ v_{00}\ (\ λ\ v_{00}\ .\ T\ )\ )\ ) \\
        &\Rightarrow (\ λ\ v_{0}\ .\ (\ v_{00}\ (\ λ\ v_{00}\ .\ A\ )\ )\ ) \Rightarrow (\ λ\ v_{0}\ .\ (\ v_{00}\ (\ λ\ v_{00}\ .\ A_{0}\ )\ )\ ) \\
        &\Rightarrow (\ λ\ v_{0}\ .\ (\ v_{00}\ (\ λ\ v_{00}\ .\ A_{00}\ )\ )\ ) \Rightarrow (\ λ\ v_{0}\ .\ (\ v_{00}\ (\ λ\ v_{00}\ .\ v_{000}\ )\ )\ ).
    \end{align*}
  \end{description}
\end{exmp}

\begin{exmp} Sea \( N = ((λv_{00}.v_{0}\, v_{00}) v_{0}) \), la cadena \( N \) no es un término lambda ya que

  \begin{description}
  \item[Por definición inductiva] Ya que \( Λ \) se definió como el \emph{conjunto más pequeño}, se demuestra que \( N \not\in Λ \) de la siguiente manera
    \begin{align*}
      (λv_{00}.v_{0}\, v_{00}), v_{0} \in Λ &\implies ((λv_{00}.v_{0}\, v_{00}) v_{0}) \in Λ; \\
      v_{00} \in V,\ v_{0}\, v_{00} \in Λ &\implies (λv_{00}.v_{0}\, v_{00}) \in Λ; \\
      v_{0}\, v_{00} \not\in Λ &\therefore ((λv_{00}.v_{0}\, v_{00}) v_{0}) \not\in Λ.
    \end{align*}
  \item[Por gramática] La gramática no es ambigua, realizando una derivación por la izquierda
    \begin{align*}
      T &\Rightarrow E \Rightarrow (\ T\ T\ ) \Rightarrow (\ F\ T\ ) \Rightarrow (\ (\ λ\ A\ .\ T\ )\ T\ ) \\
        &\Rightarrow (\ (\ λ\ A_{0}\ .\ T\ )\ T\ ) \Rightarrow (\ (\ λ\ v_{00}\ .\ T\ )\ T\ ) \\
        &\nRightarrow (\ (\ λ\ A\ .\ v_{0}\, v_{00}\ )\ T\ ).
    \end{align*}
  \end{description}
\end{exmp}

La sintaxis del cálculo lambda es uniforme, lo cual permite identificar su estructura con facilidad y evitar ambigüedades, sin embargo, suele ser tedioso escribir términos largos debido al extenso uso de paréntesis. Es por esto que en este trabajo se hacen las siguientes consideraciones sobre la notación:

\begin{enumerate}
\item \label{enum:notacion:1} El símbolo \( \synteq \) denota la equivalencia sintáctica entre dos términos lambda, esta equivalencia contempla las consideraciones de este listado.
\item \label{enum:notacion:2} Cuando se hace referencia a \emph{cualquier} término lambda se utilizan las letras mayúsculas \( M \), \( N \), \( P \), etc. Es importante establecer que si en un ejemplo, explicación, teorema o demostración se hace referencia a un término lambda con una letra mayúscula, cualquier otra aparición de esta letra hace referencia a este mismo término dentro de ese contexto.
\item \label{enum:notacion:3} Cuando se hace referencia a \emph{cualquier} átomo se utilizan las letras minúsculas \( x \), \( y \), \( z \), etc. Al igual que en el punto anterior, la aparición de una letra minúscula en un ejemplo, explicación, teorema o demostración hace referencia al mismo átomo.
\item \label{enum:notacion:4} Los paréntesis son omitidos de acuerdo a las siguientes equivalencias sintácticas:
  \begin{enumerate}
  \item \label{enum:notacion:4a} \( ((M\, N) P) \synteq M\, N\, P\), en general, se considera la aplicación de términos lambda con asociación a la izquierda. Se tiene que tener cuidado con respetar esta regla, por ejemplo \( (M(N(O\, P))) \synteq M(N(O\, P)) \not\synteq M\, N\, O\, P \).
  \item \label{enum:notacion:4b} \( (λx.(M N)) \synteq λx.(M N) \), en general, se puede escribir una abstracción omitiendo los paréntesis externos. Es necesario escribir de manera explícita los paréntesis en algunos casos, por ejemplo \( ((λx.(M\, N)) O) \synteq (λx.(M\, N)) O \not\synteq λx.(M\, N)O \) ya que el lado derecho de la equivalencia es sintácticamente equivalente a \( (λx.((M\, N)O)) \).
  \item \label{enum:notacion:4c} \( (λx.(λy.(λz.M))) \synteq (λx\, y\, z.M) \), en general, si el cuerpo de una abstracción es también una abstracción, se pueden agrupar las variables ligadas y enlazadas. Éste abuso de notación es consistente con la representación de funciones de varias variables usada por Schönfinkel \cite{Schonfinkel:Varargs}.
  \end{enumerate}
\item \label{enum:notacion:5} Para hacer referencia a una secuencia con una cantidad arbitraria de términos lambda se usa la notación \( \vec{x}=x_{1},...,x_{n} \) cuando es secuencia de átomos y \( \vec{M}=M_{1},...,M_{n} \) cuando es secuencia de términos lambda en general. Con esta notación se puede abreviar la consideración de \ref{enum:notacion:4a} como
  \[ ((\, ...\, ((M_{1}\, M_{2}) M_{3})\, ...\, ) M_{n}) \synteq \vec{M} \]
  y la consideración de \ref{enum:notacion:4c} como
  \[ (λx_{1}.(λx_{2}.(λx_{3}.\, ...\, (λx_{n}.M)\, ...\, ))) \synteq (λ\vec{x}.M) \]
  Ya que la notación no indica la cantidad de términos en la secuencia, se suele decir que \( \vec{M} \) cabe en \( \vec{N} \) cuando son secuencias con la misma cantidad de elementos.
\item Al escribir términos lambda con repetición de aplicaciones suele ser conveniente utilizar una notación más compacta. Cuando se aplica \( n \) veces un término \( F \) por la izquierda a otro término \( M \) se denota \( F^{n}\, M \). Cuando se aplica \( n \) veces un término \( M \) por la derecha a otro término \( F \) se denota \( F\, M^{\sim n}\). Por ejemplo, el término \( (f(f(f(f\, x)))) \) se puede denotar como \( (f^{4}\, x) \) y el término \( (f\, x\, x\, x\, x) \) se puede denotar como \( (f\, x^{\sim 4}) \). La definición inductiva de esta notación es:
  \begin{align}
    \label{eq:abuso:F}
    \begin{split}
      F^{n+1}\, M & \synteq F (F^{n}\, M) \\
      F^{0}\, M & \synteq M
    \end{split}
  \end{align}
  \begin{align}
    \label{eq:abuso:M}
    \begin{split}
      F\, M^{\sim n+1} & \synteq (F\, M^{\sim n}) M \\
      F\, M^{\sim 0} & \synteq F
    \end{split}
  \end{align}
\end{enumerate}

Inicialmente, estos abusos de notación pueden resultar confusos, sin embargo, al escribir términos lambda complejos resulta conveniente acortarlos. A continuación se muestran ejemplos de términos lambda asociados a términos sintácticamente equivalentes pero escritos con abuso de notación:

\begin{exmp}
  \label{exmp:notacion}
  \begin{align*}
    (((x\, y)z) (y\, x)) & \synteq x\, y\, z (y\, x) \\
    (λx.((u\, x)y)) & \synteq λx.u\, x\, y \\
    (λy.(u(λx.y))) & \synteq λy.u(λx.y) \\
    (((λy.((v\, u)u))z)y) & \synteq (λy.v\, u\, u) z\, y \\
    (((u\, x)(y\, z))(λv.(v\, y))) & \synteq u\, x(y\, z)(λv.v\, y) \\
    ((((λx.(λy.(λz.((x\, z) (y\, z))))) u) v) w) & \synteq (λx\, y\, z.x\, z(y\, z)) u\, v\, w
  \end{align*}
\end{exmp}

\subsection{Estructura}

Dado un término lambda \( M \) es deseable poder cuantificar algunas propiedades de acuerdo a su estructura, la medida más común es la de \emph{longitud}. Esta propiedad resulta importante en los razonamientos inductivos, por ejemplo, al plantear una demostración se suele usar la expresión ``por inducción sobre \( M \)'' la cual técnicamente se refiere a una inducción sobre la longitud de \( M \).

\begin{defn}[Longitud]
  La longitud de un término lambda \( M \), denotada como \( \| M \| \), es la cantidad de \emph{apariciones} de átomos en \( M \), se determina a partir de la estructura del término lambda como:
  \label{defn:longitud}
  \begin{align*}
    \|x\| & = 1 \\
    \|M\, N\| & = \|M\| + \|N\| \\
    \|λx.M\| & = 1 + \|M\|
  \end{align*}
\end{defn}

Debido a que la definición considera la cantidad de átomos en \( M \) y la longitud de un átomo es \( 1 \), se infiere que para cualquier término lambda \( M \), su longitud será estrictamente mayor a cero. Una implicación de esta observación es que al ``desbaratar'' la longitud de un término lambda de acuerdo a su estructura, en el caso de que \( M \) sea una aplicación o una abstracción, la longitud de sus partes es estríctamente menor a su longitud.

\begin{exmp} A continuación se presenta el procedimiento para calcular la longitud del término \( M \synteq (x(λy.y\, u\, x)) \) siguiendo la definición \ref{defn:longitud}
  \begin{align*}
    \| M \| &= \| (x(λy.y\, u\, x)) \| = \| (x (λy.((y\, u) x))) \\
            &= \| x \| + \| (λy.((y\, u) x)) \| = 1 + \| (λy.((y\, u) x)) \| \\
            &= 1 + ( 1 + \| ((y\, u) x) \|  ) = 2 + \| ((y\, u) x) \| \\
            &= 2 + \| (y\, u) \| + \| x \| = 2 + \| (y\, u) \| + 1 = 3 + \| (y\, u) \| \\
            &= 3 + \| y \| + \| u \| = 3 + 1 + 1 \\
            &= 5
  \end{align*}
\end{exmp}

Una cuestión importante al momento de demostrar un teorema o definir un concepto por inducción sobre un término lambda es que usualmente la inducción matemática relaciona proposiciones con números naturales. Sin embargo es posible tener dos términos diferentes \( M \) y \( N \) tal que \( \|M\| = \|N\| \), por ejemplo \( (λx.x) \) y \( (z\, z) \) tienen longitud \( 2 \).

La inducción sobre la longitud de un término lambda considera también la estructura del término, de tal manera que para una proposición \( P \) sobre un término lambda \( M \), los casos base de la inducción son aquellos en donde la estructura no es compuesta (en átomos cuya longitud siempre es \( 1 \)) y la hipótesis de inducción considera que \( P \) se cumple para los subtérminos de \( M \) cuya longitud siempre es estrictamente menor que \( \|M\| \).

En la definición de longitud se menciona de la cantidad de \emph{apariciones} de átomos en \( M \), el concepto de aparición de \( M \) en \( N \) para cualesquiera \( M \) y \( N \) se formaliza a partir del concepto de \emph{subtérmino}.

\begin{defn}[Subtérmino]
  \( M \) es un subtérmino de \( N \), denotado \( M \subset N \) si \( M \in \Sub(N) \), donde \( \Sub(N) \) es la colección de subtérminos de \( N \) definida de manera inductiva como
  \label{defn:subtermino}
  \begin{align*}
    \Sub(x) & = \{ x \} \\
    \Sub(λx.M) & = \Sub(M) \cup \{ λx.M \} \\
    \Sub(M\, N) & = \Sub(M) \cup \Sub(N) \cup \{ M\, N \}
  \end{align*}
\end{defn}

\begin{defn}[Aparición]
  La aparición de \( M \) en \( N \) implica que \( M \subset N \) o que \( M \) es \emph{el} argumento de una abstracción en \( N \).
  \label{defn:aparicion}
\end{defn}

Usualmente se habla de la aparición de \( M \) en \( N \) para referirse a una subtérmino en particular en \( N \), sin embargo, un subtérmino pude \emph{aparecer} varias veces en un término. Algunas clasificaciones de subtérminos son:

\begin{itemize}
\item Si \( M_{1} \) y \( M_{2} \) son subtérminos de \( N \) y no tienen átomos en común, se dice que son términos \emph{disjuntos} de \( N \), ya que si esta condición se cumple \( \Sub(M_{1}) \cap \Sub(M_{2}) = \emptyset \);
\item Si \( M \subset N \) y \( (M\, Z) \subset N \) se dice que \( M \) es un término \emph{activo} en \( N \), de lo contrario, se le llama \emph{pasivo};
\item Si \( M \subset N \) y \( (λx.M) \subset N \), se dice que la aparición \( M \) es el \emph{alcance} de la aparición del átomo \( x \) que acompaña a la \( λ \).
\end{itemize}

\begin{exmp}
  Sea \( M \synteq λx.x\, y (λz.y) \):
  \label{exmp:subterminos-apariciones}
  \begin{itemize}
  \item el término \( (x\, y) \subset M \);
  \item el átomo \( z \not\subset M \) pero si aparece en \( M \), debido a que \( z \) acompaña a una \( λ \);
  \item el término \( y(λz.y) \) a pesar de parecer ser un subtérmino de \( M \) no lo es, esto se puede corroborar escribiendo los términos sin el abuso de notación: \( y(λz.y) \synteq (y(λz.y)) \) y \( M \synteq λx.x y(λz.y) \synteq (λx.((x\, y)(λz.y))) \), en este caso, la clave está en observar la estructura de la aplicación \( (x\, y(λz.y)) \).
  \item Las apariciones de \( x \) y \( (λz.y) \) en \( M \) son disjuntas.
  \item Los términos \( x \) y \( (x\, y) \) son subtérminos activos de \( M \), mientras que \( y \) y \( (λz.y) \) son subtérminos pasivos.
  \end{itemize}
\end{exmp}

Los conceptos de longitud y de subtérmino nos permiten razonar de manera clara sobre la estructura de los términos lambda y con la clasificación de los subtérminos se puede caracterizar el rol que juegan las partes de un término en la estructura general.

\subsection{Clasificación}

A continuación se presentan algunos criterios para clasificar partes de los términos lambda y las propiedades que tienen los términos de acuerdo a su clasificación.

Al considerar las apariciones de átomos en un término lambda, es conveniente diferenciar a los átomos sintácticamente iguales dependiendo de el papel que juegan en el término.

\begin{defn}[Clasificación de variables]\label{defn:clasifvar}
  La aparición de un átomo \( x \) en un término \( P \) es llamada:
  \begin{itemize}
  \item \emph{variable ligada} si es un subtérmino de \( M \) en una abstracción \( (λx.M) \) en \( P \);
  \item \emph{variable enlazada} si y sólo si es la \( x \) que acompaña la \( λ \) de \( (λx.M) \) en \( P \);
  \item \emph{variable libre} en otro caso.
  \end{itemize}
\end{defn}

La diferencia entre un átomo \( x \subset M \) y una aparición de \( x \) en \( M \) es que la aparición se refiere a una variable en particular nombrada \( x \) en una parte específica de la estructura de \( M \). Por ejemplo, en el término lambda \( ((λx.x) x) \) la primera aparición del átomo \( x \) es una variable enlazada, la segunda aparición es una variable ligada y la tercera aparición es una variable libre.

Cuando se abordó el concepto de reducción en la sección \ref{sec:op-reduccion} la distinción entre una variable libre y una ligada era importante ya que las variables libres nunca son sustituídas en una reducción ya que el procedimiento relacionaba únicamente a las variables ligadas en el alcance de una abstracción activa.

\begin{exmp}
  Sea \( M \synteq x(λy.x\, y) \):
  \label{exmp:clasifvar}
  \begin{itemize}
  \item El átomo \( x \) aparece como variable libre dos veces en \( M \);
  \item El átomo \( y \) aparece como variable ligada en \( M \);
  \item El átomo \( y \) aparece como la variable enlazada de la abstracción.
  \end{itemize}
\end{exmp}

En la definición formal de algunos conceptos es conveniente hacer referencia a todas las variables libres de un término lambda.

\begin{defn}[Variables libres]
  El conjunto de variables libres de un término lambda \( M \) se denota \( \FV(M) \) y se define de manera inductiva como:
  \label{defn:varlib}
  \begin{align*}
    \FV(x) & = \{ x \} \\
    \FV(λx.M) & = \FV(M) \setminus \{ x \} \\
    \FV(M\, N) & = \FV(M) \cup \FV(N)
  \end{align*}
  Cuando \( \FV(M)=\emptyset \) se dice que \( M \) es un \emph{combinador} o \emph{término cerrado}.
\end{defn}

\begin{exmp}
  Consideremos los términos \( (x(λx.x\, y\, z)) \), \( (λx\, y\, z.y) \) y \( ((λy.x)λx.y) \).
  \label{exmp:varlib}
  \begin{itemize}
  \item \( \FV(x(λx.x\, y\, z)) = \{x,\ y,\ z\} \);
  \item \( \FV(λx\, y\, z.y)=\emptyset \), por lo tanto es un combinador;
  \item \( \FV((λy.x)λx.y)=\{ x,\ y \} \).
  \end{itemize}
\end{exmp}

En ocaciones es importante distinguir los términos lambda cerrados de aquellos que contienen variables libres, para ello se identifica el subconjunto de \( Λ \) que contiene a todos los términos cerrados:

\begin{defn}[Términos cerrados]
  Se denota como \( Λ^{0} \) al conjunto
  \label{defn:termcerr}
  \[ \{ M \in Λ \mid M \text{ es un término cerrado} \} \]
\end{defn}

La notación \( Λ^{0} \) se puede generalizar para identificar diferentes subconjuntos de \( Λ \) a partir de las variables libres de los términos lambda:

\[ Λ^{0}(\vec{x})=\{ M \in Λ \mid \FV(M) \subseteq \{ \vec{x} \} \} \]

De tal manera que:

\[ Λ^{0}=Λ^{0}(\emptyset) \]

Si consideramos un término \( M \) con variables libres, se puede encontrar otro término \( N \in Λ^{0} \) similar a \( M \), al cual se le llama clausura de \( M \).

\begin{defn}[Clausura] \label{defn:clausura}
  La clausura de un término lambda \( M \) con \( \FV(M) \not= \emptyset \) es un término lambda
  \[ (λ\vec{x}.M) \]
  con \( \vec{x}=\FV(M) \)
\end{defn}

\begin{exmp} \label{exmp:clausura}
  Sea \( M \synteq λz.x\, y\, z \)
  \begin{itemize}
  \item \( (λx\, y.λz.x\, y\, z) \) es una clausura de \( M \);
  \item \( (λy\, x\, z.x\, y\, z) \) es una clausura de \( M \);
  \item \( (λz\, x\, y.λz.x\, y\, z) \) no es una clausura de \( M \).
  \end{itemize}
\end{exmp}

\subsection{Sustitución de términos}
\paragraph{Pendiente explicar el resto de la sección}

\begin{defn}[Sustitución]
  \label{defn:sustitucion}
  Para cualesquiera términos lambda \( M \), \( N \) y \( x \), se define \( M[x:=N] \) como el resultado de sustituir cada aparición libre de \( x \) por \( N \) en \( M \) de acuerdo a las siguientes reglas:
  \begin{align*}
    x[x:=N] & \synteq N; \\
    a[x:=N] & \synteq a && a \not \synteq x; \\
    (P\, Q)[x:=N] & \synteq P[x:=N]\, Q[x:=N]; \\
    (λx.P)[x:=N] & \synteq λx.P; \\
    (λy.P)[x:=N] & \synteq λy.P && x \not\synteq y,\ x \not\in \FV(P); \\
    (λy.P)[x:=N] & \synteq λy.P[x:=N] && x \not\synteq y,\ x \in \FV(P),\ y \not\in \FV(N); \\
    (λy.P)[x:=N] & \synteq λz.P[y:=z][x:=N] && x \not\synteq y,\ x \in \FV(P),\ y \in \FV(N),\ z \not\in \FV(N P).
  \end{align*}
\end{defn}

\begin{exmp} \label{exmp:sustitucion}
  Procedimientos de sustituciones para cada uno de los casos de la definición \ref{defn:sustitucion}:
  \begin{itemize}
  \item Caso \( x[x:=N] \)
    \begin{align*}
      y[y:=λx.x] \synteq λx.x
    \end{align*}
  \item Caso \( a[x:=N] \), donde \( a \not\synteq x \)
    \begin{align*}
      z[w:=x\, x] \synteq z
    \end{align*}
  \item Caso \( (P\, Q)[x:=N] \)
    \begin{align*}
      (y\, x\, x)[x:=y] & \synteq ((y\, x) x)[x:=y] \\
                        & \synteq (y\, x)[x:=y]\, x[x:=y] \\
                        & \synteq (y[x:=y]\, x[x:=y]) y \\
                        & \synteq y\, y\, y
    \end{align*}
  \item Caso \( (λx.P)[x:=N] \)
    \begin{align*}
      (λf\, x.f\, f\, x)[f:=g] \synteq λf\, x.f\, f\, x
    \end{align*}
  \item Caso \( (λy.P)[x:=N] \), donde  \( x \not\synteq y \), \( x \not\in \FV(P) \)
    \begin{align*}
      (λf\, x.f\, f\, x)[f:=g] \synteq λf\, x.f\, f\, x
    \end{align*}
  \item Caso \( (λy.P)[x:=N] \), donde \( x \not\synteq y \), \( x \in \FV(P) \), \( y \not\in \FV(N) \)
    \begin{align*}
      (λf.x\, λx.f\, f\, x)[x:=y] & \synteq λf.(x\, λx.f\, f\, x)[x:=y] \\
                                  & \synteq λf.x[x:=y]\, (λx.f\, f\, x)[x:=y] \\
                                  & \synteq λf.y\, λx.f\, f\, x
    \end{align*}
  \item Caso \( (λy.P)[x:=N] \), donde \( x \not\synteq y \), \( x \in \FV(P) \), \( y \in \FV(N) \) y \( z \not\in \FV(N P) \)
    \begin{align*}
      (λf.x\, λx.f\, f\, x)[x:=f] & \synteq λg.(x\, λx.f\, f\, x)[f:=g][x:=f] \\
                                  & \synteq λg.(x[f:=g](λx.f\, f\, x)[f:=g])[x:=f] \\
                                  & \synteq λg.(x\, λx.(f\, f\, x)[f:=g])[x:=f] \\
                                  & \synteq λg.(x\, λx.((f\, f)[f:=g] x[f:=g]))[x:=f] \\
                                  & \synteq λg.(x\, λx.((f[f:=g]\, f[f:=g]) x))[x:=f] \\
                                  & \synteq λg.(x\, λx.g\, g\, x)[x:=f] \\
                                  & \synteq λg.x[x:=f] (λx.g\, g\, x)[x:=f] \\
                                  & \synteq λg.f\, λx.g\, g\, x
    \end{align*}
  \end{itemize}
\end{exmp}

En el último caso es importante observar que las apariciones ligadas de \( x \) no se sustituyen.

\begin{lem}
  Si \( (y\, x) \not\in \FV(L) \) y \( x \not\synteq y \), entonces
  \[ M[x:=N][y:=L] \synteq M[y:=L][x:=N[y:=L]] \]
\end{lem}

En contraste a la operación de sustitución en donde no se permite introducir o quitar referencias a variables enlazadas, el \emph{contexto} es un término con ``hoyos'':

\begin{defn}[Contexto]
  \label{defn:contexto}
  Un contexto es un término lambda denotado \( C[\quad] \) definido de manera inductiva:
  \begin{itemize}
  \item \( x \) es un contexto;
  \item \( [\quad] \) es un contexto;
  \item Si \( C_{1}[\quad] \) y \( C_{2}[\quad] \) son contextos, entonces \( C_{1}[\quad]\, C_{2}[\quad] \) y \( λx.C_{1}[\quad] \) también lo son.
  \end{itemize}
\end{defn}

Si \( C[\quad] \) es un contexto y \( M \in Λ \), entonces \( C[M] \) denota el resultado de reemplazar por \( M \) los hoyos de \( C[\quad] \). Al realizar esto, las variables libres de \( M \) pueden convertirse en variables ligadas de \( C[M] \).

\begin{exmp}
  Consideremos el contexto \( C[\quad] \synteq λx.x\, λy.[\quad] \) y el término lambda \( M \synteq (x y) \).
  
  \begin{align*}
    C[M] & \synteq (λx.x\, λy.[\quad])[(x\, y)] \\
         & \synteq (λx.x\, λy.(x\, y))
  \end{align*}
  
  El caso análogo con la sustitución es

  \begin{align*}
    (λx.x\, λy.w)[w:=(x\, y)] & \synteq λz.(x\, λy.w)[x:=z][w:=(x\, y)] \\
                              & \synteq λz.(x[x:=z] (λy.w)[x:=z])[w:=(x\, y)] \\
                              & \synteq λz.(z\, λy.w)[w:=(x y)] \\
                              & \synteq λz.z[w:=(x\, y)] (λy.w)[w:=(x\, y)] \\
                              & \synteq λz.z\, λv.w[w:=(x\, y)] \\
                              & \synteq λz.z\, λv.(x\, y)
  \end{align*}
\end{exmp}

\section{Los cálculos de la conversión lambda}
\label{sec:conversion-lambda}

El objetivo principal de esta subsección es presentar una formalización del cálculo lambda descrito en el capítulo \ref{ch:nocion-informal} desde el punto de vista de \emph{teorías formales}. El nombre técnico de la teoría formal principal de este trabajo es \( \bs{λKβ} \), se pueden realizar modificaciones y extensiones a esta teoría y los siguientes conceptos permiten estudiar las implicaciones de estos cambios.

\subsection{Teorías formales}
\label{sec:teorias-formales}

Una \emph{teoría formal} \( \mathcal{T} \) es una tripleta \( (\mathcal{F},\mathcal{A},\mathcal{R}) \) donde

\begin{itemize}
\item \( \mathcal{F} \) es el conjunto de todas las \emph{fórmulas} \( X = Y \) con \( X \) y \( Y \) elementos de un lenguaje formal;
\item \( \mathcal{A} \) es un conjunto de \emph{axiomas} y \( \mathcal{A} \subseteq \mathcal{F} \);
\item \( \mathcal{R} \) es un conjunto de \emph{reglas}.
\end{itemize}

Una regla es una función \( φ \colon \mathcal{F}^{n} \to \mathcal{F} \) con \( n \geq 1 \). Si se consideran \( n \) fórmulas \( A_{1},\ ...\ ,\ A_{n} \) tal que

\[ φ(A_{1},\ ...\ ,\ A_{n})=B \]

Se dice que la secuencia \( \langle A_{1},\ ...\ ,\ A_{n},\ B \rangle \) es una \emph{instancia} de la regla \( φ \). Las primeras \( n \) fórmulas de una instancia son llamadas \emph{premisas} y la última fórmula es llamada \emph{conclusión}. Para escribir una instancia de una regla se utiliza la notación

\[ \infer{B}{A_{1} & ... & A_{n}} \]

\begin{rem}
  En la literatura se pueden encuentrar diferentes maneras de trabajar con teorías formales, dependiendo de su ``estilo'' y definición, por ejemplo en \cite{Troelstra:ProofTheory} las reglas se definen como conjuntos de secuencias \( \langle A_{1},...,A_{n+1} \rangle \) con \( n \) premisas y una conclusión, en donde los axiomas se definen como elementos de \( \mathcal{R} \) con cero premisas. La definición de teoría formal presentada en este trabajo es del estilo Hilbert y está basada en \cite[pp.~69--70]{HindleySeldin:LambdaCalculusAndCombinators}.
\end{rem}

Si consideramos un conjunto de \emph{suposiciones} \( Γ \subseteq \mathcal{F} \), una \emph{deducción} de una fórmula \( B \) desde \( Γ \) es un árbol dirigido de fórmulas en donde los vértices de un extremo son elementos de \( \mathcal{A} \) o \( Γ \), los vértices intermedios son deducidos a partir de los vértices que inciden en ellos a partir de una regla y el vértice de el otro extremo siendo \( B \). Si y solo si existe una deducción para una fórmula \( B \), se dice que \( B \) es \emph{demostrable} en \( \mathcal{T} \) suponiendo \( Γ \), denotado

\[ \mathcal{T},Γ \vdash B \]

En caso que la deducción no tenga suposiciones, se dice que es una \emph{demostración} y que \( B \) es un \emph{teorema}. Cuando una deducción no tiene suposiciones, es decir, \( Γ = \emptyset \) se denota

\[ \mathcal{T} \vdash B \]

La relación binaria \( = \) en las fórmulas de una teoría es una relación de equivalencia, la cual por definición es \emph{reflexiva}, \emph{simétrica} y \emph{transitiva}. La \emph{clase de equivalencia} de un objeto \( x \) con respecto a \( = \) de una teoría formal \( \mc{T} \), denotado \( [x]_{\mc{T}} \), es el conjunto de todos los objetos \( y \) tal que \( x = y \) es una fórmula de \( \mc{T} \).

En el contexto de las teorias que formalizan los cálculos lambda, los objetos que se relacionan son términos lambda. La frase ``módulo convertibilidad'' se refiere al conjunto de todas las clases de equivalencia de \( Λ \) considerando la relación de equivalencia de la teoría formal con la que se esté trabajando.

Que este conjunto sea el objeto de estudio de una teoría \( \bs{λ} \) del cálculo lambda significa que cada elemento de \( Λ \) módulo convertibilidad, denotado \( Λ/\!=_{\bs{λ}} \), es distinto y representa una clase de términos lambda considerados en la teoría \( \bs{λ} \) como equivalentes. Cuando \( \bs{λ} \vdash M = N \) se dice que \( M \) y \( N \) son términos \emph{convertibles}, también denotado \( M =_{\bs{λ}} N \).

Habiendo definido una teoría \( \bs{λ} \), el interés de estudiarla es

\begin{itemize}
\item determinar los términos que son convertibles en \( \bs{λ} \);
\item estudiar las propiedades que comparten dos términos convertibles;
\item modificar a \( \bs{λ} \) y comparar la teoría modificada con la original.
\end{itemize}

La comparación entre teorías usualmente consiste en partir de una teoría \( \bs{λ} \), modificar sus fórmulas, axiomas o reglas para obtener otra teoría \( \bs{λ}^{\prime} \) y determinar si \( \bs{λ} \) y \( \bs{λ}^{\prime} \) son equivalentes. Para poder realizar esto, se debe definir formalmente cómo se modifica una teoría y cómo se demuestra que dos teorías son equivalentes.

Modificar una teoría \( \bs{λ} \) puede implicar cambiar la definición de sus términos, es decir, utilizar un lenguaje formal diferente al de \( \bs{λ} \) para expresar las fórmulas \( M=N \). Hacer cambios al lenguaje formal suele requerir modificar al conjunto \( \mc{F} \), \( \mc{A} \) y \( \mc{R} \) de la teoría.

La modificación al lenguaje formal puede ser únicamente de relevancia sintáctica, por ejemplo modificar una teoría \( \bs{λ} \) cuyo lenguaje de términos es el conjunto \( Λ^{0} \) para que los términos sean escritos con el índice de DeBruijn mostrado en \ref{exmp:debrujn} no tendría implicaciones fuertes en la convertibilidad de la teoría, ya que hay una correspondencia uno a uno entre estas dos notaciones. Por otro lado, modificar una teoría \( \bs{λ} \) con términos \( Λ \) de tal manera que se consideren únicamente los términos cerrados \( Λ^{0} \) si puede tener fuertes implicaciones en la convertibilidad de la teoría ya que habrá términos lambda no admitidos en fórmulas.

Otra manera de modificar una teoría \( \bs{λ} \) es añadir o quitar axiomas y reglas de inferencia. Para abordar la modificación de estas dos componentes de una teoría consideramos que los axiomas son reglas sin premisas.

Cuando se considera extender una teoría \( \bs{λ} \) con una nueva regla \( φ \) lo primero que se debe estudiar es si \( φ \) es \emph{derivable} en \( \bs{λ} \), es decir, si para cada instancia de \( φ \), su conclusión es deducible en \( \bs{λ} \) considerando sus premisas como suposiciones. Formalmente, para cada instancia \( \langle A_{1},\ ...\ ,\ A_{n},\ B \rangle \) de \( φ \), \( φ \) es derivable en \( \bs{λ} \) si y sólo si

\begin{equation}
  \label{eq:teorias-derivable}
  \bs{λ},\ \{ A_{1},\ ...\ ,\ A_{n} \} \vdash B
\end{equation}

Cuando añadir una regla \( φ \) a una teoría \( \bs{λ} \) no cambia el conjunto de teoremas se dice que la regla es \emph{admisible}, por ejemplo si \( φ \) es utilizada en la demostración de un teorema, pero este teorema se puede demostrar sin suponer las premisas de \( φ \), entonces añadir a \( φ \) no afecta el hecho de que el teorema exista en la teoría.

Otra manera de verificar si una regla \( φ \) es admisible en \( \bs{λ} \) es demostrando que la regla es \emph{correcta}. Una regla se dice ser correcta en una teoría \( \bs{λ} \) si y sólo si, para cada instancia \( \langle A_{1},\ ...\ ,\ A_{n},\ B \rangle \) de \( φ \):

\begin{equation}
  \label{eq:teorias-correcta}
  (\bs{λ} \vdash A_{1}),\ ...\ ,\ (\bs{λ} \vdash A_{n}) \implies (\bs{λ} \vdash B)
\end{equation}

Si una regla es derivable, entonces es admisible, sin embargo, una regla admisible no siempre es derivable. Consideremos una instancia \( \mc{r} \) de una regla admisible en \( \bs{λ} \) tal que ni las premisas, ni la conclusión de \( \mc{r} \) son demostrables en la teoría, entonces la implicación \eqref{eq:teorias-correcta} es verdadera para \( \mc{r} \), sin embargo esto no implica que se pueda demostrar la conclusión suponiendo las premisas. Por otro lado, si consideramos una instancia \( \mc{r} \) de una regla derivable en \( \bs{λ} \), entonces ya que la conclusión es demostrable suponiendo las premisas, demostrar las premisas asegura que se puede demostrar la conclusión.

Con estos conceptos se pueden definir dos criterios de equivalencia entre teorías: \emph{equivalentes en teoremas} y \emph{equivalentes en reglas}.

\begin{defn}[Equivalencia de teorías] \label{defn:teorias-equivalentes}
  Sean \( \bs{λ} \) y \( \bs{λ}^{\prime} \) dos teorías formales con el mismo conjunto de fórmulas.

  Se dice que las teorías son \emph{equivalentes en teoremas} cuando cada regla y axioma de \( \bs{λ} \) es admisible en \( \bs{λ}^{\prime} \) y viceversa.

  Se dice que las teorías son \emph{equivalentes en reglas} cuando cada regla y axioma de \( \bs{λ} \) es derivable en \( \bs{λ}^{\prime} \) y viceversa.

  La equivalencia en teoremas es una equivalencia más débil que la equivalencia en reglas.
\end{defn}

\subsection{Teoría \( \bs{λKβ} \)}
\label{sec:teorialambda}

La teoría \( \bs{λKβ} \) es la formalización del cálculo lambda que se ha tratado desde el inicio de este trabajo. Ya que es la teoría principal, a partir de este punto cuando se hable de \emph{la} teoría \( \bs{λ} \) se estará refiriendo a la teoría \( \bs{λKβ} \) y cuando se hable de \emph{las} teorías \( \bs{λ} \) se estará refiriendo a la familia de teorías que formalicen los cálculos lambda.

\begin{defn}[Teoría \( \bs{λKβ} \)]
  \label{defn:teorialambda}

  El conjunto de fórmulas \( \mc{F} \) en \( \bs{λKβ} \) tiene como elementos ecuaciones de la forma:

  \begin{align*}
    M = N & & \forall M,N \in Λ \text{ (de la definición~\ref{defn:terminos})}
  \end{align*}
  
  Los axiomas \( \mc{A} \) de \( \bs{λKβ} \) son:

  \begin{subequations}
    \begin{align}
      \label{teorialambda:alpha} \tag{\( α \)}
      λx.M & = λy.M[x:=y] & &  \forall y \not\in \FV(M) \\
      \label{teorialambda:beta} \tag{\( β \)}
      (λx.M)N & = M[x:=N] \\
      \label{teorialambda:rho} \tag{\( ρ \)}
      M & = M
    \end{align}
  \end{subequations}

  Las reglas \( \mc{R} \) de \( \bs{λKβ} \) son:

  \begin{equation}
    \label{teorialambda:mu} \tag{\( μ \)}
    \infer{Z\, M = Z\, N}{M = N}
  \end{equation}
  \begin{equation}
    \label{teorialambda:nu} \tag{\( ν \)}
    \infer{M\, Z = N\, Z}{M = N}
  \end{equation}
  \begin{equation}
    \label{teorialambda:xi} \tag{\( ξ \)}
    \infer{λx.M = λx.N}{M = N}
  \end{equation}
  \begin{equation}
    \label{teorialambda:tau} \tag{\( τ \)}
    \infer{M = P}{M = N & N = P}
  \end{equation}
  \begin{equation}
    \label{teorialambda:sigma} \tag{\( σ \)}
    \infer{N = M}{M = N}
  \end{equation}
  
\end{defn}

Consideremos la convertibilidad en \( \bs{λKβ} \) de los términos lambda
\begin{align*}
  M &\synteq (λf.x((λy.y\, f) λz.z))w \\
  N &\synteq x\, w
\end{align*}

Se demuestra que \( M=_{\bs{λKβ}}N \) construyendo un árbol de deducción como el de la figura \ref{fig:demostrabilidad}.

\begin{figure}
  \centering
  \begin{tikzpicture}[
    equat/.style={rectangle,draw},grow=up,edge from parent/.style={draw,latex-},
    level 1/.style={sibling distance=20em, level distance=5em},
    level 2/.style={sibling distance=40em},
    level 3/.style={sibling distance=20em},
    level 4/.style={sibling distance=10em}
    ]
    \node [equat] (foo1) {\( (λf.x((λy.y\, f) λz.z))w = x\, w \)}
    child {
      node [equat] (foo3) {\( (λf.x\, f)w = x\, w \)}
    }
    child {
      node [equat] (foo5) {\( (λf.x((λy.y\, f) λz.z))w = (λf.x\, f)w \)}
      child {
        node [equat] (foo7) {\( λf.x((λy.y\, f) λz.z) = λf.x\, f \)}
        child {
          node [equat] (foo9) {\( x((λy.y\, f) λz.z) = x\, f \)}
          child {
            node [equat] (foo11) {\( (λy.y\, f)λz.z = f \)}
            child {
              node [equat] (foo13) {\( (λz.z)f = f \)}
            }
            child {
              node [equat] (foo14) {\( (λy.y\, f)λz.z = (λz.z) f \)}
            }
            edge from parent [] node [right] {\( (μ) \)}
          }
          edge from parent [] node [right] {\( (ξ) \)}
        }
        edge from parent [] node [right] {\( (ν) \)}
      }
    };
    \node [above=0em of foo13] (bar1) {\( (β) \)};
    \node [above=0em of foo14] (bar2) {\( (β) \)};
    \node [above=0em of foo11] (bar3) {\( (τ) \)};
    \node [above=0em of foo1] (bar4) {\( (τ) \)};
    \node [above=0em of foo3] (bar5) {\( (β) \)};
  \end{tikzpicture}
  \caption{Árbol de deducción para demostrar la convertibilidad entre dos términos}
  \label{fig:demostrabilidad}
\end{figure}

\subsection{Combinadores \( \bs{SKI} \)}
\label{sec:combinadores-ski}

\begin{defn}[Combinadores SKI]
  \label{defn:ski}
  Tres términos lambda de suma importancia son
  \begin{align*}
    \bs{I} & \synteq λx.x \\
    \bs{K} & \synteq λx\, y.x \\
    \bs{S} & \synteq λx\, y\, z.x\, z(y\, z)
  \end{align*}
\end{defn}

\begin{cor}
  \label{cor:ski}
  Para todo término \( M,N,L \in Λ \)

  \begin{align*}
    \bs{I}\, M & =_{\bs{λ}} M \\
    \bs{K}\, M\, N & =_{\bs{λ}} M \\
    \bs{S}\, M\, N\, L & =_{\bs{λ}} M\, L (N\, L)
  \end{align*}
\end{cor}

Estos tres combinadores generan en la teoría \( \bs{λ} \) al conjunto \( Λ^{0} \) con combinaciones de aplicaciones. Debido a que \( \bs{SKK} =_{\bs{λ}} \bs{I} \), sólo es necesario combinar con aplicaciones a \( \bs{K} \) y a \( \bs{S} \) para generar cualquier término cerrado.

\begin{defn}[Bases]
  \label{defn:bases}
  \begin{enumerate}
  \item Sea \( \mathcal{X} \subset Λ \). El conjunto de términos \emph{generado} por \( \mathcal{X} \), denotado \( \mathcal{X}^{+} \), es el conjunto mas pequeño tal que
    \begin{enumerate}
    \item \( \mathcal{X} \subseteq \mathcal{X}^{+} \),
    \item \( M, N \in \mathcal{X}^{+} \implies (M N) \in \mathcal{X}^{+} \).
    \end{enumerate}
  \item Sea \( \mathcal{P} \subset Λ \). \( \mathcal{X} \subset Λ \) es una \emph{base} para \( \mathcal{P} \) si para toda \( M \in \mathcal{P} \) existe \( N \in \mathcal{X}^{+} \) tal que \( N = M \).
  \item \( \mathcal{X} \) es llamada una \emph{base} si \( \mathcal{X} \) es una base para \( Λ^{0} \).
  \end{enumerate}
\end{defn}

\begin{lem}
  \label{lem:ski}
  Sea \( λx.M \) una abstracción tal que \( \Sub(M) \) no contiene abstracciones

  \begin{enumerate}
  \item Si \( M = x \), entonces \( λx.M = \bs{I} \);
  \item Si \( x \not\in \FV(M) \), entonces \( λx.M = (\bs{K} M) \);
  \item Si \( M = P\, Q \), entonces \( λx.M = \bs{S}(λx.P)(λx.Q) \).
  \end{enumerate}
\end{lem}

\begin{proof}
  Ver \ref{defn:ski} y \ref{cor:ski}
  \begin{enumerate}
  \item
    \begin{align*}
      (\bs{I}\, N) &= N \\
                   &= ((λx.x) N) \\
                   &= ((λx.M) N)
    \end{align*}
  \item
    \begin{align*}
      ((\bs{K}\, M) N) &= (λx\, y.x)M\, N \\
                       &= (λy.M)N \\
                       &= ((λx.M)N)
    \end{align*}
  \item
    \begin{align*}
      \bs{S}(λx.P)(λx.Q) &= (λabc.(a\, c)(b\, c))(λx.P)(λx.Q) \\
                         &= λc.((λx.P)c)((λx.Q)c) \\
                         &= λc.P[x:=c]Q[x:=c] \\
                         &= λc.(P\, Q)[x:=c] \\
                         &= λx.P\, Q \\
                         &= λx.M
    \end{align*}
  \end{enumerate}
\end{proof}

\begin{prop}
  \label{prop:ski}
  \( \{ \bs{S}, \bs{K}, \bs{I} \} \) es una base, es decir, para todo término \( M \in Λ^{0} \), existe un término \( M' \) compuesto de aplicaciones de \( \bs{S} \), \( \bs{K} \) e \( \bs{I} \) tal que \( M = M' \).
\end{prop}

La demostración de la proposición \ref{prop:ski} consiste en la construcción de un algoritmo para transformar \( M \) a \( M' \).

\begin{proof}
  \label{proof:ski}
  Sea \( M \in Λ^{0} \), se construye un término \( M' \in \{ \bs{S},\bs{K},\bs{I} \}^{+} \) tal que \( M' = M \) enumerando los subtérminos en \( M \) que sean abstracciones de menor a mayor longitud.

  Sea \( λx.N \) la abstracción con menor longitud en \( M \), según la estructura de \( N \) se aplican las siguientes transformaciones:

  \begin{enumerate}
  \item Si \( N = a \)
    \begin{enumerate}
    \item \label{item:ski:1a} Si \( a = x \) se transforma \( λx.N \) a \( \bs{I} \) en \( M \).
    \item \label{item:ski:1b}Si \( a \not= x \) se transforma \( λx.N \) a \( (\bs{K}\, a) \) en \( M \).
    \end{enumerate}
  \item \( N = (P\, Q) \)
    \begin{enumerate}
    \item \label{item:ski:2a} Si \( x \not\in \FV(P) \) y \( x \not\in \FV(Q) \) se transforma \( λx.P\, Q \) a \( \bs{S} (\bs{K}\, P) (\bs{K}\, Q) \) en \( M \).
    \item \label{item:ski:2b} Si \( x \not\in \FV(P) \) y \( x \in \FV(Q) \) se transforma \( λx.P\, Q \) a \( \bs{S} (\bs{K}\, P) (λx.Q) \) en \( M \).
    \item \label{item:ski:2c} Si \( x \in \FV(P) \) y \( x \not\in \FV(Q) \) se transforma \( λx.P\, Q \) a \( \bs{S} (λx.P) (\bs{K}\, Q) \) en \( M \).
    \item \label{item:ski:2d} Si \( x \in \FV(P) \) y \( x \in \FV(Q) \) se transforma \( λx.P\, Q \) a \( \bs{S} (λx.P) (λx.Q) \) en  \( M \).
    \end{enumerate}
  \end{enumerate}

  En los casos \ref{item:ski:2a}, \ref{item:ski:2b}, \ref{item:ski:2c}, \ref{item:ski:2d} se forman abstracciones con longitud menor a \( λx.N \), por lo tanto serán las que se transformarán después. Ya que la longitud de estas abstracciones es estrictamente menor a \( λx.N \) y los casos base \ref{item:ski:1a} y \ref{item:ski:1b} de la transformación no introducen abstracciones, en una cantidad finita de pasos el término \( M \) transformado no tendrá abstracciones de la forma \( λx.N \).

  Para un término \( M \) con sólo una abstracción, \( \mathit{a} \) aplicaciones y \( \mathit{v} \) variables ligadas (no enlazadas) una cota superior para la máxima cantidad de pasos se calcula considerando que para los términos de la forma \( λx.a \) se cumple el caso \ref{item:ski:1b} (el cual aumenta la cantidad de aplicaciones en 1) y que para los términos de la forma \( λx.P\, Q \) se cumple el caso \ref{item:ski:2d} (el cual aumenta la cantidad de aplicaciones en 2) ya que en estos casos se produce el término con mas aplicaciones, las cuales determinan la cantidad de veces que se repite el algoritmo por cada abstracción en \( M \). Para calcular la cota superior de la cantidad de aplicaciones \( a' \) que produce el algoritmo para un término con \( n \) abstracciones se plantea la siguiente relación de recurrencia:

  \begin{align*}
    \mathit{a}'_{0} &= \mathit{a} \\
    \mathit{a}'_{n} &= 2 \times \mathit{a}'_{n-1} + \mathit{v}
  \end{align*}

  Esta recurrencia describe la función \( \mathit{a}' \colon \mathbb{N} \to \mathbb{N} \):

  \[ \mathit{a}'(n) = 2^{n} \times \mathit{a} + (2^{n}-1) \times \mathit{v} \]

  Para la cota superior de la cantidad de pasos realizados por el algoritmo para un término \( M \) con \( n \) abstracciones, se plantea la siguiente relación de recurrencia basada en \( \mathit{a}' \) y en el hecho de que la cantidad de variables ligadas y no enlazadas no aumenta en los pasos del algoritmo:

  \begin{align*}
    \mathit{p}_{0} &= 0 \\
    \mathit{p}_{n} &= \mathit{p}_{n-1} + \mathit{a}'(n-1) + \mathit{v}
  \end{align*}

  Esta recurrencia describe la función \( \mathit{p} \colon \mathbb{N} \to \mathbb{N} \):

  \begin{align*}
    \mathit{p}(n) &= (\mathit{a} + \mathit{v}) \times \sum_{i=0}^{n-1} 2^{i} \\
                  &= (\mathit{a} + \mathit{v}) \times (2^{n}-1)
  \end{align*}
  
\end{proof}

\paragraph{Algoritmo para compilar \( Λ^{0} \bs{\mapsto} \{ \bs{S},\ \bs{K},\ \bs{I} \}^{+} \)}

\begin{algorithm}
  \caption{SKI}
  \begin{algorithmic}
    \REQUIRE \( M \in Λ^{0} \)
    \ENSURE \( M' \in \{ \bs{S},\ \bs{K},\ \bs{I} \} \)
    
    \STATE \( M' \leftarrow M \)
    \STATE \( \mathcal{L} \leftarrow \{ A \in \Sub(M') \mid A \synteq λx.N \} \)
    
    \WHILE{\( \mathcal{L} \not= \emptyset \)}
    
    \STATE \( A \leftarrow λx.N \in \mathcal{L} \mid \| λx.N \| \leq A', \forall A' \in \mathcal{L} \)
    
    \IF{\( A \synteq λx.a \)}
    \IF{\( a \synteq x \)}
    \STATE \( M'[A] \leftarrow \bs{I} \)
    \ELSIF{\( a \not\synteq x \)}
    \STATE \( M'[A] \leftarrow \bs{K}\, a \)
    \ENDIF
    \ELSIF{\( A \synteq λx.P\, Q \)}
    \IF{\( x \not\in \FV(P) \land x \not\in \FV(Q) \)}
    \STATE \( M'[A] \leftarrow \bs{S} (\bs{K}\, P) (\bs{K}\, Q) \)
    \ELSIF{\( x \not\in \FV(P) \land x \in \FV(Q) \)}
    \STATE \( M'[A] \leftarrow \bs{S} (\bs{K}\, P) (λx.Q) \)
    \ELSIF{\( x \in \FV(P) \land x \not\in \FV(Q) \)}
    \STATE \( M'[A] \leftarrow \bs{S} (λx.P) (\bs{K}\, Q) \)
    \ELSIF{\( x \in FV(P) \land x \in \FV(Q) \)}
    \STATE \( M'[A] \leftarrow \bs{S} (λx.P) (λx.Q) \)
    \ENDIF
    \STATE \( M'[A] \leftarrow \mathrm{SKI}(\ M'[A]\ ) \)
    \ENDIF
    
    \STATE \( \mathcal{L} \leftarrow \mathcal{L} \setminus \{ A \} \)
    
    \ENDWHILE
    \RETURN \( M' \)
  \end{algorithmic}
\end{algorithm}

\subsection{Teoría \( \bs{λIβ} \)}
\label{sec:lambda-i-beta}

En el artículo \cite{Church:LambdaConversion}, Alonzo Church presenta una definición del cálculo lambda con un conjunto restringido de términos lambda. A la teoría que considera a este conjunto restringido de términos lambda (denotado \( Λ_{I} \)) y los axiomas y reglas de inferencia de la teoría \( \bs{λ} \) cambiando \( Λ \) por \( Λ_{I} \) se le conoce como teoría \( \bs{λIβ} \) (o el cálculo \( λI \)).

\begin{defn}[Términos en \( Λ_{I} \)]
  \label{defn:lambdaI}
  \begin{align*}
    x \in V & \implies x \in Λ_{I} \\
    M \in Λ_{I},\ x \in \FV{M} & \implies λx.M \in Λ_{I} \\
    M, N \in Λ_{I} & \implies M\, N \in Λ_{I}
  \end{align*}
\end{defn}

La diferencia fundamental entre las teorías \( \bs{λKβ} \) y \( \bs{λIβ} \) es el término lambda \( \bs{K} \), ya que \( \bs{K} \in Λ \setminus Λ_{I} \) pero \( \bs{K} \not\in Λ_{I} \). Esto es debido a que el subtérmino \( λy.x \) en \( \bs{K} \) de la definición \ref{defn:ski} no puede existir en \( Λ_{I} \) debido a que \( y \not\in \FV(x) \).

\subsection{Extensionalidad}
\label{sec:extensionalidad}

El concepto de igualdad de funciones usado en la mayoría de las ramas de la matemática es lo que se conoce como ``extensional'', esta propiedad de las relaciones de equivalencia hace referencia a las características externas de los objetos que compara, en el caso de las funciones, se incluye la suposición de que para funciones \( f \) y \( g \) con el mismo dominio

\[ \forall x [ f(x)=g(x) ] \implies f=g \]

Contraria a esta suposición, en la computación, el tema central son los procedimientos y procesos que describen los programas o algoritmos, cuyas igualdades ``intensional'', es decir, si dos programas computan la misma función matemática, no necesariamente se dice que son el mismo programa ya que uno pudiera ser mas eficiente que otro (la característica de eficiencia es interna a cada algorítmo).

La teoría \( \bs{λ} \) también es intensional: existen dos términos lambda \( F \) y \( G \) tales que para tódo término \( X \)

\[ \bs{λ} \vdash F\, X = G\, X \]

Pero no \( \bs{λ} \vdash F=G \). Por ejemplo, \( F \synteq y \) y \( G \synteq λx.y\, x \)

Cuando se plantea formalizar un cálculo lambda que sea extensional, surge la pregunta, ¿Qué es demostrable en el sistema extensional que no es demostrable en \( \bs{λ} \). A continuación se presentan tres diferentes agregados a la teoría \( \bs{λ} \) las cuales incluyen la propiedad de extensionalidad y que han sido propuestas en la literatura \cite{HindleySeldin:LambdaCalculusAndCombinators,Barendregt:Bible}. Las teorías extendidas son llamadas \( \bs{λζ} \), \( \bs{λ+ext} \) y \( \bs{λη} \) de acuerdo a la regla que se añade a la definición \ref{defn:teorialambda}.

\begin{defn}[Reglas de extensionalidad]
  \label{defn:extensionalidad}
  Cada una de las siguientes reglas nos permite añadir a \( \bs{λ} \) la propiedad de extensionalidad.
  \begin{description}
  \item[Reglas de inferencia]
    \begin{subequations}
      \begin{align}
        \label{extensionalidad:zeta} \tag{\( ζ \)}
        \infer{M = N}{M\, x = N\, x} & & \text{si \( x \not\in \FV(M\, N) \)} \\
        \label{extensionalidad:ext} \tag{ext}
        \infer{M = N}{M\, P = N\, P} & & \forall P \in Λ
      \end{align}
    \end{subequations}
  \item[Axiomas]
    \begin{align}
      \label{extensionalidad:eta} \tag{\( η \)}
      λx.M\, x = M & & \text{si \( x \not\in \FV(M) \)}
    \end{align}
  \end{description}
\end{defn}

La regla \eqref{extensionalidad:zeta} dice, de manera informal, que si \( M \) y \( N \) tienen el mismo efecto sobre un objeto no especificado \( x \), entonces \( M = N \). La regla \eqref{extensionalidad:ext} tiene una infinidad de premisas, una por cada término lambda \( P \), por lo tanto, las deducciones en donde se involucre esta regla serán árboles infinitos.

\paragraph{Pendiente} Comparación entre teorías formales, énfasis en \( η \) contra \( ζ \) contra \( ext \).

\section{Teoría de reducción}
\label{sec:teoriareduccion}

1.14 y 2.9

\subsection{Contracciones}
\label{sec:contracciones}

Transformaciones de términos con un paso.

\begin{defn}[Contracciones Hindley y Seldin]
  \label{defn:contraccion}
  Dado un término lambda \( X \), una \emph{contracción} en \( X \) es una tripleta \( \langle X,R,Y \rangle \), denotada \( X \contract{R} Y \), donde \( R \) es una aparición de un \emph{redex} en \( X \) y \( Y \) es el resultado de contraer \( R \) en \( X \).
\end{defn}

\begin{exmp}
  \begin{align*}
    (λx.(λy.y\, x)z)v &\contract{(λx.(λy.y\, x)z)v} (λy.y\, v)z ,\\
    (λx.(λy.y\, x)z)v &\contract{(λy.y\, x)z} (λx.z\, x)v.
  \end{align*}
\end{exmp}

\subsection{Reducciones}
\label{sec:reducciones}

Reducciones basadas en contracciones, de Barendregt

\begin{defn}[Relación compatible]
  \label{defn:compatible}
  Si dice que una relación binaria \( \bs{R} \) sobre \( Λ \) es:
  \begin{enumerate}
  \item Una \emph{relación compatible} cuando
    \[ (M,M') \in \bs{R} \implies (Z\, M,Z\, M') \in \bs{R},\ (M\, Z,M'\, Z) \in \bs{R},\ (λx.M,λx.M') \in \bs{R} \]
    para toda \( M, M', Z \in Λ \).
  \item Una \emph{relación de congruencia} cuando \( \bs{R} \) es compatible, reflexiva, transitiva y simétrica.
  \item Una \emph{relación de reducción} cuando \( \bs{R} \) es compatible, reflexiva y transitiva.
  \end{enumerate}
  \emph{compatible} cuando

\end{defn}

\begin{note}
  Una relación \( \bs{R} \subseteq Λ^{2} \) es compatible cuando

  \[ (M,M') \in \bs{R} \implies (C[M],C[M']) \in \bs{R} \]

  para toda \( M, M' \in Λ \) y todo contexto \( C[\quad] \), con un hoyo.
\end{note}

\begin{defn}
  \label{defn:nocion-reduccion}
  Una \emph{noción de reducción} en \( Λ \) es una relación binaria \( \bs{R} \) en \( Λ \).
\end{defn}

Sean \( \bs{R}_{1} \) y \( \bs{R}_{2} \) nociones de reducción, la relación \( \bs{R}_{1} \cup \bs{R}_{2} \) se denota \( \bs{R}_{1}\bs{R}_{2} \).


\begin{defn}[Reducción \( \bs{β} \)]
  La regla \eqref{teorialambda:beta} en la teoría \( \bs{λ} \) se puede definir como la reducción:
  
  \[ \bs{β} = \{ ((λx.M)N,M[x:=N]) : M, N \in Λ \} \]
\end{defn}

\begin{defn}
  Sea \( \bs{R} \) una noción de reducción en \( Λ \), \( \bs{R} \) introduce las relaciones binarias:

  \begin{itemize}
  \item R-reducción en un paso, denotada \( \contract{R} \) y definida de manera inductiva como:
    \begin{align*}
      \text{(1)} && (M,N) \in \bs{R} &\implies M \contract{R} N \\
      \text{(2)} && M \contract{R} N &\implies Z\, M \contract{R} Z\, N \\
      \text{(3)} && M \contract{R} N &\implies M\, Z \contract{R} N\, Z \\ 
      \text{(4)} && M \contract{R} N &\implies λx.M \contract{R} λx.N
    \end{align*}
  \item R-reducción, denotada \( \reduce{R} \) y definida de manera inductiva como:
    \begin{align*}
      \text{(1)} && M \contract{R} N &\implies M \reduce{R} N \\
      \text{(2)} && M \reduce{R} M \\
      \text{(3)} && M \reduce{R} N,\ N \reduce{R} L &\implies M \reduce{R} L
    \end{align*}
  \item R-convertibilidad, denotada \( \convertible{R} \) y definida de manera inductiva como:
    \begin{align*}
      \text{(1)} && M \reduce{R} N &\implies M \convertible{R} N \\
      \text{(2)} && M \convertible{R} N &\implies N \convertible{R} M \\
      \text{(3)} && M \convertible{R} N,\ N \convertible{R} L &\implies M \convertible{R} L
    \end{align*}
  \end{itemize}
\end{defn}

\begin{lem}
  Las relaciones \( \contract{R} \), \( \reduce{R} \) y \( \convertible{R} \) son compatibles. Por lo tanto \( \reduce{R} \) es una relación de reducción y \( \convertible{R} \) es una relación de congruencia.
\end{lem}

\begin{proof}
  
\end{proof}

Usualmente una noción de reducción se introduce de la siguiente manera: ``Sea \( \bs{R} \) definida por las siguientes \emph{reglas de contracción} \ \( \bs{R} \colon M \contract{} N \text{ dado que } ... \)''.

Esto significa que \( \bs{R} = \{ (M,N) : ... \} \), por ejemplo, \( \bs{β} \) pudo haber sido introducida por la siguiente regla de contracción

\[ \bs{β} : (λx.M)N \contract{} M[x:=N] \]

\begin{defn}
  \begin{enumerate}
  \item Un \( R \)-redex es un término \( M \) tal que \( (M,N) \in R \) para algún término \( N \). En este caso \( N \) es llamado un \( R \)-contractum de \( M \).
  \item Un término \( M \) es llamado una forma normal de \( R \), denotado \( R-fn \), si \( M \) no contiene algún \( R \)-redex.
  \item Un término \( N \) es una \( R-fn \) de \( M \) (o \( M \) tiene la \( R-fn \) \( N \)) si \( N \) es una \( R-fn \) y \( M \convertible{R} N \).
  \end{enumerate}
\end{defn}

El proceso de pasar de un redex a su contractum es llamado \emph{contracción}. En lugar de escribir ``\( M \) es una \( R-fn \)'' usualmente se escribe ``\( M \) está en \( R-fn \)'', pensando en una máquina que ha llegado a su estado final.

\begin{exmp}
  \( (λx.x\, x)λy.y \) es un \( β \)-redex. Por lo tanto \( (λx.x\, x)(λy.y)z \) no está en \( β-fn \); sin embargo este término tiene la \( β-fn \) \( z \).
\end{exmp}

Plantear las ideas para pasar del concepto de contracción al de reducción y posteriormente al de convertibilidad.

Gráficas de reducción.

Teorema de Church-Rosser y toda la magia necesaria para abordarlo y la magia que valga la pena mencionar en la que CR es importante va aquí.

Probablemente es buena idea separar la sección ``Noción informal del cálculo lambda'' y ``Formalización del cálculo lambda'' en dos capítulos diferentes. ¿Valdrá la pena abordar árboles de Böhm?


%%% Local Variables:
%%% mode: latex
%%% TeX-master: "main"
%%% End:


\chapter{Codificación de objetos}
\label{ch:codificacion}
Alonzo Church y Alan Turing hipotetizaron en 1936 de manera independiente que el cálculo $ λ $ y las máquinas de Turing formalizaban el concepto de cómputo \cite{Church:Unsolvable,Turing:Computable}. Esto puede parecer extraño ya que claramente los programas de computadora pueden realizar cómputos complejos que involucran números, texto, árboles, gráficas o conjuntos mientras que en el cálculo $ λ $ se está limitado a átomos, abstracciones y aplicaciones.

En este capítulo se presenta un tratamiento computacional del cálculo $ λ $ con el objetivo de explorar el tipo de conceptos matemáticos y algorítmicos que se pueden representar en él.

En la primera sección se plantean codificaciones para valores de verdad y operaciones booleanas, con esto se construye en el cálculo una operación similar a la sentencia condicional de los lenguajes de programación; en la segunda sección se exploran los números naturales y las operaciones aritméticas elementales, con esto se plantea un mecanismo de iteración; en la tercera sección se deriva un mecanismo de recursividad que permitirá representar algoritmos recursivos de manera sencilla; en la cuarta sección se presentan codificaciones de las listas, árboles y gráficas, finalmente se presenta la codificación de términos $ λ $.

Este capítulo está fuertemente influenciado por la serie de reportes técnicos llamados en la comunidad de lenguajes de programación como ``The Lambda Papers'' por Sussman y Steele \cite{Scheme:first,Steele:Imperative,Steele:Declarative,Steele:LambdaGOTO,Steele:Opcode}.

\section{Álgebra Booleana}
\label{sec:algebra-booleana}

El álgebra booleana es una rama del álgebra en donde las expresiones tienen asociado un valor de \emph{falso} o \emph{verdadero}. Estas expresiones son fundamentales en el estudio de circuitos y programas escritos en lenguajes de programación.

Los términos $ λ $ no tienen asignados un valor de verdad y las operaciones que se plantearon en los primeros dos capítulos involucraron el concepto de falso y verdadero únicamente en el metalenguaje y asociando estos valores no a los términos $ λ $ en sí, si no a propiedades de estos, por ejemplo, es falso que $ \| λx.x \| = 5 $ y es verdadero que $ (\bs{K}\, x) \reduce{β} (λx.y) $. Sin embargo es posible codificar los valores de verdad como elementos de $ Λ $ y construir abstracciones que emulen las propiedades de las operaciones booleanas bajo la $ β $-reducción. De esta manera se pueden escribir términos que, de acuerdo con la codificación establecida, representen expresiones booleanas y términos $ λ $ al mismo tiempo.

En los lenguajes de programación usualmente se mezclan las expresiones booleanas con otras expresiones y objetos a partir de \emph{predicados}, éstos son funciones con algún dominio $ X $ y codominio $ \{ \mathrm{falso},\ \mathrm{verdadero} \} $. Por ejemplo, al escribir un programa en donde se necesite tomar una desición a partir de si un número $ n $ es positivo o negativo se escribiría (en pseudocódigo):

\begin{algorithmic}
  \IF{esPositivo($ n $)}
  \STATE $ ... $
  \ELSE
  \STATE $ ... $
  \ENDIF
\end{algorithmic}

En este ejemplo \texttt{esPositivo} es un predicado que es evaluado a falso si $ n $ no es positivo y a verdadero si lo es.

La codificación de valores de verdad y operaciones booleanas es común incluso en lenguajes de programación populares, por ejemplo en C, el tipo \texttt{bool} es codificado como un entero, en donde falso es 0 y verdadero cualquier otro entero, a su vez, los enteros son codificados usualmente como secuencias de 32 bits en complemento a dos. Por lo tanto, si <<esPositivo>> fuera una función de C: \texttt{esPositivo(8)} sería evaluado a 1 y \texttt{esPositivo(-8)} sería evaluado a 0.

Al igual que el cálculo $ λ $, otras teorías que fundamentan las ciencias de la computación también carecen de expresiones y operaciones booleanas. En el caso de la máquina de Turing los cambios de estado en la ejecución de un programa se determinan a partir de su función de transición y predicados simples de igualdad entre símbolos del alfabeto de cinta se realizan en un paso, sin embargo, predicados mas complejos requieren ser codificados con estados, transiciones y anotaciones en su cinta.

\subsection{Valores de verdad}
\label{sec:valores-de-verdad}

En el álgebra booleana, los valores de las expresiones son falso y verdadero. El nombre de estos valores no es de relevancia y usualmente falso se representa como 0 y verdadero como 1. El aspecto importante de estos valores es que son distintos y si un valor $ x $ no es uno, entonces es el otro.

Podemos ignorar la representación concreta de estos valores y pensar en una situación hipotética: Una persona omnisciente y muda llamada $ P $ puede decirme si una oración que le digo es falsa o verdadera dándole una manzana y una pera; si me regresa la manzana significa que la oración es verdadera y si me regresa la pera significa que la oración es falsa. En este planteamiento irreal e hipotético, no fué necesario conocer la estructura de la verdad y la falsedad, solo fué necesario tener a alguien que tomara una desición (en este caso $ P $) y proveer dos objetos que podemos distinguir entre sí (en este caso la manzana y la pera). Las desiciones de esta persona pueden ser los conceptos de falso y verdadero si nunca podemos conocer los valores booleanos.

Detrás del concepto de falso y verdadero, está el concepto de \emph{desición}, la codificación que se desarrolla está basada en este concepto y aparece en \cite[p.~133]{Barendregt:Bible}.

Supongamos que $ P $ es un término $ λ $ el cual puede ser aplicado a una oración $ O $, al $ β $-reducir $ (P\, O) $ se obtiene una decisión $ D $ la cual al ser aplicada a dos términos $ λ $ $ M $ y $ N $ se $ β $-reduce a $ M $ si la oración $ O $ es verdadera y a $ M $ si es falsa:
\[ P\, O \reduce{β} D, \]
\[ D\, M\, N \reduce{β} \begin{cases} M & \text{si $ O $ es verdadera}\\ N & \text{si $ O $ es falsa}\end{cases} .\]
Para fines prácticos no es necesario saber cómo es $ P $ ni $ O $, lo importante es que cuando $ O $ es cierta, $ D $ eligirá $ M $ y si $ O $ es falsa, eligirá $ N $. Por lo tanto, $ (P\, O) = D $ es un término $ λ $ de la forma
\[ λx\, y.Q \]
Si $ D $ es una desición tomada por que $ O $ es verdadera, podemos asegurar que $ (D\, M\, N) = M $, por lo tanto:
\[ D \synteq λx\, y.x \]
Si $ D $ es una desición tomada por que $ O $ es falsa, podemos asegurar que $ (D\, M\, N) = N $, por lo tanto:
\[ D \synteq λx\, y.y \]
Teniendo los términos $ λ $ que representan la desición de $ P $ ante una oración falsa y ante una oración verdadera, se puede considerar que estos términos representan el concepto de falso y verdadero.

\begin{defn}[Valores de verdad]
  \label{defn:valores-verdad}
  El concepto de falso y verdadero es codificado en el cálculo $ λ $ como los términos $ \bs{T} $ y $ \bs{F} $ respectivamente.
  \begin{align*}
    \bs{T} &\synteq λx\, y.x & \bs{F} &\synteq λx\, y.y
  \end{align*}
\end{defn}

Utilizar $ \bs{T} $ y $ \bs{F} $ en términos $ λ $ es similar a imitar a $ P $ y determinar cuando $ O $ es verdadera o falsa. Esto es debido a que se pueden plantear predicados que sean conceptualmente ilógicos, por ejemplo, si <<esPositivo>> se define de tal manera que sin importar en que valor sea evaluado siempre resulte en falso, los programas que se escriban no van a funcionar suponiendo que <<esPositivo>> calcula lo que debe de calcular, sin embargo lo importante de codificar el álgebra booleana es poder manipular los valores de falso y verdadero, no representar un término $ P $ que determine verdades absolutas.

\subsection{Expresiones booleanas}
\label{sec:expresiones-booleanas}

Las expresiones booleanas se conforman de operaciones y valores de verdad. Las operaciones más básicas son la conjunción, la disyunción y la negación, también llamadas $ AND $, $ OR $, $ NOT $ y denotadas $ \land $, $ \lor $ y $ \lnot $ respectivamente.

La conjunción y la disyunción son operaciones binarias definidas en

\[ \{ \mathrm{falso},\ \mathrm{verdadero} \}^{2} \to \{ \mathrm{falso},\ \mathrm{verdadero} \} \]

y la negación es una operación unaria definida en

\[ \{ \mathrm{falso},\ \mathrm{verdadero} \} \to \{ \mathrm{falso},\ \mathrm{verdadero} \}. \]

Las tablas de verdad en la \autoref{tab:and-or-not} establecen los resultados de estas tres operaciones para cada valor en su dominio.

\begin{table}[!htbp]
  \centering
  \small
  \begin{tabular}{|c|c|c|c|}
    \hline
    $ x $ & $ y $ & $ x \land y $ & $ x \lor y $ \\ [0.5ex]
    \hline\hline
    falso & falso & falso & falso \\
    falso & verdadero & falso & verdadero \\
    verdadero & falso & falso & verdadero \\
    verdadero & verdadero & verdadero & verdadero \\
    \hline
  \end{tabular}
  \hfill
  \begin{tabular}{|c|c|}
    \hline
    $ x $ & $ \lnot x $ \\ [0.5ex]
    \hline\hline
    falso & verdadero  \\
    verdadero & falso \\
    \hline
  \end{tabular}
  \caption{Tablas de verdad para $ \land $, $ \lor $ y $ \lnot $}
  \label{tab:and-or-not}
\end{table}

En el álgebra booleana, las expresiones se escriben en notación de infijo, utilizan paréntesis para agrupar expresiones y cuando los paréntesis son omitidos la negación tiene mayor presedencia que la conjunción y la conjunción tiene mayor presedencia que la disyunción, por ejemplo:

\[ \mathrm{verdadero} \land \mathrm{falso} \lor \lnot \mathrm{falso} \]
\[ \lnot (\mathrm{falso} \lor \mathrm{falso}) \]
\[ \mathrm{verdadero} \land (\mathrm{falso} \lor \mathrm{falso}) \]

Esta notación es conveniente para escribir expresiones booleanas de manera concisa, pero es únicamente una conveniencia sintáctica del álgebra booleana. La codificación que se desarrolla de las operaciones seguirá las convenciones sintácticas del cálculo $ λ $, por ejemplo, suponiendo que $ \bs{\land} $, $ \bs{\lor} $, $ \bs{\lnot} $ son términos $ λ $, las expresiones mencionadas escribirían con notación de prefijo:
\[ \bs{\lor} (\bs{\land}\, \bs{T}\, \bs{F}) \bs{F} \]
\[ \bs{\lnot} (\bs{\lor}\, \bs{F}\, \bs{F}) \]
\[ \bs{\land}\, \bs{T} (\bs{\lor}\, \bs{F}\, \bs{F}) \]

Al igual que los valores de verdad, las operaciones básicas son codificadas como abstracciones del cálculo $ λ $. Hay varias metodologías para derivar términos $ λ $ para las operaciones booleanas a partir de $ \bs{T} $ y $ \bs{F} $, en esta sección se abordarán dos:
\begin{itemize}
\item Combinar valores de verdad
\item Programar las operaciones
\end{itemize}
La primera metodología parte de la observación de que la codificación de falso y verdadero son abstracciones, por lo tanto, es posible $ β $-reducirlas al aplicarlas a otros términos; se explora la clase de términos $ λ $ en $ \{ \bs{T},\ \bs{F} \}^{+} $.

La segunda metodología presenta la construcción del operador condicional, a partir del cual se derivan las operaciones booleanas como si fueran programas de computadora.

\subsubsection{Combinaciones de valores de verdad}
\label{sec:combinacion-valores}

Una manera de obtener términos $ λ $ a partir de $ \bs{F} $ y $ \bs{T} $ es $ β $-reducir combinaciones de aplicaciones entre estos valores. En la \autoref{tab:verdad-pares} se muestran los términos obtenidos al reducir combinaciones de dos valores de verdad.

\begin{table}[!htbp]
  \centering
  \begin{tabular}{|c||l|}
    \hline
    $ \bs{F}\, \bs{F} $ & $ (λx\, y.y)\, \bs{F} \reduce{β} λy.y \synteq \bs{I} $ \\
    \hline
    $ \bs{F}\, \bs{T} $ & $ (λx\, y.y)\, \bs{T} \reduce{β} λy.y \synteq \bs{I} $ \\
    \hline
    $ \bs{T}\, \bs{F} $ & $ (λx\, y.x)\, \bs{F} \reduce{β} λy.\bs{F} \synteq \bs{K}\, \bs{F} $ \\
    \hline
    $ \bs{T}\, \bs{T} $ & $ (λx\, y.x)\, \bs{T} \reduce{β} λy.\bs{T} \synteq \bs{K}\, \bs{T} $ \\
    \hline
  \end{tabular}
  \caption{Posibles combinaciones de valores de verdad por pares.}
  \label{tab:verdad-pares}
\end{table}

En las reducciones de la \autoref{tab:verdad-pares} se pueden observar cuatro términos, a partir de estos se puede descubrir la operación de negación:

\begin{itemize}
\item $ (\bs{F}\, \bs{F}) $ se reduce a la abstracción identidad, esto significa que para cualquier término $ M \in Λ $
  \[ \bs{λ} \vdash (\bs{F}\, \bs{F}\, M) = M \]
\item Al igual que la primera reducción $ (\bs{F}\, \bs{T}) $ se reduce a $ \bs{I} $, por lo tanto se concluye que para cualesquiera términos $ M \in Λ $, $ N \in \{ \bs{F},\ \bs{T} \} $
  \[ \bs{λ} \vdash (\bs{F}\, N\, M) = M \]
\item $ (\bs{T}\, \bs{F}) $ se reduce a la abstracción constante de $ \bs{F} $, esto significa que para cualquier término $ M \in Λ $
  \[ \bs{λ} \vdash (\bs{T}\, \bs{F}\, M) = \bs{F} \]
\item Al igual que la tercera reducción $ (\bs{T}\, \bs{T}) $ se reduce a $ (\bs{K}\, \bs{T}) $, por lo tanto se concluye que para cualesquiera términos $ M \in Λ $, $ N \in \{ \bs{F},\ \bs{T} \} $
  \[ \bs{λ} \vdash (\bs{T}\, N\, M) = N \]
\end{itemize}

Debido a las reducciones mostradas en la \autoref{tab:verdad-pares} se puede analizar que a partir de un témino $ \bs{F} $, se puede obtener $ \bs{T} $ al reducir $ (\bs{F}\, N\, \bs{T}) $ y que a partir de un término $ \bs{T} $, se puede obtener $ \bs{F} $ al reducir $ (\bs{T}\, \bs{F}\, M) $. Considerando que $ N \synteq \bs{F} $ y $ M \synteq \bs{T} $ las reducciones serían:
\[ \bs{F}\, \bs{F}\, \bs{T} \reduce{β} \bs{T} \]
\[ \bs{T}\, \bs{F}\, \bs{T} \reduce{β} \bs{F} \]

Si se considera que $ P \in \{ \bs{F},\ \bs{T} \} $
\[ P\, \bs{F}\, \bs{T} \reduce{β} \bs{\lnot}\, P \]

\begin{rem}[Sobre la $ β $-reducción]
  En el tratamiento de la codificación del álgebra booleana en el cálculo $ λ $, cuando se $ β $-reducen términos $ λ $ que tienen como subtérminos valores que suponemos son $ \bs{F} $ o $ \bs{T} $ se extiende la teoría $ \bs{λ} $ con la siguiente ecuación:
  \begin{align*}
    P\, \bs{T}\, \bs{F} &= P && \text{si $ P \in \{ \bs{F},\ \bs{T} \} $}
  \end{align*}
\end{rem}
\begin{defn}[Operación de negación]
  \label{defn:negacion}
  El término $ λ $ $ \bs{\lnot} \synteq (λp.p\, \bs{F}\, \bs{T}) $ se reduce a $ \bs{T} $ cuando es aplicado a $ \bs{F} $ y viceversa
  \begin{align*}
    \bs{\lnot}\, \bs{F} &\synteq (λp.p\, \bs{F}\, \bs{T})\, \bs{F} \\
                        &\contract{β} \bs{F}\, \bs{F}\, \bs{T} \\
                        &\reduce{β} \bs{T} \\
    \bs{\lnot}\, \bs{T} &\synteq (λp.p\, \bs{F}\, \bs{T})\, \bs{T} \\
                        &\contract{β} \bs{T}\, \bs{F}\, \bs{T} \\
                        &\reduce{β} \bs{F}
  \end{align*}
\end{defn}

Las reducciones de la \autoref{tab:verdad-pares} se pueden aplicar a $ \bs{F} $ y $ \bs{T} $ para obtener todas las posibles combinaciones de aplicaciones de valores de verdad de la forma $ ((P\, M) N) $, en la \autoref{tab:verdad-tripletas} se muestran las reducciones de las nuevas aplicaciones.

\begin{table}[!htbp]
  \centering
  \begin{tabular}{|c||l|}
    \hline
    $ \bs{F}\, \bs{F}\, \bs{F} $ & $ \bs{I}\, \bs{F} \reduce{β} \bs{F} $ \\
    \hline
    $ \bs{F}\, \bs{F}\, \bs{T} $ & $ \bs{I}\, \bs{T} \reduce{β} \bs{T} $ \\
    \hline
    $ \bs{F}\, \bs{T}\, \bs{F} $ & $ \bs{I}\, \bs{F} \reduce{β} \bs{F} $ \\
    \hline
    $ \bs{F}\, \bs{T}\, \bs{T} $ & $ \bs{I}\, \bs{T} \reduce{β} \bs{T} $ \\
    \hline
    $ \bs{T}\, \bs{F}\, \bs{F} $ & $ \bs{K}\, \bs{F}\, \bs{F} \reduce{β} \bs{F} $ \\
    \hline
    $ \bs{T}\, \bs{F}\, \bs{T} $ & $ \bs{K}\, \bs{F}\, \bs{T} \reduce{β} \bs{F} $ \\
    \hline
    $ \bs{T}\, \bs{T}\, \bs{F} $ & $ \bs{K}\, \bs{T}\, \bs{F} \reduce{β} \bs{T} $ \\
    \hline
    $ \bs{T}\, \bs{T}\, \bs{T} $ & $ \bs{K}\, \bs{T}\, \bs{T} \reduce{β} \bs{T} $ \\
    \hline
  \end{tabular}
  \caption{Posibles combinaciones de valores de verdad con asociación a la izquierda.}
  \label{tab:verdad-tripletas}
\end{table}

Al observar la \autoref{tab:verdad-tripletas}, se distinguen algúnos patrones en los resultados de las reducciones, por ejemplo, si $ P $ es un valor de verdad cualquiera, $ (P\, \bs{F}\, \bs{F}) $ se reduce a $ \bs{F} $ y $ (P\, \bs{T}\, \bs{T}) $ se reduce a $ \bs{T} $, las combinaciones mas interesantes se presentan en los renglones 2, 3, 6 y 7.

En búsqueda de las operaciones binarias de conjunción y disyunción se desarrollan tablas de verdad con las posibles combinaciones de dos términos $ P, Q \in \{ \bs{F},\ \bs{T} \} $. La cantidad de combinaciones de estos valores es $ 2 \times \binom 3 2 = 2 \times \frac{3!}{2!} = 6 $ y son
\[ (P\, Q\, \bs{F}) ,\ (P\, Q\, \bs{T}) ,\ (P\, \bs{F}\, Q) ,\ (P\, \bs{T}\, Q) ,\ (\bs{F}\, P\, Q) ,\ (\bs{T}\, P\, Q) \]
Las tablas de verdad de estas combinaciones intercambiando las posiciones de $ P $ y $ Q $ serían las mismas ya que ambos términos toman los valores de falso y verdadero en las tablas de verdad. En la \autoref{tab:verdad-pq} se muestran estas tablas.


\begin{table}[!htbp]
  \centering
  \begin{tabular}{|c|c||c|c|c|c|c|c|}
    \hline
    $ P $ & $ Q $ & $ P\, Q\, \bs{F} $ & $ P\, Q\, \bs{T} $ & $ P\, \bs{F}\, Q $ & $ P\, \bs{T}\, Q $ & $ \bs{F}\, P\, Q $ & $ \bs{T}\, P\, Q $ \\ [0.5ex]
    \hline
    \hline
    $ \bs{F} $ & $ \bs{F} $ & $ \bs{F} $ & $ \bs{T} $ & $ \bs{F} $ & $ \bs{F} $ & $ \bs{F} $ & $ \bs{F} $ \\
    $ \bs{F} $ & $ \bs{T} $ & $ \bs{F} $ & $ \bs{T} $ & $ \bs{T} $ & $ \bs{T} $ & $ \bs{T} $ & $ \bs{F} $ \\
    $ \bs{T} $ & $ \bs{F} $ & $ \bs{F} $ & $ \bs{F} $ & $ \bs{F} $ & $ \bs{T} $ & $ \bs{F} $ & $ \bs{T} $ \\
    $ \bs{T} $ & $ \bs{T} $ & $ \bs{T} $ & $ \bs{T} $ & $ \bs{F} $ & $ \bs{T} $ & $ \bs{T} $ & $ \bs{T} $ \\
    \hline
  \end{tabular}
  \caption{Tablas de verdad considerando dos variables $ P $ y $ Q $ en aplicaciones de tres términos}
  \label{tab:verdad-pq}
\end{table}

Las columnas de la combinación $ (P\, Q\, \bs{F}) $ y $ (P\, \bs{T}\, Q) $ de la \autoref{tab:verdad-pq} corresponden a la operación de conjunción y disyunción respectivamente, como aparecen en la \autoref{tab:and-or-not}. Las otras combinaciones corresponden a operaciones no básicas del álgebra booleana: $ (P\, Q\, \bs{T}) $ es la implicación material; $ (P\, \bs{F}\, Q) $ es la no implicación inversa; $ (\bs{F}\, P\, Q) $ es la proyección de $ Q $; y $ (\bs{T}\, P\, Q) $ es la proyección de $ P $.

Ya que  $ \bs{λ} \vdash (P\, Q\, \bs{F}) = (\bs{\land}\, P\, Q) $ y $ \bs{λ} \vdash (P\, \bs{T}\, Q) = (\bs{\lor}\, P\, Q) $, se construyen los términos $ \bs{\land} $ y $ \bs{\lor} $ abstrayendo a $ P $ y $ Q $ de las igualdades.

\begin{defn}[Operación de conjunción]
  \label{defn:conjuncion}
  El término $ λ $ que representa la conjunción es
  \[ \bs{\land} \synteq λp\, q.p\, q\, \bs{F} \]
  y cumple las siguientes propiedades de $ β $-reducción al ser aplicada a valores de verdad:
  \begin{align*}
    \bs{\land}\, \bs{F}\, \bs{F} &\synteq (λp\, q.p\, q\, \bs{F})\, \bs{F}\, \bs{F} \reduce{β} \bs{F}\, \bs{F}\, \bs{F} \reduce{β} \bs{F} \\
    \bs{\land}\, \bs{F}\, \bs{T} &\synteq (λp\, q.p\, q\, \bs{F})\, \bs{F}\, \bs{T} \reduce{β} \bs{F}\, \bs{T}\, \bs{F} \reduce{β} \bs{F} \\
    \bs{\land}\, \bs{T}\, \bs{F} &\synteq (λp\, q.p\, q\, \bs{F})\, \bs{T}\, \bs{F} \reduce{β} \bs{T}\, \bs{F}\, \bs{F} \reduce{β} \bs{F} \\
    \bs{\land}\, \bs{T}\, \bs{T} &\synteq (λp\, q.p\, q\, \bs{F})\, \bs{T}\, \bs{T} \reduce{β} \bs{T}\, \bs{T}\, \bs{F} \reduce{β} \bs{T}
  \end{align*}
\end{defn}

\begin{defn}[Operación de disyunción]
  \label{defn:disyuncion}
  El término $ λ $ que representa la disyunción es
  \[ \bs{\lor} \synteq λp\, q.p\, \bs{T}\, q \]
  y cumple las siguientes propiedades de $ β $-reducción al ser aplicada a valores de verdad:
  \begin{align*}
    \bs{\lor}\, \bs{F}\, \bs{F} &\synteq (λp\, q.p\, \bs{T}\, q) \bs{F}\, \bs{F} \reduce{β} \bs{F}\, \bs{T}\, \bs{F} \reduce{β} \bs{F} \\
    \bs{\lor}\, \bs{F}\, \bs{T} &\synteq (λp\, q.p\, \bs{T}\, q) \bs{F}\, \bs{T} \reduce{β} \bs{F}\, \bs{T}\, \bs{T} \reduce{β} \bs{T} \\
    \bs{\lor}\, \bs{T}\, \bs{F} &\synteq (λp\, q.p\, \bs{T}\, q) \bs{T}\, \bs{F} \reduce{β} \bs{T}\, \bs{T}\, \bs{F} \reduce{β} \bs{T} \\
    \bs{\lor}\, \bs{T}\, \bs{T} &\synteq (λp\, q.p\, \bs{T}\, q) \bs{T}\, \bs{T} \reduce{β} \bs{T}\, \bs{T}\, \bs{T} \reduce{β} \bs{T}
  \end{align*}
\end{defn}

Esta metodología para encontrar operaciones del álgebra booleana, aplicando los términos codificados de los valores de verdad, es tediosa pero hasta cierto grado efectiva. Como en los casos de las operaciones no básicas mostradas en la \autoref{tab:verdad-pq}, operaciones del álgebra booleana pueden ser ``descubiertas'' y no construidas. Ya que la negación, la conjunción y la disyunción fueron descubiertas con este método, cualquier operación booleana eventualmente será encontrada como combinación de valores de verdad. Sin embargo, descubrir la codificación de una operación booleana complicada utilizando este método es un proceso muy tardado.

\subsubsection{Programación de operaciones booleanas}
\label{sec:programacion-operaciones}

Otra metodología que permite construir las operaciones booleanas como términos $ λ $ es la de partir de un algoritmo que las describa. Usualmente las operaciones booleanas no son definidas como procedimientos, si no como operaciones primitivas del lenguaje utilizado para describirlos.

Consideremos dos términos $ M $ y $ N $. Ya que $ (\bs{T}\, M\, N) \reduce{β} M $ y $ (\bs{F}\, M\, N) \reduce{β} N $, si $ M \reduce{β} M' $ y $ N \reduce{β} N' $, entonces
\[ \bs{T}\, M\, N \reduce{β} M' \]
\[ \bs{F}\, M\, N \reduce{β} N' \]

Es decir, si $ P \in \{ \bs{F},\ \bs{T} \} $:
\[ P\, M\, N \reduce{β} \begin{cases} M' & P \synteq \bs{T} \\ N' & P \synteq \bs{F} \end{cases} \]

Esta aplicación de un valor de verdad a dos términos $ λ $ cualquiera permite capturar el concepto de una expresión o sentencia condicional, usualmente llamada en los lenguajes de programación como sentencia \texttt{if-then-else}.

\begin{defn}[Expresión condicional]
  \label{defn:condicional}
  El término $ λ $ que representa a la expresión condicional es
  \[ \bs{\prec} \synteq λp\, m\, n.p\, m\, n \]
  Y si $ P $ es un valor de verdad, entonces
  \begin{align*}
    \bs{\prec}\, P\, M\, N &\synteq (λp\, m\, n.p\, m\, n)\, P\, M\, N \\
                           &\reduce{β} P\, M\, N
  \end{align*}
  Un programa de la forma
  \begin{algorithmic}
    \IF{$ P $}
    \STATE $ M $
    \ELSE
    \STATE $ N $
    \ENDIF
  \end{algorithmic}
  Puede ser traducido a $ (\bs{\prec}\, P\, M\, N) $
\end{defn}

Consideremos la siguiente definición en pseudocódigo de la operación de negación:
\begin{algorithm}
  \caption{Negación de $ p $}
  \label{alg:negacion}
  \begin{algorithmic}
    \REQUIRE $ p \in \{ \mathrm{falso},\ \mathrm{verdadero} \} $
    \ENSURE $ \lnot p $
    \IF{$ p $}
    \RETURN falso
    \ELSE
    \RETURN verdadero
    \ENDIF
  \end{algorithmic}
\end{algorithm}
El pseudocódigo se traduce al cálculo $ λ $ como
\[ \bs{\lnot} \synteq λp.\bs{\prec}\, p\, \bs{F}\, \bs{T} \]
El cuerpo de la abstracción puede ser $ β $-reducido para obtener el término de la \autoref{defn:negacion}
\begin{align*}
  λp.\bs{\prec}\, p\, \bs{F}\, \bs{T} &\synteq λp.(λp\, m\, n.p\, m\, n)\, p\, \bs{F}\, \bs{T} \\
                                      &\reduce{β} λp.p\, \bs{F}\, \bs{T}
\end{align*}

Para la operación de conjunción, se considera el siguiente pseudocódigo:
\begin{algorithm}
  \caption{Conjunción de $ p_{1} $ y $ p_{2} $}
  \label{alg:conjuncion}
  \begin{algorithmic}
    \REQUIRE $ p_{1}, p_{2} \in \{ \mathrm{falso},\ \mathrm{verdadero} \} $
    \ENSURE $ p_{1} \land p_{2} $
    \IF{$ p_{1} $}
    \IF{$ p_{2} $}
    \RETURN verdadero
    \ELSE
    \RETURN false
    \ENDIF
    \ELSE
    \RETURN falso
    \ENDIF
  \end{algorithmic}
\end{algorithm}

Traducido al cálculo $ λ $ como
\[ \bs{\land} \synteq λp_{1}\, p_{2}.\bs{\prec}\, p_{1}\, (\bs{\prec}\, p_{2}\, \bs{T}\, \bs{F})\, \bs{F} \]
Al $ β $-reducir el cuerpo de la abstracción se obtiene el término de la \autoref{defn:conjuncion}
\begin{align*}
  λp_{1}\, p_{2}.\bs{\prec}\, p_{1}\, (\bs{\prec}\, p_{2}\, \bs{T}\, \bs{F})\, \bs{F}
  &\synteq λp_{1}\, p_{2}.(λp\, m\, n.p\, m\, n)\, p_{1}\, ((λp\, m\, n.p\, m\, n)\, p_{2}\, \bs{T}\, \bs{F})\, \bs{F} \\
  &\reduce{β} λp_{1}\, p_{2}.p_{1}\, ((λp\, m\, n.p\, m\, n)\, p_{2}\, \bs{T}\, \bs{F})\, \bs{F} \\
  &\reduce{β} λp_{1}\, p_{2}.p_{1}\, (p_{2}\, \bs{T}\, \bs{F})\, \bs{F} \\
  &=_{\bs{λ}} λp_{1}\, p_{2}.p_{1}\, p_{2}\, \bs{F}
\end{align*}

De igual manera, considerando el siguiente pseudocódigo de la operación de disyunción:
\begin{algorithm}
  \caption{Disyunción de $ p_{1} $ y $ p_{2} $}
  \label{alg:disyuncion}
  \begin{algorithmic}
    \REQUIRE $ p_{1}, p_{2} \in \{ \mathrm{falso},\ \mathrm{verdadero} \} $
    \ENSURE $ p_{1} \lor p_{2} $
    \IF{$ p_{1} $}
    \RETURN verdadero
    \ELSE
    \IF{$ p_{2} $}
    \RETURN verdadero
    \ELSE
    \RETURN falso
    \ENDIF
    \ENDIF
  \end{algorithmic}
\end{algorithm}

Se traduce al cálculo $ λ $ como
\[ \bs{\lor} \synteq λp_{1}\, p_{2}.\bs{\prec}\, p_{1}\, \bs{T}\, (\bs{\prec}\, p_{2}\, \bs{T}\, \bs{F}) \]
Y al $ β $-reducir el cuerpo de la abstracción se obtiene el término de la \autoref{defn:disyuncion}

\begin{align*}
  λp_{1}\, p_{2}.\bs{\prec}\, p_{1}\, \bs{T}\, (\bs{\prec}\, p_{2}\, \bs{T}\, \bs{F})
  &\synteq λp_{1}\, p_{2}.(λp\, m\, n.p\, m\, n)\, p_{1}\, \bs{T}\, ((λp\, m\, n.p\, m\, n)\, p_{2}\, \bs{T}\, \bs{F}) \\
  &\reduce{β} λp_{1}\, p_{2}.p_{1}\, \bs{T}\, ((λp\, m\, n.p\, m\, n)\, p_{2}\, \bs{T}\, \bs{F}) \\
  &\reduce{β} λp_{1}\, p_{2}.p_{1}\, \bs{T}\, (p_{2}\, \bs{T}\, \bs{F}) \\
  &=_{\bs{λ}} λp_{1}\, p_{2}.p_{1}\, \bs{T}\, p_{2}
\end{align*}

Utilizando esta técnica, se puede obtener el término $ λ $ para una operación a partir del pseudocódigo basado en valores de verdad y la sentencia \verb!if-then-else!. Teniendo estos resultados resulta natural, generalizar el pseudocódigo para construir un término $ λ $ que a partir de la tabla de verdad de una operación booleana binaria, resulte en la abstracción que codifica la operación.

\begin{defn}[Traducción de operaciones booleanas binarias]
  \label{defn:op-bool-bin-lambda}
  Sea $ \bs{\odot} $ una operación booleana binaria con la siguiente tabla de verdad

  \begin{center}
    \begin{tabular}{|c|c||c|}
      \hline
      $ P $ & $ Q $ & $ P \bs{\odot} Q $ \\ [0.5ex]
      \hline\hline
      $ \bs{F} $ & $ \bs{F} $ & $ x_{1} $ \\
      \hline
      $ \bs{F} $ & $ \bs{T} $ & $ x_{2} $ \\
      \hline
      $ \bs{T} $ & $ \bs{F} $ & $ x_{3} $ \\
      \hline
      $ \bs{T} $ & $ \bs{T} $ & $ x_{4} $ \\
      \hline
    \end{tabular}
  \end{center}

  El procedimiento generalizado es

  \begin{algorithm}
    \caption{Operación booleana $ \odot $ dado $ x_{1} $, $ x_{2} $, $ x_{3} $, $ x_{3} $}
    \label{alg:bool-bin-gen}
    \begin{algorithmic}
      \REQUIRE $ p_{1}, p_{2} \in \{ \mathrm{falso},\ \mathrm{verdadero} \} $
      \ENSURE Valor $ x_{i} $ de la tabla de verdad
      \IF{$ p_{1} $}
      \IF{$ p_{2} $}
      \RETURN $ x_{4} $
      \ELSE
      \RETURN $ x_{3} $
      \ENDIF
      \ELSE
      \IF{$ p_{2} $}
      \RETURN $ x_{2} $
      \ELSE
      \RETURN $ x_{1} $
      \ENDIF
      \ENDIF
    \end{algorithmic}
  \end{algorithm}

  Y la traducción al cálculo $ λ $ es
  \[ λx_{1}\, x_{2}\, x_{3}\, x_{4}.(λp_{1}\, p_{2}.(\bs{\prec}\, p_{1} (\bs{\prec}\, p_{2}\, x_{4}\, x_{3}) (\bs{\prec}\, p_{2}\, x_{2}\, x_{1}))) \]
  
\end{defn}

Las operaciones booleanas binarias \emph{NAND} y \emph{NOR} conforman los conjuntos unitarios $ \{ \mathrm{NAND} \} $ y $ \{ \mathrm{NOR} \} $ los cuales son conjuntos funcionalmene completos, es decir, únicamente con la operación NAND se puede emular cualquier operación booleana y únicamente con la operación NOR se puede emular cualquier operación booleana.

La operación \emph{NAND} se denota $ P \uparrow Q $ y tiene la siguiente tabla de verdad
\begin{center}
  \begin{tabular}{|c|c||c|}
    \hline
    $ P $ & $ Q $ & $ P \uparrow Q $ \\ [0.5ex]
    \hline\hline
    $ \bs{F} $ & $ \bs{F} $ & $ \bs{T} $ \\
    \hline
    $ \bs{F} $ & $ \bs{T} $ & $ \bs{T} $ \\
    \hline
    $ \bs{T} $ & $ \bs{F} $ & $ \bs{T} $ \\
    \hline
    $ \bs{T} $ & $ \bs{T} $ & $ \bs{F} $ \\
    \hline
  \end{tabular}
\end{center}
Con el proceso de traducción mostrado en la \autoref{defn:op-bool-bin-lambda}, el término $ \bs{\uparrow} $ que codifica la operación NAND sería
\begin{align*}
  \bs{\uparrow}
  &\synteq λp_{1}\, p_{2}. \bs{\prec}\, p_{1}\, (\bs{\prec}\, p_{2}\, \bs{F}\, \bs{T})\, (\bs{\prec}\, p_{2}\, \bs{T}\, \bs{T}) \\
  &\reduce{β} λp_{1}\, p_{2}.p_{1}\, (p_{2}\, \bs{F}\, \bs{T})\, \bs{T}
\end{align*}

La operación \emph{NOR} se denota $ P \downarrow Q $ y tiene la siguiente tabla de verdad
\begin{center}
  \begin{tabular}{|c|c||c|}
    \hline
    $ P $ & $ Q $ & $ P \downarrow Q $ \\ [0.5ex]
    \hline\hline
    $ \bs{F} $ & $ \bs{F} $ & $ \bs{T} $ \\
    \hline
    $ \bs{F} $ & $ \bs{T} $ & $ \bs{F} $ \\
    \hline
    $ \bs{T} $ & $ \bs{F} $ & $ \bs{F} $ \\
    \hline
    $ \bs{T} $ & $ \bs{T} $ & $ \bs{F} $ \\
    \hline
  \end{tabular}
\end{center}
Usando el mismo proceso de traducción que con la operación NAND, se obtiene
\begin{align*}
  \bs{\downarrow}
  &\synteq λp_{1}\, p_{2}. \bs{\prec}\, p_{1} (\bs{\prec}\, p_{2}\, \bs{F}\, \bs{F}) (\bs{\prec}\, p_{2}\, \bs{F}\, \bs{T}) \\
  &\reduce{β} λp_{1}\, p_{2}.p_{1}\, \bs{F} (p_{2}\, \bs{F}\, \bs{T})
\end{align*}

Un aspecto interesante de este método de traducción de operaciones booleanas es que se puede adaptar para operaciones $ n $-árias. Un bosquejo de la forma de estas generalizaciones es
\begin{center}
  \begin{tikzpicture}[level/.style={sibling distance=100mm/#1, level distance=10mm}]
    \node (args1) {$ λx_{1}\, ...\, x_{2^{n}} $}
    child {
      node (args2) {$ λp_{1}\, ...\, p_{n} $}
      child {
        node (p1) {$ \bs{\prec}\, p_{1} $}
        child {
          node (p2a) {$ \bs{\prec}\, p_{2} $}
          child {
            node (dots1) {$ ... $}
            child {
              node (pna) {$ \bs{\prec}\, p_{n} $}
              child {
                node (x2n) {$ x_{2^{n}} $}
              }
              child {
                node (x2n1) {$ x_{2^{n}-1} $}
              }
            }
            child {
              node (dots3) {$ ... $}
            }
          }
          child {
            node (dots2) {$ ... $}
          }
        }
        child {
          node (p2b) {$ \bs{\prec}\, p_{2} $}
          child {
            node (dots4) {$ ... $}
          }
          child {
            node (dots5) {$ ... $}
            child {
              node (dots6) {$ ... $}
            }
            child {
              node (pnb) {$ \bs{\prec}\, p_{n} $}
              child {
                node (x2) {$ x_{2} $}
              }
              child {
                node (x1) {$ x_{1} $}
              }
            }
          }
        }
      }
    };
  \end{tikzpicture}
\end{center}

\subsection{Extensiones al álgebra booleana}
\label{sec:boolean-extensiones}

Conociendo el proceso de codificación del álgebra booleana en el cálculo $ λ $, resulta simple adaptar la codificación.

Consideremos el caso en donde, además de tener los valores de falso y verdadero, se desea incorporar un valor ``desconocido'' utilizado para representar un valor que no es ni falso, ni verdadero. La interpretación de estos valores es similar a la \autoref{defn:valores-verdad}, pero en lugar de decidir sobre dos términos, se decide sobre tres.

\begin{defn}[Valores de álgebra trivalente]
  La codificación en términos $ λ $ de los valores de ésta álgebra trivalente son
  \begin{align*}
    \bs{T} &\synteq λx\, y\, z.x \\
    \bs{F} &\synteq λx\, y\, z.y \\
    \bs{U} &\synteq λx\, y\, z.z 
  \end{align*}
\end{defn}

Al igual que en la codificación bivalente, se puede codificar un término $ \bs{\prec_{3}} $, similar a $ \bs{\prec} $ de la \autoref{defn:condicional} pero con tres ramificaciones
\begin{defn}[Condicional trivalente]
  \[ \bs{\prec_{3}} \synteq λp\, m\, n\, o.p\, m\, n\, o \]
  De tal manera que, si $ P \in \{ \bs{T},\ \bs{F},\ \bs{U} \} $
  \[ (\bs{\prec_{3}}\, P\, M\, N\, O) \reduce{β} \begin{cases} M & P \synteq \bs{T}; \\ N & P \synteq \bs{F}; \\ O & P \synteq \bs{U}. \end{cases} \]
\end{defn}

Sea $ \odot $ una operación trivalente binaria con la siguiente tabla de valores
\begin{center}
  \begin{tabular}{|c|c||c|}
    \hline
    $ P $ & $ Q $ & $ P \odot Q $ \\ [0.5ex] \hline\hline
    $ \bs{T} $ & $ \bs{T} $ & $ x_{1} $ \\ \hline
    $ \bs{T} $ & $ \bs{F} $ & $ x_{2} $ \\ \hline
    $ \bs{T} $ & $ \bs{U} $ & $ x_{3} $ \\ \hline
    $ \bs{F} $ & $ \bs{T} $ & $ x_{4} $ \\ \hline
    $ \bs{F} $ & $ \bs{F} $ & $ x_{5} $ \\ \hline
    $ \bs{F} $ & $ \bs{U} $ & $ x_{6} $ \\ \hline
    $ \bs{U} $ & $ \bs{T} $ & $ x_{7} $ \\ \hline
    $ \bs{U} $ & $ \bs{F} $ & $ x_{8} $ \\ \hline
    $ \bs{U} $ & $ \bs{U} $ & $ x_{9} $ \\ \hline
  \end{tabular}
\end{center}

El procedimiento en pseudocódigo que la describe se muestra en el \autoref{alg:op-bin-gen-2} y es traducido al cálculo $ λ $ como
\[ λx_{1}\, x_{2}\, x_{3}\, x_{4}\, x_{5}\, x_{6}\, x_{7}\, x_{8}\, x_{9}.(λp_{1}\, p_{2}.(\bs{\prec_{3}}\, p_{1}\, R_{1}\, R_{2}\, R_{3})) \]
Donde
\begin{align*}
  R_{1} &\synteq (\bs{\prec_{3}}\, p_{2}\, x_{1}\, x_{2}\, x_{3}) \\
  R_{2} &\synteq (\bs{\prec_{3}}\,  p_{2}\, x_{4}\, x_{5}\, x_{6}) \\
  R_{3} &\synteq (\bs{\prec_{3}}\, p_{2}\, x_{7}\, x_{8}\, x_{9})
\end{align*}

\begin{algorithm}[ht]
  \caption{Operación booleana $ \odot $ dado $ x_{1} $, $ x_{2} $, ... , $ x_{9} $}
  \label{alg:op-bin-gen-2}
  \begin{algorithmic}
    \REQUIRE $ p_{1}, p_{2} \in \{ \mathrm{verdadero},\ \mathrm{falso},\ \mathrm{desconocido} \} $
    \ENSURE Valor $ x_{i} $ de la tabla de verdad
    \IF{$ p_{1} = \mathrm{verdadero} $}
    \IF{$ p_{2} = \mathrm{verdadero} $}
    \RETURN $ x_{1} $
    \ELSIF{$ p_{2} = \mathrm{falso} $}
    \RETURN $ x_{2} $
    \ELSIF{$ p_{2} = \mathrm{desconocido} $}
    \RETURN $ x_{3} $
    \ENDIF
    \ELSIF{$ p_{1} = \mathrm{falso} $}
    \IF{$ p_{2} = \mathrm{verdadero} $}
    \RETURN $ x_{4} $
    \ELSIF{$ p_{2} = \mathrm{falso} $}
    \RETURN $ x_{5} $
    \ELSIF{$ p_{2} = \mathrm{desconocido} $}
    \RETURN $ x_{6} $
    \ENDIF
    \ELSIF{$ p_{1} = \mathrm{desconocido} $}
    \IF{$ p_{2} = \mathrm{verdadero} $}
    \RETURN $ x_{7} $
    \ELSIF{$ p_{2} = \mathrm{falso} $}
    \RETURN $ x_{8} $
    \ELSIF{$ p_{2} = \mathrm{desconocido} $}
    \RETURN $ x_{9} $
    \ENDIF
    \ENDIF
  \end{algorithmic}
\end{algorithm}

\section{Aritmética}
\label{sec:aritmetica}

La aritmética es una de las ramas más antiguas de las matemáticas. Consiste en el estudio de los números y de las operaciones elementales como la suma y la multiplicación. El manejo de expresiones aritméticas es ubicuo en la vida cotidiana y es una parte fundamental de la formación básica en matemáticas.

En el cálculo $ λ $, los números naturales no son términos $ λ $, sin embargo, desde el metalenguaje se pueden manejan números naturales y expresiones aritméticas como por ejemplo en la \autoref{defn:longitud} de longitud. Al igual que el álgebra boolena, las expresiones aritméticas pueden ser codificadas como términos $ λ $.

En los lenguajes de programación los números naturales y las operaciones aritméticas son de los objetos más utilizados para expresar la mayoría de los cómputos. Virtualmente todo programa no trivial ejecutable en una computadora hace uso de números y operaciones sobre ellos. Como se menciona al inicio de la sección anterior, el concepto de número es codificado usualmente como una secuencia de bits de longitud fija y las operaciones aritméticas terminan siendo traducidas a instrucciones ejecutadas por la unidad aritmética lógica de la computadora.

En esta sección se plantea la codificación de expresiones aritméticas en el lenguaje del cálculo $ λ $ de manera similar a como se abordó en la \autoref{sec:algebra-booleana}, también se aborda la representación de la noción de iteración y algunos mecanismos que nos permiten abstraer el cómputo de las operaciones elementales.

\subsection{Numerales de Church}
\label{sec:numerales}

Los números naturales son los objetos más básicos para representar expresiones aritméticas. En este trabajo se considera que $ \mathbb{N} $ contiene el número 0, por lo que el conjunto de números naturales es
\[ \mathbb{N} = \left\{ 0,\ 1,\ 2,\ 3,\ ... \right\} \]
En la codificación del álgebra booleana se presenta la representación de valores de verdad como una decisión entre dos valores dados. En la \autoref{sec:boolean-extensiones} se extiende la representación de la decisión a tres valores y siguiendo el mismo procedimiento se puede extender a $ n $ valores. Esta representación no es útil al tratar con los números naturales ya que no se tiene un conjunto finito de valores, sin embargo, si se establece una cota superior para la cantidad de números naturales representables es posible utilizar esta codificación.

Por cuestiones de eficiencia, en las computadoras se limita la cantidad de naturales representables a valores entre 0 y $ 2^{64}-1 $, por lo tanto, es posible representar números en este rango como abstracciones de $ 2^{64} $ variables enlazadas. Utilizar esta codificación no es conveniente ya que las operaciones deberán ser definidas para cada posible combinación de naturales.

La codificación más utilizada para números naturales es la de \emph{numerales de Church}, esta codificación fue propuesta por Alonzo Church en 1941 \cite[p.~28]{Church:LambdaConversion}.

\begin{defn}[Numerales de Church]
  \label{defn:numerales-church}
  El numeral de Church, denotado $ \cn{n} $, asociado al número natural $ n $ es
  \begin{equation}
    \label{eq:numeral}
    \cn{n} \synteq (λx\, y.x^{n}\, y)
  \end{equation}
\end{defn}

Los primeros 5 numerales son:
\begin{align*}
  \cn{0} &\synteq λx\, y.x^{0}\, y \synteq λx\, y.y \\
  \cn{1} &\synteq λx\, y.x^{1}\, y \synteq λx\, y.x\, y \\
  \cn{2} &\synteq λx\, y.x^{2}\, y \synteq λx\, y.x\, (x\, y) \\
  \cn{3} &\synteq λx\, y.x^{3}\, y \synteq λx\, y.x\, (x\, (x\, y)) \\
  \cn{4} &\synteq λx\, y.x^{4}\, y \synteq λx\, y.x\, (x\, (x\, (x\, y)))
\end{align*}

Al manipular numerales de Church, se debe tener cuidado en la manera en que se reducen aplicaciones con otros términos. Con la codificación de valores de verdad es relativamente sencillo experimentar con la manera en la que $ \bs{T} $ y $ \bs{F} $ se combinan y corroborar manualmente que una combinación se reduce a otra. Sin embargo, al aplicar un numeral de Church $ \cn{n} $ a un término cualquiera $ M $, el término resultante de la contracción de dicha aplicación tendría aproximadamente una longitud de $ \| M \|\times n $, lo cual resulta inconveniente de escribir en cada paso de una reducción.

Para facilitar el desarrollo de reducciones se muestran algunas propiedades de los numerales de Church y reducciones que involucran términos de la forma $ (P^{n}\, Q) $.

Consideremos un numeral $ \cn{n} $ y términos cualesquiera $ P $ y $ Q $.
\begin{equation}
  \label{eq:numeral:P}
  \cn{n}\, P \contract{β} λx.P^{n}\, x
\end{equation} \begin{equation}
  \label{eq:numeral:PQ}
  \cn{n}\, P\, Q \contract{β} (λx.P^{n}\, x)\, Q \contract{β} P^{n}\, Q
\end{equation}

Sea $ \cn{n} $ un numeral de Church, $ P $, $ Q $ y $ R $ términos cualesquiera y $ m $ un número natural. Se aborda la reducción de las aplicaciones $ ((\cn{n}\, P\, Q)^{m}\, R) $, $ ((\cn{n}\, P)^{m}\, Q) $ y $ (\cn{n}^{m}\, P) $, las cuales corresponden a la aplicación $ (F^{m}\, X) $ donde $ F $ es el término de las ecuaciones \eqref{eq:numeral:PQ}, \eqref{eq:numeral:P} y \eqref{eq:numeral}.
\begin{align}
  \label{eq:numeral:PQm}
  (\cn{n}\, P\, Q)^{m}\, R &\reduce{β} (P^{n}\, Q)^{m}\, R &\text{Por \eqref{eq:numeral:PQ}}
\end{align}
Ya que no se hicieron suposiciones adicionales sobre $ P $ y $ Q $ no podemos asegurar que el término final de la reducción \eqref{eq:numeral:PQm} pueda ser reducido más.
\begin{align}
  \label{eq:numeral:PmQ}
  (\cn{n}\, P)^{m}\, Q \synteq &(\cn{n}\, P)^{m-1}\, (\cn{n}\, P\, Q) &\text{Por \eqref{eq:abuso:F}} \\
  \reduce{β} &(\cn{n}\, P)^{m-1}\, (P^{n}\, Q) &\text{Por \eqref{eq:numeral:PQ}} \nonumber \\
  \synteq &(\cn{n}\, P)^{m-2}\, (\cn{n}\, P\, (P^{n}\, Q)) &\text{Por \eqref{eq:abuso:F}} \nonumber \\
  \reduce{β} &(\cn{n}\, P)^{m-2}\, (P^{n}\, (P^{n}\, Q)) &\text{Por \eqref{eq:numeral:PQ}} \nonumber \\
  \synteq &(\cn{n}\, P)^{m-2}\, (P^{2\times n}\, Q) \nonumber \\
                               &\text{Repitiendo para $ m-3, ..., m-m $} \nonumber \\
  \reduce{β} &(\cn{n}\, P)^{m-m}\, (P^{m\times n}\, Q) \nonumber \\
  \synteq &P^{m\times n}\, Q &\text{Por \eqref{eq:abuso:F}} \nonumber
\end{align}

\begin{align}
  \label{eq:numeral:mP}
  \cn{n}^{m}\, P \synteq &\cn{n}^{m-1} (\cn{n}\, P) &\text{Por \eqref{eq:abuso:F}} \\
                \synteq &\cn{n}^{m-2} (\cn{n} (\cn{n}\, P)) &\text{Por \eqref{eq:abuso:F}} \nonumber \\
           \contract{β} &\cn{n}^{m-2} (λx.(\cn{n}\, P)^{n} x) &\text{Por \eqref{eq:numeral:P}} \nonumber \\
             \reduce{β} &\cn{n}^{m-2} (λx.P^{n\times n}\, x) &\text{Por \eqref{eq:numeral:PmQ}} \nonumber \\
                \synteq &\cn{n}^{m-2} (\cn{n\times n}\, P) &\text{Por \eqref{eq:numeral:P}} \nonumber \\
                        &... &\text{Repitiendo para $ m-3, ..., m-m $} \nonumber \\
             \reduce{β} &\cn{n}^{m-m} (\cn{n^{m}}\, P) \nonumber \\
                \synteq &(\cn{n^{m}}\, P) &\text{Por \eqref{eq:abuso:F}} \nonumber
\end{align}

Consideremos la reducción de una aplicación de numerales $ (\cn{n}\, \cn{m}) $. Cuando $ \cn{n} \synteq \cn{0} $ se tiene que para todo natural $ m $, $ (\cn{0}\, \cn{m}) \reduce{β} \bs{I} $ debido a que $ \cn{0} \synteq \bs{F} $ y $ (\bs{F}\, M) \reduce{β} \bs{I} $. Cuando $ \cn{n} \synteq \cn{1} $ las reducciones siguen siendo manejables, para los casos donde $ \cn{m} \synteq \cn{0}, \cn{1}, \cn{2} $ se obtienen las siguientes reducciones:
\begin{align*}
  \cn{1}\, \cn{0} &\contract{β} (λx.\cn{0}^{1}\, x) &\text{Por \eqref{eq:numeral:P}} \\
                  &\reduce{β} (λx.\cn{0^{1}}\, x) &\text{Por \eqref{eq:numeral:mP}} \\
                  &\contract{β} (λx\, y.x^{0}\, y) &\text{Por \eqref{eq:numeral:P}} \\
                  &\synteq \cn{0} \\
  \cn{1}\, \cn{1} &\contract{β} (λx.\cn{1}^{1}\, x) &\text{Por \eqref{eq:numeral:P}} \\
                  &\reduce{β} (λx.\cn{1^{1}}\, x) &\text{Por \eqref{eq:numeral:mP}} \\
                  &\contract{β} (λx\, y.x^{1}\, y) &\text{Por \eqref{eq:numeral:P}} \\
                  &\synteq \cn{1} \\
  \cn{1}\, \cn{2} &\contract{β} (λx.\cn{2}^{1}\, x) &\text{Por \eqref{eq:numeral:P}} \\
                  &\reduce{β} (λx.\cn{2^{1}}\, x) &\text{Por \eqref{eq:numeral:mP}} \\
                  &\contract{β} (λx\, y.x^{2}\, y) &\text{Por \eqref{eq:numeral:P}} \\
                  &\synteq \cn{2} \\
\end{align*}
Para cualquier numeral $ \cn{m} $:
\[ \cn{1}\, \cn{m} \contract{β} λx.\cn{m}^{1}\, x \reduce{β} λx.\cn{m^{1}}\, x \contract{β} λx\, y.x^{m}\, y \synteq \cn{m} \]

Cuando se considera $ \cn{n} \synteq \cn{2} $ las reducciones siguen los mismos pasos que en el caso anterior:
\begin{align*}
  \cn{2}\, \cn{m} &\contract{β} λx.\cn{m}^{2}\, x \reduce{β} λx.\cn{m^{2}}\, x \contract{β} λx\, y.x^{m^{2}}\, y \synteq \cn{m^{2}}
\end{align*}
Lo cual nos lleva a concluír que en el caso general, para cualesquiera numerales $ \cn{n} $ y $ \cn{m} $:
\begin{equation}
  \label{eq:numeral:nm}
  \cn{n}\, \cn{m} \contract{β} λx.\cn{m}^{n}\, x \reduce{β} λx.\cn{m^{n}}\, x \contract{β} λx\, y.x^{m^{n}}\, y \synteq \cn{m^{n}}
\end{equation}

Es curioso observar como la reducción de una aplicación sencilla entre dos numerales nos permite computar una operación aritmética relativamente compleja como la exponenciación. Este resultado pudiera parecer también preocupante, si la codificación de la operación $ n^{m} $ es tan sencillo como reducir la aplicación $ (\cn{m}\, \cn{n}) $, ¿Cómo se implementan operaciones mas simples como la suma y la multiplicación?.

En lo que resta de la sección se presentan procedimientos sistemáticos para codificar las operaciones elementales de la aritmética.

\subsection{Operaciones elementales}
\label{sec:aritmetica-elemental}

En la ecuación \eqref{eq:numeral:nm} se muestra como la aplicación de dos numerales se relaciona directamente con la operación de exponenciación. La primera aproximación a la codificación de las operaciones aritméticas seguirá un procedimiento inverso a cómo se dió con este resultado.

Las operaciones que se codificarán son la suma, la multiplicación y la exponenciación. Estas tres operaciones son binarias, es decir, a partir de dos números calculan otro. Para comenzar a codificar este tipo de operaciones consideremos una operación aritmética binaria $ \odot $ que realiza algún cálculo.

La convención para nombrar numerales será utilizada para las operaciones aritméticas, por lo tanto $ m \odot n $ se codifica como $ (\cn{\odot}\, \cn{m}\, \cn{n}) $. Ya que $ \cn{\odot} $ es una abstracción que espera ser aplicada a dos numerales, se propone que la operación codificada tenga la forma $ \cn{\odot} \synteq (λm\, n.M) $, donde $ M $ es un término que al reducir la aplicación $ (\cn{\odot}\, \cn{a}\, \cn{b}) $ es equivalente a $ (λx\, y.x^{a \odot b}\, y) $.

En el caso de la codificación de la exponenciación, denotada $ \cn{\uparrow} $, se tiene que $ (\cn{\uparrow}\, \cn{m}\, \cn{n}) $ debe reducirse a
\[ λx\, y.x^{m^{n}}\, y \]
Se puede derivar el término $ \cn{\uparrow} $ observando que
\begin{align*}
  λx\, y.x^{m^{n}}\, y &\synteq \cn{m^{n}} &\text{Por \eqref{eq:numeral}} \\
                       &\convertible{β} \cn{n}\, \cn{m} &\text{Por \eqref{eq:numeral:nm}} \\
                       &\convertible{β} λx\, y.\cn{n}\, \cn{m}\, x\, y
\end{align*}
Por lo tanto, la codificación de la exponenciación es
\begin{equation}
  \label{eq:numeral:exp1}
  \cn{\uparrow} \synteq λm\, n.λx\, y.n\, m\, x\, y
\end{equation}

En el caso de la codificación de la multiplicación, denotada $ \cn{\times} $, se tiene que $ (\cn{\times}\, \cn{m}\, \cn{n}) $ debe reducirse a
\[ λx\, y.x^{n\times m}\, y \]
Para derivar el término $ \cn{\times} $ se observa que
\begin{align*}
  λx\, y.x^{n\times m}\, y &\convertible{β} λx\, y.(\cn{m}\, x)^{n}\, y &\text{Por \eqref{eq:numeral:PmQ}} \\
                           &\convertible{β} λx\, y.\cn{n}\, (\cn{m}\, x)\, y &\text{Por \eqref{eq:numeral:PQ}}
\end{align*}
Por lo tanto, la codificación de la multiplicación es
\begin{equation}
  \label{eq:numeral:mul1}
  \cn{\times} \synteq λm\, n.λx\, y.n(m\, x)y
\end{equation}

Finalmente, con la operación de adición, denotada $ \cn{+} $, se tiene que $ (\cn{+}\, \cn{m}\, \cn{n}) $ debe reducirse a
\[ λx\, y.x^{n+m}\, y \]
Para derivar el término $ \cn{+} $ se observa que
\begin{align*}
  λx\, y.x^{n+m}\, y &\synteq λx\, y.x^{n}\, (x^{m}\, y) &\text{Por \eqref{eq:abuso:F}} \\
                     &\convertible{β} λx\, y.x^{n}\, (\cn{m}\, x\, y) &\text{Por \eqref{eq:numeral:PQ}} \\
                     &\convertible{β} λx\, y.\cn{n}\, x\, (\cn{m}\, x\, y) &\text{Por \eqref{eq:numeral:PQ}}
\end{align*}
Por lo tanto, la codificación de la adición es
\begin{equation}
  \label{eq:numeral:sum1}
  \cn{+} \synteq λm\, n.λx\, y.n\, x\, (m\, x\, y)
\end{equation}
Las codificaciones \eqref{eq:numeral:exp1}, \eqref{eq:numeral:mul1} y \eqref{eq:numeral:sum1} fueron construídas a partir de las reducciones mostradas en \eqref{eq:numeral:PmQ}, \eqref{eq:numeral:mP} y \eqref{eq:numeral:nm}, las cuales a su vez fueron obtenidas a partir del abuso de notación definido en \eqref{eq:abuso:F} el cual refleja la estructura de los numerales de Church, por lo tanto, las codificaciones mostradas se basan únicamente en la estructura de los numerales. Sin embargo, las operaciones de adición, multiplicación y exponenciación no son únicamente cálculos independientes que permiten expresar expresiones aritméticas. Estas tres operaciones se encuentran conceptualmente relacionadas.

\begin{figure}[!htbp]
  \begin{align*}
    \cn{+} & &\synteq& & \mathcolor{gray}{λm\, n.λx\, y.}\mathcolor{red}{n}\, x\, (\mathcolor{blue}{m}\, x\, y) & &\synteq& & \mathcolor{gray}{(λm.(λn.(λx.(λy.}((n\, x)\, \mathcolor{magenta}{((m\, x)\, y)})\mathcolor{gray}{))))} \\
    \cn{\times} & &\synteq& & \mathcolor{gray}{λm\, n.λx\, y.}\mathcolor{red}{n}\, (\mathcolor{blue}{m}\, x)y & &\synteq& & \mathcolor{gray}{(λm.(λn.(λx.(λy.}((n\, \mathcolor{magenta}{(m\, x)})\, y)\mathcolor{gray}{))))} \\
    \cn{\uparrow} & &\synteq& &\mathcolor{gray}{λm\, n.λx\, y.}\mathcolor{red}{n}\, \mathcolor{blue}{m}\, x\, y & &\synteq& & \mathcolor{gray}{(λm.(λn.(λx.(λy.}(((n\, \mathcolor{magenta}{m})\, x)\, y)\mathcolor{gray}{))))}
  \end{align*}
  \caption{Codificaciones de adición, multiplicación y exponenciación}
  \label{fig:numeral:cod1comp}
\end{figure}

En la \autoref{fig:numeral:cod1comp} se puede apreciar la diferencia estructural entre las codificaciones definidas, las similitudes se encuentran coloreadas en gris.

La segunda columna muestra las tres codificaciones escritas de manera compacta, se observa que en los tres casos, la aparición de $ n $ se encuentra antes que la aparición de $ m $. Debido a que los numerales de Church son abstracciones, al reducir la aplicación de una operación a dos numerales $ \cn{m} $ y $ \cn{n} $, la estructura del resultado en su forma normal se basará principalmente en la estructura de $ \cn{n} $. Esto no es muy relevante en el caso de $ \cn{+} $ y $ \cn{\times} $ ya que las operaciones son conmutativas, por lo tanto, no es importante si se intercambian las apariciones de $ n $ y $ m $, sin embargo, la exponenciación no es una operación conmutativa, el numeral base y el numeral exponente juegan papeles diferentes en la operación.

La tercera columna muestra las tres codificaciones escritas sin abuso de notación, se observa que la aparición de $ m $ se agrupa con las variables $ x $ y $ y $ de manera similar a las ecuaciones \eqref{eq:numeral}, \eqref{eq:numeral:P} y \eqref{eq:numeral:PQ}, las cuales fueron utilizadas en las ecuaciones \eqref{eq:numeral:PmQ}, \eqref{eq:numeral:mP} y \eqref{eq:numeral:nm}.

La figura nos permiten razonar sobre la manera en como se $ β $-reduce la aplicación de las operaciones, sin embargo, las similitudes en la estructura de las codificaciones no refleja las similitudes de las operaciones, por lo que es difícil razonar sobre las operaciones a partir de su definición.

En la \autoref{sec:algebra-booleana}, las codificaciones desarrolladas se basaron en una relación fundamental entre los valores de verdad y una desición, esto permitió construir abstracciones componibles que facilitaron codificar y razonar sobre las expresiones booleanas, a tal grado que se estableció una correspondencia directa entre las expreciones condicionales de los lenguajes de programación y las operaciones booleanas.

Para lograr este mismo efecto con la codificación de la aritmética, se deben hacer observaciones más fundamentales sobre la estructura de los numerales de Church y las nociones de las operaciones aritméticas.

Los números naturales nacieron a la par de la necesidad humana de \emph{contar}. De manera similar a la analogía presentada al inicio de la \autoref{sec:valores-de-verdad} se plantea la siguiente situación hipotética:

Una persona omnisciente y muda llamada $ P $ puede decirme la cantidad de objetos en el mundo si le planteo una pregunta con una respuesta contable y le doy un martillo y un clavo; la cantidad de objetos va a corresponder a la cantidad de veces que $ P $ golpea el clavo con el martillo. En este planteamiento irreal e hipotético, no es necesario conocer la estructura del número, sólo es necesario tener a alguien que pueda contar y proveer dos objetos sabiendo que la persona va a realizar algo con el primero sobre el segundo (en este caso, golpear con el martillo al clavo). El procedimiento que realiza esta persona puede representar valores numéricos si nunca podemos conocer a los números naturales.

Detrás del concepto de contar, está el concepto de \emph{repetición}, la estructura de los numerales de Church se puede interpretar como la analogía entre repetición y número.

Sea $ P $ un término $ λ $ el cual puede ser aplicado a una pregunta $ Q $, al $ β $-reducir $ (P\, Q) $ se obtiene una repetición $ R $ la cual al ser aplicada a una acción $ A $ y un objeto $ O $ se $ β $-reduce a realizar la acción $ A $ sobre $ O $ y repetir el procedimiento con el resultado hasta haber realizado cierta cantidad de acciones:

\[ P\, Q \reduce{β} R, \]
\[ R\, A\, O \reduce{β} \underbrace{A(A(\ ...\ (A(A\, O))\ ...\ ))}_\text{$ n $ veces} \]

Para fines prácticos no es necesario conocer la estructura de $ P $ ni de $ Q $, lo importante es que $ A $ se realice cierta cantidad de veces sobre $ O $. Por lo tanto $ R $ es un término $ λ $ de la forma
\[ λx\, y.x\, (x\, (\ ...\ (x\, (x\, y))\ ...\ )) \]
La cual corresponde a la estructura de los numerales de Church.

Teniendo una justificación conceptual e informal para considerar a los números como repeticiones podemos estudiar las operaciones aritméticas a partir de esta perspectiva.

Al inicio de esta sección se construyeron las codificaciones de las operaciones aritméticas en un orden peculiar. Primero la exponenciación, después la multiplicación y al final la adición. Esto es bastante raro debido a que la exponenciación suele ser considerada una operación más compleja que la multiplicación y a su vez esta más compleja que la adición, la estructura de las codificaciones parece aumentar en complejidad entre menos complejas son las operaciones que describen.

La percepción de complejidad de operaciones aritméticas se remonta a la manera en cómo se enseña la aritmética en la educación básica. Después de aprender a contar, se aprende a sumar y después a multiplicar. A pesar de ser en un inicio un proceso de memorización, el acto de sumar y multiplicar números pequeños termina siendo un acto trivial, empleando algoritmos y heurísticas de estimación cuando los números son grandes. En el caso de la exponenciación, los computólogos suelen adquirir esta misma capacidad cuando se trata de operaciones de la forma $ 2^{n} $ debido a la repetida utilización de números en base 2. Sin embargo, esta percepción tiene también una justificación algorítmica.

La operación de multiplicación puede ser definida en función de la operación de adición. Sean $ m $ y $ n $ dos números naturales, la operación $ m \times n $ es equivalente a sumar $ m $ consigo mismo $ n $ veces.
\begin{equation}
  \label{eq:numeral:muldef}
  m \times n = \underbrace{m + m + ... + m}_\text{$ n $ veces} = \sum_{i=1}^{n} m
\end{equation}
De manera análoga, la operación de exponenciación puede ser definida en función de la operación de multiplicación. Sean $ m $ y $ n $ dos números naturales, la operación $ m^{n} $ es equivalente a multiplicar $ m $ consigo mismo $ n $ veces.
\begin{equation}
  \label{eq:numeral:expdef}
  m^{n} = \underbrace{m \times m \times ... \times m}_\text{$ n $ veces} = \prod_{i=1}^{n} m
\end{equation}
De esta manera, una operación compleja como la exponenciación se define en términos de una operación más fundamental como la multiplicación. Esta observación trae a colación la pregunta, ¿cuál es la operación aritmética más fundamental?.

La respuesta a esta pregunta no es fácil de encontrar, se pudiera pensar que la adición es la operación más fundamental, sin embargo, la adición puede ser definida en función de la operación unaria sucesor y esta a su vez es un caso particular de la adición. Sea $ +_{1} $ el operador unario sucesor
\[ m + n = \underbrace{+_{1} +_{1} ... {} +_{1}}_\text{$ n $ veces} m = \underbrace{1 + 1 + ... + 1}_\text{$ n $ veces}+m \]
En la segunda aproximación de las codificaciones, se considera que el operador de sucesor es más fundamental que la adición debido a que es fácil codificar la sucesión sin basarse en resultados previos.

La definición de la operación de sucesor consiste en ``añadir'' una variable $ x $ a un número, ya que
\[ \cn{n} \synteq λx\, y.\underbrace{x\, (x\, (\ ...\ (x\, y)))}_\text{$ n $ apariciones de $ x $} \]
solo se necesita obtener el cuerpo del numeral con la aplicación $ (\cn{n}\, x\, y) $ y aplicar $ x $ al resultado de la reducción. La codificación del operador $ \cn{+}_{1} $ es
\begin{equation}
  \label{eq:numeral:suc2}
  \cn{+}_{1} \synteq λn.λx\, y.x\, (n\, x\, y)
\end{equation}
Para demostrar que esta definición es correcta, consideremos la reducción de la aplicación de $ \cn{+}_{1} $ en un numeral cualquiera $ \cn{n} $:
\begin{align}
  \label{eq:numeral:suc2dem}
  \cn{+}_{1}\, \cn{n} \synteq &(λn.λx\, y.x\, (n\, x\, y))\, \cn{n} &\text{Por \eqref{eq:numeral:suc2}} \\
                \contract{β} &λx\, y.x\, (\cn{n}\, x\, y) \nonumber \\
                  \reduce{β} &λx\, y.x\, (x^{n}\, y) &\text{Por \eqref{eq:numeral:PQ}} \nonumber \\
                     \synteq &λx\, y.x^{n+1}\, y &\text{Por \eqref{eq:abuso:F}} \nonumber \\
                     \synteq &\cn{n+1} \nonumber
\end{align}
Ahora se debe plantear una manera de aplicar el concepto de repetición de sucesores para obtener la adición. La operación $ \cn{+} $ deberá tomar dos numerales $ \cn{m} $ y $ \cn{n} $ y repetir $ \cn{n} $ veces la operación de sucesor sobre $ \cn{m} $. Ya que la aplicación de $ \cn{+}_{1} $ a un numeral, resulta en un numeral basta con aplicar $ \cn{n} $ al operador $ \cn{+}_{1} $ y al numeral $ \cn{m} $. Por ejemplo
\begin{align*}
  (\cn{3}\, \cn{+}_{1}\, \cn{4}) \reduce{β} &\cn{+}_{1}\, (\cn{+}_{1}\, (\cn{+}_{1}\, \cn{4})) &\text{Por \eqref{eq:numeral:PQ}} \\
                                 \reduce{β} &\cn{+}_{1}\, (\cn{+}_{1}\, \cn{5}) &\text{Por \eqref{eq:numeral:suc2dem}} \\
                                 \reduce{β} &\cn{+}_{1}\, \cn{6} &\text{Por \eqref{eq:numeral:suc2dem}} \\
                                 \reduce{β} &\cn{7} &\text{Por \eqref{eq:numeral:suc2dem}}
\end{align*}
En general, para cualesquiera $ \cn{m} $ y $ \cn{n} $, la aplicación $ (\cn{n}\, \cn{+}_{1}\, \cn{m}) $ se reduce a:
\begin{align}
  \label{eq:numeral:sum2dem}
  \cn{n}\, \cn{+}_{1}\, \cn{m} \reduce{β} &\cn{+}_{1}^{n}\, \cn{m} &\text{Por \eqref{eq:numeral:PQ}} \\
                                 \synteq &\cn{+}_{1}^{n-1}\, (\cn{+}_{1}\, \cn{m}) &\text{Por \eqref{eq:abuso:F}} \nonumber \\
                              \reduce{β} &\cn{+}_{1}^{n-1}\, \cn{m+1} &\text{Por \eqref{eq:numeral:suc2dem}} \nonumber \\
                                      ...& \nonumber \\
                              \reduce{β} &\cn{+}_{1}^{n-n}\, \cn{m+n} \nonumber \\
                                 \synteq &\cn{m+n} &\text{Por \eqref{eq:abuso:F}} \nonumber
\end{align}
La codificación del operador $ \cn{+} $ es
\begin{equation}
  \label{eq:numeral:sum2}
  \cn{+} \synteq λm\, n.n\, \cn{+}_{1}\, m
\end{equation}
Para codificar la operación de multiplicación y exponenciación se puede seguir el mismo patrón: un numeral $ \cn{n} $ determina una cantidad de repeticiones, es aplicado a una operación unaria que será aplicada $ n $ veces a un término. Hay dos detalles importantes que considerar, primero, cómo convertir una codificación de una operación binaria a unaria y qué valor aplicar al final.

En el caso de la codificación de la multiplicación se debe convertir a $ \cn{+} $ en una operación unaria, por la definición de multiplicación \eqref{eq:numeral:muldef} se tiene que
\[ m + m + ... + m = m + m + ... + m + 0 = (m + (m + ... + (m + 0) ...)) \]
Esto es, se repite la aplicación de una abstracción que toma un numeral y computa la suma de el numeral y $ m $, cierta cantidad de veces, comenzando con el numeral $ \cn{0} $. Para construir la versión unaria de $ \cn{+} $ se puede plantear la abstracción $ λn.(\cn{+}\, n\, \cn{m}) $, sin embargo hay una manera más conveniente de escribir esta abstracción. Si consideramos la definición \eqref{eq:numeral:sum2} y la aplicamos únicamente a un numeral, se reduce a
\[ \cn{+}\, \cn{m} \synteq (λm\, n.n\, \cn{+}_{1}\, m)\, \cn{m} \contract{β} λn.n\, \cn{+}_{1}\, \cn{m} \]
El cual al ser aplicado a algun numeral $ \cn{n} $ será reducido a un término $ β $-convertible a $ (\cn{+}\, \cn{m}\, \cn{n}) $. Por lo tanto, la aplicación $ (\cn{n}\, (\cn{+}\, \cn{m})\, \cn{0}) $ computa la multiplicación de $ \cn{m} $ y $ \cn{n} $. Por ejemplo
\begin{align*}
  \cn{3}\, (\cn{+}\, \cn{4}) \cn{0} \reduce{β} &(\cn{+}\, \cn{4})^{3}\, \cn{0} &\text{Por \eqref{eq:numeral:PQ}} \\
                                     \synteq &(\cn{+}\, \cn{4})^{2}\, (\cn{+}\, \cn{4}\, \cn{0}) &\text{Por \eqref{eq:abuso:F}} \\
                                  \reduce{β} &(\cn{+}\, \cn{4})^{2}\, \cn{4} &\text{Por \eqref{eq:numeral:sum2dem}} \\
                                     \synteq &(\cn{+}\, \cn{4})\, (\cn{+}\, \cn{4}\, \cn{4}) &\text{Por \eqref{eq:abuso:F}} \\
                                  \reduce{β} &(\cn{+}\, \cn{4})\, \cn{8} &\text{Por \eqref{eq:numeral:sum2dem}} \\
                                     \synteq &(\cn{+}\, \cn{4})^{0}\, (\cn{+}\, \cn{4}\, \cn{8}) &\text{Por \eqref{eq:abuso:F}} \\
                                  \reduce{β} &(\cn{+}\, \cn{4})^{0}\, \cn{12} &\text{Por \eqref{eq:numeral:sum2dem}} \\
                                     \synteq &\cn{12} &\text{Por \eqref{eq:abuso:F}}
\end{align*}
En general, para cualesquiera $ \cn{m} $ y $ \cn{n} $, la aplicación $ (\cn{n}\, (\cn{+}\, \cn{m})\, \cn{0}) $ se reduce a:
\begin{align}
  \label{eq:numeral:mul2dem}
  \cn{n}\, (\cn{+}\, \cn{m}) \cn{0} \reduce{β}& (\cn{+}\, \cn{m})^{n}\, \cn{0} &\text{Por \eqref{eq:numeral:PQ}} \\
                                       \synteq& (\cn{+}\, \cn{m})^{n-1}\, (\cn{+}\, \cn{m}\, \cn{0}) &\text{Por \eqref{eq:abuso:F}} \nonumber \\
                                    \reduce{β}& (\cn{+}\, \cn{m})^{n-1}\, \cn{m+0} &\text{Por \eqref{eq:numeral:sum2dem}} \nonumber \\
                                       \synteq& (\cn{+}\, \cn{m})^{n-2}\, (\cn{+}\, \cn{m}\, \cn{m+0}) &\text{Por \eqref{eq:abuso:F}} \nonumber \\
                                    \reduce{β}& (\cn{+}\, \cn{m})^{n-2}\, \cn{m\times 2 + 0} &\text{Por \eqref{eq:numeral:sum2dem}} \nonumber \\
                                           ...& \nonumber \\
                                    \reduce{β}& (\cn{+}\, \cn{m})^{n-n}\, \cn{m\times n + 0} \nonumber \\
                                       \synteq& \cn{m\times n} &\text{Por \eqref{eq:abuso:F}} \nonumber
\end{align}
La codificación del operador $ \cn{\times} $ es
\begin{equation}
  \label{eq:numeral:mul2}
  \cn{\times} \synteq λm\, n.n\, (\cn{+}\, m)\, \cn{0}
\end{equation}

De manera análoga, se utiliza la definición de exponenciación \eqref{eq:numeral:expdef} para definir su codificación en función de $ \cn{\times} $. Sean $ \cn{m} $ y $ \cn{n} $ dos numerales cualesquiera, la aplicación $ (\cn{\uparrow}\, \cn{m}\, \cn{n}) $ debe repetir la multiplicación de la base $ \cn{m} $ una cantidad de veces determinada por el exponente $ \cn{n} $. La codificación es muy similar a la de multiplicación, sólo que utilizando como término final el numeral $ \cn{1} $ ya que $ \prod_{i=1}^{n} m = 1 \times \prod_{i=1}^{n} m $. Para corroborar que la aplicación $ (\cn{n} (\cn{\times}\, \cn{m})\, \cn{1}) $ computa la exponenciación de $ \cn{m} $ a la $ \cn{n} $ se desarrolla el siguiente ejemplo
\begin{align*}
  \cn{3}\, (\cn{\times}\, \cn{4})\, \cn{1} \reduce{β} &(\cn{\times}\, \cn{4})^{3}\, \cn{1} &\text{Por \eqref{eq:numeral:PQ}} \\
                                          \synteq &(\cn{\times}\, \cn{4})^{2}\, (\cn{\times}\, \cn{4}\, \cn{1}) &\text{Por \eqref{eq:abuso:F}} \\
                                       \reduce{β} &(\cn{\times}\, \cn{4})^{2}\, \cn{4} &\text{Por \eqref{eq:numeral:mul2dem}} \\
                                          \synteq &(\cn{\times}\, \cn{4})^{1}\, (\cn{\times}\, \cn{4}\, \cn{4}) &\text{Por \eqref{eq:abuso:F}} \\
                                       \reduce{β} &(\cn{\times}\, \cn{4})^{1}\, \cn{16} &\text{Por \eqref{eq:numeral:mul2dem}} \\
                                          \synteq &(\cn{\times}\, \cn{4})^{0}\, (\cn{\times}\, \cn{4}\, \cn{16}) &\text{Por \eqref{eq:abuso:F}} \\
                                       \reduce{β} &(\cn{\times}\, \cn{4})^{0}\, \cn{64} &\text{Por \eqref{eq:numeral:mul2dem}} \\
                                          \synteq &\cn{64}
\end{align*}

En general, para cualesquiera $ \cn{m} $ y $ \cn{n} $, la aplicación $ (\cn{n} (\cn{\times}\, \cn{m}) \cn{1}) $ se reduce a:
\begin{align}
  \label{eq:numeral:exp2dem}
  \cn{n}\, (\cn{\times}\, \cn{m})\, \cn{1} \reduce{β} &(\cn{\times}\, \cn{m})^{n}\, \cn{1} &\text{Por \eqref{eq:numeral:PQ}} \\
                                          \synteq &(\cn{\times}\, \cn{m})^{n-1}\, (\cn{\times}\, \cn{m}\, \cn{1}) &\text{Por \eqref{eq:abuso:F}} \nonumber \\
                                       \reduce{β} &(\cn{\times}\, \cn{m})^{n-1}\, \cn{m\times 1} &\text{Por \eqref{eq:numeral:mul2dem}} \nonumber \\
                                          \synteq &(\cn{\times}\, \cn{m})^{n-2}\, (\cn{\times}\, \cn{m}\, \cn{m\times 1}) &\text{Por \eqref{eq:abuso:F}} \nonumber \\
                                       \reduce{β} &(\cn{\times}\, \cn{m})^{n-2}\, \cn{m^{2}\times 1} &\text{Por \eqref{eq:numeral:mul2dem}} \nonumber \\
                                              ... &\nonumber \\
                                       \reduce{β} &(\cn{\times}\, \cn{m})^{n-n}\, \cn{m^{n}\times 1} \nonumber \\
                                          \synteq &\cn{m^{n}} \nonumber
\end{align}

La codificación del operador $ \cn{\uparrow} $ es
\begin{equation}
  \label{eq:numeral:exp2}
  \cn{\uparrow} \synteq λm\, n.n\, (\cn{\times}\, m)\, \cn{1}
\end{equation}

Comparando esta segunda aproximación de las codicicaciones de $ \cn{+} $, $ \cn{\times} $ y $ \cn{\uparrow} $ se pueden observar relaciones tanto en estructura como en significado.

En la \autoref{fig:numeral:cod2comp} se puede apreciar la diferencia estrucural entre las codificaciones definidas, las similitudes menos importantes se encuentran coloreadas en gris.

\begin{figure}[!htbp]
  \begin{align*}
    \cn{+} &&\synteq&& \mathcolor{gray}{λm\, n.}\mathcolor{red}{n}\, \cn{+}_{1}\, \mathcolor{blue}{m}  &&\synteq&& \mathcolor{gray}{(λm.(λn.}((n\, \mathcolor{magenta}{\cn{+}_{1}})\, m)\mathcolor{gray}{))} \\
    \cn{\times} &&\synteq&& \mathcolor{gray}{λm\, n.}\mathcolor{red}{n}\, (\cn{+}\, m)\, \mathcolor{blue}{\cn{0}}  &&\synteq&& \mathcolor{gray}{(λm.(λn.}((n\, \mathcolor{magenta}{(\cn{+}\, m)})\, \cn{0})\mathcolor{gray}{))} \\
    \cn{\uparrow} &&\synteq&& \mathcolor{gray}{λm\, n.}\mathcolor{red}{n}\, (\cn{\times}\, m)\, \mathcolor{blue}{\cn{1}}  &&\synteq&& \mathcolor{gray}{(λm.(λn.}((n\, \mathcolor{magenta}{(\cn{\times}\, m)})\, \cn{1})\mathcolor{gray}{))}
  \end{align*}
  \caption{Codificaciones de adición, multiplicación y exponenciación}
  \label{fig:numeral:cod2comp}
\end{figure}

La segunda columna muestra las tres codificaciones escritas de manera compacta, se observa que en los tres casos el átomo $ n $ corresponde al operando derecho de la operación y es el que determina la cantidad de veces que se aplicará un procedimiento. Coloreados con azúl se encuentran los valores iniciales a los que se aplica el procedimiento, estos corresponden al caso trivial de la operación, es decir, si $ n $ es cero, entonces el resultado de la suma es $ m $, el de la multiplicación es $ 0 $ y el de la exponenciación es $ 1 $.

En la tercera columna se encuentran las codificaciones escritas sin abuso de notación, coloreado en magenta están los términos a los que $ n $ es aplicado primero, esto es, los términos que serán aplicados una y otra vez. Estos términos se encuentran en función de la operación anterior (en orden de menor a mayor complejidad). Al ver las definiciones se puede saber que la expoenciación es repetición de multiplicaciones con caso base $ 1 $, la multiplicación es repetición de adiciones con caso base $ 0 $ y la adición es repetición de sucesiones con caso base $ m $.

Teniendo codificaciones definidas de manera compacta y elegante, solo queda preguntarnos cómo obtener el resto de las operaciones aritméticas elementales, es decir, la sustracción, la división y para completar las inversas, el logaritmo y la raíz.

Estas operaciones inversas pueden ser vistas de manera similar a la adición, multiplicación y exponenciación, solo que en lugar de añadir aplicaciones, eliminar aplicaciones. Esto se puede lograr con la operación \emph{predecesor}, definida como una operación unaria cerrada en los naturales como
\begin{align*}
  -_{1} 0 &= 0 \\
  -_{1} n &= n-1
\end{align*}
La estructura de los numerales de Church favorece los mecanismos que añaden aplicaciones. En la codificación del sucesor fue relativamente sencillo ``revincular'' las variables ligadas en $ \cn{n} $ de tal manera que sólo se necesitaba aplicar $ x $ a $ (x^{n}\, y) $ para obtener $ (x^{n+1}\, y) $. Sin embargo, para codificar el predecesor es necesario ``eliminar'' una $ x $ de la aplicación y no hay manera sencilla de lograr esto.

Henk Barendreght, en el artículo titulado ``The Impact of Lambda Calculus in Logic and Computer Science'' \cite{Barendregt:Impact}, menciona que la codificación del predecesor en el cálculo $ λ $ fue un problema abierto. Alonzo Church pudo codificar la adición, la multiplicación y la exponenciación, sin embargo, la función predecesor resultaba ser extremadamente difícil de encontrar con sus numerales.

Stephen Kleene, estudiante de Alonzo Church, encontró la solución de la misteriosa codificación del predecesor. Sin embargo se tuvo que auxiliar de una representación alternativa para los números naturales. De acuerdo a \cite{Barendregt:Impact}, Kleene hizo uso de una codificación de pares de números $ \langle n-1,n \rangle $. Se inicia con $ \langle 0,0 \rangle $ y el sucesor de $ \langle a,b \rangle $ es $ \langle b,b+1 \rangle $. Cuando Kleene le llevó la propuesta a Church, este ya se había convencido que el cálculo $ λ $ era un sistema demasiado débil para representar el predecesor; es entonces que Church, habiendo aprendido que el predecesor era definible en el cálculo $ λ $, se convenció de que todas las funciones que eran intuitivamente computables, eran definibles en el cálculo $ λ $.

Para definir la codificación del predecesor, no se hará uso de la técnica de Kleene, pero si se introducirá otra notación para los números naturales.

Lo que hace que la estructura de los numerales de Church no sea adecuada es que no se tiene una manera sencilla de remover aplicaciones. Sin embargo, podemos considerar una modificación a los numerales de Church, de tal manera que el mecanismo para quitar y añadir aplicaciones sea sencillo.

Sea $ \cn{n} $ un numeral de Church, su estructura $ (λx\, y.x^{n}\, y) $ se modifica para que una de las apariciones de $ x $ se enlace a una variable diferente, por ejemplo $ z $, de tal manera que la cantidad de $ x $ sumada a la cantidad de $ z $ sea el número representado. Consideremos que el numeral modificado $ \cn{n}' $ tiene la última $ x $ de $ \cn{n} $ como $ z $, su definición sería:
\[ \cn{n}^{\prime} \synteq λx\, y\, z.x^{n-1}\, (z\, y) \]
En esta nueva codificación, los primeros cinco números son codificados como
\begin{align*}
  \cn{0}^{\prime} &\synteq λx\, y\, z.y \\
  \cn{1}^{\prime} &\synteq λx\, y\, z.z\, y \\
  \cn{2}^{\prime} &\synteq λx\, y\, z.x\, (z\, y) \\
  \cn{3}^{\prime} &\synteq λx\, y\, z.x\, (x\, (z\, y)) \\
  \cn{4}^{\prime} &\synteq λx\, y\, z.x\, (x\, (x\, (z\, y)))
\end{align*}
La clave de utilizar esta representación modificada está en observar que es fácil pasar de $ \cn{n}^{\prime} $ a $ \cn{n-1} $. Sea $ \cn{n}^{\prime} $ un numeral en la codificación modificada, la reducción de la aplicación $ (\cn{n}^{\prime}\, x\, y\, \bs{I}) $ resulta en $ (x^{n-1}\, y) $.

\begin{align*}
  \cn{n}^{\prime}\, x\, y\, \bs{I} \synteq &(λx\, y\, z.x^{n-1}\, (z\, y))\, x\, y\, \bs{I} \\
                              \reduce{β} &x^{n-1}\, (\bs{I}\, y) \\
                            \contract{β} &x^{n-1}\, y
\end{align*}

De tal manera que la definición del predecesor $ \cn{-}_{1} $ para la codificación de Church puede ser escrita
\[ \cn{-}_{1} \synteq λn.λx\, y.\mc{T}[n\mapsto n^{\prime}]\, x\, y\, \bs{I} \]
Donde $ \mc{T}[n\mapsto n^{\prime}] $ es una transformación que a partir de $ n $ obtiene el mísmo número pero con la codificación modificada. Con la cual es sencillo encontrar $ n-1 $.

Para construir el término $ \mc{T}[n\mapsto n^{\prime}] $ se debe encontrar una manera de contar desde $ 0 $ hasta $ n $ en la codificación modificada. Esto se puede lograr utilizando la interpretación de los numerales de Church como operadores de repetición. Si se construye una codificación del sucesor $ \cn{+}_{1}^{\prime} $ para los numerales modificados, entonces $ \mc{T}[n\mapsto n^{\prime}] $ puede ser definida como $ n $ aplicaciones de $ \cn{+}^{\prime} $ con el caso base $ \cn{0}^{\prime} $, es decir
\[ \mc{T}[n\mapsto n^{\prime}] \synteq n\, \cn{+}_{1}^{\prime}\, \cn{0}^{\prime} \]
El problema de codificar el predecesor se reduce ahora a la construcción del sucesor de un número con la codificación modificada. Esta construcción resulta ser casi tan sencilla como la codificación de $ \cn{+}_{1} $. Primero se analiza cómo cambia la estructura de la codificación de $ \cn{n}^{\prime} $ a $ \cn{n+1}^{\prime} $:
\begin{align*}
  \cn{+}_{1}^{\prime}\, \cn{0}^{\prime} && \synteq && \cn{+}_{1}^{\prime}\, (λx\, y\, z.y) && \reduce{β} && λx\, y\, z.z\, y \\
  \cn{+}_{1}^{\prime}\, \cn{1}^{\prime} && \synteq && \cn{+}_{1}^{\prime}\, (λx\, y\, z.z\, y) && \reduce{β} && λx\, y\, z.x(z\, y)\\
  \cn{+}_{1}^{\prime}\, \cn{2}^{\prime} && \synteq && \cn{+}_{1}^{\prime}\, (λx\, y\, z.x\, (z\, y)) && \reduce{β} && λx\, y\, z.x(x(z\, y))\\
  \cn{+}_{1}^{\prime}\, \cn{3}^{\prime} && \synteq && \cn{+}_{1}^{\prime}\, (λx\, y\, z.x\, (x\, (z\, y))) && \reduce{β} && λx\, y\, z.x\, (x\, (x\, (z\, y)))
\end{align*}

Cuando se computa el sucesor de $ \cn{0}^{\prime} $, la única variable enlazada es $ y $ y el resultado es $ z\, y $, por lo tanto, la $ y $ será sustituida por $ (z\, y) $ en el sucesor. Cuando se computa el sucesor de $ \cn{1}^{\prime} $, se tienen variables enlazadas $ z $ y $ y $, si $ y $ es sustituido por $ (z\, y) $, la $ z $ deberá ser sustituida por $ x $ en el sucesor. Cuando se computa el sucesor de $ \cn{2}^{\prime} $, se tienen variables enlazadas $ x $, $ z $ y $ y $, si se suponen las sustituciones de los otros dos casos, la $ x $ deberá ser sustituida por $ x $ en el sucesor. Para corroborar que estas sustituciones son correctas para un caso concreto, se considera la aplicación $ (\cn{3}^{\prime}\, x\, (z\, y)\, x) $:
\begin{align*}
  \cn{3}^{\prime}\, x\, (z\, y)\, x &\reduce{β} x^{2}\, (x\, (z\, y)) \\
                                    &\synteq x^{3}\, (z\, y)
\end{align*}
Y en general, aplicar estos términos a un numeral modificado $ \cn{n}^{\prime} $ se reduce a
\begin{align*}
  \cn{n}^{\prime}\, x\, (z\, y) x &\reduce{β} x^{n-1}\, (x\, (z\, y)) \\
                                &\synteq x^{n}\, (z\, y)
\end{align*}
Término que corresponde al cuerpo del numeral $ \cn{4}^{\prime} $. Por lo tanto, la codificación de la operación $ \cn{+}_{1}^{\prime} $ es:
\[ \cn{+}_{1}^{\prime} \synteq λn^{\prime}.λx\, y\, z.n^{\prime}\, x\, (z\, y)\, x \]
Con estas piezas, la codificación de la operación predecesor se define como
\begin{equation}
  \label{eq:numeral:pred2}
  \cn{-}_{1} \synteq λn.λx\, y.(n\, \cn{+}_{1}^{\prime}\, \cn{0}^{\prime})\, x\, y\, \bs{I}
\end{equation}
Para corroborar que la codificación computa el resultado deseado, consideremos los casos $ \cn{n} \synteq \cn{0} $ y $ \cn{n} \synteq \cn{k} $ en la reducción de la aplicación $ (\cn{-}_{1}\, \cn{n}) $:
\begin{align*}
  \cn{-}_{1}\, \cn{0} \contract{β} &λx\, y.(\cn{0}\, \cn{+}_{1}^{\prime}\, \cn{0}^{\prime})\, x\, y\, \bs{I} \\
                        \reduce{β} &λx\, y.(\bs{I} \cn{0}^{\prime})\, x\, y\, \bs{I} \\
                      \contract{β} &λx\, y.\cn{0}^{\prime}\, x\, y\, \bs{I} \\
                           \synteq &λx\, y.(λx\, y\, z.y)\, x\, y\, \bs{I} \\
                        \reduce{β} &λx\, y.y \\
                           \synteq &\cn{0} \\
  \cn{-}_{1}\, \cn{k} \contract{β} &λx\, y.(\cn{k}\, \cn{+}_{1}^{\prime}\, \cn{0}^{\prime})\, x\, y\, \bs{I} \\
                        \reduce{β} &λx\, y.\cn{+}_{1}^{\prime k}\, \cn{0}^{\prime}\, x\, y\, \bs{I} \\
                        \reduce{β} &λx\, y.\cn{k}^{\prime}\, x\, y\, \bs{I} \\
                           \synteq &λx\, y.(λx\, y\, z.x^{k-1}\, (z\, y))\, x\, y\, \bs{I} \\
                        \reduce{β} &λx\, y.x^{k-1}\, (\bs{I}\, y) \\
                      \contract{β} &λx\, y.x^{k-1}\, y \\
                           \synteq &\cn{k-1}
\end{align*}

Teniendo la codificación del predecesor se puede plantear una codificación de la resta, después de todo, restar $ n $ de $ m $ es aplicar la función predecesor $ n $ veces a $ m $.
\begin{equation}
  \label{eq:numeral:sub2}
  \cn{-} \synteq λm\, n.n\, \cn{-}_{1}\, m
\end{equation}
Para corroborar que esta definición es correcta, se reduce la aplicación $ (\cn{-}\, \cn{m}\, \cn{n}) $ la cuál deberá resultar en la codificación del número $ m-n $.
\begin{align*}
  \cn{-}\, \cn{m}\, \cn{n} &\synteq (λm\, n.n\, \cn{-}_{1}\, m)\, \cn{m}\, \cn{n} \\
                           &\reduce{β} \cn{n}\, \cn{-}_{1}\, \cn{m} \\
                           &\reduce{β} \cn{-}_{1}^{n}\, \cn{m} \\
                           &\synteq \cn{-}_{1}^{n-1}\, (\cn{-}_{1}\, \cn{m}) \\
                           &\reduce{β} \cn{-}_{1}^{n-1}\, \cn{m-1} \\
                           &... \\
                           &\reduce{β} \cn{-}_{1}^{n-n}\, \cn{m-n} \\
                           &\synteq \cn{m-n}
\end{align*}

La idea de definir operaciones aritméticas complejas en función de otras más sencillas también puede ser aplicada a la definición de la división. Sin embargo, no podemos definir $ m/n $ como $ (\cn{n}\, (\cn{-}\, \cn{m})\, \cn{0}) $ ya que el procedimiento de reducción consistiría en primero calcular $ m-0 = m $, después calcular $ m-m = 0 $, después calcular $ m-0=m $, y así sucesivamente hasta dejar de repetir el procedimiento. En pocas palabras, $ (\cn{n}\, (\cn{-}\, \cn{m})\, \cn{0}) $ se reduce a $ \cn{m} $ cuando $ \cn{n} $ es impar y a $ \cn{0} $ cuando $ \cn{n} $ es par.

La idea de definir la división como repetición de restas se puede interpretar considerando el siguiente ejemplo. El resultado de dividir a 12 en 3 es 4 porque 4 es la \emph{cantidad} de veces que hay que restarle al 12 el 3 hasta llegar al cero:
\[ 12-3-3-3-3=12-4\times 3=12-12=0 \]
Esto presenta un aumento de complejidad en la codificación de la operación, no solo se debe restar hasta llegar a $ \cn{0} $, también se debe mantener un conteo de la cantidad de veces que se ha restado.

Para el caso de la función logaritmo, el problema es similar a la división. El resultado de calcular el logaritmo base 3 de 81 es 4 porque 4 es la cantidad de veces que hay que dividir al 81 en 3 hasta llegar al 1 (el caso trivial para la división):
\[ (((81/3)/3)/3)/3 = 81/3^{4} = 81/81 = 1 \]

Para el caso de la función raíz, el problema es aún más grande que con la división y el logaritmo. Ya que $ \sqrt[n]{m} $ es calculada como la base a la que debemos elevar por $ n $ para obtener $ m $. Poniendo el mismo ejemplo que con el logaritmo, la idea de la raíz es partir del $ 81 $ y el $ 4 $ y calcular el número al que se tiene que dividir $ 81 $, $ 4 $ veces, hasta llegar al $ 1 $. El problema con esto es que no se puede reducir el problema a uno más sencillo basándonos únicamente en contar y en la operación de división. Sin embargo, es posible utilizar métodos de aproximación y calcular $ \sqrt[4]{81} $ intentando divisiones entre 1 y fallar, luego entre 2 y fallar, luego entre 3 y encontrar que es la respuesta.

En la siguiente subsección se construye la manera en la que se podrán plantear mecanismos más complejos de cómputo y lograr definir las operaciones de división, logaritmo y raíz.



\subsection{Iteración}
\label{sec:iteracion}

La técnica de utilizar los números naturales como mecanismo de repetición es de mucha utilidad. En el diseño de algoritmos, la repetición es usualmente representada con mecanismos de \emph{iteración}, usualmente en cada paso de la iteración una o más variables en el contexto del algoritmo cambian, hasta obtener en una de ellas el resultado final.

Por ejemplo, un algoritmo para computar el factorial de un número $ n $ puede ser expresado de manera iterativa como:

\begin{algorithm}
  \caption{Factorial de $ n $}
  \label{alg:factorial}
  \begin{algorithmic}
    \REQUIRE $ n \in \mathbb{N} $
    \ENSURE $ n! $
    \STATE $ r \leftarrow copia(n) $
    \STATE $ a \leftarrow 1 $
    \WHILE{$ r \centernot= 0 $}
    \STATE $ a \leftarrow a \times r $
    \STATE $ r \leftarrow r - 1 $
    \ENDWHILE
    \RETURN $ a $
  \end{algorithmic}
\end{algorithm}

El mecanismo utilizado en este algoritmo para iterar es el \emph{mientras}, éste se acompaña con una condición, si la condición se satisface, el cuerpo del \emph{mientras} es ejecutado, de lo contrario, se detiene la iteración y se prosigue con el resto de los pasos del algoritmo.

Por fortuna, este tipo de algoritmos pueden ser codificados de manera sencilla en el cálculo $ λ $. En lugar de codificar el \emph{mientras}, se puede utilizar $ \cn{n} $ para repetir un procedimiento, después de todo, el algoritmo inicia con $ r = n $ y en cada iteración $ r $ disminuye en 1, por lo que se realizan $ n $ iteraciones. La variable $ a $ inicia en 1 y en cada iteración es multiplicada por $ r $, al terminar los pasos del algoritmo, el valor de $ a $ es el resultado $ n! $.

La clave para codificar este algoritmo es determinar los valores que representan el cómputo. Las variables $ r $ y $ a $ describen el estado del cómputo en el caso trivial $ n = 0 $. Cuando $ n \centernot= 0 $, en cada iteración, toda la información del cómputo sigue estando en las variables $ r $ y $ a $, aún más, los valores de estas variables describen una propiedad interesante del algoritmo: Antes y después de cada iteración, se cumple que $ n! = a \times r! $.

Ya que $ \cn{n} $ será el mecanismo de iteración, se deben determinar dos cosas: el término que será aplicado repetidas veces $ P $ y el término inicial $ B $, de tal manera que $ (\cn{n}\, P\, B) $ se $ β $-reduzca al estado final del cómputo.

El \emph{estado del cómputo} consiste de la codificación de $ r $ y $ a $ como numerales de Church. Lo que debemos codificar para representar el algoritmo es una manera de crear un estado a partir de dos números y una manera de obtener el primer y el segundo valor de un estado, es decir, el \emph{constructor} y los \emph{selectores}. Ya que son únicamente dos valores, se pueden utilizar las codificaciones $ \bs{T} $ y $ \bs{F} $ para \emph{decidir} el valor del estado que se desea obtener.

Sean $ \cn{r} $ y $ \cn{a} $ dos numerales, el término $ M \synteq λp.p\, \cn{r}\, \cn{a} $ puede representar al estado, de tal manera que $ (M\, \bs{T}) \reduce{β} \cn{r} $ y $ (M\, \bs{F}) \reduce{β} \cn{a} $.

Para construir un estado, basta con codificar un término $ λ $ que al ser aplicado a dos numerales, se reduzca a un término como $ M $. Se define el constructor $ S $ de un estado como
\begin{equation}
  \label{eq:iter:const1}
  S \synteq λn_{1}\, n_{2}.λp.p\, n_{1}\, n_{2}
\end{equation}
Para seleccionar de un determinado valor a partir de un estado, se plantean abstracciones $ λ $ que al ser aplicadas a un término como $ M $, se reduzcan al valor deseado. Se definen los selectores $ S_{r} $ y $ S_{a} $ de un estado como
\begin{equation}
  \label{eq:iter:selec1r}
  S_{r} \synteq λs.s\, \bs{T}
\end{equation}
\begin{equation}
  \label{eq:iter:selec1a}
  S_{a} \synteq λs.s\, \bs{F}
\end{equation}

Con estas codificaciones, se facilita la escritura de los términos $ P $ y $ B $. El estado inicial es $ r = n $ y $ a = 1 $, por lo que, conociéndo un numeral $ \cn{n} $, el término $ B $ se codifica como
\begin{equation}
  \label{eq:iter:base1}
  B \synteq S\, \cn{n}\, \cn{1}
\end{equation}
El término $ P $ debe ser una abstracción que sea aplicada a un estado y sea reducida a un estado, ya que en el cálculo $ λ $ no hay una noción de asignación, en cada repetición de la aplicación de $ P $ se crea un estado nuevo con sus valores en función del estado anterior. Si $ \cn{r} $ y $ \cn{a} $ son los valores del estado previo, el término $ P $ debe reducirse a un estado en donde el primer elemento sea $ (\cn{-_{1}}\, \cn{r}) $ y el segundo sea $ (\cn{\times}\, \cn{a}\, \cn{r}) $. Utilizando los selectores $ S_{r} $ y $ S_{a} $, el término $ P $ se codifica como

\begin{equation}
  \label{eq:iter:proc1}
  P \synteq λs.S\, (\cn{-}_{1}\, (S_{r}\, s))\, (\cn{\times}\, (S_{a}\, s)\, (S_{r}\, s))
\end{equation}

La codificación completa del algoritmo factorial, utilizando las definiciones \eqref{eq:iter:const1}, \eqref{eq:iter:selec1r} y \eqref{eq:iter:selec1a}, se define
\begin{equation}
  \label{eq:iter:fact1}
  \cn{!} \synteq λn.n\, (λs.S (\cn{-}_{1}\, (S_{r}\, s))\, (\cn{\times}\, (S_{a}\, s)\, (S_{r}\, s)))\, (S\, n\, \cn{1})
\end{equation}

Para poder integrar esta definición y componer expresiones algebraicas con las codificaciones de las operaciones elementales, la aplicación de $ \cn{!} $ a un numeral $ \cn{n} $ debe $ β $-reducirse al numeral $ \cn{n!} $, sin embargo, al obtener la forma normal de $ (\cn{!}\, \cn{n}) $ el resultado es un estado equivalente a $ (S\, \cn{0}\, \cn{1}) $ si $ \cn{n} \convertible{β} \cn{0} $ o a $ (S\, \cn{1}\, \cn{n!}) $ en otro caso. Por lo tanto $ \cn{!} $ debe estar codificado de tal manera que después de computar el algoritmo, seleccione el segundo elemento del estado resultante.
\begin{equation}
  \label{eq:iter:fact2}
  \cn{!} \synteq λn.S_{a}\, (n\, (λs.S\, (\cn{-}_{1}\, (S_{r}\, s))\, (\cn{\times}\, (S_{a}\, s)\, (S_{r}\, s)))\, (S\, n\, \cn{1}))
\end{equation}

Lamentablemente, no siempre es posible expresar algoritmos en donde todo el estado se resuma en dos valores. Sin embargo, utilizando una extensión similar a la \autoref{sec:boolean-extensiones} se puede generalizar la codificación de algoritmos cuyos estados tienen $ n $ componentes.

Sean $ {}_{n}v_{1}, {}_{n}v_{2}, ..., {}_{n}v_{n} $, términos de la forma

\[ {}_{n}v_{i} \synteq λx_{1}\, ...\, x_{n}.x_{i} \]

El constructor $ {}_{n}S $ de un estado con $ n $ valores se define como
\begin{equation}
  \label{eq:state-cons}
  {}_{n}S \synteq λx_{1}\, ...\, x_{n}.λp.p\, x_{1}\, ...\, x_{n}
\end{equation}
El selector del $ i $-ésimo valor de un estado, se define
\begin{equation}
  \label{eq:state-selec}
  {}_{n}S_{i} \synteq λs.s\, v_{i}
\end{equation}

La codificación de las operaciones aritméticas de división, logaritmo y raíz, serán basadas en algoritmos similares al del factorial. Para tener la habilidad de escribir algoritmos aritméticos, es necesario complementar los numerales de Church con predicados, por ejemplo, para determinar si un numeral es cero o si dos numerales son iguales.

El primer predicado que se define es el que a partir de un numeral $ \cn{n} $, determina si es el $ \cn{0} $. Este predicado es muy utilizado ya que en muchos algoritmos, la condición de paro es cuando un valor numérico tiene el valor cero. A partir de la estructura de $ \cn{n} \synteq (λx\, y.x^{n}\, y) $ se puede encontrar una manera de aplicarle dos términos $ P $ y $ Q $ a $ \cn{n} $, tal que al reducirse resulte en $ \bs{T} $ si $ n=0 $ y a $ \bs{F} $ si $ n>0 $. El $ \cn{0} $ no tiene aplicaciones internas y simplemente es reducido al segundo término al que fue aplicado, por lo tanto $ (\cn{n}\, P\, \bs{T}) $ debe reducirse a $ \bs{T} $ cuando $ n=0 $. Cuando el numeral es mayor a cero, la primera aplicación de $ (P\, \bs{T}) $ debe reducirse a $ \bs{F} $ y las siguientes aplicaciones deben ser $ (P\, \bs{F}) $ y reducirse también a $ \bs{F} $. Un término que al ser aplicado a cualquier término es reducido a $ \bs{F} $ es $ (\bs{K}\, \bs{F}) $. Por lo tanto, la codificación de este predicado se define
\begin{equation}
  \label{eq:numeral:pred0}
  \cn{0}_{?} \synteq λn.n\, (\bs{K}\, \bs{F})\, \bs{T}
\end{equation}

Para corroborar que este predicado es correcto al ser aplicado a un numeral, se reducen las siguientes aplicaciones:
\begin{align*}
  \cn{0}_{?}\, \cn{0} \synteq &(λn.n\, (\bs{K}\, \bs{F})\, \bs{T})\, \cn{0} \\
                \contract{β} &\cn{0}\, (\bs{K}\, \bs{F})\, \bs{T} \\
                  \reduce{β} &(\bs{K}\, \bs{F})^{0}\, \bs{T} \\
                     \synteq &\bs{T} \\
  \cn{0}_{?}\, \cn{n} \synteq &(λn.n\, (\bs{K}\, \bs{F})\, \bs{T})\, \cn{n} \\
                \contract{β} &\cn{n}\, (\bs{K}\, \bs{F})\, \bs{T} \\
                  \reduce{β} &(\bs{K}\, \bs{F})^{n}\, \bs{T} \\
                     \synteq &(\bs{K}\, \bs{F})^{n-1}\, (\bs{K}\, \bs{F}\, \bs{T}) \\
                  \reduce{β} &(\bs{K}\, \bs{F})^{n-1}\, \bs{F} \\
                     \synteq &(\bs{K}\, \bs{F})^{n-2}\, (\bs{K}\, \bs{F}\, \bs{F}) \\
                  \reduce{β} &(\bs{K}\, \bs{F})^{n-2}\, \bs{F} \\
                          ...& \\
                  \reduce{β} &(\bs{K}\, \bs{F})^{n-n}\, \bs{F} \\
                     \synteq &\bs{F}
\end{align*}

El siguiente predicado que se define es el que a partir de dos numerales $ \cn{m} $ y $ \cn{n} $, determina si $ m \leq n $. La codificación de este predicado se basa en la observación de al restar $ \cn{n} $ de $ \cn{m} $ el resultado será $ \cn{0} $ si $ \cn{m} $ es menor o igual a $ \cn{n} $. Por lo tanto, la codificación de este predicado se define
\begin{equation}
  \label{eq:numeral:predleq}
  \cn{\leq}_{?} \synteq λm\, n.\cn{0}_{?}\, (\cn{-}\, m\, n)
\end{equation}

Ya que estos predicados son reducidos a valores booleanos cuando se aplican a numerales de Church, pueden ser combinados utilizando las operaciones del álgebra booleana. Otros predicados pueden codificarse haciéndo uso de propiedades numéricas.
\begin{itemize}
\item Si $ n \leq m $, entonces $ m \geq n $:
  \begin{equation}
    \label{eq:numeral:predgeq}
    \cn{\geq}_{?} \synteq λm\, n.\cn{\leq}_{?}\, n\, m
  \end{equation}
\item Si $ m \leq n $ y $ m \geq n $, entonces $ m = n $:
  \begin{equation}
    \label{eq:numeral:predeq}
    \cn{=}_{?} \synteq λm\, n.\bs{\land}\, (\cn{\leq}_{?}\, m\, n)\, (\cn{\geq}_{?}\, m\, n)
  \end{equation}
\item Si $ m \centernot\leq n $, entonces $ m > n $:
  \begin{equation}
    \label{eq:numeral:predgt}
    \cn{>}_{?} \synteq λm\, n.\bs{\lnot}\, (\cn{\leq}_{?}\, m\, n)
  \end{equation}
\item Si $ m \centernot\geq n $, entonces $ m < n $:
  \begin{equation}
    \label{eq:numeral:predlt}
    \cn{<}_{?} \synteq λm\, n.\bs{\lnot}\, (\cn{\geq}_{?}\, m\, n)
  \end{equation}
\end{itemize}

A la vez, a partir de estos predicados se pueden definir otros términos muy utilizados en algoritmos aritméticos:

\begin{itemize}
\item Si $ m < n $, entonces $ min(m,n)=m $, de lo contrario $ min(m,n)=n $:
  \begin{equation}
    \label{eq:numeral:min}
    \cn{min} \synteq λm\, n.\bs{\prec}\, (\cn{<}_{?}\, m\, n) m\, n
  \end{equation}
\item Si $ m > n $, entonces $ max(m,n)=m $, de lo contrario $ max(m,n)=n $:
  \begin{equation}
    \label{eq:numeral:max}
    \cn{max} \synteq λm\, n.\bs{\prec}\, (\cn{>}_{?}\, m\, n) m\, n
  \end{equation}
\end{itemize}

Para codificaciones de $ \cn{min} $ y $ \cn{max} $ que se reduzcan correctamente al ser aplicados a más de dos numerales, se puede utilizar una técnica como la mostrada en la \autoref{defn:op-bool-bin-lambda}.

Con estos nuevos términos, es más amena la codificación de algoritmos, de hecho, la manera de codificarlos es casi tan sencillo como programar los algoritmos en lenguajes aptos para la programación funcional como \texttt{Lisp}, \texttt{ML} o \texttt{Haskell}.

La estrategia general para la codificación de las operaciones faltantes se basa en la observación de que la sustracción es la función inversa de la adición, la división es la función inversa de la multiplicación y el logaritmo y la raíz son las inversas de la exponenciación. Sean $ m $, $ n $ y $ k $ tres números naturales
\begin{align*}
  m-n = x &\iff m = x+n \\
  m/n = x &\iff m = x\times n\\
  \log_{n}m = x &\iff m = n^{x} \\
  \sqrt[n]m = x &\iff x^{n}
\end{align*}
Estas operaciones deben de ser tratadas con mucho cuidado ya que no son operaciones cerradas, es decir, existen números naturales $ m $ y $ n $ tal que, para alguna operación $ \odot $ de las cuatro mencionadas, $ m \odot n $ no es un número natural. Por ejemplo, con la sustracción se pueden calcular números negativos, con la división números racionales, con la raíz números reales y con el logaritmo no únicamente números reales, también el valor $ -\infty $. Por lo tanto, las codificaciones que se definen, serán versiones discretas, cerradas y por lo tanto inexactas de las que usualmente se utilizan.

El algoritmo en el que se basan estas cuatro operaciones considera dos números $ m $ y $ n $; una operación inversa $ \odot $; una condición de trivialidad $ {\rm err} $; y un valor de trivialidad $ t $. La idea del algoritmo es regresar $ t $ cuando $ {\rm err}(m,n) $ es verdadero, de lo contrario iterar a partir de $ x = 0 $, calculando el resultado $ x \odot n $ hasta obtener un valor mayor o igual a $ m $, en donde $ x $ será el cálculo de la operación.

\begin{algorithm}
  \caption{Cálculo de $ m \odot^{\, -1} n $}
  \label{alg:inversas}
  \begin{algorithmic}
    \REQUIRE $ m,\ n,\ t \in \mathbb{N},\ \odot \colon \mathbb{N} \times \mathbb{N} \to \mathbb{N},\ err \colon \mathbb{N} \times \mathbb{N} \to \{ \mathrm{verdadero},\ \mathrm{falso} \} $
    \ENSURE $ m \odot^{\, -1} n $
    \IF{$ err(m,n) $}
    \RETURN $ t $
    \ELSE
    \STATE $ x \leftarrow 0 $
    \STATE $ a \leftarrow x \odot n $
    \WHILE{$ a < m $}
    \STATE $ a \leftarrow x \odot n $
    \STATE $ x \leftarrow x + 1 $
    \ENDWHILE
    \RETURN $ a $
    \ENDIF
  \end{algorithmic}
\end{algorithm}

Para la sustracción la condición de trivialidad es $ m \leq n $ y el valor de trivialidad $ 0 $. Para la división la condición de trivialidad es $ m < n $ y el valor de trivialidad $ 1 $. Para el logaritmo la condición de trivialidad es $ m=0 $ y el valor de trivialidad $ 0 $. Para la raíz la condición de trivialidad es $ n=0 $ y el valor de trivialidad $ 1 $.

Una limitación que tiene la iteración en base a los numerales es que se debe conocer la cantidad de veces que se repetirá un proceso, en el caso del \autoref{alg:inversas}, la condición de paro es verificada de manera dinámica, mientras cuando los cálculos se están realizando. Sin embargo, es posible establecer una cota superior a la cantidad de pasos en base a las propiedades de las operaciones establecidas.

En el caso de la sustracción, $ m-n $ debe ser un número natural, y ya que la iteración sucede cuando $ m>n $, entonces $ x\leq m $. En el caso de la división, $ m/n $ debe ser un número natural, y ya que la iteración sucede cuando $ m\geq n $, entonces $ x\leq m $. En el caso del logaritmo, $ \log_{n}m $ debe ser un número natural, y ya que la iteración sucede cuando $ m\geq n $, entonces $ x\leq m $. En el caso de la raíz, ya que $ m $, $ n $ y $ x $ siempre serán naturales, y la iteración sucede cuando $ n \centernot= 0 $, entonces $ x\leq m $. Por lo tanto, se puede utilizar a $ m $ para determinar la cantidad de iteraciones, de tal manera que cuando la condición $ a < m $ se cumpla, siempre se reduzca al valor de $ a $. El \autoref{alg:inversas2} es una modificación del \autoref{alg:inversas} para que su codificación sea más directa.

\begin{algorithm}
  \caption{Cálculo de $ m \odot^{\, -1} n $}
  \label{alg:inversas2}
  \begin{algorithmic}
    \REQUIRE $ m,\ n,\ t \in \mathbb{N},\ \odot \colon \mathbb{N} \times \mathbb{N} \to \mathbb{N},\ err \colon \mathbb{N} \times \mathbb{N} \to \{ \mathrm{verdadero},\ \mathrm{falso} \} $
    \ENSURE $ m \odot^{\, -1} n $
    \IF{$ err(m,n) $}
    \RETURN $ t $
    \ELSE
    \STATE $ x \leftarrow 0 $
    \STATE $ a \leftarrow x \odot n $
    \FOR{iteraciones \TO $ m $}
    \IF{$ a < m $}
    \STATE $ a \leftarrow x \odot n $
    \STATE $ x \leftarrow x + 1 $
    \ELSE
    \STATE $ a \leftarrow a $
    \ENDIF
    \ENDFOR
    \RETURN $ a $
    \ENDIF
  \end{algorithmic}
\end{algorithm}

Para acortar la definición de la codificación del algoritmo de inversa, se definen los siguientes términos auxiliares:
\begin{align*}
  \bs{:} &\synteq {}_{2}S \\
  \bs{A} &\synteq {}_{2}S_{1} \\
  \bs{X} &\synteq {}_{2}S_{2}
\end{align*}

La codificación del \autoref{alg:inversas2} es
\begin{align}
  \label{eq:numeral:inversas}
  \cn{\odot}^{\, -1} \synteq λ\odot\, e\, t.λm\, n.\bs{\prec}\, (e\, m\, n) \ \ \ \ \ \ \ \ \ &\\
  t \ \ \ \ \ \ \ \ \ \ \ \ \ \ \ \ \ \ &\nonumber \\
  (\bs{X}\, (m\, (λs.\bs{\prec}\, &(\cn{<}_{?}\, (\bs{A}\, s)\, m)\nonumber \\
  &(\bs{:}\, (\odot\, (\bs{X}\, s)\, n)\, (\cn{+}_{1}\, (\bs{X}\, s)))\nonumber \\
  &(\bs{:}\, (\bs{A}\, s)\, (\bs{X}\, s)))\nonumber \\
  (\bs{:}\, (\odot\, \cn{0}&\, n)\, \cn{0})))\nonumber
\end{align}

La definición es difícil de leer y comprender si se escribe en un solo renglón, por ello, se utilizaron varios renglones para escribir el término, la convención para escribirlo fue: los saltos de renglón se dan en cada término condicional para que la condición esté en el mismo renglón que $ \bs{\prec} $, el consecuente en el siguiente renglón horizontalmente alineado con la condición y la alternativa en el siguiente renglón horizontalmente alineado con el consecuente. También hay un salto de renglón en la aplicación de $ m $, de tal manera que los dos términos a los que es aplicado estén alineados horizontalmente.

Utilizando el término $ \cn{\odot}^{\, -1} $ y las condiciones y valores de trivialidad mencionados anteriormente, se definen las codificaciones de las funciones inversas
\begin{equation}
  \label{eq:numeral:subs-inv}
  \cn{-} \synteq \cn{\odot}^{\, -1}\, \cn{+}\, \cn{\leq}_{?}\, \cn{0}
\end{equation}
\begin{equation}
  \label{eq:numeral:div-inv}
  \cn{/} \synteq \cn{\odot}^{\, -1}\, \cn{\times}\, \cn{<}_{?}\, \cn{1}
\end{equation}
\begin{equation}
  \label{eq:numeral:log-inv}
  \cn{\log} \synteq \cn{\odot}^{\, -1}\, \cn{\uparrow} (λm\, n.\cn{0}_{?}\, m) \cn{0}
\end{equation}
\begin{equation}
  \label{eq:numeral:root-inv}
  \cn{\mathrm{root}} \synteq \cn{\odot}^{\, -1} (λx\, n.\cn{\uparrow}\, n\, x) (λm\, n.\cn{0}_{?}\, n) \cn{1}
\end{equation}

El \autoref{alg:inversas2} describe un método de aproximación bastante pobre e ineficiente, sin embargo, es posible codificar otras maneras de calcular estas operaciones utilizando las técnicas vistas hasta el momento, siempre y cuando se establezca una cota superior antes de realizar las iteraciones.

\subsection{Hiperoperaciones}
\label{sec:hiperoperaciones}

Las definiciones \eqref{eq:numeral:sum2}, \eqref{eq:numeral:mul2}, \eqref{eq:numeral:exp2} describen de manera clara y consisa la relación entre la adición, la multiplicación y la exponenciación. Conociendo una cantidad de repeticiones, una operación de agregación y el valor neutro de dicha agregación, fué posible definir una operación aritmética en función de otra operación más simple, hasta tener el caso base con la adición.

En el artículo ``Mathematics and Computer Science: Coping with Finiteness'' \cite{Knuth:Arrow}, Donald E. Knuth introduce la notación de flecha para expresar números finitos gigantescos. La notación de flecha ya fue utilizada en el término de la exponenciación, ya que, de acuerdo a Knuth, $ x \mathbin{\uparrow} n = x^{n} $.

La definición de la notación de flecha es la siguiente:
\begin{align*}
  x \mathbin{\uparrow} n &= x^{n} \\
  x \mathbin{\uparrow^{k}} n = x \underbrace{\uparrow ... \uparrow}_\text{$ k $ flechas} n &= \underbrace{(x \overbrace{\uparrow ... \uparrow}^\text{$ k-1 $ flechas} (x \overbrace{\uparrow ... \uparrow}^\text{$ k-1 $ flechas} (\cdots \overbrace{\uparrow ... \uparrow}^\text{$ k-1 $ flechas} x) \cdots))}_\text{$ n $ veces}
\end{align*}

Esta notación introduce una secuencia infinita de operaciones cuya definición es recursiva y consistente con las definiciones de la adición, multiplicación y exponenciación presentadas. Sea $ \cn{\mc{H}(i)} $ el $ i $-ésimo elemento de esta secuencia
\begin{align}
  \label{eq:hyper}
  \cn{\mc{H}(1)} \synteq \cn{\uparrow} &\synteq (λm\, n.n\, (\cn{\times}\, m)\, \cn{1}) \\
  \cn{\mc{H}(i)} \synteq \cn{\uparrow}_{i} &\synteq (λm\, n.n\, (\cn{\mc{H}(i-1)}\, m)\, \cn{1}) \nonumber
\end{align}
El valor neutro siempre es $ \cn{1} $ ya que, para todo $ k > 1 $
\[ x \mathbin{\uparrow^{k}} 1 = x \mathbin{\uparrow^{k-1}} 1 \]
El primer paso para codificar estas secuencia es generalizar la estructura del término $ \cn{\uparrow} $. Ya que en la definición \eqref{eq:hyper} lo único que cambia en la estructura es la operación previa, se puede colocar una variable enlazada $ f $ que denote la operación anterior. De esta manera, la codificación de un término que dada una codificación de $ \mc{H}(i) $ es reducido a $ \mc{H}(i+1) $ es
\[ λf.λm\, n.n(f\, m) \cn{1} \]
Se definen los términos $ \cn{\uparrow}_{i} $ de la siguiente manera
\begin{align*}
  \cn{\uparrow}_{1} &\synteq (λf.λm\, n.n\, (f\, m)\, \cn{1})\, \cn{\times}\, \reduce{β} λm\, n.n\, (\cn{\times}\, m)\, \cn{1} \\
  \cn{\uparrow}_{2} &\synteq (λf.λm\, n.n\, (f\, m)\, \cn{1})\, \cn{\uparrow}_{1} \reduce{β} λm\, n.n\, (\cn{\uparrow}_{1}\, m)\, \cn{1} \\
                    &...\\
  \cn{\uparrow}_{i} &\synteq (λf.λm\, n.n\, (f\, m)\, \cn{1})\, \cn{\uparrow}_{i-1} \reduce{β} λm\, n.n\, (\cn{\uparrow}_{i-1}\, m)\, \cn{1}
\end{align*}
Este procedimiento es correcto, sin embargo, en caso que se requiera utilizar el término $ \cn{\uparrow}_{1000} $ se deberán de escribir manualmente las definiciones de $ \cn{\uparrow}_{999} $ hasta $ \cn{\uparrow}_{1} $ lo cuál es inconveniente y tedioso. Con las técnicas que se han desarrollado previamente, se puede construir un término que dado un numeral $ \cn{n} $, se $ β $-reduzca al $ n $-ésimo operador de la secuencia, es decir, codificar $ \cn{\mc{H}} $, tal que $ (\cn{\mc{H}}\, \cn{n}) \reduce{β} \cn{\mc{H}(n)} $ donde
\begin{equation}
  \label{eq:hyper-flow}
  \cn{\mc{H}} \synteq λn.n (λf.λm\, n.n(f\, m) \cn{1}) \cn{\times}
\end{equation}

Para corroborar que la ecuación \eqref{eq:hyper-flow} es correcta, se computan las reducciones
\begin{align*}
  \cn{\mc{H}}\, \cn{1} &\synteq (λn.n\, (λf.λm\, n.n\, (f\, m)\, \cn{1})\, \cn{\times})\, \cn{1} \\
                       &\contract{β} \cn{1}\, (λf.λm\, n.n\, (f\, m)\, \cn{1})\, \cn{\times} \\
                       &\reduce{β} (λf.λm\, n.n\, (f\, m)\, \cn{1})\, \cn{\times} \\
                       &\contract{β} λm\, n.n\, (\cn{\times}\, m)\, \cn{1} \synteq \cn{\uparrow} \synteq \cn{\uparrow}_{1}
\end{align*}

\begin{align*}
  \cn{\mc{H}}\, \cn{i} &\synteq (λn.n\, (λf.λm\, n.n\, (f\, m)\, \cn{1})\, \cn{\times})\, \cn{i} \\
                       &\contract{β} \cn{i}\, (λf.λm\, n.n\, (f\, m)\, \cn{1})\, \cn{\times} \\
                       &\reduce{β} (λf.λm\, n.n\, (f\, m)\, \cn{1})^{i}\, \cn{\times} \\
                       &\synteq (λf.λm\, n.n\, (f\, m)\, \cn{1})^{i-1}\, ((λf.λm\, n.n\, (f\, m)\, \cn{1})\, \cn{\times}) \\
                       &\contract{β} (λf.λm\, n.n\, (f\, m)\, \cn{1})^{i-1}\, (λm\, n.n\, (\cn{\times}\, m)\, \cn{1}) \\
                       &\synteq (λf.λm\, n.n\, (f\, m)\, \cn{1})^{i-1}\, \cn{\uparrow}_{1} \\
                       &... \\
                       &\reduce{β} (λf.λm\, n.n\, (f\, m)\, \cn{1})^{i-i}\, \cn{\uparrow}_{i} \synteq \cn{\uparrow}_{i}
\end{align*}

Ejemplos de reducciones concretas no serán dados debido a la naturaleza de la secuencia, tan solo $ 5 \mathbin{\uparrow\uparrow} 3 $ tiene 2185 dígitos y $ 5 \mathbin{\uparrow\uparrow} 4 $ tiene $ 1335740483872137\times 10^{2169} $ dígitos.

\section{Procesos recursivos}
\label{sec:procesos-recursivos}

Para complementar el mecanismo de iteración presentado en la \autoref{sec:iteracion}, se plantea la manera de codificar algoritmos que describen procesos \emph{recursivos}.

La recursividad está detrás de una gran cantidad de definiciones, problemas y algoritmos en matemáticas y ciencias de la computación. La idea básica de la recursividad es plantear un concepto en términos de sí mismo \cite{knuth:Concrete}.

Un ejemplo de definición recursiva es el de términos $ λ $. Una aplicación es un término y se compone de otros dos términos, mientras que una abstracción es un término y se compone de un átomo y otro término. Incluso la definición de los números naturales, los cuales nos permiten iterar, es recursiva, ya que $ (x^{n}\, y) $ es definido como un abuso de notación de $ (x\, (x^{n-1}\, y)) $.

Hay problemas clásicos cuyas soluciones son también recursivas, por ejemplo, el rompecabezas de \emph{las torres de Hanoi} o el problema de \emph{Flavio Josefo} \cite{knuth:Concrete}. Sus soluciones consisten en suponer que el problema ya fue resuelto para una versión más simple y resolver la diferencia del problema simple al original. De esta manera, se reduce el problema hasta llegar a una versión muy simple, en donde la solución es ``trivial''.

Los algoritmos con definiciones recursivas en algunos casos son más concisos que sus contrapartes iterativas. En particular cuando los algoritmos manipulan estructuras definidas de manera recursiva, su especificación suele seguir un patrón similar a la de las estructuras que manipula. Por ejemplo, los procedimientos para encontrar los subtérminos de un término $ λ $ o calcular su longitud, por la \autoref{defn:longitud} y la \autoref{defn:subtermino}, son recursivos.

El mecanismo de iteración presentado en este trabajo está más asociado al concepto de iteración en matemáticas que en computación. La estructura de los numerales de Church, capturan la idea de la aplicación de una \emph{función iterativa} cuyo dominio y rango son el mismo conjunto, de tal manera que estas funciones se pueden componer consigo mismas.

En computación, el concepto de iteración es usualmente asociado a la manera en cómo se expresa un procedimiento; la distinción entre la iteración y otros mecanismos para codificar algoritmos se vuelve entonces en una cuestión lingüística, describiendo así la iteración con palabras como \emph{repetir mientras} o \emph{repetir para}. Al traducir un \emph{programa simbólico} a una secuencia de instrucciones que una máquina abstracta (como la máquina de Turing) o real (como las computadoras) pueda entender, la diferencia entre un procedimiento recursivo y uno iterativo se evapora \cite[p.~73]{Aho:Dragon} \cite{Steele:LambdaGOTO}.

\subsection{Procedimientos v.s. procesos}
\label{sec:procedimientos-procesos}

La codificación \eqref{eq:iter:fact1} del \autoref{alg:factorial} cumple con la descripción mencionada de procedimiento recursivo. Se tiene un valor numérico que se desea calcular y para encontrarlo se emplea una abstracción que parte de un estado con una solución parcial, esta abstracción sólo debe realizar dos operaciones aritméticas para encontrar otra solución parcial más cercana a la respuesta y delegar el trabajo de computar el resto a otra abstracción que realizará lo mismo.

Un algoritmo que describe de manera más precisa la manera con como \eqref{eq:iter:fact1} se reduce es:

\begin{algorithm}
  \caption{Procedimiento $ \mathrm{factorial}(n,r,a) $}
  \label{alg:factorial2}
  \begin{algorithmic}
    \REQUIRE $ n,\ r,\ a\in \mathbb{N} $, inicialmente $ r=n $ y $ a=1 $
    \ENSURE $ n! = a\times r! $
    \IF{$ r=0 $}
    \RETURN $ a $
    \ELSE
    \RETURN $ \mathrm{factorial}(n,r-1,a\times r) $
    \ENDIF
  \end{algorithmic}
\end{algorithm}

A pesar de poder escribir un procedimiento iterativo y otro recursivo para el cálculo del factorial, los dos algoritmos describen el mismo \emph{proceso computacional}. Estos procesos no son definidos a partir del lenguaje utilizado para describir el algoritmo, si no a partir de las acciones que se realizan para computar el resultado \cite{AbelsonSussman:Wizard}. Tanto el \autoref{alg:factorial} como el \autoref{alg:factorial2} como la codificación \eqref{eq:iter:fact1} describen el mismo proceso computacional.

Un tercer algoritmo para el cálculo del factorial es el \autoref{alg:factorial3}. A pesar de ser expresado como un procedimiento recursivo al igual que el \autoref{alg:factorial2}, no describe el mismo proceso computacional.

\begin{algorithm}
  \caption{Procedimiento $ \mathrm{factorial}(n) $}
  \label{alg:factorial3}
  \begin{algorithmic}
    \REQUIRE $ n\in \mathbb{N} $
    \ENSURE $ n! $
    \IF{$ n=0 $}
    \RETURN $ 1 $
    \ELSE
    \RETURN $ n\times \mathrm{factorial}(n-1) $
    \ENDIF
  \end{algorithmic}
\end{algorithm}

Los dos procesos mostrados comparten ciertas características:

\begin{itemize}
\item Realizan la misma cantidad de multiplicaciones y restas en cada paso iterativo/recursivo;
\item Realizan una cantidad de operaciones proporcional al $ n $;
\item En cada paso iterativo/recursivo, una solución parcial al problema es calculada.
\end{itemize}

La diferencia fundamental entre estos los dos procesos es que en los Algoritmos \ref{alg:factorial}, \ref{alg:factorial2} y en la codificación \eqref{eq:iter:fact1}, en cada paso se conoce el estado completo; mientras que en el \autoref{alg:factorial3}, al realizar el paso recursivo, se pierde la información de lo que ya se ha computado.

Así como se distinguen los procedimientos recursivos de los iterativos por la manera en como son expresados. Los procesos también se pueden distinguir en recursivos e iterativos. En general, un proceso iterativo es aquel en donde el estado puede ser capturado por una cantidad fija de valores, junto con una regla fija que describe como estos valores evolucionan a lo largo del cómputo. Por otro lado, los procesos recursivos suspenden el cálculo de las operaciones hasta tener todos los valores necesarios para computar el resultado \cite{AbelsonSussman:Wizard}.

Desde un aspecto operativo, el proceso recursivo del factorial multiplica una vez que el subproblema ha sido resuelto, mientras que el proceso iterativo del factorial, multiplica conforme los subproblemas son resueltos.

\subsection{Derivación de un mecanismo de recursividad}
\label{sec:deriv-recursividad}

Las técnicas para la codificación de algoritmos que se han tratado hasta este punto, sirven para aquellos que describen un proceso iterativo. En esta sección se aborda la manera en la que se pueden codificar algoritmos que describen procesos recursivos. Para desarrollar esta técnica, se considera como ejemplo la definición recursiva de la función factorial:
\[ n! =
  \begin{cases}
    1 &n=0;\\
    n\times (n-1)! &n>0.
  \end{cases}
\]

Todas las componentes de ésta definición están codificadas, ya sea como expresiones booleanas o como expresiones aritméticas. Una pseudo-definición de esta codificación es:
\begin{align}
  \label{eq:fact1}
  \cn{!} \synteq λn.(\bs{\prec}\ &(\cn{0}_{?}\, n) \\
                                 &\cn{1} \nonumber \\
                                 &(\cn{\times}\, n\, (\cn{!}\, (\cn{-}_{1}\, n)))) \nonumber
\end{align}

El problema con esta definición es que antes de definir $ \cn{!} $, se hace referencia a ella en $ (\cn{!}\, (\cn{-}_{1}\, n)) $, por lo que no es posible establecer el valor del término factorial antes de terminar de escribir su definición.

Hay algunos trucos que se pueden implementar para simular que se tiene definida la codificación de factorial antes de definirla. Por ejemplo, si $ \cn{!} $ fuera una abstracción la cual espera ser aplicada a sí misma y a un número, pudiera definirse exactamente como \eqref{eq:fact1}, pero con una variable enlazada $ f $ que será sustituída por el término $ \cn{!} $ dentro de la definición.
\begin{align}
  \label{eq:fact2}
  \cn{!} \synteq λf\, n.(\bs{\prec}\ &(\cn{0}_{?}\, n) \\
                                     &\cn{1} \nonumber \\
                                     &(\cn{\times}\, n\, (f\, (\cn{-}_{1}\, n)))) \nonumber
\end{align}

El problema con \eqref{eq:fact2} es que si se reduce $ (\cn{!}\, \cn{!}\, \cn{n}) $ y $ n \centernot= 0 $, ocurre lo siguiente:
\begin{align*}
  \cn{!}\, \cn{!}\, \cn{n} &\reduce{β} (\cn{\times}\, \cn{n}\, (\cn{!}\, (\cn{-}_{1}\, \cn{n})))
\end{align*}

La multiplicación de $ \cn{n} $ debe realizarse con otro numeral, sin embargo, $ \cn{!} $ espera ser aplicado a $ \cn{!} $ y a $ (\cn{-}_{1}\, \cn{n}) $, sin embargo, en la reducción el término $ \cn{!} $ no es aplicado a sí mismo.
\[ (\ \cn{\times}\ \cn{n}\ (\ \cn{!}\ \underbrace{[\quad]}_\text{Debe de ir $ \cn{!} $}\ (\ \cn{-}_{1}\ \cn{n}\ ))\ ) \]

Para escapar de este problema, se debe de aplicar $ (f\, f\, (\cn{-}_{1}\, \cn{n})) $ en el cuerpo de la definición. El factorial modificado es
\begin{align}
  \label{eq:fact3}
  \cn{!} \synteq λf\, n.(\bs{\prec}\ &(\cn{0}_{?}\, n) \\
                                     &\cn{1} \nonumber \\
                                     &(\cn{\times}\, n\, (f\, f\, (\cn{-}_{1}\, n)))) \nonumber
\end{align}

Esta definición se escapa del problema de la definición recursiva, puede reducirse a la codificación de la función factorial aplicando
\[ \cn{!}\, \cn{!} \reduce{β} λn.(\bs{\prec}\, (\cn{0}_{?}\, n)\, \cn{1}\, (\cn{\times}\, n\, (\cn{!}\, \cn{!}\, (\cn{-}_{1}\, n)))) \]

En efecto, no se necesita el nombrar la abstracción con un símbolo como $ \cn{!} $ para expresar la codificación del factorial. La reducción anterior es exactamente la misma a
\begin{align*}
  \begin{array}{r@{\mskip\thickmuskip}l}
    ((λf\, n.(\bs{\prec}\ &(\cn{0}_{?}\, n) \\
                        &\cn{1} \\
                        &(\cn{\times}\, n\, (f\, f\, (\cn{-}_{1}\, n))))) \\
  (λf\, n.(\bs{\prec}\ &(\cn{0}_{?}\, n) \\
                        &\cn{1} \\
                        &(\cn{\times}\, n\, (f\, f\, (\cn{-}_{1}\, n))))))
  \end{array}
                          \quad \reduce{β} \quad
                          \begin{array}{r@{\mskip\thickmuskip}l}
                            (λn.(\bs{\prec}\, (\cn{0}_{?}\, n)\ \ \ \ \ \ \ \ \ \ \ \ \ &\\
                                            \cn{1}\ \ \ \ \ \ \ \ \ \ \ \ \ \ \ \ \ \ \ &\\
                                            (\cn{\times}\, n\ \ \ \ \ \ \ \ \ \ \ \ \ \ &\\
                            ((λf\, n.(\bs{\prec}\, &(\cn{0}_{?}\, n)\\
                                                 &\cn{1}\\
                                                 &(\cn{\times}\, n\, (f\, f\, (\cn{-}_{1}\, n)))))\\
                             (λf\, n.(\bs{\prec}\, &(\cn{0}_{?}\, n)\\
                                                 &\cn{1}\\
                                                 &(\cn{\times}\, n\, (f\, f\, (\cn{-}_{1}\, n)))))\\
                             (\cn{-}_{1}\, n)))))&
                          \end{array}
\end{align*}

\subsection{Combinador genérico de recursividad}
\label{sec:combinador-recursividad}

A pesar de tener este problema resuelto, la solución no es buena. Esta técnica obliga a que en cada algoritmo recursivo que se codifique, cada aplicación recursiva se deba tener el primer argumento aplicado a sí mismo. Lo que se necesita para tener un buen mecanismo de recursividad es separar la auto-aplicación de una codificación recursiva de la definición misma. Lo ideal es poder codificar un algoritmo similar a la definición \eqref{eq:fact2} y mediante algún procedimiento genérico, hacer que se auto-aplique el procedimiento a si mismo.

La ``factorización'' del mecanismo de recursión se puede lograr combinando la idea de la auto-aplicación de la ecuación \eqref{eq:fact3} y la idea de escribir los términos recursivos como en la ecuación \eqref{eq:fact2}.

En \eqref{eq:fact2} se esperaba que $ \cn{!} $ sea aplicado a sí mismo, sin embargo en su definición solo aplica $ f $ a $ (\cn{-}_{1}\, n) $, por lo que es conveniente tratar de reducir la aplicación de $ \cn{!} $ a $ (\cn{!}\, \cn{!}) $, de tal manera que la reducción sea
\begin{align*}
  \cn{!}\, (\cn{!}\, \cn{!}) &\synteq (λf\, n.(\bs{\prec}\, (\cn{0}_{?}\, n)\, \cn{1}\, (\cn{\times}\, n\, (f\, (\cn{-}_{1}\, n)))))\, (\cn{!}\, \cn{!}) \\
                           &\contract{β} λn.(\bs{\prec}\, (\cn{0}_{?}\, n)\, \cn{1}\, (\cn{\times}\, n\, (\cn{!}\ \cn{!}\, (\cn{-}_{1}\, n))))
\end{align*}

Esto hace que la codificación funcione en el primer paso recursivo. Sin embargo, $ (\cn{!}\ \cn{!}\, (\cn{-}_{1}\, n)) $ será reducido de tal manera que el siguiente paso recursivo no se aplique $ \cn{!} $ a si mismo.

\begin{align*}
  \cn{!}\ \cn{!}\, (\cn{-}_{1}\, n) &\synteq (λf\, n.(\bs{\prec}\, (\cn{0}_{?}\, n)\, \cn{1}\, (\cn{\times}\, n\, (f\, (\cn{-}_{1}\, n)))))\, \cn{!}\, (\cn{-}_{1}\, n) \\
                                 &\contract{β} (λn.(\bs{\prec}\, (\cn{0}_{?}\, n)\, \cn{1}\, (\cn{\times}\, n\, (\cn{!}\, (\cn{-}_{1}\, n)))))\, (\cn{-}_{1}\, n)
\end{align*}

El término $ (\cn{!}\, (\cn{-}_{1}\, n)) $ debería de ser $ (\cn{!}\ \cn{!}\, (\cn{-}_{1}\, n)) $ de nuevo para que el siguiente paso recursivo funcione.

En la \autoref{sec:expresiones} se mostró un término que es útil para este tipo de situaciones: \[ ω \synteq λx.x\, x \] La propiedad interesante de $ ω $ es que $ (ω\, ω) \reduce{β} (ω\, ω) $, que es justo lo que deseamos en nuestro mecanismo recursivo, el combinador que se necesita es uno similar a $ ω $, llamado $ ω^{\prime} $ tal que $ (ω^{\prime}\, ω^{\prime}) \reduce{β} (\cn{!} (ω^{\prime}\, ω^{\prime})) $.

Suponiendo que $ \cn{!} $ ya es una variable enlazada, el combinador $ ω^{\prime} $ debe ser una abstracción que espera ser aplicada a si misma, tiene la forma $ λω^{\prime}.M $. Ya que la reducción de $ ω^{\prime} $ aplicada a sí misma resulta en $ (\cn{!} (ω^{\prime}\, ω^{\prime})) $, $ M $ debe de ser la aplicación $ (\cn{!}\, N) $, para completar la regla de reducción $ N \synteq (ω^{\prime}\, ω^{\prime})$.
\[ ω^{\prime} \synteq λω^{\prime}.\cn{!}\, (ω^{\prime}\, ω^{\prime}) \convertible{α} λx.\cn{!}\, (x\, x) \]

Para completar la definición del mecanismo de recursividad se plantea un combinador que espere ser aplicado a un término como $ \cn{!} $ de \eqref{eq:fact2} e internamente aplique $ ω^{\prime} $ a sí misma. Este mecanismo es llamado combinador $ \bs{Y} $.
\begin{equation}
  \label{eq:recur:Y}
  \bs{Y} \synteq λf.(λx.f\, (x\, x))\, (λx.f\, (x\, x))
\end{equation}

Al aplicar el combinador $ \bs{Y} $ a la definición \eqref{eq:fact2} del factorial, se obtiene que
\begin{align*}
  \bs{Y}\, \cn{!} &\convertible{β} \cn{!}\, (\bs{Y}\, \cn{!}) \\
                  &\synteq (λf\, n.(\bs{\prec}\, (\cn{0}_{?}\, n)\, \cn{1}\, (\cn{\times}\, n\, (f (\cn{-}_{1}\, n))))) (\bs{Y}\, \cn{!}) \\
                  &\contract{β} λn.(\bs{\prec}\, (\cn{0}_{?}\, n)\, \cn{1}\, (\cn{\times}\, n\, (\bs{Y}\, \cn{!}\, (\cn{-}_{1}\, n))))
\end{align*}

Al reducir la aplicación de este término en un numeral mayor a cero se obtiene
\begin{align*}
  \begin{array}{r@{\mskip\thickmuskip}l}
    (λn.(\bs{\prec}\, &(\cn{0}_{?}\, n)\\
                      &\cn{1}\\
                      &(\cn{\times}\, n\, (\bs{Y}\, \cn{!}\, (\cn{-}_{1}\, n)))))\, \cn{n}
  \end{array}
  \quad \reduce{β} \quad
  \begin{array}{r@{\mskip\thickmuskip}l}
    (\cn{\times}\, \cn{n}\, (\bs{Y}\, \cn{!}\, (\cn{-}_{1}\, \cn{n})))
  \end{array}
\end{align*}

Es entonces que $ (\bs{Y}\, \cn{!}) $ puede ser $ β $-convertido nuevamente a $ (\cn{!}\, (\bs{Y}\, \cn{!})) $ para continuar al siguiente paso recursivo de la misma manera.

\subsection{Combinadores de punto fijo}
\label{sec:fixed-point}

Un \emph{punto fijo} de un operador o función es un objeto el cuál no cambia cuando el operador le es aplicado. Por ejemplo, la función $ f(x)=x^{2} $ tiene dos puntos fijos 0 y 1, ya que $ 0^{2} = 0 $ y $ 1^{2} = 1 $. Hay operadores que no tienen punto fijo, por ejemplo el sucesor de un número, ya que $ n+1 \centernot= n $ para toda $ n $.

El \autoref{thm:punto-fijo} es uno de los resultados básicos en el cálculo $ λ $.
\begin{thm}[Teorema de punto fijo]
  \label{thm:punto-fijo}
  $ \forall F\, \exists X \colon  F\, X \convertible{β} X $

  \begin{proof}[Demostración]
    Sea $ W\synteq λx.F\, (x\, x) $ y $ X \synteq W\, W $. Entonces
    \[ X \synteq (λx.F\, (x\, x))\, W \contract{β} F\, (W\, W) \synteq F\, X \]
  \end{proof}
\end{thm}

Esta demostración es algo peculiar ya que se inicia con $ X $ y se reduce este término a $ (F\, x) $, en lugar de partir hacer el proceso inverso. Sin embargo, la reducción presentada es válida de acuerdo a la definición de $ β $-convertibilidad.

El combinador $ \bs{Y} $ derivado en la \autoref{sec:combinador-recursividad} pertenece a una clase de combinadores interesantes llamados \emph{combinadores de punto fijo}. Estos combinadores tienen la propiedad de que al ser aplicados a cualquier término $ F $ ``encuentran'' un punto fijo para $ F $ \cite[p.~34]{HindleySeldin:LambdaCalculusAndCombinators}, es decir, si $ M $ es un combinador de punto fijo, entonces, para toda $ F $
\begin{equation}
  \label{eq:fixed:prop}
  F\, (M\, F) \convertible{β} M\, F
\end{equation}

A partir de la ecuación \eqref{eq:recur:Y}, se puede corroborar que $ \bs{Y} $ es un combinador de punto fijo de la siguiente manera
\begin{align}
  \label{eq:fixed:Ydem}
  \bs{Y}\, F &\synteq (λf.(λx.f\, (x\, x))\, (λx.f\, (x\, x)))\, F \\
             &\contract{β} (λx.F\, (x\, x))\, (λx.F\, (x\, x)) \nonumber \\
             &\contract{β} F\, ((λx.F\, (x\, x))\, (λx.F\, (x\, x))) \nonumber \\
             &\convertible{β} F\, (\bs{Y}\, F) \nonumber
\end{align}

Existen combinadores con una propiedad más fuerte que \eqref{eq:fixed:prop}. Alan Turing descubrió el combinador $ \bs{Θ} $ en 1937, su definicón es

\begin{equation}
  \label{eq:fixed:Turing}
  \begin{array}{r@{\mskip\thickmuskip}l}
    \bs{Θ} \synteq U\, U
  \end{array}
  \quad \text{donde} \quad
  \begin{array}{r@{\mskip\thickmuskip}l}
    U \synteq λu\, x.x\, (u\, u\, x)
  \end{array}
\end{equation}

La propiedad interesante de $ \bs{Θ} $ es que puede computar puntos fijos únicamente con la $ β $-reducción. Se corrobora esto de la siguiente manera

\begin{align}
  \label{eq:fixed:Udem}
  \bs{Θ}\, F &\synteq U\, U\, F \\
             &\contract{β} (λx.x\, (U\, U\, x))\, F \nonumber \\
             &\contract{β} F\, (U\, U\, F) \nonumber \\
             &\synteq F\, (\bs{Θ}\, F) \nonumber
\end{align}

Esta propiedad no la tiene $ \bs{Y} $, para realizar el último paso del desarrollo \eqref{eq:fixed:Ydem}, se tuvo que hacer una reducción inversa.

La utilidad de los combinadores de punto fijo va más allá que el de permitir la codificación de procesos recursivos. Estos combinadores son especialmente útiles para resolver el siguiente tipo de problema:

Sea $ Z $ un término $ λ $, con variables libres $ f $ y $ \vec{x} $, encuentra el término $ F $ tal que

\[ F\, \vec{M} \convertible{β} \subst{\subst{Z}{f}{F}}{\vec{x}}{\vec{M}} \]

En este planteamiento $ F $ puede no aparecer en $ Z $ y la solución es el mismo término a que si apareciera. Una instancia de este problema puede ser una reformulación de la codificación recursiva del factorial. Sea $ Z \synteq (\bs{\prec}\, (\cn{0}_{?}\, n)\, \cn{1}\, (\cn{\times}\, n\, (f\, (\cn{-}_{1}\, n)))) $, encuentra el término $ \cn{!} $ tal que
\[ \cn{!}\, \cn{n} \convertible{β} \subst{\subst{Z}{f}{\cn{!}}}{n}{\cn{n}} \]
\begin{proof}[Solución]
  Por la regla $ (β) $ se tiene que
  \[ \cn{!}\, \cn{n} \convertible{β} (λf\, n.\bs{\prec}\, (\cn{0}_{?}\, n)\, \cn{1}\, (\cn{\times}\, n\, (\cn{!}\, (\cn{-}_{1}\, n))))\, \cn{!}\, \cn{n} \]
  Por la regla $ (ν) $ se tiene que
  \[ \cn{!} \convertible{β} (λf\, n.\bs{\prec}\, (\cn{0}_{?}\, n)\, \cn{1}\, (\cn{\times}\, n\, (\cn{!}\, (\cn{-}_{1}\, n))))\, \cn{!} \]
  Esta ecuación tiene la forma $ A \convertible{β} (B\, A) $, al considerar a $ A $ de la forma $ (\bs{Y}\, B) $, se cumple la propiedad de \eqref{eq:fixed:prop} y por lo tanto
  \[ \cn{!} \synteq \bs{Y}\, (λf\, n.\bs{\prec}\, (\cn{0}_{?}\, n)\, \cn{1}\, (\cn{\times}\, n\, (f (\cn{-}_{1}\, n)))) \]
\end{proof}

En general, la solución para este tipo de problemas es
\begin{equation}
  \label{eq:fixed:prala}
  F\synteq \bs{Y} (λf\, \vec{x}.Z)
\end{equation}

\section{Estructuras recursivas}
\label{sec:estructuras-recursivas}

Utilizando combinadores de punto fijo como $ \bs{Y} $ y $ \bs{Θ} $ y la solución general \eqref{eq:fixed:prala} se pueden codificar procedimientos recursivos como el \autoref{alg:factorial3} del factorial. Sin embargo, los procesos recursivos se tornan más interesantes cuando la información que manipulan es también recursiva. En esta sección se muestran técnicas para codificar estructuras recursivas en el cálculo $ λ $.

Una variedad de lenguajes de programación, cuentan con un operador o función fundamental para la construcción de estructuras compuestas. Este operador usualmente es llamado \emph{cons}, el cual es una abreviación de la palabra ``construír en memoria''. Este operador toma dos objetos $ a $ y $ d $ y construye en memoria un objeto que contiene a ambos.

Matemáticamente, el objeto resultante de aplicar el operador \emph{cons} es un par ordenado. La notación que se utiliza en este trabajo para escribir el par conformado por $ a $ y $ d $ es $ \langle a : d \rangle $.

Un par ordenado no es más que un estado con dos elementos, para codificarse en el cálculo $ λ $, se deben plantear los mecanismos para construír pares y obtener sus elementos. A continuación se construyen las definiciones del constructor $ \mc{p} $ y los selectores $ \mc{a} $ y $ \mc{d} $ para el par ordenado (basadas en las definiciones \eqref{eq:state-cons} y \eqref{eq:state-selec}).

\begin{equation}
  \label{eq:cons1}
  \mc{p} \synteq λa\, d.λq.q\, a\, d
\end{equation}
\begin{equation}
  \label{eq:car1}
  \mc{a} \synteq λc.c\, (λa\, d.a)
\end{equation}
\begin{equation}
  \label{eq:cdr1}
  \mc{d} \synteq λc.c\, (λa\, d.d)
\end{equation}

Estas tres ecuaciones cumplen con las reducciones
\[ (\mc{a}\, (\mc{p}\, M\, N)) \reduce{β} M \]
\[ (\mc{d}\, (\mc{p}\, M\, N)) \reduce{β} N \]
para cualesquiera términos $ λ $ $ M $ y $ N $. Para corroborar esto, se desarrollan las reducciones:
\begin{align}
  \label{eq:car1-prop}
  \mc{a} (\mc{p}\, M\, N) &\synteq (λc.c\, (λa\, d.a))\, (\mc{p}\, M\, N) \\
                          &\contract{β} (\mc{p}\, M\, N)\, (λa\, d.a) \nonumber \\
                          &\synteq ((λa\, d.λq.q\, a\, d)\, M\, N)\, (λa\, d.a) \nonumber \\
                          &\reduce{β} (λq.q\, M\, N)\, (λa\, d.a) \nonumber \\
                          &\contract{β} (λa\, d.a)\, M\, N \nonumber \\
                          &\reduce{β} M \nonumber
\end{align}
\begin{align}
  \label{eq:cdr1-prop}
  \mc{d} (\mc{p}\, M\, N) &\synteq (λc.c\, (λa\, d.d))\, (\mc{p}\, M\, N) \\
                          &\contract{β} (\mc{p}\, M\, N)\, (λa\, d.d) \nonumber \\
                          &\synteq ((λa\, d.λq.q\, a\, d)\, M\, N)\, (λa\, d.d) \nonumber \\
                          &\reduce{β} (λq.q\, M\, N)\, (λa\, d.d) \nonumber \\
                          &\contract{β} (λa\, d.d)\, M\, N \nonumber \\
                          &\reduce{β} N \nonumber
\end{align}

En el resto de esta sección se abordan diferentes maneras en las que se puede emplear la estructura del par para construír estructuras más complejas.

\subsection{Listas}
\label{sec:estructura-listas}

Las listas son secuencias de valores, en donde cada valor en la lista tiene una posición fija. La codificación de las listas en el cálculo $ λ $ se asemeja a la lista enlazada comunmente estudiada en estructuras de datos. Si consideramos que el par ordenado contiene como primer elemento un numeral de Church y como segundo elemento otro par ordenado, se puede representar una lista de números, en donde el último par contiene como segundo elemento una codificación que represente el valor nulo $ \emptyset $.

\[ \langle \cn{n}_{1} : \langle \cn{n}_{2} : \langle \cn{n}_{3} : \langle ... \langle \cn{n}_{k} : \emptyset \rangle ... \rangle \rangle \rangle \rangle \]

Denotada de manera abreviada como

\[ \langle \cn{n}_{1},\ \cn{n}_{2},\ \cn{n}_{3},\ ...\ ,\ \cn{n}_{k} \rangle \]

La codificación de $ \emptyset $ debe elegirse con mucho cuidado. En los algoritmos que manipulan listas es crucial determinar cuando se ha llegado al final de la lista, por lo tanto se hace uso de un predicado para determinar si un determinado objeto es un par o es el valor nulo (de manera similar a la comparación de un número con el cero en los algoritmos aritméticos).

Así como el predicado $ \cn{0}_{?} $ fue construído asumiendo que sería aplicado a un numeral, el término $ \emptyset $ se construye asumiendo que será aplicado a una lista de números. Formalmente una lista de numeros, o es un par cuyo primer elemento es un numeral de Church y cuyo segundo elemento es otra lista, o es el término nulo (lista con cero elementos).

\begin{align}
  \label{eq:lis-def1}
  \mathcal{L} \longrightarrow \langle \cn{n} : \mathcal{L} \rangle \mid \emptyset
\end{align}

Por lo tanto, se espera que el predicado $ \emptyset_{?} $ sea aplicado a un término de la forma $ (λq.q\, \cn{n}\, \mc{L}) $ o al término $ \emptyset $, de tal manera que

\begin{align*}
  \emptyset_{?} (λq.q\, \cn{n}\, \mc{L}) &\reduce{β} \bs{F} \\
  \emptyset_{?} \emptyset &\reduce{β} \bs{T}
\end{align*}

Una manera de convertir un término par a $ \bs{F} $ es reduciendo la aplicación $ (\bs{K}\, \bs{F}\, (λq.q\, \cn{n}\, \mc{L})) $, sin embargo, este resultado es el mismo para cualquier valor al que se aplique $ (\bs{K}\, \bs{F}) $ y el objetivo es poder discriminar entre un par y $ \emptyset $. Se puede considerar un término similar a $ (\bs{K}\, \bs{F}) $ pero que cancele los siguientes dos términos a los que sea aplicado:
\[ (λx\, y.\bs{F})M\, N \reduce{β} \bs{F} \]
De esta manera, aplicar un par a este término resulta en la reducción
\begin{align*}
  (λq.q\, \cn{n}\, \mc{L})\, (λx\, y.\bs{F}) &\contract{β} (λx\, y.\bs{F})\, \cn{n}\, \mc{L} \\
                                             &\reduce{β} \bs{F}
\end{align*}
Por lo tanto, la codificación de $ \emptyset_{?} $ que es reducida a falso al ser aplicada a un par es
\begin{equation}
  \label{eq:lis-emptyp}
  \emptyset_{?} \synteq (λl.l\, (λx\, y.\bs{F}))
\end{equation}
Con esta definición, la propiedad que debe cumplir la codificación de $ \emptyset $ es $ (\emptyset_{?}\, \emptyset) \reduce{β} \bs{T} $, lo cuál resulta ser el término $ (\bs{K}\, \bs{T}) $ ya que
\begin{align*}
  \emptyset_{?}\, (\bs{K}\, \bs{T}) &\synteq (λl.l\, (λx\, y.\bs{F}))\, (\bs{K}\, \bs{T}) \\
                                    &\contract{β} \bs{K}\, \bs{T}\, (λx\, y.\bs{F}) \\
                                    &\reduce{β} \bs{T}
\end{align*}
Por lo tanto
\begin{equation}
  \label{eq:lis-empty}
  \emptyset \synteq \bs{K}\, \bs{T}
\end{equation}

\begin{rem}[Sobre predicados]
  La aplicación de un predicado como $ \cn{0}_{?} $ o $ \emptyset_{?} $ a un término $ λ $ $ M $ no necesariamente se $ β $-reduce a una codificación de valor de verdad. Estos predicados son construídos para ser aplicados a numerales o listas respectivamente y el resultado de reducir otro tipo de términos no es de importancia para la codificación de algoritmos.
\end{rem}

Con estas codificaciones se pueden construír algoritmos que manipulen listas de números. Consideremos el \autoref{alg:lis-doubles} que dada una lista de números compute una lista con la misma cantidad de elementos pero con cada número de la lista original multiplicado por 2. En cada paso recursivo se verifica que la lista $ \mc{L} $ no sea el valor nulo, en cuyo caso, será un par cuyo primer elemento es $ n $ y segundo elemento es otra lista $ \mc{L}^{\prime} $; se construye otro par con el primer elemento multiplicado por 2 y con el segundo elemento el resultado de realizar este mismo proceso con $ \mc{L}^{\prime} $.

\begin{algorithm}
  \caption{Procedimiento recursivo $ \mathrm{dobles}(\mc{L}) $}
  \label{alg:lis-doubles}
  \begin{algorithmic}
    \REQUIRE $ \mc{L} \in \{ \langle n_{1},\ ...\ ,\ n_{k} \rangle \mid n\in \mathbb{N} \} \cup \emptyset $
    \ENSURE $ \langle 2\times n_{1},\ ...\ ,\ 2\times n_{k} \rangle $ ó $ \emptyset $
    \IF{$ \mc{L} = \emptyset $}
    \RETURN $ \emptyset $
    \ELSE
    \STATE $ \langle n : \mc{L}^{\prime} \rangle \leftarrow \mc{L} $
    \RETURN $ \langle 2\times n : \mathrm{dobles}(\mc{L}^{\prime}) \rangle $
    \ENDIF
  \end{algorithmic}
\end{algorithm}

El predicado $ \emptyset_{?} $ es utilizado con la condicional booleana $ \bs{\prec} $  sobre la lista $ \mc{L} $. Cuando la lista sea la codificación de nulo, el resultado es nulo; de lo contrario, se asume que $ \mc{L} $ es un par y se construye el par correspondiente utilizando los términos $ \mc{p} $, $ \mc{a} $ y $ \mc{d} $. La codificación resultante es \eqref{eq:lis-doubles}.
\begin{align}
  \label{eq:lis-doubles}
  \bs{Y}\, (λf\, \mc{L}.\bs{\prec}\, (\emptyset_{?}&\, \mc{L})\\
  \emptyset\ \ \ & \nonumber \\
  (\mc{p}\  &(\cn{\times}\, \cn{2}\, (\mc{a}\, \mc{L})) \nonumber \\
                                               &(f\, (\mc{d}\, \mc{L})))) \nonumber
\end{align}
Teniendo esta codificación, es relativamente fácil generalizarla para que a cada número en la lista se le aplique algún término $ g $ dado, el cual puede ser reducido a otro número.
\begin{align}
  \label{eq:lis-map}
  \bs{Y}\, (λf\, g\, \mc{L}.\bs{\prec}\, (\emptyset_{?}&\, \mc{L})\\
  \emptyset\ \ \ & \nonumber \\
  (\mc{p}\  &(g\, (\mc{a}\, \mc{L})) \nonumber \\
                                                   &(f\, g\, (\mc{d}\, \mc{L})))) \nonumber
\end{align}
Aún más, se puede generalizar la estructura del resultado al abstraer el término final $ \emptyset $ y el constructor del par $ \mc{p} $ de la siguiente manera.
\begin{align}
  \label{eq:lis-fold}
  \bs{Y}\, (λf\, \mc{p}\, \emptyset\, g\, \mc{L}.\bs{\prec}\, (\emptyset_{?}&\, \mc{L})\\
  \emptyset\ \ \ & \nonumber \\
  (\mc{p}\  &(g\, (\mc{a}\, \mc{L})) \nonumber \\
                                                   &(f\, \mc{p}\, \emptyset\, g\, (\mc{d}\, \mc{L})))) \nonumber
\end{align}
Si $ \mc{F} $ es la abstracción \eqref{eq:lis-fold}, $ \mc{M} $ es la abstracción \eqref{eq:lis-map}, $ \mc{D} $ es la abstracción \eqref{eq:lis-doubles} y $ \mc{L} $ es el término $ λ $ que codifica la lista $ \langle n_{1},\ ...\ ,\ n_{k} \rangle $, entonces
\[ \mc{F}\, \mc{p}\, \mc{\emptyset}\, (\cn{\times}\, \cn{2})\, \mc{L} \convertible{β} \mc{M}\, (\cn{\times}\, \cn{2})\, \mc{L} \convertible{β} \mc{D}\, \mc{L} \]

Consideremos ahora el \autoref{alg:lis-impares} que dada una lista de números compute una lista únicamente con los números de la lista original que son impares. La estructura de este algoritmo es similar al \autoref{alg:lis-doubles} pero la alternativa de la primer condicional verifica además si el primer elemento $ n $ de $ \mc{L} $ es impar, en cuyo caso $ n $ será elemento de un par resultante; de lo contrario $ n $ es ignorado y no se construye un par en este paso recursivo.

\begin{algorithm}
  \caption{Procedimiento recursivo $ \mathrm{impares}(\mc{L}) $}
  \label{alg:lis-impares}
  \begin{algorithmic}
    \REQUIRE $ \mc{L} \in \{ \langle n_{1},\ ...\ ,\ n_{k} \rangle \mid n\in \mathbb{N} \} \cup \emptyset $
    \ENSURE $ \langle n^{\prime}_{1},\ ...\ ,\ n^{\prime}_{k^{\prime}} \rangle $ ó $ \emptyset \mid n^{\prime}_{i} \in \mc{L},\ n^{\prime}_{i}$ impar
    \IF{$ \mc{L} = \emptyset $}
    \RETURN $ \emptyset $
    \ELSE
    \STATE $ \langle n : \mc{L}^{\prime} \rangle \leftarrow \mc{L} $
    \IF{$ n $ impar}
    \RETURN $ \langle n : \mathrm{impares}(\mc{L}^{\prime}) \rangle $
    \ELSE
    \RETURN $ \mathrm{impares}(\mc{L}^{\prime}) $
    \ENDIF
    \ENDIF
  \end{algorithmic}
\end{algorithm}

La codificación de este algoritmo es bastante directa también, sin embargo, se necesita codificar un predicado que determine si un numeral de Church es impar o no. Un algoritmo recursivo puede ser elegido para la codificación del predicado impar, sin embargo, es más fácil en este caso utilizar la estructura de los numerales para lograr el resultado deseado. Sea $ \cn{imp}_{?} $ la codificación del predicado impar, se deben de satisfacer las siguientes reducciones:
\begin{align*}
  \cn{imp}_{?} (λx\, y.y) &\reduce{β} \bs{F} \\
  \cn{imp}_{?} (λx\, y.x\, y) &\reduce{β} \bs{T} \\
  \cn{imp}_{?} (λx\, y.x\, (x\, y)) &\reduce{β} \bs{F} \\
  \cn{imp}_{?} (λx\, y.x\, (x\, (x\, y))) &\reduce{β} \bs{T} \\
                          &...
\end{align*}
Una manera sencilla de codificar este término es
\[ \cn{imp}_{?} \synteq λn.n\, \bs{\lnot}\, \bs{F} \]
De esta manera
\begin{align*}
  \cn{imp}_{?} (λx\, y.y) &\reduce{β} \bs{F} \\
  \cn{imp}_{?} (λx\, y.x\, y) &\reduce{β} \bs{\lnot}\, \bs{F} \\
  \cn{imp}_{?} (λx\, y.x\, (x\, y)) &\reduce{β} \bs{\lnot}\,  (\bs{\lnot}\, \bs{F}) \\
  \cn{imp}_{?} (λx\, y.x\, (x\, (x\, y))) &\reduce{β} \bs{\lnot}\, (\bs{\lnot}\, (\bs{\lnot}\, \bs{F})) \\
                          &...
\end{align*}
Con este nuevo predicado, la codificación del \autoref{alg:lis-impares} es \eqref{eq:lis-impares}.
\begin{align}
  \label{eq:lis-impares}
  \bs{Y}(λf\, \mc{L}.\bs{\prec} (\emptyset_{?}\, &\mc{L}) \\
                                \emptyset\ \ \ & \nonumber \\
                                (\bs{\prec}\ &(\cn{imp}_{?}\, (\mc{a}\, \mc{L})) \nonumber \\
                                            &(\mc{p}\, (\mc{a}\, \mc{L})\, (f\, (\mc{d}\, \mc{L}))) \nonumber \\
                                            &(f\, (\mc{d}\, \mc{L})))) \nonumber
\end{align}

Esta abstracción también es posible generalizarla con pocas modificaciones para que, además de una lista considere un predicado $ g_{?} $ que sea reducido a una codificación booleana cuando sea aplicado a un numeral de Church. El término \eqref{eq:lis-filter} muestra este término más general.

\begin{align}
  \label{eq:lis-filter}
  \bs{Y}(λf\, g_{?}\, \mc{L}.\bs{\prec}\, (\emptyset_{?}\, &\mc{L}) \\
                                \emptyset\ \ \ & \nonumber \\
                                (\bs{\prec}\ &(g_{?}\, (\mc{a}\, \mc{L})) \nonumber \\
                                            &(\mc{p}\, (\mc{a}\, \mc{L})\, (f\, g_{?}\, (\mc{d}\, \mc{L}))) \nonumber \\
                                            &(f\, g_{?}\, (\mc{d}\, \mc{L})))) \nonumber
\end{align}

Si $ \mc{F} $ es la abstracción \eqref{eq:lis-filter}, $ \mc{I} $ es la abstracción \eqref{eq:lis-impares} y $ \mc{L} $ es el término $ λ $ que codifica la lista $ \langle n_{1},\ ...\ ,\ n_{k} \rangle $, entonces

\[ \mc{F}\, \cn{imp}_{?}\, \mc{L} \convertible{β} \mc{I}\, \mc{L} \]

Algoritmos de procesamiento de listas más complejos pueden ser codificados ya sea utilizando \eqref{eq:lis-map}, \eqref{eq:lis-fold}, \eqref{eq:lis-filter} o términos con una estructura similar. Algo que es importante notar de las generalizaciones planteadas en los ejemplos es que la lista $ \mc{L} $ puede tener términos $ λ $ que no sean numerales, al utilizar los algoritmos genéricos sólo se debe tener cuidado con que la forma de los términos en $ \mc{L} $ sea conocida para $ g $ y $ g_{?} $.

\subsection{Árboles}
\label{sec:estructura-arboles}

Los árboles son estructuras no-lineales jerárquicas compuestos de \emph{vértices} (también llamados \emph{nodos}) y \emph{aristas} que establecen una relación entre dos vértices.

Utilizando pares y listas es posible construír árboles. Consideremos el árbol mostrado en la \autoref{fig:tree}.
\begin{figure}[!htbp]
  \centering
  \begin{tikzpicture}
    \node[circle,draw] (11) {};
    \node (label11) [right= 0cm of 11] {$ v_{1,1} $};
    \node[circle,draw] (21) [below left= .75cm and 1cm of 11] {};
    \node (label21) [left= 0cm of 21] {$ v_{2,1} $};
    \node[circle,draw] (22) [below of=11] {};
    \node (label22) [right= 0cm of 22] {$ v_{2,2} $};
    \node[circle,draw] (23) [below right= .75cm and 1cm of 11] {};
    \node (label23) [right= 0cm of 23] {$ v_{2,3} $};
    \node[circle,draw] (31) [below left=.5cm and .1cm of 21] {};
    \node (label31) [left= 0cm of 31] {$ v_{3,1} $};
    \node[circle,draw] (32) [below right=.5cm and .1cm of 21] {};
    \node (label32) [below= 0cm of 32] {$ v_{3,2} $};
    \node[circle,draw] (33) [below=.43cm of 22] {};
    \node (label33) [right= 0cm of 33] {$ v_{3,3} $};
    \node[circle,draw] (41) [below=.43cm of 31] {};
    \node (label11) [left= 0cm of 41] {$ v_{4,1} $};
    
    \path[draw,thick]
    (11) edge node {} (21)
    (11) edge node {} (22)
    (11) edge node {} (23);

    \path[draw,thick]
    (21) edge node {} (31)
    (21) edge node {} (32);
    
    \path[draw,thick]
    (22) edge node {} (33);

    \path[draw,thick]
    (31) edge node {} (41);
  \end{tikzpicture}
  \caption{Ejemplo de árbol}
  \label{fig:tree}
\end{figure}

La definición usual de un árbol como el mostrado en la figura es como un conjunto de vértices y un conjunto de aristas. Sin embargo, una definición recursiva puede ser expresada como. Un árbol $ \mc{T} $ puede ser o un vértice $ v $ acompañado de una lista de árboles $ \mc{L}_{\mc{T}} $ o un valor nulo $ \emptyset $, es decir

\[ \mc{T} \longrightarrow (v,\ \langle \mc{T}_{1},\ ...\ ,\ \mc{T}_{k} \rangle) \mid \emptyset \]

Esta definición nos permite codificar árboles como el valor nulo o como un par cuyo primer elemento sea un vértice $ v $ y cuyo segundo elemento sea una lista de los vértices en los que $ v $ incide. El vértice puede ser representado por cualquier término $ λ $, en ocaciones es útil asociar un término a cada vértice del árbol para construír, por ejemplo, árboles de numerales.

El árbol de la \autoref{fig:tree} se representa de la siguiente manera:
\begin{align*}
  \langle v_{1,1} : \langle \langle v_{2,1} : \langle &\langle v_{3,1} : \langle \langle v_{4,1} : \emptyset \rangle \rangle \rangle \\
                                                      &\langle v_{3,2} : \emptyset \rangle \rangle \rangle\\
                           \langle v_{2,2} : \langle &\langle v_{3,3} : \emptyset \rangle \rangle \rangle \\
                           \langle v_{2,3} : \emptyset& \rangle \rangle \rangle
\end{align*}

La agrupación entre un vértice y sus subárboles se puede realizar con la estructura par, de tal manera que

\[ \mc{T} \synteq \langle v : \langle \mc{T}_{1},\ ...\ ,\ \mc{T}_{k} \rangle \rangle \convertible{β} \langle v,\ \mc{T}_{1},\ ...\ ,\ \mc{T}_{k} \rangle \]

Por lo que el predicado $ \emptyset_{?} $ funciona para diferenciar árboles nulos de árboles con vértices y si $ \mc{T} $ es un árbol no nulo, entonces $ (\mc{a}\, \mc{T}) $ se reduce al vértice del árbol y $ (\mc{d}\, \mc{T}) $ se reduce a los subárboles del árbol.

Esta similitud en la representación de árboles y listas corresponde con la codificación de listas como ``pares de pares'' y la codificación de árboles como ``listas de listas''. Para mostrar las posibilidades que estas relaciones nos permiten se considera el \autoref{alg:tree-count}, el cual calcula la cantidad de vértices en un árbol. El procedimiento $ \mathrm{suma} $ calcula la suma de los elementos de una lista de numeros y el procedimiento $ \mathrm{map} $ es el que es codificado por \eqref{eq:lis-map}.

\begin{algorithm}
  \caption{Procedimiento recursivo $ \mathrm{cuenta}(\mc{T}) $}
  \label{alg:tree-count}
  \begin{algorithmic}
    \REQUIRE $ \mc{T} \in \{ \langle v : \langle \mc{T}_{1},\ ...\ ,\ \mc{T}_{k} \rangle \rangle \} \cup \emptyset $
    \ENSURE $ n = $ cantidad de vértices en $ \mc{T} $
    \IF{$ \mc{T} = \emptyset $}
    \RETURN $ 0 $
    \ELSE
    \STATE $ \langle v : \mc{L}_{\mc{T}} \rangle \leftarrow \mc{T} $
    \RETURN $ 1 + \mathrm{fold}(+,\ 0,\ \mathrm{cuenta},\ \mc{L}_{\mc{T}}) $
    \ENDIF
  \end{algorithmic}
\end{algorithm}

La codificación del procedimiento $ \mathrm{fold} $ es el término \eqref{eq:lis-fold}. Por lo que el \autoref{alg:tree-count} se define en el cálculo $ λ $ como
\begin{align*}
  \bs{Y}\, (λf\, \mc{T}.\bs{\prec}\, &(\emptyset_{?}\, \mc{T})\\
                                &\cn{0}\\
                                &(\cn{+}_{1}\, (\mc{F}\, \cn{+}\, f\, (\mc{d}\, \mc{T}))))
\end{align*}

\subsection{Gráficas}
\label{sec:estructura-graficas}

Las gráficas son una generalización de árboles en donde se admiten \emph{ciclos}, es decir, a partir de un vértice, se pudieran encontrar dos caminos de aristas para llegar a otro vértice. Consideremos la gráfica mostrada en la \autoref{fig:graph}.

\begin{figure}[!htbp]
  \centering
  \begin{tikzpicture}
    \node[circle,draw] (1) {};
    \node (label1) [above=0cm of 1] {$ v_{1} $};
    \node[circle,draw] (2) [below left of=1] {};
    \node (label2) [left=0cm of 2] {$ v_{2} $};
    \node[circle,draw] (3) [below right of=1] {};
    \node (label3) [below=0cm of 3] {$ v_{3} $};
    \node[circle,draw] (4) [right of=3] {};
    \node (label4) [right=0cm of 4] {$ v_{4} $};

    \path[draw,thick]
    (1) edge node {} (2)
    (1) edge node {} (3)
    (2) edge node {} (3)
    (3) edge node {} (4);
  \end{tikzpicture}
  \caption{Ejemplo de gráfica}
  \label{fig:graph}
\end{figure}

Una posible codificación para las gráficas es como \emph{lista de adyacencia}: Una gráfica $ \mc{G} $ es representada como una lista con un elemento por vértice; cada elemento de esta lista es un par $ \langle v : \mc{L}_{v} \rangle $ donde $ v $ es un término que representa al vértice y $ \mc{L}_{v} $ es una lista con un elemento por vértice en el que $ v $ incida; cada elemento de $ \mc{L}_{v} $ es un término que representa a un vértice.

La gráfica de la \autoref{fig:graph} se representa como lista de adyacencia de la siguiente manera:
\begin{align*}
  \langle &\langle v_{1} : \langle v_{2},\ v_{3} \rangle \rangle\\
          &\langle v_{2} : \langle v_{1},\ v_{3} \rangle \rangle\\
          &\langle v_{3} : \langle v_{1},\ v_{2},\ v_{4} \rangle \rangle\\
          &\langle v_{4} : \langle v_{3} \rangle \rangle \rangle
\end{align*}

La única restricción que se debe incorporar a la codificación es que los vértices puedan ser distinguidos entre sí, esto se puede lograr por ejemplo, utilizando numerales de Church para representar vértices.

Ya que las gráficas son codificadas utilizando términos $ λ $ conocidos, los algoritmos que manipular gráficas pueden ser codificados de manera similar a los presentados en la \autoref{sec:estructura-listas} y \ref{sec:estructura-arboles}.

\subsection{Términos \texorpdfstring{$ \bs{λ} $}{lambda}}
\label{sec:estructura-lambda}

En esta sección se aborda el mecanismo mediante el cual es posible codificar términos $ λ $ dentro del cálculo $ λ $ y de esta manera, tener las herramientas para codificar algoritmos que manipulen términos $ λ $ como la sustitución, la $ α $-contracción, la $ β $-contracción, encontrar los subtérminos de un término, etc.

El hecho de poder codificar términos $ λ $ en el cálculo $ λ $ no introduce problema algúno. Este técnica de representar un lenguaje en sí mismo es común, por ejemplo, en compiladores de \texttt{C} escritos en \texttt{C} o en intérpretes de \texttt{Lisp} escritos en \texttt{Lisp}.

Partiendo de la \autoref{defn:terminos} de término $ λ $, se modifica el conjunto $ V $ para que en lugar de ser $ \{v_{0},\ v_{00},\ v_{000},\ ... \} $, sea $ \{ 0,\ 1,\ 2,\ ... \} $, es decir, que los átomos sean numeros naturales; además en lugar de construír las abstracciones y aplicaciones a partir de símbolos, se construyen a partir de pares.
\begin{defn}[Términos $ λ $]
  El conjunto $ Λ^{\prime} $ tiene elementos que son pares y números naturales. $ Λ^{\prime} $ es el conjunto más pequeño que satisface:
  \label{defn:cod-terminos}
  \begin{subequations}
    \begin{align}
      \label{cod-terminos:atomos} \tag{a}
      n \in \mathbb{N} & \implies n \in Λ^{\prime} \\
      \label{cod-terminos:abstracciones} \tag{b}
      M \in Λ^{\prime},\ n \in \mathbb{N} & \implies  \langle n : M \rangle \in Λ^{\prime} \\
      \label{cod-terminos:aplicaciones} \tag{c}
      M,\ N \in Λ^{\prime} & \implies \langle M : N \rangle \in Λ^{\prime}
    \end{align}
  \end{subequations}
\end{defn}

El problema con esta definición de términos $ λ $ es que no es posible distinguir al término $ (λx.x) $ de $ (x\, x) $. Además, la mayoría de los algoritmos que manipulan términos $ λ $ es crucial poder diferenciar cuando un término $ M $ es un átomo, una abstracción o una aplicación. Para arreglar la \autoref{defn:cod-terminos} se pueden utilizar números que etiqueten cada tipo de término, asignando el número 1 a los átomos, el 2 a las abstracciones y el 3 a las aplicaciones, se tiene una correspondencia uno a uno entre $ Λ $ y $ Λ^{\prime} $.

\begin{defn}[Términos $ λ $]
  El conjunto $ Λ^{\prime} $ tiene elementos que son pares. $ Λ^{\prime} $ es el conjunto más pequeño que satisface:
  \label{defn:cod-terminos2}
  \begin{subequations}
    \begin{align}
      \label{cod-terminos:atomos2} \tag{a}
      n \in \mathbb{N} & \implies \langle 1 : n \rangle \in Λ^{\prime} \\
      \label{cod-terminos:abstracciones2} \tag{b}
      M \in Λ^{\prime},\ n \in \mathbb{N} & \implies \langle 2 : \langle n : M \rangle \rangle \in Λ^{\prime} \\
      \label{cod-terminos:aplicaciones2} \tag{c}
      M,\ N \in Λ^{\prime} & \implies \langle 3 : \langle M : N \rangle \rangle \in Λ^{\prime}
    \end{align}
  \end{subequations}
\end{defn}

Con esta definición, el término $ (λx.x) $ se puede representar como $ \langle 2 : \langle \langle 1 : n \rangle : \langle 1 : n \rangle \rangle \rangle $ y el término $ (x\, x) $ se puede representar como $ \langle 3 : \langle \langle 1 : n \rangle : \langle 1 : n \rangle \rangle \rangle $.

Codificaciones adecuadas para los constructores y selectores de esta representación de términos $ λ $ se basan en las técnicas abordadas a lo largo de este capítulo.

Sea $ \mc{T} $ la codificación de un término $ λ $, los predicados $ \mathrm{atomo}_{?} $, $ \mathrm{abstraccion}_{?} $ y $ \mathrm{aplicacion}_{?} $ permiten determinar la clase de término que es $ \mc{T} $:
\begin{align}
  \label{eq:cod-lambda-preds}
  \mathrm{atomo}_{?} &\synteq λ\mc{T}.\cn{=}_{?}\, \cn{1} (\mc{a}\, \mc{T})\\
  \mathrm{abstraccion}_{?} &\synteq λ\mc{T}.\cn{=}_{?}\, \cn{2} (\mc{a}\, \mc{T})\\
  \mathrm{aplicacion}_{?} &\synteq λ\mc{T}.\cn{=}_{?}\, \cn{3} (\mc{a}\, \mc{T})
\end{align}

Utilizando estos tres predicados y el término $ \bs{\prec} $ se puede codificar un término que funcione como una estructura de control similar a $ \bs{\prec} $ que a partir de un término codificado $ \mc{T} $ y cuatro términos $ M_{1} $, $ M_{2} $, $ M_{3} $ y $ M_{4} $ compute $ M_{1} $ si $ \mc{T} $ es un átomo, $ M_{2} $ si es una abstracción, $ M_{3} $ si es una aplicación y $ M_{4} $ en otro caso:
\begin{equation}
  \label{eq:cod-lambda-match}
  \mathrm{ramifica} \synteq λ\mc{T}\, x\, y\, z\, e.\bs{\prec} (\mathrm{atomo}_{?}\, \mc{T}) x (\bs{\prec} (\mathrm{abstraccion}_{?}\, \mc{T}) y (\bs{\prec} (\mathrm{aplicacion}_{?}\, \mc{T}) z\, e))
\end{equation}

Los constructores para cada clase de término $ λ $ siguen la \autoref{defn:cod-terminos2}:
\begin{align}
  \label{eq:cod-lambda-cons}
  \mathrm{atomo} &\synteq λn.\mc{p}\, \cn{1}\, n \\
  \mathrm{abstraccion} &\synteq λn\, x.\mc{p}\, \cn{2} (\mc{p}\, n\, x) \\
  \mathrm{aplicacion} &\synteq λx\, y.\mc{p}\, \cn{3} (\mc{p}\, x\, y)
\end{align}

Los constructores en \eqref{eq:cod-lambda-cons} deben satisfacer los predicados en \eqref{eq:cod-lambda-preds}, se corrobora esto con los siguientes desarrollos.
\begin{align*}
  \mathrm{atomo}_{?}\, (\mathrm{atomo}\, \cn{n}) &\synteq (λ\mc{T}.\cn{=}_{?}\, \cn{1}\, (\mc{a}\, \mc{T}))\, (\mathrm{atomo}\, \cn{n}) \\
                                               &\contract{β} \cn{=}_{?}\, \cn{1}\, (\mc{a} (\mathrm{atomo}\, \cn{n})) \\
                                               &\synteq \cn{=}_{?}\, \cn{1}\, (\mc{a}\, ((λn.\mc{p}\, \cn{1}\, n)\, \cn{n})) \\
                                               &\contract{β} \cn{=}_{?}\, \cn{1}\, (\mc{a}\, (\mc{p}\, \cn{1}\, \cn{n})) \\
                                               &\reduce{β} \cn{=}_{?}\, \cn{1}\, \cn{1} \reduce{β} \bs{T}
\end{align*}
\begin{align*}
  \mathrm{atomo}_{?}\, (\mathrm{aplicacion}\, M\, N) &\synteq (λ\mc{T}.\cn{=}_{?}\, \cn{1}\, (\mc{a}\, \mc{T}))\, (\mathrm{aplicacion}\, M\, N) \\
                                               &\contract{β} \cn{=}_{?}\, \cn{1}\, (\mc{a}\, (\mathrm{aplicacion}\, M\, N)) \\
                                               &\synteq \cn{=}_{?}\, \cn{1}\, (\mc{a}\, ((λx\, y.\mc{p}\, \cn{3}\, (\mc{p}\, x\, y))\, M\, N)) \\
                                               &\reduce{β} \cn{=}_{?}\, \cn{1}\, (\mc{a}\, (\mc{p}\, \cn{3}\, (\mc{p}\, M\, N))) \\
                                               &\reduce{β} \cn{=}_{?}\, \cn{1}\, \cn{3} \reduce{β} \bs{F}
\end{align*}
Corroborar que las otras combinaciones de aplicaciones se reducen de manera correcta implica un desarrollo similar.

Los selectores para cada clase de término $ λ $ se codifican en función de la cantidad de elementos que conforman al término. Los átomos tienen un selector, las abstracciones dos y las aplicaciones dos. Debido a que las abstracciones comparten la misma estructura, sus selectores son los mismos.

\begin{align}
  \label{eq:cod-lambda-selec}
  \mathrm{atomo}_{n} &\synteq λx.\mc{d}\, x \\
  \mathrm{abstraccion}_{v} &\synteq λx.\mc{a} (\mc{d}\, x) \\
  \mathrm{abstraccion}_{t} &\synteq λx.\mc{d} (\mc{d}\, x) \\
  \mathrm{aplicacion}_{o} &\synteq λx.\mc{a} (\mc{d}\, x) \\
  \mathrm{aplicacion}_{a} &\synteq λx.\mc{d} (\mc{d}\, x)
\end{align}

La técnica de etiquetar pares para poder determinar el ``tipo'' de estructura es muy utilizada en la implementación de lenguajes de programación con verificación dinámica de tipos. La flexibilidad de la técnica permite codificar todos los objetos que se han abordado en este trabajo como pares etiquetados, los valores de verdad pueden etiquetarse con el 4, los numerales de Church con el 5, los pares con el 6 y así sucesivamente. De esta manera la codificación de algoritmos en el cálculo $ λ $  puede ser más robusta, evitando errores que puedan surgir al reducir los términos.

%%% Local Variables:
%%% mode: latex
%%% TeX-master: "../main"
%%% End:


\chapter*{Conclusión}
\addcontentsline{toc}{chapter}{Conclusión}
\markright{Conclusión}
\label{ch:conclusion}
En este trabajo se presentaron las ideas generales del cálculo \( λ \), se formalizaron utilizando como herramientas matemáticas las teorías formales y los sistemas de reducción, también se desarrollaron varias maneras para representar información compleja y algoritmos tanto iterativos como recursivos utilizando únicamente funciones y aplicación de funciones. Sin embargo, este trabajo comprende sólo una pequeña parte del estudio del cálculo \( λ \).

Este trabajo puede ser ameno de leer para algunas personas y para otras no. Intenté mantener un balance entre los aspectos conceptuales y formales, pero el contenido del trabajo terminó siendo más técnico de lo que esperaba. Los aspectos pragmáticos y las aplicaciones del cálculo \( λ \) están ausentes del trabajo y esto puede resultan contraproducente para interesar a un lector que nunca ha sido expuesto al cálculo \( λ \).

Algunos temas que se pueden estudiar después de leer este trabajo son las extensiones del cálculo con lógica ilativa \cite[pp.~573--576]{Barendregt:Bible}, teoría de tipos \cite{Pierce:TypesAndPLangs}, semántica denotacional \cite{Stoy:Semantics} y programación funcional.

%%% Local Variables:
%%% mode: latex
%%% TeX-master: "main"
%%% End:


\appendix

\chapter{Programación de codificaciones}
\label{ap:lambda-scheme}
El siguiente código es una implementación literal de los combinadores \( \bs{S} \), \( \bs{K} \) e \( \bs{I} \) descritos en \ref{defn:ski}; los valores de verdad y las operaciones booleanas descritas en \ref{sec:algebra-booleana}; los numerales de Church y las operaciones aritméticas descritas en \ref{sec:aritmetica}.

El lenguaje de programación utilizado en esta implementación es \texttt{Scheme}.

\begin{lstlisting}[language=scheme]
;; 
;; Combinadores SKI
;;

(define :I
  ;; Combinador identidad @<\( \mathcolor{gray}{\bs{I}} \)>@
  (lambda (:x) :x))

(define :K
  ;; Combinador constante @<\( \mathcolor{gray}{ \bs{K}  } \)>@
  (lambda (:x) (lambda (:y) :x)))

(define :S
  ;; Combinador de sustitucion @<\( \mathcolor{gray}{ \bs{S}  } \)>@
  (lambda (:x) (lambda (:y) (lambda (:z) ((:x :z) (:y :z))))))

;; 
;; Algebra booleana
;;

(define :T
  ;; Combinador verdadero @<\( \mathcolor{gray}{ \bs{T}  } \)>@
  :K)

(define :F
  ;; Combinador falso @<\( \mathcolor{gray}{ \bs{F}  } \)>@
  (lambda (:x) (lambda (:y) :y)))

(define (boolean->church b)
  ;; procedimiento que toma un objeto booleano de scheme y regresa el
  ;; combinador que codifica al valor
  (if b :T :F))

(define (church->boolean :p)
  ;; Procedimiento que toma un combinador booleano y regresa el objeto
  ;; booleano de scheme que representa al valor
  ((:p #t) #f))

(define :IF
  ;; Combinador condicional @<\( \mathcolor{gray}{ \bs{\prec}  } \)>@
  (lambda (:p) (lambda (:m) (lambda (:n) ((:p :m) :n)))))

(define :NOT
  ;; Combinador negacion @<\( \mathcolor{gray}{ \bs{\lnot}  } \)>@
  (lambda (:p) (((:IF :p) :F) :T)))

(define :OR
  ;; Combinador disyuncion @<\( \mathcolor{gray}{ \bs{\lor}  } \)>@
  (lambda (:p1) (lambda (:p2) (((:IF :p1) :T)(((:IF :p2) :T) :F)))))

(define :AND
  ;; Combinador conjuncion @<\( \mathcolor{gray}{ \bs{\land}  } \)>@
  (lambda (:p1) (lambda (:p2) (((:IF :p1) (((:IF :p2) :T) :F) :F)))))

;; 
;; Aritmetica
;; 
(define :0
  ;; Combinador cero @<\( \mathcolor{gray}{ \cn{0}  } \)>@
  :F)

(define :0?
  ;; Combinador predicado @<\( \mathcolor{gray}{ \cn{0}?  } \)>@  para determinar si un termino lambda es el
  ;; combinador cero
  (lambda (:n) ((:n (:K :F)) :T)))

(define :SUCC
  ;; Combinador sucesor @<\( \mathcolor{gray}{ \cn{+}_{1}  } \)>@
  (lambda (:n) (lambda (:x) (lambda (:y) (:x ((:n :x) :y))))))

(define :1
  ;; Combinador uno @<\( \mathcolor{gray}{ \cn{1}  } \)>@
  (:SUCC :0))

(define :2
  ;; Combinador dos @<\( \mathcolor{gray}{ \cn{2}  } \)>@
  (:SUCC :1))

(define :3
  ;; Combinador tres @<\( \mathcolor{gray}{ \cn{3}  } \)>@
  (:SUCC :2))

(define :4
  ;; Combinador cuatro @<\( \mathcolor{gray}{ \cn{4}  } \)>@
  (:SUCC :3))

(define :+
  ;; Combinador adicion @<\( \mathcolor{gray}{ \cn{+}  } \)>@
  (lambda (:m) (lambda (:n) ((:n :SUCC) :m))))

(define :*
  ;; Combinador multiplicacion @<\( \mathcolor{gray}{ \cn{\times}  } \)>@
  (lambda (:m) (lambda (:n) ((:n (:+ :m)) :0))))

(define :^
  ;; Combinador exponenciacion @<\( \mathcolor{gray}{ \cn{\uparrow}  } \)>@
  (lambda (:m) (lambda (:n) ((:n (:* :m)) :1))))

(define (number->church n)
  ;; Procedimiento que toma un objeto numerico de scheme y regresa el
  ;; combinador que codifica al valor
  (if (zero? n)
      :0
      (:SUCC (number->church (- n 1)))))

(define (church->number :n)
  ;; Procedimiento que toma un combinador numerico y regresa el valor de
  ;; scheme que representa el valor
  ((:n (lambda (x) (+ 1 x))) 0))

(define :0*
  ;; Combinador cero modificado @<\( \mathcolor{gray}{ \cn{0}^{\prime}  } \)>@ (utilizado para computar el predecesor)
  (lambda (:x) (lambda (:y) (lambda (:z) :y))))

(define :SUCC*
  ;; Combinador sucesor modificado @<\( \mathcolor{gray}{ \cn{+}_{1}^{\prime}  } \)>@  (utilizado para computar el predecesor)
  (lambda (:n*) (lambda (:x) (lambda (:y) (lambda (:z) (((:n* :x) (:z :y)) :x))))))

(define :PRED
  ;; Combinador predecesor @<\( \mathcolor{gray}{ \cn{-}_{1}  } \)>@
  (lambda (:n) (lambda (:x) (lambda (:y) (((((:n :SUCC*) :0*) :x) :y) :I)))))

(define :-
  ;; Combinador sustraccion @<\( \mathcolor{gray}{ \cn{-}  } \)>@
  (lambda (:m) (lambda (:n) ((:n :PRED) :m))))

\end{lstlisting}

%%% Local Variables:
%%% mode: latex
%%% TeX-master: "main"
%%% End:


\chapter{Intérprete Lambda}
\label{ap:lambda}
Este apéndice consiste de un programa literario\footnote{La programación literaria es un paradigma propuesto por Donald E. Knuth en donde los programas son escritos en un orden que favorece la manera de pensar del programador más que el orden de ejecución del código por la computadora.} conformado por un intérprete y un editor estructural para el cálculo \( λ \). Se utilizó la herramienta \texttt{Noweb}, escrita por Norman Ramsey\footnote{Para más información, visitar el sitio \url{http://www.cs.tufts.edu/~nr/noweb/}}, para extraer del documento el código en \texttt{Racket} del programa ejecutable y el código en~\LaTeX~ para la composición tipográfica del presente apéndice.

La versión de \texttt{Racket} para la que este programa fue programado y probado es la \( 6.6 \). La versión del programa \texttt{Noweb} es la \( 2.11 \) adaptada al Español.

\includepdf[pages = -, pagecommand = {\thispagestyle{empty}}]{programas/lambda/main.pdf}

\nocite{*}
\bibliographystyle{acm}
\thispagestyle{empty}
{\small
\bibliography{bibliografia}}
\addcontentsline{toc}{chapter}{Bibliografía}

\end{document}


%%% Local Variables:
%%% mode: latex
%%% TeX-master: t
%%% End:
