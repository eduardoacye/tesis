\documentclass[10pt,letterpaper]{article}

\usepackage[utf8]{inputenc}
\usepackage[spanish]{babel}
\usepackage[T1]{fontenc}



\usepackage{microtype}
\usepackage{kpfonts}
\usepackage{tocenter}
\ToCenter{30pc}{50.5pc}

\usepackage{newunicodechar}
\newunicodechar{λ}{\lambda}

\newcommand{\RomanNumeral}[1]{\uppercase\expandafter{\romannumeral #1\relax}}
\newcommand{\namesigdate}[2][6cm]{%
  \begin{tabular}{@{}p{#1}@{}}
    #2 \\[2\normalbaselineskip] \hrule \\[0pt]
    {\small \textit{Firma}} \\[2\normalbaselineskip] \hrule \\[0pt]
    {\small \textit{Fecha}}
  \end{tabular}
}

\newcommand{\titulo}{Proyecto de tesis}
\newcommand{\tesis}{El cálculo \( λ \) y los fundamentos de la computación}
\newcommand{\estudiante}{Eduardo Acuña Yeomans}
\newcommand{\director}{Martín Eduardo Frías Armenta}

\begin{document}

\thispagestyle{empty}

\begin{center}
  \Large{\textsc{\tesis}}\\
  \normalsize
  \vspace*{25pt}
  \begin{tabular}{r l}
    \textbf{\textsc{Proponente}} & \estudiante \\
    \textbf{\textsc{Director}} & \director
  \end{tabular}
  \vspace*{25pt}
\end{center}

\begin{abstract}
  El cálculo \( λ \) es un lenguaje y herramienta para el estudio del cómputo, ha sido utilizado para explorar las cuestiones más fundamentales de la teoría de la computación como el significado de algoritmo y las limitaciones intrínsecas de ellos. Este trabajo presenta una introducción al estudio del cálculo \( λ \) puro, abordando desde los aspectos superficiales como la notación y las estructuras sintácticas que permite expresar, hasta el significado de las expresiones y las ideas subyacentes a estas. Se presenta un tratamiento matemático del tema, utilizando dos perspectivas de formalización: con teorías formales y con sistemas de reducción. También se presenta un tratamiento computacional del tema, desarrollando representaciones algorítmicas del álgebra booleana, la aritmética elemental, los procesos recursivos y las estructuras recursivas.
\end{abstract}

\begin{center}
  \large
  \textsc{\RomanNumeral{1}. Contenido del trabajo}
\end{center}

El trabajo consiste en una revisión de la literatura sobre el cálculo \( λ \) y los temas matemáticos y computacionales asociados a él. Se presentan distintos enfoques y metodologías para el estudio formal del cálculo y para su uso como herramienta matemática y computacional.

El trabajo se constituye de tres capítulos y dos apéndices:
\begin{itemize}
\item El primer capítulo presenta las nociones informales del cálculo \( λ \) y se aborda la notación, las operaciones y las equivalencias del cálculo de forma introductoria.
\item El segundo capítulo presenta la formalización matemática del cálculo \( λ \) haciendo uso de teorías formales y sistemas de reducción.
\item El tercer capítulo presenta la codificación de objetos matemáticos en el cálculo, enfizando aspectos computacionales.
\item El primer apéndice presenta la implementación de los resultados del tercer capítulo.
\item El segundo apéndice presenta la implementación de un intérprete y editor estructural de expresiones del cálculo \( λ \) para la exploración interactiva de los temas abordados en el cuerpo de la tesis.
\end{itemize}
  

\begin{center}
  \large
  \textsc{\RomanNumeral{2}. Objetivos y metas}
\end{center}

\paragraph{\RomanNumeral{2}.\RomanNumeral{1}. Objetivo general.} Introducir el cálculo \( λ \) y enfatizar su relación con los fundamentos de la computación.

\paragraph{\RomanNumeral{2}.\RomanNumeral{2}. Objetivos particulares.}
\begin{itemize}
\item Estudiar de manera informal y formal el cálculo \( λ \),
\item Desarrollar ejemplos y mostrar resultados interesantes del cálculo \( λ \),
\item Explorar aplicaciones del cálculo \( λ \).
\end{itemize}

\paragraph{\RomanNumeral{2}.\RomanNumeral{3}. Metas.}
\begin{itemize}
\item Redacción clara y entendible para personas con una formación teórica básica de Computación y Matemáticas,
\item Interesar a los lectores del trabajo en el tema.
\end{itemize}

\begin{center}
  \large
  \textsc{\RomanNumeral{3}. Cronología de actividades}
\end{center}

\paragraph{\RomanNumeral{6}. Cronograma de actividades.} El desarrollo de este trabajo comprende desde la segunda mitad del 2014 hasta la primera mitad del 2016. El trabajo realizado entre la segunda mitad del 2014 y la primer mitad del 2015 consistió en estudiar el tema y establecer el contenido general del trabajo. El trabajo realizado entre la segunda mitad del 2015 y la primer mitad del 2016 consistió en redactar un manuscrito. A lo largo de estos dos años se escribieron varios programas para facilitar la exploración de los temas.

\begin{center}
  \large
  \textsc{\RomanNumeral{4}. Firma del estudiante y del director de tesis}
\end{center}

\vspace*{50pt}

\noindent \namesigdate{\estudiante} \hfill \namesigdate{\director}



\end{document}

%%% Local Variables:
%%% mode: latex
%%% TeX-master: t
%%% End: