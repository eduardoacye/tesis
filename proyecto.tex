\documentclass[12pt,letterpaper]{article}

\usepackage[utf8]{inputenc}
\usepackage[spanish]{babel}
\usepackage[T1]{fontenc}

\usepackage{microtype}
\usepackage{kpfonts}
\usepackage{tocenter}
\ToCenter{30pc}{50.5pc}

\usepackage{newunicodechar}
\newunicodechar{λ}{\lambda}

\newcommand{\RomanNumeral}[1]{\uppercase\expandafter{\romannumeral #1\relax}}
\newcommand{\namesigdate}[2][6cm]{%
  \begin{tabular}{@{}p{#1}@{}}
    #2 \\[2\normalbaselineskip] \hrule \\[0pt]
    {\small \textit{Firma}} \\[2\normalbaselineskip] \hrule \\[0pt]
    {\small \textit{Fecha}}
  \end{tabular}
}

\newcommand{\titulo}{Proyecto de tesis}
\newcommand{\tesis}{El cálculo \( λ \) y los fundamentos de la computación}
\newcommand{\estudiante}{Eduardo Acuña Yeomans}
\newcommand{\director}{Martín Eduardo Frías Armenta}

\begin{document}

\thispagestyle{empty}

\begin{center}
  \Huge{\textbf{\titulo}}\\
  \vspace*{10pt}
  \small{\emph{Licenciatura en Ciencias de la Computación}}\\
  \small{\emph{Departamento de Matemáticas}}
  \vspace*{25pt}
\end{center}

\paragraph{\RomanNumeral{1}. Título tentativo de tesis.} `` \tesis ''

\paragraph{\RomanNumeral{2}. Nombre del estudiante que la realiza.} \estudiante

\paragraph{\RomanNumeral{3}. Nombre del director de tesis.} \director

\paragraph{\RomanNumeral{4}. Resumen del trabajo que se pretende realizar.} El trabajo consiste en una revisión de la literatura sobre el cálculo \( λ \) y los temas computacionales asociados a él. Se presentarán distintos enfoques y metodologías para el estudio formal del cálculo y para su uso como herramienta matemática y computacional. 

\paragraph{\RomanNumeral{5}.\RomanNumeral{1}. Objetivo general.} Introducir el cálculo \( λ \) a estudiantes de la Licenciatura en Ciencias de la Computación que no han sido expuestos al tema.

\paragraph{\RomanNumeral{5}.\RomanNumeral{2}. Objetivos particulares.}
\begin{itemize}
\item Presentar de manera informal y formal el cálculo \( λ \),
\item Desarrollar ejemplos y mostrar resultados interesantes del cálculo \( λ \),
\item Presentar aplicaciones del cálculo \( λ \).
\end{itemize}


\paragraph{\RomanNumeral{5}.\RomanNumeral{3}. Metas.}
\begin{itemize}
\item Redacción clara y entendible para personas con una formación teórica básica de Computación y Matemáticas,
\item Interesar a los lectores del trabajo en el tema.
\end{itemize}

\paragraph{\RomanNumeral{5}.\RomanNumeral{4}. Alcances.} Pretendo presentar un trabajo que abarque los conceptos básicos sobre el cálculo \( λ \), al menos dos aspectos teóricos y dos de aplicación, al menos un programa de computadora que aborde algún aspecto tratado en el trabajo.

\paragraph{\RomanNumeral{6}. Cronograma de actividades.} Pendiente...

\paragraph{\RomanNumeral{7}. Firma del estudiante y del director de tesis.}\

\vspace*{50pt}

\noindent \namesigdate{\estudiante} \hfill \namesigdate{\director}



\end{document}

%%% Local Variables:
%%% mode: latex
%%% TeX-master: t
%%% End: