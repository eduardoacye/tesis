\paragraph{Con respecto a comentarios} Para hacer comentarios en el documento propongo que se utilice el entorno que corresponde al nombre del revisor (\texttt{Gutu}, \texttt{Waissman}, \texttt{Frias}, \texttt{Jesus}, \texttt{Eduardo}), en caso que se desee hacer un comentario anónimo, hay un entorno anónimo (\texttt{Anonimo}) también. Los comentarios pueden ser anidados para comentar comentarios. Ejemplo de como se ven y como se escriben:
\begin{Gutu}
  Comentario de Olivia
  \begin{Waissman}
    Comentario de Julio
    \begin{Frias}
      Comentario de Martín
    \end{Frias}
  \end{Waissman}
  \begin{Jesus}
    Comentario de Jesús
  \end{Jesus}
\end{Gutu}
\begin{Anonimo}
  Comentario anónimo
\end{Anonimo}

\begin{lstlisting}[language={[LaTeX]TeX},literate={ñ}{{\~n}}{1} {Ñ}{{\~N}}{1} {á}{{\'a}}{1} {Á}{{\'A}}{1} {é}{{\'e}}{1} {É}{{\'E}}{1} {í}{{\'i}}{1} {Í}{{\'I}}{1} {ó}{{\'o}}{1} {Ó}{{\'O}}{1} {ú}{{\'u}}{1} {Ú}{{\'U}}{1}]
\begin{Gutu}
  Comentario de Olivia
  \begin{Waissman}
    Comentario de Julio
    \begin{Frias}
      Comentario de Martín
    \end{Frias}
  \end{Waissman}
  \begin{Jesus}
    Comentario de Jesús
  \end{Jesus}
\end{Gutu}
\begin{Anonimo}
  Comentario anónimo
\end{Anonimo}
\end{lstlisting}

\paragraph{Con respecto al contenido} Hay muchos aspectos del trabajo que me gustaría mejorar, creo que la redacción pudiera ser mucho más clara y concisa. Además me da la impresión que el trabajo está ``sobreestructurado'', presentar los aspectos informales antes de la formalización y las codificaciones me parece lo más apropiado, pero al leer el trabajo desde el inicio al final me quedo con la impresión de que los cambios de capítulo son muy repentinos.

Tengo algunas autocríticas de la manera en como presento el contenido del trabajo pero quisiera esperar a recibir retroalimentación desde otros puntos de vista.

\paragraph{Con respecto a la forma} Esto lo puse como última prioridad, lo que le hace falta al trabajo es uniformizar las enumeraciones de términos \( λ \), en ocasiones utilizo letras, en ocasiones las enumero por capítulo. Otra cuestión algo superficial es determinar si es conveniente que los ejemplos sean especificados como bloques de contenido, inicialmente los escribí en su propio entorno de \LaTeX~ para poder hacer referencia a ellos, pero creo que casi no hago referencias a ejemplos y al leer el trabajo siento que se pierde el flujo de la lectura.

\vfill
\begin{center}
Gracias por darse el tiempo de leer el trabajo.

\qquad ---Eduardo
\end{center}